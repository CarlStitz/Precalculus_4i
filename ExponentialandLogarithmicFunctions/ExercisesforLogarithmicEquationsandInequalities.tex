\label{ExercisesforLogarithmicEquationsandInequalities}

In Exercises \ref{solvelogeqexfirst} - \ref{solvelogeqexlast}, solve the equation analytically.

\begin{multicols}{2}
\begin{enumerate}

\item $\log(3x-1) = \log(4-x)$  \phantom{$\log_{2}\left(x^{3}\right) = \log_{2}(x)$} \label{solvelogeqexfirst}

\item $\log_{2}\left(x^{3}\right) = \log_{2}(x)$

\setcounter{HW}{\value{enumi}}
\end{enumerate}
\end{multicols}

\begin{multicols}{2}
\begin{enumerate}
\setcounter{enumi}{\value{HW}}

\item $\ln\left(8-t^2\right)=\ln(2-t)$ \vphantom{$\log_{5}\left(18-t^2\right) = \log_{5}(6-t)$}

\item $\log_{5}\left(18-t^2\right) = \log_{5}(6-t)$

\setcounter{HW}{\value{enumi}}
\end{enumerate}
\end{multicols}

\begin{multicols}{2}
\begin{enumerate}
\setcounter{enumi}{\value{HW}}

\item $\log_{3}(7-2x) = 2$ \vphantom{$\log_{\frac{1}{2}} (2x-1) = -3$}
\item $\log_{\frac{1}{2}} (2x-1) = -3$

\setcounter{HW}{\value{enumi}}
\end{enumerate}
\end{multicols}

\begin{multicols}{2}
\begin{enumerate}
\setcounter{enumi}{\value{HW}}

\item $\ln\left(t^2-99\right) = 0$
\item $\log(t^2-3t) = 1$

\setcounter{HW}{\value{enumi}}
\end{enumerate}
\end{multicols}

\begin{multicols}{2}
\begin{enumerate}
\setcounter{enumi}{\value{HW}}

\item $\log_{125} \left(\dfrac{3x-2}{2x+3}\right)=\dfrac{1}{3}$

\item $\log\left(\dfrac{x}{10^{-3}}\right) = 4.7$ \vphantom{$\log_{125} \left(\dfrac{3x-2}{2x+3}\right)$} \label{sixfourRichterequ}


\setcounter{HW}{\value{enumi}}
\end{enumerate}
\end{multicols}


\begin{multicols}{2}
\begin{enumerate}
\setcounter{enumi}{\value{HW}}

\item $-\log(x) = 5.4$ \vphantom{$10\log\left(\dfrac{t}{10^{-12}}\right)$} \label{sixfourpHequ}
\item $10\log\left(\dfrac{x}{10^{-12}}\right) = 150$ \label{sixfourdecibelequ}

\setcounter{HW}{\value{enumi}}
\end{enumerate}
\end{multicols}

\begin{multicols}{2}
\begin{enumerate}
\setcounter{enumi}{\value{HW}}

\item $6-3\log_{5}(2t)=0$
\item $3\ln(t)-2=1-\ln(t)$

\setcounter{HW}{\value{enumi}}
\end{enumerate}
\end{multicols}

\begin{multicols}{2}
\begin{enumerate}
\setcounter{enumi}{\value{HW}}

\item $\log_{3}(t - 4) + \log_{3}(t + 4) = 2$

\item $\log_{5}(2t + 1) + \log_{5}(t + 2) = 1$

\setcounter{HW}{\value{enumi}}
\end{enumerate}
\end{multicols}

\begin{multicols}{2}
\begin{enumerate}
\setcounter{enumi}{\value{HW}}

\item $\log_{169}(3x + 7) - \log_{169}(5x - 9) = \dfrac{1}{2}$

\item $\ln(x+1) - \ln(x) = 3$ \vphantom{$\log_{169}(3x + 7)$}

\setcounter{HW}{\value{enumi}}
\end{enumerate}
\end{multicols}

\begin{multicols}{2}
\begin{enumerate}
\setcounter{enumi}{\value{HW}}

\item $2\log_{7}(t) = \log_{7}(2) + \log_{7}(t+12)$

\item $\log(t) - \log(2) = \log(t+8)  - \log(t+2)$

\setcounter{HW}{\value{enumi}}
\end{enumerate}
\end{multicols}

\begin{multicols}{2}
\begin{enumerate}
\setcounter{enumi}{\value{HW}}

\item $\log_{3}(x) = \log_{\frac{1}{3}}(x) + 8$

\item $\ln(\ln(x)) = 3$

\setcounter{HW}{\value{enumi}}
\end{enumerate}
\end{multicols}

\begin{multicols}{2}
\begin{enumerate}
\setcounter{enumi}{\value{HW}}

\item $\left(\log(t)\right)^2=2\log(t)+15$

\item $\ln(t^{2}) = (\ln(t))^{2}$ \label{solvelogeqexlast}

\setcounter{HW}{\value{enumi}}
\end{enumerate}
\end{multicols}


In Exercises \ref{solvelogineqexfirst} - \ref{solvelogineqexlast}, solve the inequality analytically.

\begin{multicols}{2}
\begin{enumerate}
\setcounter{enumi}{\value{HW}}

\item $\dfrac{1 - \ln(t)}{t^{2}} < 0$ \label{solvelogineqexfirst}
\item $t\ln(t) - t > 0$ \phantom{$\dfrac{1 - \ln(x)}{x^{2}} < 0$}  


\setcounter{HW}{\value{enumi}}
\end{enumerate}
\end{multicols}

\begin{multicols}{2}
\begin{enumerate}
\setcounter{enumi}{\value{HW}}

\item $10\log\left(\dfrac{x}{10^{-12}}\right) \geq 90$ \label{sixfourdecibelineq} 
\item $5.6 \leq \log\left(\dfrac{x}{10^{-3}}\right) \leq 7.1$ \label{sixfourRichterineq}


\setcounter{HW}{\value{enumi}}
\end{enumerate}
\end{multicols}

\begin{multicols}{2}
\begin{enumerate}
\setcounter{enumi}{\value{HW}}


\item $2.3 < -\log(x) < 5.4$ \label{sixfourpHineq} 

\item $\ln(t^{2}) \leq (\ln(t))^{2}$ \label{solvelogineqexlast} 

\setcounter{HW}{\value{enumi}}
\end{enumerate}
\end{multicols}

\pagebreak

In Exercises \ref{logeqcalcexfirst} - \ref{logeqcalcexlast}, use a graphing utility to help you solve the equation or  inequality.

\begin{multicols}{2}
\begin{enumerate}
\setcounter{enumi}{\value{HW}}

\item $\ln(t) = e^{-t}$ \label{logeqcalcexfirst} 
\item $\ln(x) = \sqrt[4]{x}$ 

\setcounter{HW}{\value{enumi}}
\end{enumerate}
\end{multicols}

\begin{multicols}{2}
\begin{enumerate}
\setcounter{enumi}{\value{HW}}

\item $\ln(t^{2} + 1) \geq 5$
\item $\ln(-2x^{3} - x^{2} + 13x - 6) < 0$ \label{logeqcalcexlast} 

\setcounter{HW}{\value{enumi}}
\end{enumerate}
\end{multicols}


In Exercises \ref{domaincomplicatedlogfirst} - \ref{domaincomplicatedloglast},  find the domain of the function.

\begin{multicols}{2} 
\begin{enumerate}
\setcounter{enumi}{\value{HW}}

\item \label{domaincomplicatedlogfirst}  $r(x) =   \dfrac{x}{1 - \ln(x)}$  %(-\infty, e) \cup (e, \infty)$

\item   $R(x) = \dfrac{x \ln(x)}{1 - \ln(x)}$   % $(0,e) \cup (e, \infty)$

\setcounter{HW}{\value{enumi}}
\end{enumerate}
\end{multicols}


\begin{multicols}{2} 
\begin{enumerate}
\setcounter{enumi}{\value{HW}}

\item     $s(t) = \sqrt{2 - \log(t)}$  \vphantom{$c(t) =  (2 \ln(t) -1)^{\frac{2}{3}}$} %$(0, 100]$
\item     $c(t) =  (2 \ln(t) -1)^{\frac{2}{3}}$  %$(0, \infty)$

\setcounter{HW}{\value{enumi}}
\end{enumerate}
\end{multicols}

\begin{multicols}{2} 
\begin{enumerate}
\setcounter{enumi}{\value{HW}}
  
\item     $\ell(t) = \ln( \ln(t))$  \vphantom{$L(x) = \log\left( \dfrac{x \ln(x)}{1 - \ln(x)} \right)$} %$(1, \infty)$    

\item  \label{domaincomplicatedloglast}    $L(x) = \log\left( \dfrac{x \ln(x)}{1 - \ln(x)} \right)$  %$(1,e)$ 


\setcounter{HW}{\value{enumi}}
\end{enumerate}
\end{multicols}


\begin{enumerate}
\setcounter{enumi}{\value{HW}}

\item \label{onetooneexpexercise} Since $f(x) = e^{x}$ is a strictly increasing function, if $a < b$ then $e^{a} < e^{b}$.  Use this fact to solve the inequality $\ln(2x + 1) < 3$ without a sign diagram. Use this technique to solve the inequalities in Exercises \ref{sixfourdecibelineq} - \ref{sixfourpHineq}. (Compare this to Exercise  \ref{onetoonelogexercise} in Section \ref{ExponentialEquationsandInequalities}.)

\item Solve $\ln(3 - y) - \ln(y) = 2x + \ln(5)$ for $y$.

\item In Example \ref{logfracinverse} we found the inverse of $f(x) = \dfrac{\log(x)}{1-\log(x)}$ to be $f^{-1}(x) = 10^{\frac{x}{x+1}}$.

\begin{enumerate}

\item Algebraically check our answer by verifying  $\left(f^{-1} \circ f\right)(x) = x$ for all $x$ in the domain of $f$ and that $\left(f \circ f^{-1}\right)(x) = x$ for all $x$ in the domain of $f^{-1}$.

\item Find the range of $f$ by finding the domain of $f^{-1}$.

\item Let $g(x) = \dfrac{x}{1 - x}$ and $h(x) = \log(x)$.  Show that $f = g \circ h$ and $(g \circ h)^{-1} = h^{-1} \circ g^{-1}$.\\


NOTE:  We know this is true in general by Exercise \ref{fcircginverse} in Section \ref{InverseFunctions}, but it's nice to see a specific example of the property.

\end{enumerate}

\item \label{inversehyptangent} Let $f(x) = \dfrac{1}{2}\ln\left(\dfrac{1 + x}{1 - x}\right)$.  Compute $f^{-1}(x)$ and find its domain and range.

\item Explain the equation in Exercise \ref{sixfourRichterequ} and the inequality in Exercise \ref{sixfourRichterineq} above in terms of the Richter scale for earthquake magnitude.  (See Exercise \ref{Richterexercise} in Section \ref{ExponentialFunctions}.)

\item Explain the equation in Exercise \ref{sixfourdecibelequ} and the inequality in Exercise \ref{sixfourdecibelineq} above in terms of sound intensity level as measured in decibels.  (See Exercise \ref{decibelexercise} in Section \ref{ExponentialFunctions}.)

\item Explain the equation in Exercise \ref{sixfourpHequ} and the inequality in Exercise \ref{sixfourpHineq} above in terms of the pH of a solution.  (See Exercise \ref{pHexercise} in Section \ref{ExponentialFunctions}.)

\item \label{powerloggrowthex} \begin{enumerate} \item\label{numericalinvestigationlimitlnxoverx} With the help of your classmates, numerically and graphically investigate $\ds{\lim_{x \rightarrow \infty}}$ $\frac{\ln(x)}{x^{p}}$ for various real number powers, $p > 0$.

 \item\label{numericalinvestigationlimitlnxtimesx} With the help of your classmates, numerically and graphically investigate $\ds{\lim_{x \rightarrow 0^{+}}}$ $x^{p} \, \ln(x) $ for various real number powers, $p > 0$.

\item  What do \ref{numericalinvestigationlimitlnxoverx} and \ref{numericalinvestigationlimitlnxtimesx} suggest about the relative growth rates of powers of $x$ and $\ln(x)$?

\end{enumerate}  

\setcounter{HW}{\value{enumi}}
\end{enumerate}

In Exercises \ref{logcurvesketchfirst}  - \ref{logcurvesketchlast} a function $f$ along with its derivatives $f'$ and $f''$ are given.

\begin{itemize}

\item  Find the domain of $f$.

\item  Find the $x$- and $y$-intercepts of the graph of each function, if any.

\item  Use limits to determine the end behavior and behavior at the endpoints of the domain.

\item  Use $f'$ to determine the open intervals over which $f$ is increasing or decreasing.

\item Determine the local extrema, if any.

\item  Use $f''$ to determine the open intervals over which the graph of $f$  is concave up or concave down.

\item  Determine the inflection points of the graph, if any.

\end{itemize}

\begin{enumerate}
\setcounter{enumi}{\value{HW}}

\item\label{logcurvesketchfirst}  $f(x) = \ln(x) - \ln(5-x)$,  $f'(x) = \dfrac{1}{x} + \dfrac{1}{5-x}$, $f''(x) = \dfrac{1}{(5-x)^2} - \dfrac{1}{x^2}$

\smallskip

\item\label{logcurvesketchlast}  $f(x) = \dfrac{\ln(x)}{x}$, $f'(x) = \dfrac{1 - \ln(x)}{x^2}$, $f''(x) = \dfrac{2 \ln(x) - 3}{x^3}$.


\setcounter{HW}{\value{enumi}}
\end{enumerate}



\newpage

\subsection{Answers}
\begin{multicols}{3}
\begin{enumerate}

\item $x = \frac{5}{4}$
\item $x = 1$
\item $t=-2$

\setcounter{HW}{\value{enumi}}
\end{enumerate}
\end{multicols}

\begin{multicols}{3}
\begin{enumerate}
\setcounter{enumi}{\value{HW}}

\item $t=-3,\, 4$
\item $x=-1$
\item $x=\frac{9}{2}$

\setcounter{HW}{\value{enumi}}
\end{enumerate}
\end{multicols}

\begin{multicols}{3}
\begin{enumerate}
\setcounter{enumi}{\value{HW}}

\item $t=\pm 10$
\item $t=-2,\, 5$
\item $x = -\frac{17}{7}$

\setcounter{HW}{\value{enumi}}
\end{enumerate}
\end{multicols}

\begin{multicols}{3}
\begin{enumerate}
\setcounter{enumi}{\value{HW}}

\item $x = 10^{1.7}$
\item $x = 10^{-5.4}$
\item $x = 10^{3}$

\setcounter{HW}{\value{enumi}}
\end{enumerate}
\end{multicols}

\begin{multicols}{3}
\begin{enumerate}
\setcounter{enumi}{\value{HW}}

\item $t=\frac{25}{2}$
\item $t=e^{3/4}$
\item $t = 5$

\setcounter{HW}{\value{enumi}}
\end{enumerate}
\end{multicols}

\begin{multicols}{3}
\begin{enumerate}
\setcounter{enumi}{\value{HW}}

\item $t = \frac{1}{2}$
\item $x = 2$
\item $x = \frac{1}{e^3-1}$

\setcounter{HW}{\value{enumi}}
\end{enumerate}
\end{multicols}

\begin{multicols}{3}
\begin{enumerate}
\setcounter{enumi}{\value{HW}}

\item $t=6$
\item $t=4$
\item $x = 81$

\setcounter{HW}{\value{enumi}}
\end{enumerate}
\end{multicols}

\begin{multicols}{3}
\begin{enumerate}
\setcounter{enumi}{\value{HW}}

\item $x = e^{e^3}$
\item $t=10^{-3}, \, 10^{5}$
\item $t = 1, \, x = e^{2}$

\setcounter{HW}{\value{enumi}}
\end{enumerate}
\end{multicols}

\begin{multicols}{3}
\begin{enumerate}
\setcounter{enumi}{\value{HW}}

\item $(e, \infty)$
\item $(e, \infty)$
\item $\left[10^{-3}, \infty \right)$

\setcounter{HW}{\value{enumi}}
\end{enumerate}
\end{multicols}

\begin{multicols}{3}
\begin{enumerate}
\setcounter{enumi}{\value{HW}}

\item $\left[10^{2.6}, 10^{4.1}\right]$

\item $\left(10^{-5.4}, 10^{-2.3}\right)$
\item $(0, 1] \cup [e^{2}, \infty)$

\setcounter{HW}{\value{enumi}}
\end{enumerate}
\end{multicols}

\begin{multicols}{2}
\begin{enumerate}
\setcounter{enumi}{\value{HW}}

\item $t \approx 1.3098$
\item $x \approx 4.177, \, x \approx 5503.665$

\setcounter{HW}{\value{enumi}}
\end{enumerate}
\end{multicols}

\begin{multicols}{2}
\begin{enumerate}
\setcounter{enumi}{\value{HW}}

\item $\approx (-\infty, -12.1414) \cup (12.1414, \infty)$
\item $\approx (-3.0281, -3) \cup (0.5, 0.5991) \cup (1.9299, 2)$

\setcounter{HW}{\value{enumi}}
\end{enumerate}
\end{multicols}

\begin{multicols}{3} 
\begin{enumerate}
\setcounter{enumi}{\value{HW}}

\item  $(-\infty, e) \cup (e, \infty)$

\item   $(0,e) \cup (e, \infty)$

\item  $(0, 100]$

\setcounter{HW}{\value{enumi}}
\end{enumerate}
\end{multicols}


\begin{multicols}{3} 
\begin{enumerate}
\setcounter{enumi}{\value{HW}}

\item    $(0, \infty)$
  
\item    $(1, \infty)$    

\item    $(1,e)$ 


\setcounter{HW}{\value{enumi}}
\end{enumerate}
\end{multicols}


\begin{multicols}{2}
\begin{enumerate}
\setcounter{enumi}{\value{HW}}

\item $-\dfrac{1}{2} < x < \dfrac{e^{3} - 1}{2}$, so $\left( -\dfrac{1}{2}, \dfrac{e^{3} - 1}{2}\right)$

\item $y = \dfrac{3}{5e^{2x} + 1}$ \vphantom{$\dfrac{e^{3} - 1}{2}$}

\setcounter{HW}{\value{enumi}}
\end{enumerate}
\end{multicols}

\begin{enumerate}
\setcounter{enumi}{\value{HW}}
\addtocounter{enumi}{1}

\item $f^{-1}(x) = \dfrac{e^{2x} - 1}{e^{2x} + 1} = \dfrac{e^{x} - e^{-x}}{e^{x} + e^{-x}}$. 

To see why we rewrite this in this form, see  Exercise \ref{andtheresthyperbolic} in Section \ref{ParametricEquations}.

 The domain of $f^{-1}$ is $(-\infty, \infty)$ and its range is the same as the domain of $f$, namely $(-1, 1)$.

\setcounter{HW}{\value{enumi}}
\end{enumerate}

\newpage

\begin{enumerate}
\setcounter{enumi}{\value{HW}}

\addtocounter{enumi}{4}

\item  \begin{itemize}  \item  Domain:  $(0, 5)$.

\item $x$-intercept:  $\left( \frac{5}{2}, 0\right)$;  there is no $y$-intercept.

\item $\ds{\lim_{x \rightarrow 0^{+}} f(x) = -\infty}$, $\ds{\lim_{x \rightarrow 5^{-}} f(x) = \infty}$;  we have two vertical asymptotes:  $x = 0$ and $x = 5$.

\item  $f$ is always increasing: $(0, 5)$.

\item There are no local extrema.

\item  The graph of $f$ is concave up on $\left(0, \frac{5}{2}\right)$ and concave down on $\left( \frac{5}{2}, 5\right)$.

\item  The inflection point is $\left( \frac{5}{2}, 0\right)$.

\end{itemize}




\item   \begin{itemize}  \item  Domain:  $(0, \infty)$.

\item $x$-intercept:  $\left( 1 , 0\right)$;  there is no $y$-intercept.

\item $\ds{\lim_{x \rightarrow 0^{+}} f(x) = -\infty}$, $\ds{\lim_{x \rightarrow \infty} f(x) = 0}$;  we have a vertical and horizontal asymptote: $x = 0$ and  $y = 0$.

\item  $f$ is  increasing on  $(0, e)$ and decreasing on $(e, \infty)$.

\item  There is a local (absolute) max at $\left(e, \frac{1}{e}\right)$.

\item  The graph of $f$ is concave up on $\left(e^{\frac{3}{2}}, \infty \right)$ and concave down on $\left(0 ,  e^{\frac{3}{2}} \right)$.

\item  The inflection point is $\left( e^{\frac{3}{2}},  \frac{3}{2 e^{\frac{3}{2}}} \right)$.

\end{itemize}


\setcounter{HW}{\value{enumi}}
\end{enumerate}


