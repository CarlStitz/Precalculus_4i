\label{ExercisesforExponentialEquationsandInequalities}

In Exercises \ref{expeqnfirst} - \ref{expeqnlast}, solve the equation analytically.

\begin{multicols}{3}
\begin{enumerate}

\item $2^{4x} = 8$  \label{expeqnfirst} 
\item $3^{(x - 1)} = 27$  
\item $5^{2x-1} = 125$ 

\setcounter{HW}{\value{enumi}}
\end{enumerate}
\end{multicols}

\begin{multicols}{3}
\begin{enumerate}
\setcounter{enumi}{\value{HW}}

\item $4^{2t} = \frac{1}{2}$
\item $8^{t} = \frac{1}{128}$ 
\item $2^{(t^{3} - t)} = 1$ \vphantom{ $8^{t} = \frac{1}{128}$}

\setcounter{HW}{\value{enumi}}
\end{enumerate}
\end{multicols}

\begin{multicols}{3}
\begin{enumerate}
\setcounter{enumi}{\value{HW}}

\item $3^{7x} = 81^{4-2x}$ 
\item $9 \cdot 3^{7x} = \left(\frac{1}{9}\right)^{2x}$ 
\item $3^{2x} = 5$ 

\setcounter{HW}{\value{enumi}}
\end{enumerate}
\end{multicols}

\begin{multicols}{3}
\begin{enumerate}
\setcounter{enumi}{\value{HW}}

\item $5^{-t} = 2$ 
\item $5^{t} = -2$  
\item $3^{(t - 1)} = 29$  

\setcounter{HW}{\value{enumi}}
\end{enumerate}
\end{multicols}

\begin{multicols}{3}
\begin{enumerate}
\setcounter{enumi}{\value{HW}}

\item $(1.005)^{12x} = 3$
\item $e^{-5730k} = \frac{1}{2}$ 
\item $2000e^{0.1t} = 4000$  

\setcounter{HW}{\value{enumi}}
\end{enumerate}
\end{multicols}

\begin{multicols}{3}
\begin{enumerate}
\setcounter{enumi}{\value{HW}}


\item $500\left(1-e^{2t}\right) = 250$
\item $70 + 90e^{-0.1t} = 75$ 
\item $30-6e^{-0.1t}=20$ 


\setcounter{HW}{\value{enumi}}
\end{enumerate}
\end{multicols}

\begin{multicols}{3}
\begin{enumerate}
\setcounter{enumi}{\value{HW}}

\item $\dfrac{100e^{x}}{e^{x}+2}=50$ 
\item $\dfrac{5000}{1+2e^{-3t}}=2500$ 
\item $\dfrac{150}{1 + 29e^{-0.8t}} = 75$ 


\setcounter{HW}{\value{enumi}}
\end{enumerate}
\end{multicols}

\begin{multicols}{3}
\begin{enumerate}
\setcounter{enumi}{\value{HW}}

\item $25\left(\frac{4}{5}\right)^{x} = 10$  

\item $e^{2x} = 2e^{x}$ 
\item  $7e^{2t} = 28e^{-6t}$ 

\setcounter{HW}{\value{enumi}}
\end{enumerate}
\end{multicols}

\begin{multicols}{3}
\begin{enumerate}
\setcounter{enumi}{\value{HW}}

\item $3^{(x - 1)} = 2^{x}$ 
\item $3^{(x - 1)} = \left(\frac{1}{2}\right)^{(x + 5)}$ 
\item  $7^{3+7x} = 3^{4-2x}$  

\setcounter{HW}{\value{enumi}}
\end{enumerate}
\end{multicols}

\begin{multicols}{3}
\begin{enumerate}
\setcounter{enumi}{\value{HW}}

\item $e^{2t} - 3e^{t}-10=0$ %Ans $t=\ln(5)$
\item $e^{2t} = e^{t}+6$ %Ans $t=\ln(2)$
\item $4^{t} + 2^{t} = 12$ %Ans $x=\dfrac{\ln(3)}{\ln(2)}$


\setcounter{HW}{\value{enumi}}
\end{enumerate}
\end{multicols}

\begin{multicols}{3}
\begin{enumerate}
\setcounter{enumi}{\value{HW}}

\item $e^{x}-3e^{-x}=2$ %Ans $x=\ln(3)$
\item $e^{x}+15e^{-x}=8$ %Ans $x=\ln(2)$, $\ln(5)$
\item $3^{x}+25\cdot3^{-x}=10$ %Ans $x=\dfrac{\ln(5)}{\ln(3)}$
\label{expeqnlast} 

\setcounter{HW}{\value{enumi}}
\end{enumerate}
\end{multicols}

In Exercises \ref{expineqfirst} - \ref{expineqlast}, solve the inequality analytically.

\begin{multicols}{2} 
\begin{enumerate}
\setcounter{enumi}{\value{HW}}

\item $e^{x} > 53$ \label{expineqfirst} 
\item $1000\left(1.005\right)^{12t} \geq 3000$ 

\setcounter{HW}{\value{enumi}}
\end{enumerate}
\end{multicols}

\begin{multicols}{2} 
\begin{enumerate}
\setcounter{enumi}{\value{HW}}

\item $2^{(x^{3} - x)} < 1$
\item $25\left(\dfrac{4}{5}\right)^{x} \geq 10$

\setcounter{HW}{\value{enumi}}
\end{enumerate}
\end{multicols}

\begin{multicols}{2} 
\begin{enumerate}
\setcounter{enumi}{\value{HW}}

\item $\dfrac{150}{1 + 29e^{-0.8t}} \leq 130$

\item $\vphantom{\dfrac{150}{1 + 29e^{-0.8t}}} 70 + 90e^{-0.1t} \leq 75$

\setcounter{HW}{\value{enumi}}
\end{enumerate}
\end{multicols}

\begin{multicols}{2} 
\begin{enumerate}
\setcounter{enumi}{\value{HW}}

\item $e^{-x} - xe^{-x} \geq 0$
\item $(1-e^{t}) t^{-1} \leq 0$ \label{expineqlast}

\setcounter{HW}{\value{enumi}}
\end{enumerate}
\end{multicols}

In Exercises \ref{calcexpineqfirst} - \ref{calcexpineqlast},  use a graphing utility to help you solve the equation or  inequality.

\begin{multicols}{3} 
\begin{enumerate}
\setcounter{enumi}{\value{HW}}

\item $2^{x} = x^2$ \label{calcexpineqfirst} 
\item $e^{t} = \ln(t) + 5$   
\item $e^{\sqrt{x}} = x + 1$ 

\setcounter{HW}{\value{enumi}}
\end{enumerate}
\end{multicols}

\begin{multicols}{3} 
\begin{enumerate}
\setcounter{enumi}{\value{HW}}

\item  $e^{-2t}-te^{-t} \leq 0$
\item $3^{(x - 1)} < 2^{x}$ 
\item $e^{t} < t^{3} - t$ \label{calcexpineqlast} 

\setcounter{HW}{\value{enumi}}
\end{enumerate}
\end{multicols}

\pagebreak

In Exercises \ref{domaincomplicatedexpfirst} - \ref{domaincomplicatedexplast},  find the domain of the function.

\begin{multicols}{2} 
\begin{enumerate}
\setcounter{enumi}{\value{HW}}

\item  $T(x) = \dfrac{e^{x} - e^{-x}}{e^{x} + e^{-x}}$     \label{domaincomplicatedexpfirst}

\item   $C(x) = \dfrac{e^{x}  + e^{-x}}{e^{x}  - e^{-x}}$ 

\setcounter{HW}{\value{enumi}}
\end{enumerate}
\end{multicols}


\begin{multicols}{2} 
\begin{enumerate}
\setcounter{enumi}{\value{HW}}

\item     $s(t) = \sqrt{e^{2t} - 3}$
\item     $c(t) = \sqrt[3]{e^{2t} - 3}$

\setcounter{HW}{\value{enumi}}
\end{enumerate}
\end{multicols}

\begin{multicols}{2} 
\begin{enumerate}
\setcounter{enumi}{\value{HW}}
  
\item     $L(x) = \log\left( 3 - e^{x} \right)$  \vphantom{$\ell(x) = \ln\left( \dfrac{e^{2x}}{e^{x}-2} \right)$}

\item    $\ell(x) = \ln\left( \dfrac{e^{2x}}{e^{x}-2} \right)$  \label{domaincomplicatedexplast}

\setcounter{HW}{\value{enumi}}
\end{enumerate}
\end{multicols}




\begin{enumerate}
\setcounter{enumi}{\value{HW}}

\item \label{onetoonelogexercise} Since $f(x) = \ln(x)$ is a strictly increasing function, if $0 < a < b$ then $\ln(a) < \ln(b)$.  Use this fact to solve the inequality $e^{(3x - 1)} > 6$ without a sign diagram. Use this technique to solve the inequalities in Exercises \ref{expineqfirst} - \ref{expineqlast}. (NOTE:  Isolate the exponential function first!)

\item \label{hyperbolicsine} Compute the inverse of $f(x) = \dfrac{e^{x} - e^{-x}}{2}$.  State the domain and range of both $f$ and $f^{-1}$. 

\item \label{checkingexpfracinverse} In Example \ref{expfracinverse}, we found that the inverse of $f(x) = \dfrac{5e^{x}}{e^{x}+1}$ was $f^{-1}(x) = \ln\left(\dfrac{x}{5-x}\right)$ but we left a few loose ends for you to tie up.  

\begin{enumerate}

\item Algebraically check our answer by verifying: $\left(f^{-1} \circ f\right)(x) = x$ for all $x$ in the domain of $f$ and that $\left(f \circ f^{-1}\right)(x) = x$ for all $x$ in the domain of $f^{-1}$.

\item Find the range of $f$ by finding the domain of $f^{-1}$.

\item With help of a graphing utility, graph $y = f(x)$,  $y = f^{-1}(x)$ and $y = x$ on the same set of axes.  How does this help to verify our answer?

\item Let $g(x) = \dfrac{5x}{x+1}$ and $h(x) = e^{x}$.  Show that $f = g \circ h$ and that $(g \circ h)^{-1} = h^{-1} \circ g^{-1}$. 

NOTE:  We know this is true in general by Exercise \ref{fcircginverse} in Section \ref{InverseFunctions}, but it's nice to see a specific example of the property.

\end{enumerate}

\item \label{powerexponentialgrowthex} \begin{enumerate} \item\label{numericalinvestigationlimitxpoverex} With the help of your classmates, numerically and graphically investigate $\ds{\lim_{x \rightarrow \infty}}$ $\frac{x^{p}}{e^{x}}$ for various real number powers, $p$.

\item  What does part \ref{numericalinvestigationlimitxpoverex} suggest about the relative growth rates of powers of $x$ as opposed to $e^{x}$?

\item\label{numericalxpoverexinequ}  For each power $p$ you investigated in part \ref{numericalinvestigationlimitxpoverex}, solve the inequality:  $\frac{x^{p}}{e^{x}} < \frac{1}{x}$. 

\item  Use your results from part \ref{numericalxpoverexinequ} to show that for each real number $p$ you investigated in part \ref{numericalinvestigationlimitxpoverex}, there is a real number $M$ so that if $x > M$, $0 < \frac{x^{p}}{e^{x}} < \frac{1}{x}$.  

\smallskip

Since  $\ds{\lim_{x \rightarrow \infty}}$ $\frac{1}{x} = 0$,  what do you conclude about $\ds{\lim_{x \rightarrow \infty}}$  $\frac{x^{p}}{e^{x}}$? 

\smallskip

 (This Exercise foreshadows the celebrated \index{Squeeze Theorem}\index{Theorem ! Squeeze}\textbf{Squeeze Theorem}, Theorem \ref{squeezeth}  which we'll formally introduce in Section \ref{Sequences}.)

\end{enumerate}  

\setcounter{HW}{\value{enumi}}
\end{enumerate}

\newpage

In Exercises \ref{exponentialcurvesketchfirst}  - \ref{exponentialcurvesketchlast} a function $f$ along with its derivatives $f'$ and $f''$ are given.

\begin{itemize}

\item  Find the $x$- and $y$-intercepts of the graph of each function, if any.

\item  Use limits to determine the end behavior.

\item  Use $f'$ to determine the open intervals over which $f$ is increasing or decreasing.

\item Determine the local extrema, if any.

\item  Use $f''$ to determine the open intervals over which the graph of $f$  is concave up or concave down.

\item  Determine the inflection points of the graph, if any.

\end{itemize}

\begin{enumerate}
\setcounter{enumi}{\value{HW}}

\item\label{exponentialcurvesketchfirst}  $f(x) = \dfrac{5}{1 + e^{-x}}$,  $f'(x) = \dfrac{5 e^{-x}}{\left(1 + e^{-x} \right)^2}$, $f''(x) = \dfrac{5e^{-x}\left(e^{-x}-1\right)}{\left(1+e^{-x}\right)^3}$

\smallskip

\item\label{exponentialcurvesketchlast}  $f(x) = e^{-x} - e^{-2x}$, $f'(x) = 2e^{-2x} - e^{-x}$, $f''(x) = e^{-x} - 4 e^{-2x}$


\setcounter{HW}{\value{enumi}}
\end{enumerate}


\newpage

\subsection{Answers}


\begin{multicols}{3}
\begin{enumerate}

\item $x = \frac{3}{4}$
\item $x = 4$
\item $x=2$

\setcounter{HW}{\value{enumi}}
\end{enumerate}
\end{multicols}

\begin{multicols}{3}
\begin{enumerate}
\setcounter{enumi}{\value{HW}}

\item $t = -\frac{1}{4}$
\item $t = -\frac{7}{3}$
\item $t = -1, \, 0, \, 1$

\setcounter{HW}{\value{enumi}}
\end{enumerate}
\end{multicols}

\begin{multicols}{3}
\begin{enumerate}
\setcounter{enumi}{\value{HW}}

\item $x = \frac{16}{15}$
\item $x=-\frac{2}{11}$  \vphantom{$x = \dfrac{\ln(5)}{2\ln(3)}$}
\item $x = \dfrac{\ln(5)}{2\ln(3)}$

\setcounter{HW}{\value{enumi}}
\end{enumerate}
\end{multicols}

\begin{multicols}{3}
\begin{enumerate}
\setcounter{enumi}{\value{HW}}

\item $t = -\dfrac{\ln(2)}{\ln(5)}$
\item No solution. \vphantom{ $t = \dfrac{\ln(29) + \ln(3)}{\ln(3)}$}
\item $t = \dfrac{\ln(29) + \ln(3)}{\ln(3)}$

\setcounter{HW}{\value{enumi}}
\end{enumerate}
\end{multicols}

\begin{multicols}{3}
\begin{enumerate}
\setcounter{enumi}{\value{HW}}

\item $x = \dfrac{\ln(3)}{12\ln(1.005)}$ \vphantom{$k = \dfrac{\ln\left(\frac{1}{2}\right)}{-5730} = \dfrac{\ln(2)}{5730} $}
\item $k = \dfrac{\ln\left(\frac{1}{2}\right)}{-5730} = \dfrac{\ln(2)}{5730} $
\item $t=\dfrac{\ln(2)}{0.1} = 10\ln(2)$ \vphantom{$k = \dfrac{\ln\left(\frac{1}{2}\right)}{-5730} = \dfrac{\ln(2)}{5730} $}

\setcounter{HW}{\value{enumi}}
\end{enumerate}
\end{multicols}

\begin{multicols}{2}
\begin{enumerate}
\setcounter{enumi}{\value{HW}}


\item $t=\frac{1}{2}\ln\left(\frac{1}{2}\right) = -\frac{1}{2}\ln(2)$ \vphantom{$t = \dfrac{\ln\left(\frac{1}{18}\right)}{-0.1} =10 \ln(18)$}
\item $t = \dfrac{\ln\left(\frac{1}{18}\right)}{-0.1} =10 \ln(18)$

\setcounter{HW}{\value{enumi}}
\end{enumerate}
\end{multicols}

\begin{multicols}{2}
\begin{enumerate}
\setcounter{enumi}{\value{HW}}


\item $t=-10\ln\left(\frac{5}{3}\right) = 10\ln\left(\frac{3}{5}\right)$
\item$x=\ln(2)$ \vphantom{$t=-10\ln\left(\frac{5}{3}\right) = 10\ln\left(\frac{3}{5}\right)$}

\setcounter{HW}{\value{enumi}}
\end{enumerate}
\end{multicols}

\begin{multicols}{2}
\begin{enumerate}
\setcounter{enumi}{\value{HW}}

\item $t=\frac{1}{3}\ln(2)$ \vphantom{$t = \dfrac{\ln\left(\frac{1}{29}\right)}{-0.8} = \dfrac{5}{4}\ln(29)$}

\item $t = \dfrac{\ln\left(\frac{1}{29}\right)}{-0.8} = \dfrac{5}{4}\ln(29)$

\setcounter{HW}{\value{enumi}}
\end{enumerate}
\end{multicols}

\begin{multicols}{2}
\begin{enumerate}
\setcounter{enumi}{\value{HW}}

\item $x = \dfrac{\ln\left(\frac{2}{5}\right)}{\ln\left(\frac{4}{5}\right)} = \dfrac{\ln(2)-\ln(5)}{\ln(4) - \ln(5)}$

\item $x =  \ln(2)$ \vphantom{ $x = \dfrac{\ln\left(\frac{2}{5}\right)}{\ln\left(\frac{4}{5}\right)} = \dfrac{\ln(2)-\ln(5)}{\ln(4) - \ln(5)}$}


\setcounter{HW}{\value{enumi}}
\end{enumerate}
\end{multicols}

\begin{multicols}{2}
\begin{enumerate}
\setcounter{enumi}{\value{HW}}


\item  $t = -\frac{1}{8} \ln\left(\frac{1}{4} \right) = \frac{1}{4}\ln(2)$ \vphantom{$x = \dfrac{\ln(3)}{\ln(3) - \ln(2)}$}

\item $x = \dfrac{\ln(3)}{\ln(3) - \ln(2)}$

\setcounter{HW}{\value{enumi}}
\end{enumerate}
\end{multicols}

\begin{multicols}{2}
\begin{enumerate}
\setcounter{enumi}{\value{HW}}


\item $x = \dfrac{\ln(3) + 5\ln\left(\frac{1}{2}\right)}{\ln(3) - \ln\left(\frac{1}{2}\right)} = \dfrac{\ln(3)-5\ln(2)}{\ln(3)+\ln(2)}$
\item  $x = \dfrac{4 \ln(3) - 3 \ln(7)}{7 \ln(7) + 2 \ln(3)}$ \vphantom{$x = \dfrac{\ln(3) + 5\ln\left(\frac{1}{2}\right)}{\ln(3) - \ln\left(\frac{1}{2}\right)} = \dfrac{\ln(3)-5\ln(2)}{\ln(3)+\ln(2)}$}

\setcounter{HW}{\value{enumi}}
\end{enumerate}
\end{multicols}

\begin{multicols}{3}
\begin{enumerate}
\setcounter{enumi}{\value{HW}}

\item $t=\ln(5)$ \vphantom{$t=\dfrac{\ln(3)}{\ln(2)}$}
\item $t=\ln(3)$  \vphantom{$t=\dfrac{\ln(3)}{\ln(2)}$}
\item $t=\dfrac{\ln(3)}{\ln(2)}$


\setcounter{HW}{\value{enumi}}
\end{enumerate}
\end{multicols}

\begin{multicols}{3}
\begin{enumerate}
\setcounter{enumi}{\value{HW}}

\item $x=\ln(3)$  \vphantom{$x=\dfrac{\ln(5)}{\ln(3)}$}
\item $x=\ln(3)$, $\ln(5)$  \vphantom{$x=\dfrac{\ln(5)}{\ln(3)}$}
\item $x=\dfrac{\ln(5)}{\ln(3)}$


\setcounter{HW}{\value{enumi}}
\end{enumerate}
\end{multicols}

\begin{multicols}{2} 
\begin{enumerate}
\setcounter{enumi}{\value{HW}}

\item $(\ln(53), \infty)$ \vphantom{$\left[\dfrac{\ln(3)}{12\ln(1.005)}, \infty\right)$}
\item $\left[\dfrac{\ln(3)}{12\ln(1.005)}, \infty\right)$

\setcounter{HW}{\value{enumi}}
\end{enumerate}
\end{multicols}

\begin{multicols}{2} 
\begin{enumerate}
\setcounter{enumi}{\value{HW}}

\item $(-\infty, -1) \cup (0, 1)$ \vphantom{$\left(-\infty, \dfrac{\ln\left(\frac{2}{5}\right)}{\ln\left(\frac{4}{5}\right)} \right] = \left(-\infty, \dfrac{\ln(2)-\ln(5)}{\ln(4)-\ln(5)} \right]$} 

\item $\left(-\infty, \dfrac{\ln\left(\frac{2}{5}\right)}{\ln\left(\frac{4}{5}\right)} \right] = \left(-\infty, \dfrac{\ln(2)-\ln(5)}{\ln(4)-\ln(5)} \right]$

\setcounter{HW}{\value{enumi}}
\end{enumerate}
\end{multicols}

\begin{multicols}{2} 
\begin{enumerate}
\setcounter{enumi}{\value{HW}}

\item $\left(-\infty, \dfrac{\ln\left(\frac{2}{377}\right)}{-0.8} \right] = \left(-\infty, \frac{5}{4}\ln\left(\dfrac{377}{2}\right) \right]$
\item $\left[\dfrac{\ln\left(\frac{1}{18}\right)}{-0.1}, \infty\right) = [10\ln(18), \infty)$

\setcounter{HW}{\value{enumi}}
\end{enumerate}
\end{multicols}


\begin{multicols}{2} 
\begin{enumerate}
\setcounter{enumi}{\value{HW}}

\item $(-\infty, 1]$
\item $(-\infty, 0) \cup (0, \infty)$


\setcounter{HW}{\value{enumi}}
\end{enumerate}
\end{multicols}

\begin{multicols}{2} 
\begin{enumerate}
\setcounter{enumi}{\value{HW}}

\item $x \approx -0.76666, \, x = 2, \, x = 4$
\item $x \approx 0.01866, \, x \approx 1.7115$


\setcounter{HW}{\value{enumi}}
\end{enumerate}
\end{multicols}

\begin{multicols}{2} 
\begin{enumerate}
\setcounter{enumi}{\value{HW}}

\item $x = 0$
\item $\approx [0.567, \infty)$

\setcounter{HW}{\value{enumi}}
\end{enumerate}
\end{multicols}

\begin{multicols}{2} 
\begin{enumerate}
\setcounter{enumi}{\value{HW}}

\item $\approx (-\infty, 2.7095)$
\item $\approx (2.3217, 4.3717)$


\setcounter{HW}{\value{enumi}}
\end{enumerate}
\end{multicols}


\begin{multicols}{3} 
\begin{enumerate}
\setcounter{enumi}{\value{HW}}

\item  $(-\infty, \infty)$   \vphantom{$\left( \frac{1}{2} \ln(3), \infty \right)$}
\item   $(-\infty, 0) \cup (0, \infty)$  \vphantom{$\left( \frac{1}{2} \ln(3), \infty \right)$}
\item     $\left( \frac{1}{2} \ln(3), \infty \right)$

\setcounter{HW}{\value{enumi}}
\end{enumerate}
\end{multicols}


\begin{multicols}{3} 
\begin{enumerate}
\setcounter{enumi}{\value{HW}}


\item    $(-\infty, \infty)$  

\item     $(-\infty, \ln(3))$

\item    $(\ln(2), \infty) $  

\setcounter{HW}{\value{enumi}}
\end{enumerate}
\end{multicols}


\begin{enumerate}
\setcounter{enumi}{\value{HW}}


\item $x > \frac{1}{3}(\ln(6) + 1)$, so $\left(\frac{1}{3}(\ln(6) + 1), \infty \right)$ 

\item  $f^{-1} = \ln\left(x + \sqrt{x^{2} + 1}\right)$. Both $f$ and $f^{-1}$ have domain $(-\infty, \infty)$ and range $(-\infty, \infty)$.



\setcounter{HW}{\value{enumi}}
\end{enumerate}

\begin{enumerate}
\setcounter{enumi}{\value{HW}}

\addtocounter{enumi}{2}

\item  \begin{itemize} \item  There are no $x$-intercepts;  the $y$-intercept is $(0,5)$.

\item  $\ds{\lim_{x \rightarrow -\infty} f(x) = 0}$ and $\ds{\lim_{x \rightarrow \infty} f(x) = 5}$;  we have two horizontal asymptotes: $y = 0$ and $y = 5$.

\item  $f$ is always increasing:  $(-\infty, \infty)$.

\item There are no local extrema.

\item The graph of $f$ is concave up on $(-\infty, 0)$ and concave down on $(0, \infty)$.

\item  The inflection point is $(0, 5)$.


\end{itemize}

\item  \begin{itemize} \item  The $x$- and $y$-intercept is $(0,0)$.

\item  $\ds{\lim_{x \rightarrow -\infty} f(x) = -\infty}$ and $\ds{\lim_{x \rightarrow \infty} f(x) = 0}$;  we have a horizontal asymptote $y = 0$.

\item  $f$ is increasing on $(-\infty, \ln(2))$ and decreasing on $(\ln(2), \infty)$.

\item There is a local (absolute) maximum at $\left( \ln(2), \frac{1}{4}\right)$.

\item The graph of $f$ is concave up on $(\ln(4), \infty)$ and concave down on $(-\infty, \ln(4))$.

\item  The inflection point is $\left(\ln(4), \frac{3}{16} \right)$.


\end{itemize}

\end{enumerate}
