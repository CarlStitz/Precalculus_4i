\label{ExercisesforAugMatrices}

In Exercises \ref{rreffirst} - \ref{rreflast}, state whether the given matrix is in reduced row echelon form, row echelon form only or in neither of those forms.

\begin{multicols}{3} 
\begin{enumerate}


\item $\left[ \begin{array}{rr|r} 
1 & 0 & 3 \\ 
0 & 1 & 3  \\ 
\end{array} \right]$ \label{rreffirst}

\item $\left[ \begin{array}{rrr|r} 
3 & -1 & \hphantom{-}1 & 3 \\ 
2 & -4 & 3 & 16 \\ 
1 & -1 & 1 & 5  \\
\end{array} \right]$

\item $\left[ \begin{array}{rrr|r} 
1 & 1 & 4 & 3 \\ 
0 & 1 & 3 & 6 \\ 
0 & 0 & 0 & 1  \\
\end{array} \right]$

\setcounter{HW}{\value{enumi}}
\end{enumerate}
\end{multicols}

\begin{multicols}{3}
\begin{enumerate}
\setcounter{enumi}{\value{HW}}


\item $\left[ \begin{array}{rrr|r} 
1 & 0 & 0 & 0 \\ 
0 & 1 & 0 & 0 \\ 
0 & 0 & 0 & 1  \\
\end{array} \right]$

\item $\left[ \begin{array}{rrrr|r} 
1 & 0 & 4 & 3 & 0 \\ 
0 & 1 & 3 & 6 & 0 \\ 
0 & 0 & 0 & 0 & 0 
\end{array} \right]$

\item $\left[ \begin{array}{rrr|r} 
1 & 1 & 4 & 3 \\ 
0 & 1 & 3 & 6 \\
\end{array} \right]$ \label{rreflast}

\setcounter{HW}{\value{enumi}}
\end{enumerate}
\end{multicols}

In Exercises \ref{decodefirst} - \ref{decodelast}, the following matrices are in reduced row echelon form.  Determine the solution of the corresponding system of linear equations or state that the system is inconsistent.  


\begin{multicols}{3}
\begin{enumerate}
\setcounter{enumi}{\value{HW}}

\item $\left[ \begin{array}{rr|r} 
1 & 0 & -2 \\ 
0 & 1 & 7  \\ 
\end{array} \right]$  \label{decodefirst}

\item $\left[ \begin{array}{rrr|r} 
1 & 0 & 0 & -3 \\ 
0 & 1 & 0 & 20 \\ 
0 & 0 & 1 & 19  
\end{array} \right]$

\item $\left[ \begin{array}{rrrr|r} 
1 & 0 & 0 & 3 & 4 \\ 
0 & 1 & 0 & 6 & -6 \\ 
0 & 0 & 1 & 0 & 2 
\end{array} \right]$

\setcounter{HW}{\value{enumi}}
\end{enumerate}
\end{multicols}

\begin{multicols}{3}
\begin{enumerate}
\setcounter{enumi}{\value{HW}}

\item $\left[ \begin{array}{rrrr|r} 
1 & 0 & 0 & 3 & 0 \\ 
0 & 1 & 2 & 6 & 0 \\ 
0 & 0 & 0 & 0 & 1 
\end{array} \right]$

\item $\left[ \begin{array}{rrrr|r} 
1 & \hphantom{-}0 & -8 & 1 & 7 \\ 
0 & 1 & 4 & -3 & 2 \\ 
0 & 0 & 0 & 0 & 0 \\
0 & 0 & 0 & 0 & 0 
\end{array} \right]$

\item $\left[ \begin{array}{rrr|r} 
1 & \hphantom{-}0 & 9 & -3 \\ 
0 & 1 & -4 & 20 \\ 
0 & 0 & 0 & 0  
\end{array} \right]$ \label{decodelast}

\setcounter{HW}{\value{enumi}}
\end{enumerate}
\end{multicols}




In Exercises \ref{solveaugfirst} - \ref{solveauglast}, solve the following systems of linear equations using the techniques discussed in this section.  Compare and contrast these techniques with those you used to solve the systems in the Exercises in Section \ref{LinSystems}.

\begin{multicols}{2}
\begin{enumerate}
\setcounter{enumi}{\value{HW}}


\item $\left\{ \begin{array}{rcr} -5x + y & = & 17  \\ x + y & = & 5  \end{array} \right.$ \label{solveaugfirst}
\item $\left\{ \begin{array}{rcr} x + y + z & = & 3 \\ 2x - y + z & = & 0 \\ -3x + 5y + 7z & = & 7  \end{array} \right.$

\setcounter{HW}{\value{enumi}}
\end{enumerate}
\end{multicols}


\begin{multicols}{2}
\begin{enumerate}
\setcounter{enumi}{\value{HW}}


\item $\left\{ \begin{array}{rcr} 4x - y + z & = & 5 \\ 2y + 6z & = & 30 \\ x + z & = & 5  \end{array} \right.$

\item $\left\{ \begin{array}{rcr} x-2y+3z & = & 7 \\ -3x+y+2z & = & -5 \\ 2x+2y+z & = & 3  \end{array} \right.$

\setcounter{HW}{\value{enumi}}
\end{enumerate}
\end{multicols}


\begin{multicols}{2}
\begin{enumerate}
\setcounter{enumi}{\value{HW}}


\item $\left\{ \begin{array}{rcr} 3x-2y+z & = & -5 \\ x+3y-z & = & 12 \\ x+y+2z & = & 0  \end{array} \right.$
\item $\left\{ \begin{array}{rcr} 2x-y+z& = & -1 \\ 4x+3y+5z & = & 1 \\  5y+3z & = & 4 \end{array} \right.$

\setcounter{HW}{\value{enumi}}
\end{enumerate}
\end{multicols}


\begin{multicols}{2}
\begin{enumerate}
\setcounter{enumi}{\value{HW}}


\item $\left\{ \begin{array}{rcr} x-y+z & = & -4 \\ -3x+2y+4z & = & -5 \\ x-5y+2z & = & -18  \end{array} \right.$
\item $\left\{ \begin{array}{rcr} 2x-4y+z & = & -7 \\ x-2y+2z & = & -2 \\ -x+4y-2z & = & 3  \end{array} \right.$

\setcounter{HW}{\value{enumi}}
\end{enumerate}
\end{multicols}


\begin{multicols}{2}
\begin{enumerate}
\setcounter{enumi}{\value{HW}}


\item $\left\{ \begin{array}{rcr} 2x-y+z & = & 1 \\ 2x+2y-z & = & 1 \\ 3x+6y+4z & = & 9  \end{array} \right.$
\item $\left\{ \begin{array}{rcr} x-3y-4z & = & 3 \\ 3x+4y-z & = & 13 \\ 2x-19y-19z & = & 2  \end{array} \right.$

\setcounter{HW}{\value{enumi}}
\end{enumerate}
\end{multicols}


\begin{multicols}{2}
\begin{enumerate}
\setcounter{enumi}{\value{HW}}


\item $\left\{ \begin{array}{rcr} x+y+z & = & 4 \\ 2x-4y-z& = & -1 \\ x-y & = & 2 \end{array} \right.$
\item $\left\{ \begin{array}{rcr} x-y+z & = & 8 \\ 3x+3y-9z & = & -6 \\  7x-2y+5z & = & 39 \end{array} \right.$

\setcounter{HW}{\value{enumi}}
\end{enumerate}
\end{multicols}


\begin{multicols}{2}
\begin{enumerate}
\setcounter{enumi}{\value{HW}}


\item $\left\{ \begin{array}{rcr} 2x-3y+z & = & -1 \\ 4x-4y+4z & = & -13 \\ 6x-5y+7z & = & -25  \end{array} \right.$

\item  $\left\{ \begin{array}{rcr} x_{\mbox{\tiny$1$}} - x_{\mbox{\tiny$3$}} & = & -2 \\ 
2x_{\mbox{\tiny$2$}} - x_{\mbox{\tiny$4$}} & = & 0  \\  
x_{\mbox{\tiny$1$}} -  2x_{\mbox{\tiny$2$}} + x_{\mbox{\tiny$3$}} & = & 0 \\
-x_{\mbox{\tiny$3$}} + x_{\mbox{\tiny$4$}} & = & 1  \end{array} \right.$ \label{solveauglast}

\setcounter{HW}{\value{enumi}}
\end{enumerate}
\end{multicols}



\begin{enumerate}
\setcounter{enumi}{\value{HW}}

\item  It's time for another meal at our local buffet.  This time, 22 diners (5 of whom were children) feasted for $\$162.25$, before taxes.  If the kids buffet is $\$4.50$, the basic buffet is $\$7.50$, and the deluxe buffet (with crab legs) is $\$9.25$, find out how many diners chose the deluxe buffet. 

\item Carl wants to make a party mix consisting of almonds (which cost $\$7$ per pound), cashews (which cost $\$5$ per pound), and peanuts (which cost $\$2$ per pound.)  If he wants to make a $10$ pound mix with a budget of $\$35$, what are the possible combinations almonds, cashews, and peanuts?  (You may find it helpful to review Example \ref{lucasmixex} in Section \ref{LinSystems}.)


\item  \label{threepointsmatrixfunctionfitex}Using Example \ref{matrixcurvefitting} as a guide,  determine the values of coefficients $a$, $b$, and $c$ so the graph of the given function below contains the points $(-2,1)$, $(1,4)$, $(3,-2)$:

\begin{enumerate}

\item a quadratic function: $f(x) = ax^2+bx+c$

\item  a function of the form: $g(x) = ax^3+bx+c$ 

\item a function of the form:  $h(x) = ax^{-1}+bx^2+c$

\end{enumerate}


\item  At 9 PM, the temperature was $60^{\circ}$F; at midnight, the temperature was $50^{\circ}$F; and at 6 AM, the temperature was $70^{\circ}$F .  Use the technique in Example \ref{matrixcurvefitting} to fit a quadratic function to these data with the temperature, $T$, measured in degrees Fahrenheit, as the dependent variable, and the number of hours after 9 PM, $t$, measured in hours, as the independent variable. What was the coldest temperature of the night?  When did it occur? 

\item The price for admission into the Stitz-Zeager Sasquatch Museum and Research Station is \$15 for adults and \$8 for kids 13 years old and younger. When the Zahlenreich family visits the museum their bill is \$38 and when the Nullsatz family visits their bill is \$39.  One day both families went together and took an adult babysitter along to watch the kids and the total admission charge was \$92.  Later that summer, the adults from both families went without the kids and the bill was \$45.  

\smallskip

Is that enough information to determine how many adults and children are in each family?  If not, state whether the resulting system is inconsistent or consistent dependent.  In the latter case, give at least two plausible solutions.  

\item Use the technique in Example \ref{matrixcurvefitting} to find the line between the points $(-3, 4)$ and $(6, 1)$. How does your answer compare to the slope-intercept form of the line in Equation \ref{slopeintercept}?

\item With the help of your classmates, find at least two different row echelon forms for the matrix \[\left[ \begin{array}{rr|r} 
1 & 2 & 3 \\ 
4 & 12 & 8  \\ 
\end{array} \right]\]

\end{enumerate}

\newpage

\subsection{Answers}

\begin{multicols}{2}
\begin{enumerate}

\item Reduced row echelon form
\item Neither

\setcounter{HW}{\value{enumi}}
\end{enumerate}
\end{multicols}


\begin{multicols}{2}
\begin{enumerate}
\setcounter{enumi}{\value{HW}}



\item Row echelon form only
\item Reduced row echelon form

\setcounter{HW}{\value{enumi}}
\end{enumerate}
\end{multicols}


\begin{multicols}{2}
\begin{enumerate}
\setcounter{enumi}{\value{HW}}

\item Reduced row echelon form
\item Row echelon form only

\setcounter{HW}{\value{enumi}}
\end{enumerate}
\end{multicols}

\begin{multicols}{2}
\begin{enumerate}
\setcounter{enumi}{\value{HW}}

\item $(-2, 7)$
\item $(-3, 20, 19)$

\setcounter{HW}{\value{enumi}}
\end{enumerate}
\end{multicols}


\begin{multicols}{2}
\begin{enumerate}
\setcounter{enumi}{\value{HW}}

\item $(-3t + 4, -6t - 6, 2, t)$ \\
for all real numbers $t$
\item Inconsistent


\setcounter{HW}{\value{enumi}}
\end{enumerate}
\end{multicols}


\begin{multicols}{2}
\begin{enumerate}
\setcounter{enumi}{\value{HW}}


\item $(8s - t + 7, -4s + 3t + 2, s, t)$ \\ for all real numbers $s$ and $t$
\item $(-9t - 3, 4t + 20, t)$ \\ for all real numbers $t$


\setcounter{HW}{\value{enumi}}
\end{enumerate}
\end{multicols}


\begin{multicols}{2}
\begin{enumerate}
\setcounter{enumi}{\value{HW}}

\item $(-2, 7)$
\item $(1, 2, 0)$

\setcounter{HW}{\value{enumi}}
\end{enumerate}
\end{multicols}


\begin{multicols}{2}
\begin{enumerate}
\setcounter{enumi}{\value{HW}}


\item $(-t + 5, -3t + 15, t)$\\
for all real numbers $t$
\item $(2,-1,1)$


\setcounter{HW}{\value{enumi}}
\end{enumerate}
\end{multicols}


\begin{multicols}{2}
\begin{enumerate}
\setcounter{enumi}{\value{HW}}

\item $(1,3,-2)$
\item Inconsistent


\setcounter{HW}{\value{enumi}}
\end{enumerate}
\end{multicols}


\begin{multicols}{2}
\begin{enumerate}
\setcounter{enumi}{\value{HW}}


\item $(1,3,-2)$
\item $\left(-3,\frac{1}{2},1\right)$

\setcounter{HW}{\value{enumi}}
\end{enumerate}
\end{multicols}


\begin{multicols}{2}
\begin{enumerate}
\setcounter{enumi}{\value{HW}}



\item  $\left(\frac{1}{3},\frac{2}{3},1\right)$
\item  $\left(\frac{19}{13} t + \frac{51}{13},-\frac{11}{13} t+\frac{4}{13},t\right)$\\
for all real numbers $t$

\setcounter{HW}{\value{enumi}}
\end{enumerate}
\end{multicols}


\begin{multicols}{2}
\begin{enumerate}
\setcounter{enumi}{\value{HW}}

\item Inconsistent
\item $\left(4,-3,1\right)$

\setcounter{HW}{\value{enumi}}
\end{enumerate}
\end{multicols}


\begin{multicols}{2}
\begin{enumerate}
\setcounter{enumi}{\value{HW}}


\item $\left(-2t - \frac{35}{4},-t - \frac{11}{2},t\right)$\\
for all real numbers $t$
\item $(1, 2, 3, 4)$

\setcounter{HW}{\value{enumi}}
\end{enumerate}
\end{multicols}

\begin{enumerate}
\setcounter{enumi}{\value{HW}}

\item  This time, 7 diners chose the deluxe buffet.

\item  If $t$ represents the amount (in pounds) of peanuts, then we need $1.5 t - 7.5$ pounds of almonds and $17.5 - 2.5t$ pounds of cashews.  Since we can't have a negative amount of nuts, $5 \leq t \leq 7$. 

\item 

\begin{multicols}{3}

 \begin{enumerate}

\item $f(x) = -\frac{4}{5} x^2+\frac{1}{5} x + \frac{23}{5}$

\item  $g(x) = -0.4 x^3 + 2.2 x + 2.2$

\item $h(x) = 0.6 x^{-1} - 0.7 x^2 + 4.1$

\end{enumerate}

\end{multicols}

\item  $T(t) = \frac{20}{27} t^2 - \frac{50}{9} t + 60$.  Lowest temperature of the evening $\frac{595}{12} \approx 49.58^{\circ}$F at 12:45 AM.

\newpage

\item Let $x_{\mbox{\tiny$1$}}$ and $x_{\mbox{\tiny$2$}}$ be the numbers of adults and children, respectively, in the Zahlenreich family and let $x_{\mbox{\tiny$3$}}$ and $x_{\mbox{\tiny$4$}}$ be the numbers of adults and children, respectively, in the Nullsatz family.  The system of equations determined by the given information is 

$\left\{ \begin{array}{rcr} 15x_{\mbox{\tiny$1$}} + 8x_{\mbox{\tiny$2$}} & = & 38 \\ 
15x_{\mbox{\tiny$3$}} + 8x_{\mbox{\tiny$4$}} & = & 39  \\  
15x_{\mbox{\tiny$1$}} +  8x_{\mbox{\tiny$2$}} + 15x_{\mbox{\tiny$3$}} + 8x_{\mbox{\tiny$4$}} & = & 77 \\
15x_{\mbox{\tiny$1$}} + 15x_{\mbox{\tiny$3$}} & = & 45  \end{array} \right.$

We subtracted the cost of the babysitter in E3 so the constant is 77, not 92.  This system is consistent dependent and its solution is $\left(\frac{8}{15}t + \frac{2}{5}, -t + 4, -\frac{8}{15}t + \frac{13}{5}, t \right)$.  Our variables represent numbers of adults and children so they must be whole numbers.  Running through the values $t = 0, 1, 2, 3, 4$ yields only one solution where all four variables are whole numbers; $t = 3$ gives us $(2, 1, 1, 3)$.  Thus there are 2 adults and 1 child in the Zahlenreichs and 1 adult and 3 kids in the Nullsatzs.


\end{enumerate}