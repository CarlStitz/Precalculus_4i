\label{ExercisesforDeterminants}

Exercise ideas: revisit fitting curves to three points - function condition means det not zero (?)

Determinant formula for line

Geometry of determinant  - wait to vectors (?)

Follow up in inverse determinant section:  What about $ax^m + bx^n + cx^p$ for these three points?

In Exercises \ref{finddetfirst} - \ref{finddetlast},  compute the determinant of the given matrix.  (Some of these matrices appeared in Exercises \ref{findmatinversefirst} - \ref{findmatinverselast} in Section \ref{MatMethods}.)

\begin{multicols}{2}
\begin{enumerate}

\item $B = \left[ \begin{array}{rr} 12 & -7 \\ -5 & 3 \end{array} \right]$ \label{finddetfirst}
\item $C = \left[ \begin{array}{rr} 6 & 15 \\ 14 & 35 \end{array} \right]$ \label{matrixC}

\setcounter{HW}{\value{enumi}}
\end{enumerate}
\end{multicols}

\begin{multicols}{2}
\begin{enumerate}
\setcounter{enumi}{\value{HW}}

\item $Q = \left[ \begin{array}{rr} x & x^{2} \\ 1 & 2x \end{array} \right] \vphantom{ \left[ \begin{array}{rr} \dfrac{1}{x^{3}} & \dfrac{\ln(x)}{x^{3}} \\[10pt] -\dfrac{3}{x^{4}} & \dfrac{1 - 3\ln(x)}{x^{4}} \end{array} \right]}$
\item $L = \left[ \begin{array}{rr} \dfrac{1}{x^{3}} & \dfrac{\ln(x)}{x^{3}} \\[10pt] -\dfrac{3}{x^{4}} & \dfrac{1 - 3\ln(x)}{x^{4}} \end{array} \right]$


\setcounter{HW}{\value{enumi}}
\end{enumerate}
\end{multicols}

\begin{multicols}{2}
\begin{enumerate}
\setcounter{enumi}{\value{HW}}

\item $F = \left[ \begin{array}{rrr} 4 & \hphantom{-}6 & -3 \\ 3 & 4 & -3 \\ 1 & 2 & 6 \end{array} \right]$
\item $G = \left[ \begin{array}{rrr} 1 & \hphantom{1}2 & 3 \\ 2 & 3 & 11 \\ 3 & 4 & 19 \end{array} \right]$ \label{matrixG}

\setcounter{HW}{\value{enumi}}
\end{enumerate}
\end{multicols}

\begin{multicols}{2}
\begin{enumerate}
\setcounter{enumi}{\value{HW}}

\item $V = \left[ \begin{array}{rrr} i & j & k \\ -1 & 0 & 5 \\ 9 & -4 & -2 \end{array} \right] \vphantom{\left[ \begin{array}{rrrr} 1 & 0 & -3 & 0 \\ 2 & -2 & 8 & 7 \\ -5 & 0 & 16 & 0 \\ 1 & 0 & 4 & 1 \end{array} \right]}$
\item $H = \left[ \begin{array}{rrrr} 1 & 0 & -3 & 0 \\ 2 & -2 & 8 & 7 \\ -5 & 0 & 16 & 0 \\ 1 & 0 & 4 & 1 \end{array} \right]$ \label{finddetlast}


\setcounter{HW}{\value{enumi}}
\end{enumerate}
\end{multicols}


In Exercises \ref{solvecramerfirst} - \ref{solvecramerlast},   use Cramer's Rule to solve the system of linear equations.

\begin{multicols}{2}
\begin{enumerate}
\setcounter{enumi}{\value{HW}}

\item $\left\{ \begin{array}{rcr}   3x + 7y & = & 26 \\ 5x + 12y & = & 39  \end{array} \right.$ \label{solvecramerfirst}

\item $\left\{ \begin{array}{rcr}   2x-4y & = & 5 \\ 10x + 13y & = & -6  \end{array} \right.$

\setcounter{HW}{\value{enumi}}
\end{enumerate}
\end{multicols}

\begin{multicols}{2}
\begin{enumerate}
\setcounter{enumi}{\value{HW}}

\item $\left\{ \begin{array}{rcr}   x + y & = & 8000 \\ 0.03x + 0.05y & = & 250  \end{array} \right.$

\item $\left\{ \begin{array}{rcr}   \frac{1}{2}x  - \frac{1}{5}y & = & 1 \\ 6x +7y & = & 3  \end{array} \right.$



\setcounter{HW}{\value{enumi}}
\end{enumerate}
\end{multicols}

\begin{multicols}{2}
\begin{enumerate}
\setcounter{enumi}{\value{HW}}

\item $\left\{ \begin{array}{rcr} x + y + z & = & 3 \\ 2x - y + z & = & 0 \\ -3x + 5y + 7z & = & 7  \end{array} \right.$

\item $\left\{ \begin{array}{rcr} 3x + y - 2z & = & 10 \\ 4x - y + z & = & 5 \\ x -3y - 4z & = & -1  \end{array} \right.$ \label{solvecramerlast}

\setcounter{HW}{\value{enumi}}
\end{enumerate}
\end{multicols}

In Exercises \ref{cramersinglefirst} - \ref{cramersinglelast},  use Cramer's Rule to solve for $x_{\mbox{\tiny$4$}}$.


\begin{multicols}{2}
\begin{enumerate}
\setcounter{enumi}{\value{HW}}

\item $\left\{ \begin{array}{rcr} x_{\mbox{\tiny$1$}} - x_{\mbox{\tiny$3$}} & = & -2 \\ 
2x_{\mbox{\tiny$2$}} - x_{\mbox{\tiny$4$}} & = & 0  \\  
x_{\mbox{\tiny$1$}} -  2x_{\mbox{\tiny$2$}} + x_{\mbox{\tiny$3$}} & = & 0 \\
-x_{\mbox{\tiny$3$}} + x_{\mbox{\tiny$4$}} & = & 1  \end{array} \right.$  \label{cramersinglefirst}

\item $\left\{ \begin{array}{rcr} 4x_{\mbox{\tiny$1$}} + x_{\mbox{\tiny$2$}} & = & 4 \\ 
x_{\mbox{\tiny$2$}} - 3x_{\mbox{\tiny$3$}} & = & 1  \\  
10x_{\mbox{\tiny$1$}} +x_{\mbox{\tiny$3$}} + x_{\mbox{\tiny$4$}} & = & 0 \\
-x_{\mbox{\tiny$2$}} + x_{\mbox{\tiny$3$}} & = & -3  \end{array} \right.$  \label{cramersinglelast}

\setcounter{HW}{\value{enumi}}
\end{enumerate}
\end{multicols}

\pagebreak

In Exercises \ref{invadjfirst} - \ref{invadjlast}, find the inverse of the given matrix using their determinants and adjoints.

\begin{multicols}{2}
\begin{enumerate}
\setcounter{enumi}{\value{HW}}

\item $B = \left[ \begin{array}{rr} 12 & -7 \\ -5 & 3 \end{array} \right] \vphantom{\left[ \begin{array}{rrr} 4 & \hphantom{-}6 & -3 \\ 3 & 4 & -3 \\ 1 & 2 & 6 \end{array} \right]}$ \label{invadjfirst}
\item $F = \left[ \begin{array}{rrr} 4 & \hphantom{-}6 & -3 \\ 3 & 4 & -3 \\ 1 & 2 & 6 \end{array} \right]$ \label{invadjlast}

\setcounter{HW}{\value{enumi}}
\end{enumerate}
\end{multicols}


\begin{enumerate}
\setcounter{enumi}{\value{HW}}


\item  Carl's Sasquatch Attack! Game Card Collection is a mixture of common and rare cards.  Each common card is worth $\$0.25$ while each rare card is worth $\$0.75$. If his entire 117 card collection is worth $\$48.75$, how many of each kind of card does he own?

\item  How much of a 5 gallon $40\%$ salt solution should be replaced with pure water to obtain 5 gallons of a $15 \%$ solution?

\item  How much of a 10 liter $30\%$ acid solution must be replaced with pure acid to obtain 10 liters of a $50\%$ solution?

\item  Daniel's Exotic Animal Rescue houses snakes, tarantulas and scorpions.  When asked how many animals of each kind he boards, Daniel answered:  `We board 49 total animals, and I am responsible for each of their 272 legs and 28 tails.'  How many of each animal does the Rescue board?  (Recall:  tarantulas have 8 legs and no tails,  scorpions have 8 legs and one tail, and snakes have no legs and one tail.)

\item  This exercise is a continuation of Exercise \ref{SasquatchDiet} in Section \ref{MatMethods}.  Just because a system is consistent independent doesn't mean it will admit a solution that makes sense in an applied setting. Using the nutrient values given for Ippizuti Fish, Misty Mushrooms, and Sun Berries, use Cramer's Rule to determine the number of servings of Ippizuti Fish needed to meet the needs of a daily diet which requires 2500 calories, 1000 grams of protein, and 400 milligrams of Vitamin X. Now use Cramer's Rule to find the number of servings of Misty Mushrooms required. Does a solution to this diet problem exist?    


\item Let $R = \left[ \begin{array}{rr} -7 & 3 \\ 11 & \hphantom{-} 2 \end{array} \right], \;\;\; S = \left[ \begin{array}{rr} 1 & -5 \\ 6 & 9 \end{array} \right] \;\;\; T = \left[ \begin{array}{rr} 11 & \hphantom{-} 2 \\ -7 & 3  \end{array} \right], \mbox{ and } U = \left[ \begin{array}{rr} -3 & 15 \\ 6 & 9 \end{array} \right]$

\begin{enumerate}

\item Show that $\det(RS) = \det(R)\det(S)$
\item Show that $\det(T) = -\det(R)$
\item Show that $\det(U) = -3\det(S)$

\end{enumerate}

\item For $M$,  $N$, and $P$ below, show that $\det(M) = 0$, $\det(N) = 0$ and $\det(P) = 0$. \[M = \left[ \begin{array}{rrr} 1 & 2 & 3 \\ 0 & 0 & 0 \\ 7 & 8 & 9 \end{array} \right], \quad N = \left[ \begin{array}{rrr} 1 & 2 & 3 \\ 1 & 2 & 3 \\ 4 & 5 & 6 \end{array} \right] , \quad  P =  \left[ \begin{array}{rrr} 1 & 2 & 3 \\ -2 & -4 & -6 \\ 7 & 8 & 9 \end{array} \right]  \]

\item  This exercise is a follow-up to Exercise \ref{threepointsmatrixfunctionfitex} in Section \ref{AugMatrices}.   Suppose you wish to determine coefficients $a$, $b$, and $c$ so the the graph of $f(x) = ax^{m} + bx^{n} + cx^{p}$ contains the points  $(-2,1)$, $(1,4)$, $(3,-2)$.  With help from your classmates, discuss if there a unique solution for every selection of $m$, $n$, and $p$?  If not, under what conditions is there a unique solution?

\item Let $A$ be an arbitrary invertible $3 \times 3$ matrix.  

\begin{enumerate}

\item Show that $\det(I_{\mbox{\tiny$3$}}) = 1$. (See footnote\footnote{If you think about it for just a moment, you'll see that $\det(I_{n}) = 1$ for any natural number $n$.  The formal proof of this fact requires the Principle of Mathematical Induction (Section \ref{Induction}) so we'll stick with $n = 3$ for the time being.} below.)

\item Using the facts that $AA^{-1} = I_{3}$ and $\det(AA^{-1}) = \det(A)\det(A^{-1})$, show that \[\det(A^{-1}) = \dfrac{1}{\det(A)}\]

\end{enumerate}

\setcounter{HW}{\value{enumi}}
\end{enumerate}

The purpose of Exercises \ref{eigenfirst} - \ref{eigenlast} is to introduce you to the eigenvalues and eigenvectors of a matrix.\footnote{This material is usually given its own chapter in a Linear Algebra book so clearly we're not able to tell you everything you need to know about eigenvalues and eigenvectors.  They are a nice application of determinants, though, so we're going to give you enough background so that you can start playing around with them.}  We begin with an example using a $2 \times 2$ matrix and then guide you through some exercises using a $3 \times 3$ matrix.  Consider the matrix \[C = \left[ \begin{array}{rr} 6 & 15 \\ 14 & 35 \end{array} \right]\] from Exercise \ref{matrixC}.  We know that $\det(C) = 0$ which means that $CX = 0_{\mbox{\tiny$2$} \times \mbox{\tiny$2$}}$ does not have a unique solution.  So there is a nonzero matrix $Y$ with $CY = 0_{\mbox{\tiny$2$} \times \mbox{\tiny$2$}}$.  In fact, every matrix of the form \[Y = \left[ \begin{array}{r} -\frac{5}{2}t \\[3pt] t  \end{array} \right]\] is a solution to $CX = 0_{\mbox{\tiny$2$} \times \mbox{\tiny$2$}}$, so there are infinitely many matrices such that $CX = 0_{\mbox{\tiny$2$} \times \mbox{\tiny$2$}}$.  But consider the matrix \[X_{\mbox{\tiny$41$}} = \left[ \begin{array}{r} 3 \\ 7  \end{array} \right]\]  It is NOT a solution to $CX = 0_{\mbox{\tiny$2$} \times \mbox{\tiny$2$}}$, but rather, \[CX_{\mbox{\tiny$41$}}= \left[ \begin{array}{rr} 6 & 15 \\ 14 & 35 \end{array} \right] \left[ \begin{array}{r} 3 \\ 7  \end{array} \right] = \left[ \begin{array}{r} 123 \\ 287  \end{array} \right] = 41\left[ \begin{array}{r} 3 \\ 7  \end{array} \right]\] In fact, if $Z$ is of the form \[Z = \left[ \begin{array}{r} \frac{3}{7}t \\[3pt] t  \end{array} \right]\] then \[CZ = \left[ \begin{array}{rr} 6 & 15 \\ 14 & 35 \end{array} \right] \left[ \begin{array}{r} \frac{3}{7}t \\[3pt] t  \end{array} \right] = \left[ \begin{array}{r} \frac{123}{7}t \\[3pt] 41t  \end{array} \right] = 41\left[ \begin{array}{r} \frac{3}{7}t \\[3pt] t \end{array} \right] = 41Z\] for all $t$.  The big question is ``How did we know to use $41$?'' 

\smallskip

We need a number $\lambda$ such that $CX = \lambda X$ has nonzero solutions.  We have demonstrated that $\lambda = 0$ and $\lambda = 41$ both worked.  Are there others?  If we look at the matrix equation more closely, what we \emph{really} wanted was a nonzero solution to $(C - \lambda I_{\mbox{\tiny$2$}})X = 0_{\mbox{\tiny$2$} \times \mbox{\tiny$2$}}$ which we know exists if and only if the determinant of $C - \lambda I_{\mbox{\tiny$2$}}$ is zero.\footnote{Think about this.}  So we computed \[\det(C - \lambda I_{\mbox{\tiny$2$}}) = \det\left(\left[ \begin{array}{rr} 6 - \lambda & 15 \\ 14 & 35 - \lambda \end{array} \right] \right) = (6 - \lambda)(35 - \lambda) - 14 \cdot 15 = \lambda^{2} - 41 \lambda\]  This is called the {\bf characteristic polynomial} \index{characteristic polynomial} \index{matrix ! characteristic polynomial} of the matrix $C$ and it has two zeros: $\lambda = 0$ and $\lambda = 41$.  That's how we knew to use $41$ in our work above.  The fact that $\lambda = 0$ showed up as one of the zeros of the characteristic polynomial just means that $C$ itself had determinant zero which we already knew.  Those two numbers are called the {\bf eigenvalues} of $C$.  The corresponding matrix solutions to $CX = \lambda X$ are called the {\bf eigenvectors} of $C$ and the `vector' portion of the name will make more sense after you've studied vectors. \index{eigenvalue} \index{eigenvector}

\smallskip

Now it's your turn. In the following exercises, you'll be using the matrix $G$ from Exercise \ref{matrixG}.\[G = \left[ \begin{array}{rrr} 1 & \hphantom{1}2 & 3 \\ 2 & 3 & 11 \\ 3 & 4 & 19 \end{array} \right]\] 

\begin{enumerate}
\setcounter{enumi}{\value{HW}}

\item Show that the characteristic polynomial of $G$ is $p(\lambda) = -\lambda(\lambda - 1)(\lambda - 22)$.  That is, compute $\text{det}\left(G - \lambda I_{\mbox{\tiny$3$}}\right)$. \label{eigenfirst} 

\item Let $G_{\mbox{\tiny$0$}} = G$.  Find the parametric description of the solution to the system of linear equations given by $GX = 0_{\mbox{\tiny$3$} \times \mbox{\tiny$3$}}$.

\item Let $G_{\mbox{\tiny$1$}} = G - I_{\mbox{\tiny$3$}}$.  Find the parametric description of the solution to the system of linear equations given by $G_{\mbox{\tiny$1$}}X = 0_{\mbox{\tiny$3$} \times \mbox{\tiny$3$}}$.  Show that any solution to $G_{\mbox{\tiny$1$}}X = 0_{\mbox{\tiny$3$} \times \mbox{\tiny$3$}}$ also has the property that $GX = 1X$.

\item Let $G_{\mbox{\tiny$22$}} = G - 22 I_{\mbox{\tiny$3$}}$.  Find the parametric description of the solution to the system of linear equations given by $G_{\mbox{\tiny$22$}}X = 0_{\mbox{\tiny$3$} \times \mbox{\tiny$3$}}$.  Show that any solution to $G_{\mbox{\tiny$22$}}X = 0_{\mbox{\tiny$3$} \times \mbox{\tiny$3$}}$ also has the property that $GX = 22X$. \label{eigenlast}

\end{enumerate}

\newpage

\subsection{Answers}

\begin{multicols}{2}
\begin{enumerate}

\item $\det(B) = 1$
\item $\det(C) = 0$

\setcounter{HW}{\value{enumi}}
\end{enumerate}
\end{multicols}

\begin{multicols}{2}
\begin{enumerate}
\setcounter{enumi}{\value{HW}}

\item $\det(Q) = x^{2} \phantom{\dfrac{1}{x^{7}}}$
\item $\det(L) = \dfrac{1}{x^{7}}$

\setcounter{HW}{\value{enumi}}
\end{enumerate}
\end{multicols}

\begin{multicols}{2}
\begin{enumerate}
\setcounter{enumi}{\value{HW}}

\item $\det(F) = -12$
\item $\det(G) = 0$

\setcounter{HW}{\value{enumi}}
\end{enumerate}
\end{multicols}

\begin{multicols}{2}
\begin{enumerate}
\setcounter{enumi}{\value{HW}}

\item $\det(V) = 20i + 43j + 4k$
\item $\det(H) = -2$

\setcounter{HW}{\value{enumi}}
\end{enumerate}
\end{multicols}

\begin{multicols}{2}
\begin{enumerate}
\setcounter{enumi}{\value{HW}}

\item $x = 39, \; y = -13$
\item $x = \frac{41}{66}, \; y=-\frac{31}{33}$

\setcounter{HW}{\value{enumi}}
\end{enumerate}
\end{multicols}

\begin{multicols}{2}
\begin{enumerate}
\setcounter{enumi}{\value{HW}}

\item  $x=7500, \; y=500$
\item  $x = \frac{76}{47}, \; y=-\frac{45}{47}$


\setcounter{HW}{\value{enumi}}
\end{enumerate}
\end{multicols}

\begin{multicols}{2}
\begin{enumerate}
\setcounter{enumi}{\value{HW}}

\item $x = 1, \; y = 2, \; z = 0$
\item $x = \frac{121}{60}, \; y = \frac{131}{60}, \; z = -\frac{53}{60}$


\setcounter{HW}{\value{enumi}}
\end{enumerate}
\end{multicols}

\begin{multicols}{2}
\begin{enumerate}
\setcounter{enumi}{\value{HW}}

\item $x_{\mbox{\tiny$4$}} = 4$ 

\item  $x_{\mbox{\tiny$4$}} = -1$ 

\setcounter{HW}{\value{enumi}}
\end{enumerate}
\end{multicols}


\begin{enumerate}
\setcounter{enumi}{\value{HW}}

\item $B^{-1} = \left[ \begin{array}{rr} 3 & 7 \\ 5 & 12 \end{array} \right]$
\item $F^{-1} = \left[ \begin{array}{rrr} -\frac{5}{2} & \frac{7}{2} & \frac{1}{2} \\[3pt] \frac{7}{4} & -\frac{9}{4} & -\frac{1}{4} \\[3pt] -\frac{1}{6} & \frac{1}{6} & \frac{1}{6} \end{array} \right]$

\setcounter{HW}{\value{enumi}}
\end{enumerate}

\begin{enumerate}
\setcounter{enumi}{\value{HW}}

\item  Carl owns 78 common cards and 39 rare cards.

\item  $3.125$ gallons.

\item  $\frac{20}{7} \approx 2.85$ liters.

\item  The rescue houses 15 snakes, 21 tarantulas and 13 scorpions.

\item  Using Cramer's Rule, we find we need 53 servings of Ippizuti Fish to satisfy the dietary requirements.  The number of servings of Misty Mushrooms required, however, is $-1120$.  Since it's impossible to have a negative number of servings, there is no solution to the applied problem, despite there being a solution to the mathematical problem.  A cautionary tale about using Cramer's Rule:  just because you are guaranteed a mathematical answer for each variable doesn't mean the solution will make sense in the `real' world.


\setcounter{HW}{\value{enumi}}
\end{enumerate}
