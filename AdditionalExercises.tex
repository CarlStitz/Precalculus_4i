\documentclass[11pt]{book}
\oddsidemargin 0in
\evensidemargin 0in
\marginparwidth 0in
\textheight 8in
\textwidth 6.5in
\topmargin 0in
\headheight 14pt
\usepackage{amssymb,amsmath,amsthm,fancyhdr,supertabular,longtable,hhline}
\usepackage{colortbl}
\usepackage{import, multicol,boxedminipage}
\usepackage{chapterfolder}
\usepackage[metapost,truebbox]{mfpic}
\usepackage[pdflatex]{graphicx}
\usepackage{makeidx}
\usepackage[colorlinks, hyperindex, plainpages=false, linkcolor=blue, urlcolor=blue, pdfpagelabels]{hyperref}
\usepackage[all]{hypcap}
\usepackage{cancel}
\usepackage{sectsty}
\usepackage{textcomp}
\allsectionsfont{\mdseries \scshape}
\definecolor{ResultColor}{gray}{0.9}
\theoremstyle{definition}  % this prevents the text in definitions, theorems, and corollaries from being italicized
\newtheorem{defn}{\sc Definition}[chapter]
\newtheorem{thm}{\sc Theorem}[chapter]
\newtheorem{cor}[thm]{\sc Corollary}
\newtheorem{eqn}{\sc Equation}[chapter]
\newtheorem{ex}{\sc Example}[section]
\newtheorem{fig}{\sc Figure}[chapter]
\setlength{\parindent}{0in}
\newcommand{\bbm}{\begin{boxedminipage}{6.41in}}
\newcommand{\ebm}{\end{boxedminipage}}
\usepackage{array}
\setlength{\extrarowheight}{2pt}
\allowdisplaybreaks[2]
\allsectionsfont{\mdseries \scshape}
%Below is for Helvetica (scaled): 
\usepackage[scaled=.92]{helvet}   
\renewcommand{\familydefault}{\sfdefault}  %makes the text of the book sans serif
\usepackage[helvet]{sfmath}  %makes the math in the book sans serif
\allsectionsfont{\sffamily}  %makes the chapter and section titles sans serif
\opengraphsfile{AdditionalExercises}
\begin{document}
\newcounter{HW}
\newcounter{HWindent}
\renewcommand{\textinterrobang}{$! \! \! ?$}






\S 5.3


\begin{enumerate}

\item 

\begin{enumerate}

\item  Fill in the table below.

\[ \begin{array}{|c||c|c|}

\hline

f(x) & |f(x)| & f\left(|x| \right)  \\ \hline

x+2 &      \hphantom{\sqrt{x+3}-2}   &         \hphantom{\sqrt{x+3}-2}      \\ \hline

x^2-4x &      \hphantom{\sqrt{x+3}-2}      &      \hphantom{\sqrt{x+3}-2}       \\  \hline

x^3-3x^2 &     \hphantom{\sqrt{x+3}-2}       &    \hphantom{\sqrt{x+3}-2}       \\  \hline  

(x+1)^{-1}  &      \hphantom{\sqrt{x+3}-2}      &    \hphantom{\sqrt{x+3}-2}       \\  \hline   

\sqrt{x+2}-3&    \hphantom{\sqrt{x+3}-2}        &     \hphantom{\sqrt{x+3}-2}      \\  \hline   

 \end{array} \]
 
 \[ \begin{array}{|c||c|c|}

\hline

f(x) & |f(x)| & f\left(|x| \right)  \\ \hline

x+2 &     |x+2|   &       |x|+2      \\ \hline

x^2-4x &    |x^2-4x|      &     |x|^2-3|x| = x^2-4|x|      \\  \hline

x^3-3x^2 &    |x^3-3x^2|     &  |x|^3-3|x|^2 = |x|^3-3x^2      \\  \hline  

(x+1)^{-1}  &     |(x+1)^{-1}|      &    (|x|+1)^{-1}       \\  \hline   

\sqrt{x+2}-3&    | \sqrt{x+2}-3 |       &    \sqrt{|x|+2}-3    \\  \hline   

 \end{array} \]



\item For each function $f$ above, graph $y = f(x)$ and $y=|f(x)|$ using a graphing utility.

\begin{itemize}

\item Write a sentence (or two!) explaining how to obtain the graph of $y=|f(x)|$ from $y = f(x)$.  

To obtain the graph of $y=|f(x)|$ from $y=f(x)$, reflect about the $x$-axis any portion of the graph of $y=f(x)$ which is below the $x$-axis.

\item  How does your explanation relate to Definition \ref{absolutevaluepiecewise}?

If the graph is below the $x$-axis, then $f(x) < 0$.  Since $|f(x)| = -f(x)$ if $f(x) < 0$, we are graphing $y=-f(x)$ for these values of $x$ which is a reflection across the $x$-axis.

\end{itemize}

\item  For each function $f$ above, graph $y = f(x)$ and $y = f(|x|)$ using a graphing utility.

\begin{itemize}

\item Write a sentence (or two!) explaining how to obtain the graph of $y=f(|x|)$ from $y = f(x)$.  

To obtain the graph of $y=f(|x|)$ from $y=f(x)$, replace the portion of the graph of $y=f(x)$ for $x \leq 0$ with the reflection about the $y$-axis of the portion of the graph of $y=f(x)$ for $x \geq 0$.

\item  How does your explanation relate to Definition \ref{absolutevaluepiecewise}?

If $x < 0$, then $|x| = -x$, so $f(|x|) = f(-x)$.  Since if $x < 0$, $-x > 0$,  this  means we reflect the graph of $y=f(x)$ about the $y$-axis for $x>0$ only.

\end{itemize}

\newpage

\item Use the graph of $y=f(x)$ below to graph $y = |f(x)|$ and $y = f(|x|)$.

\begin{center}

\begin{mfpic}[15]{-5}{5}{-4}{4}
\tlabel[cc](-2.25,-3.5){\scriptsize $\left( -2, -3 \right)$}
\tlabel[cc](2,3.5){\scriptsize $\left(2, 3 \right)$}
\tlabel[cc](-4.25,0.5){\scriptsize $\left(-4, 0 \right)$}
\tlabel[cc](0.75,-0.5){\scriptsize $\left(0, 0 \right)$}
\axes
\tcaption{ \scriptsize$y = f(x)$}
\tlabel[cc](5,-0.5){\scriptsize $x$}
\tlabel[cc](0.5,4){\scriptsize $y$}
\xmarks{-4,-3,-2,-1,1,2,3,4}
\ymarks{-3,-2,-1,1,2,3}
\tlpointsep{5pt}
\scriptsize
\axislabels {x}{ {$-3 \hspace{7pt}$} -3, {$-2 \hspace{7pt}$} -2, {$-1 \hspace{7pt}$} -1, {$2$} 2, {$3$} 3, {$4$} 4}
\axislabels {y}{ {$-3$} -3, {$-2$} -2,  {$1$} 1, {$2$} 2, {$3$} 3}
\normalsize
\penwd{1.25pt}
\function{-4,2,.1}{3*sin(3.14159265*x/4)}
\point[4pt]{(-2,-3), (2,3),  (-4,0), (0,0)}
\end{mfpic}

\end{center}

\begin{multicols}{2}

\begin{mfpic}[15]{-5}{5}{-4}{4}
\tlabel[cc](-2.25,3.5){\scriptsize $\left( -2, -3 \right)$}
\tlabel[cc](2,3.5){\scriptsize $\left(2, 3 \right)$}
\tlabel[cc](-4.5,-0.5){\scriptsize $\left(-4, 0 \right)$}
\tlabel[cc](0.75,-0.5){\scriptsize $\left(0, 0 \right)$}
\axes
\tcaption{ \scriptsize$y = |f(x)|$}
\tlabel[cc](5,-0.5){\scriptsize $x$}
\tlabel[cc](0.5,4){\scriptsize $y$}
\xmarks{-4,-3,-2,-1,1,2,3,4}
\ymarks{-3,-2,-1,1,2,3}
\tlpointsep{5pt}
\scriptsize
\axislabels {x}{ {$-3 \hspace{7pt}$} -3,{$-2 \hspace{7pt}$} -2, {$-1 \hspace{7pt}$} -1, {$2$} 2, {$3$} 3, {$4$} 4}
\axislabels {y}{ {$-3$} -3, {$-2$} -2,  {$1$} 1, {$2$} 2, {$3$} 3}
\normalsize
\penwd{1.25pt}
\function{-4,2,.1}{3*abs(sin(3.14159265*x/4))}
\point[4pt]{(-2,3), (2,3),  (-4,0), (0,0)}
\end{mfpic}

\begin{mfpic}[15]{-5}{5}{-4}{4}
\tlabel[cc](-2.25,3.5){\scriptsize $\left( -2, -3 \right)$}
\tlabel[cc](2,3.5){\scriptsize $\left(2, 3 \right)$}
\tlabel[cc](0.75,-0.5){\scriptsize $\left(0, 0 \right)$}
\axes
\tcaption{ \scriptsize$y = f(|x|)$ - note the domain}
\tlabel[cc](5,-0.5){\scriptsize $x$}
\tlabel[cc](0.5,4){\scriptsize $y$}
\xmarks{-4,-3,-2,-1,1,2,3,4}
\ymarks{-3,-2,-1,1,2,3}
\tlpointsep{5pt}
\scriptsize
\axislabels {x}{ {$-3 \hspace{7pt}$} -3, {$-2 \hspace{7pt}$} -2, {$-1 \hspace{7pt}$} -1, {$2$} 2, {$3$} 3}
\axislabels {y}{ {$-3$} -3, {$-2$} -2,  {$-1$} -1, {$2$} 2, {$3$} 3}
\normalsize
\penwd{1.25pt}
\function{-2,2,.1}{3*sin(3.14159265*abs(x)/4)}
\point[4pt]{(-2,3), (2,3),  (0,0)}
\end{mfpic}

\end{multicols}
\end{enumerate}

\end{enumerate}


\end{document}