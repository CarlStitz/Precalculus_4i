\documentclass{ximera}

\begin{document}
	\author{Stitz-Zeager}
	\xmtitle{TITLE}


\mfpicnumber{1}

\opengraphsfile{AppLinearSystems}

\setcounter{footnote}{0}

\label{AppLinearSystems}

\setlength{\extrarowheight}{0pt}

This section of the Appendix combines ideas from Section \ref{AppLinearEqIneq} and \ref{AppLines} so that we can start to solve systems of linear equations.  Before we get ahead of ourselves, let's review a few definitions.

\medskip

\colorbox{ResultColor}{\bbm

\begin{defn}  \label{lineareqntwovariables}  A \index{equation ! linear of two variables}\index{linear equation ! two variables}\textbf{linear equation in two variables} is an equation of the form $a_{\mbox{\tiny$1$}} x + a_{\mbox{\tiny$2$}} y = c$ where $a_{\mbox{\tiny$1$}}$, $a_{\mbox{\tiny$2$}}$ and $c$ are real numbers and at least one of $a_{\mbox{\tiny$1$}}$ and $a_{\mbox{\tiny$2$}}$ is nonzero.

\end{defn}

\ebm}

\medskip

For reasons which will become clear when you study Chapter \ref{SystemsofEquationsandMatrices}, we are using subscripts in Definition \ref{lineareqntwovariables} to indicate different, but fixed, real numbers and those subscripts have no mathematical meaning beyond that.  For example, $3x - \frac{y}{2} = 0.1$ is a linear equation in two variables with $a_{\mbox{\tiny$1$}} = 3$, $a_{\mbox{\tiny$2$}} = -\frac{1}{2}$ and $c = 0.1$.  We can also consider $x = 5$ to be a linear equation in two variables\footnote{Critics may argue that $x=5$ is clearly an equation in one variable.  It can also be considered an equation in $117$ variables with the coefficients of $116$ variables set to $0$.  As with many conventions in Mathematics, the context will clarify the situation.} by identifying $a_{\mbox{\tiny$1$}} = 1$, $a_{\mbox{\tiny$2$}} = 0$, and $c = 5$.  

\medskip

If $a_{\mbox{\tiny$1$}}$ and $a_{\mbox{\tiny$2$}}$ are both $0$, then depending on $c$, we get either an equation which is \textit{always} true, called an \index{equation ! identity}\index{identity ! statement which is always true}\textbf{identity}, or an equation which is \textit{never} true, called a \index{equation ! contradiction}\index{contradiction}\textbf{contradiction}. (If $c = 0$, then we get $0 = 0$, which is always true.  If $c \neq 0$, then we'd have  $0 \neq 0$, which is never true.)  Even though identities and contradictions have a large role to play throughout Chapter \ref{SystemsofEquationsandMatrices}, we do not consider them linear equations.  The key to identifying linear equations is to note that the variables involved are to the first power and that the coefficients of the variables are numbers.  Some examples of equations which are non-linear are $x^2 + y = 1$, $xy = 5$ and $e^{2x} + \ln(y) = 1$.  The reader should consider why these do not satisfy Definition \ref{lineareqntwovariables}.  

\medskip

We know from our work is Sections \ref{AppLines} that the graphs of linear equations are lines.  If we couple two or more linear equations together, in effect to find the points of intersection of two or more lines, we obtain a \textbf{system of linear equations in two variables}. \index{system of equations ! linear ! two variables} Our first example explores the basic techniques for solving these systems.  Remember - if we are looking for points in the plane, then both the $x$ and $y$ values are important.  This is a key distinction between solving one equation and solving a \emph{system} of equations.

\begin{ex} \label{reviewsubelim}  Solve the following systems of equations.  Check your answer algebraically and graphically.  (Said another way, make sure both $x$ and $y$ are correct!)

\begin{multicols}{3}
\begin{enumerate}

\item  $\left\{ \begin{array}{rcr} 2x - y & = & 1 \\ y & = & 3 \\ \end{array} \right.$ \vphantom{$\left\{ \begin{array}{rcr} \frac{x}{3} -\frac{4y}{5} & = & \frac{7}{5} \\ [3pt] 
\frac{2x}{9} + \frac{y}{3} & = & \frac{1}{2} \\ \end{array} \right.$}

\item  $\left\{ \begin{array}{rcr} 3x+4y & = & -2  \\ -3x-y & = & 5 \\ \end{array} \right.$ \vphantom{$\left\{ \begin{array}{rcr} \frac{x}{3} -\frac{4y}{5} & = & \frac{7}{5}\\[3pt] \frac{2x}{9} + \frac{y}{3} & = & \frac{1}{2} \\ \end{array} \right.$}

\item  $\left\{ \begin{array}{rcr} \frac{x}{3} -\frac{4y}{5} & = & \frac{7}{5} \\ [3pt] 
\frac{2x}{9} + \frac{y}{3} & = & \frac{1}{2} \\ \end{array} \right.$

\setcounter{HW}{\value{enumi}}
\end{enumerate}
\end{multicols}

\begin{multicols}{3}
\begin{enumerate}
\setcounter{enumi}{\value{HW}}

\item  $\left\{ \begin{array}{rcr} 2x - 4y & = & 6 \\ 3x -6y & = & 9\\ \end{array} \right.$ \vphantom{$\left\{ \begin{array}{rcr} x - y & = & 0 \\ x + y & = & 2 \\ -2x + y & = & -2 \end{array} \right.$}

\item  $\left\{ \begin{array}{rcr} 6x + 3y & = & 9 \\ 4x + 2y & = & 12 \\ \end{array} \right.$ \vphantom{$\left\{ \begin{array}{rcr} x - y & = & 0 \\ x + y & = & 2 \\ -2x + y & = & -2 \end{array} \right.$}

\item  $\left\{ \begin{array}{rcr} x - y & = & 0 \\ x + y & = & 2 \\ -2x + y & = & -2 \end{array} \right.$

\end{enumerate}
\end{multicols}

{\bf Solution.}

\begin{enumerate}

\item  Our first system is nearly solved for us.  The second equation tells us that $y=3$.  To find the corresponding value of $x$, we \textbf{substitute} this value for $y$ into the the first equation to obtain $2x -  3 = 1$, so that $x = 2$.  Our solution to the system is $(2,3)$.  To check this algebraically, we substitute $x=2$ and $y=3$ into each equation and see that they are satisfied.  We see $2(2) - 3 = 1$, and $3=3$, as required.  To check our answer graphically, we graph the lines $2x-y = 1$ and $y=3$ and verify that they intersect at $(2,3)$.

\setlength{\extrarowheight}{2pt}

\item  To solve the second system, we use the \textbf{addition} method to \textbf{eliminate} the variable $x$.  We take the two equations as given and `add equals to equals' to obtain \[ \begin{array}{lrcr} & 3x+4y & = & -2  \\ + & (-3x-y  & = & 5 ) \\ \hline & 3y & = & 3\end{array}\] This gives us $y = 1$.  We now substitute $y=1$ into either of the two equations, say $-3x-y = 5$, to get $-3x-1 = 5$ so that $x = -2$.  Our solution is $(-2,1)$.  Substituting $x=-2$ and $y=1$ into the first equation gives $3(-2) + 4(1) = -2$, which is true, and, likewise, when we check $(-2, 1)$ in the second equation, we get $-3(-2) - 1 = 5$, which is also true.  Geometrically, the lines $3x+4y = -2$ and $-3x-y=5$ intersect at $(-2,1)$.

\setlength{\extrarowheight}{0pt}

\begin{center}

\begin{tabular}{m{.5in}m{2in}m{.5in}m{2in}}

$~$

&

\begin{mfpic}[15]{-2}{5}{-2}{5}
\arrow \reverse \arrow \polyline{(-0.5,-2), (3,5)}
\point[3pt]{(2,3)}
\axes
\tlabel[cc](2.5,2.5){\tiny $(2,3)$}
\xmarks{-1,1,2,3,4}
\ymarks{-1,1,2,3,4}
\tlabel(5,-0.5){\scriptsize $x$}
\tlabel(0.5,5){\scriptsize $y$}
\tcaption{\scriptsize \centerline{$2x-y=1$} \\ \centerline{\boldmath $y=3$}}
\tlpointsep{4pt}
\axislabels {x}{{\tiny $-1 \hspace{7pt}$} -1, {\tiny $1$} 1, {\tiny $2$} 2, {\tiny $3$} 3, {\tiny $4$} 4}
\axislabels {y}{{\tiny $1$} 1, {\tiny $2$} 2, {\tiny $4$} 4}
\penwd{1.1pt}
\arrow \reverse \arrow \polyline{(-2,3), (5,3)}
\end{mfpic}

&

$~$
 
&

\begin{mfpic}[15]{-5}{1}{-3}{3}
\arrow \reverse \arrow \polyline{(-4.666,3), (1,-1.25)}
\point[3pt]{(-2,1)}
\axes
\tlabel[cc](-3,0.5){\tiny $(-2,1)$}
\xmarks{-4,-3,-2,-1}
\ymarks{-2,-1,1,2}
\tlabel(1,-0.5){\scriptsize $x$}
\tlabel(0.5,3){\scriptsize $y$}
\tcaption{\scriptsize \centerline{$3x+4y=-2$} \\ \centerline{\boldmath $-3x-y=5$}}
\tlpointsep{4pt}
\axislabels {x}{{\tiny $-4 \hspace{7pt}$} -4,{\tiny $-3 \hspace{7pt}$} -3,{\tiny $-2 \hspace{7pt}$} -2,{\tiny $-1 \hspace{7pt}$} -1}
\axislabels {y}{{\tiny $-2$} -2,{\tiny $-1$} -1,{\tiny $1$} 1, {\tiny $2$} 2}
\penwd{1.1pt}
\arrow \reverse \arrow \polyline{(-2.666,3), (-0.666,-3)}
\end{mfpic}

\\

\end{tabular}

\end{center}

\setlength{\extrarowheight}{2pt} 

\item  The equations in the third system are more approachable if we clear denominators.  We multiply both sides of the first equation by $15$ and both sides of the second equation by $18$ to obtain the kinder, gentler system \[\left\{ \begin{array}{rcr} 5x - 12y & = & 21  \\ 4x  + 6y & = & 9 \\ \end{array} \right.\]  Adding these two equations directly fails to eliminate either of the variables, but we note that if we multiply the first equation by $4$ and the second by $-5$, we will be in a position to eliminate the $x$ term \[ \begin{array}{lrcr} & 20x-48y & = & 84  \\ + & (-20x-30y  & = & -45 ) \\ \hline  & -78y & = & 39\end{array}\] From this we get $y = -\frac{1}{2}$.  We can temporarily avoid too much unpleasantness by choosing to substitute $y = -\frac{1}{2}$ into one of the equivalent equations we found by clearing denominators, say into $5x - 12y  =  21$.  We get $5x + 6 = 21$ which gives $x=3$.  Our answer is $\left(3, -\frac{1}{2}\right)$.  At this point, we have no choice $-$ in order to check an answer algebraically, we must see if the answer satisfies both of the \textit{original} equations, so we substitute $x = 3$ and $y = -\frac{1}{2}$ into both $\frac{x}{3} -\frac{4y}{5} = \frac{7}{5}$ and $\frac{2x}{9} + \frac{y}{3} = \frac{1}{2}$.  We leave it to the reader to verify that the solution is correct.  Graphing both of the lines involved with considerable care yields an intersection point of $\left(3, -\frac{1}{2}\right)$.  (The picture is on the next page.)


\item  An eerie calm settles over us as we cautiously approach our fourth system.  Do its friendly integer coefficients belie something more sinister?  We note that if we multiply both sides of the first equation by $3$ and both sides of the second equation by $-2$, we are ready to eliminate the $x$ \[ \begin{array}{lrcr} & 6x-12y & = & 18  \\ + & (-6x+12y  & = & -18 ) \\ \hline  & 0 & = & 0\end{array}\] We eliminated not only the $x$, but the $y$ as well and we are left with the identity $0=0$.  This means that these two different linear equations are, in fact, equivalent.  In other words, if an ordered pair $(x,y)$ satisfies the equation $2x-4y = 6$, it \textit{automatically} satisfies the equation $3x-6y = 9$.  

\setlength{\extrarowheight}{0pt} 

\medskip

This system has infinitely many solutions and one way to describe the solution set to this system is to use the roster method\footnote{See Section \ref{AppSetTheory} for a review of this.} and write $\{(x,y) \, | \, 2x-4y = 6\}$.  While this is correct (and corresponds exactly to what's happening graphically, as we shall see shortly), we take this opportunity to introduce the notion of a \index{parametric solution} \index{system of equations ! parametric solution}\textbf{parametric solution to a system}.  

\medskip

Our first step is to solve $2x-4y = 6$ for one of the variables, say $y = \frac{1}{2} x - \frac{3}{2}$.  For each value of $x$, the formula $y = \frac{1}{2} x - \frac{3}{2}$ determines the corresponding $y$-value of a solution.  Since we have no restriction on $x$, it is called a \index{system of equations ! free variable}\index{free variable}\textbf{free variable}.  We let $x=t$, a so-called `parameter', and get $y = \frac{1}{2} t - \frac{3}{2}$. Our set of solutions can then be described as $\left\{ \left(t, \frac{1}{2} t - \frac{3}{2}\right) \, | \, -\infty < t < \infty \right\}$.\footnote{Note that we could have just as easily chosen to solve $2x-4y = 6$ for $x$ to obtain $x = 2y + 3$.  Letting $y$ be the parameter $t$, we have that for any value of $t$, $x = 2t+3$, which gives $\{(2t+3, t) \, | \, - \infty < t < \infty\}$.  There is no one correct way to parameterize the solution set, which is why it is always best to check your answer.}  

\medskip

For specific values of $t$, we can generate solutions.  For example, $t=0$ gives us the solution $\left(0,-\frac{3}{2}\right)$;  $t = 117$ gives us $(117,57)$, and while we can  check that each of these particular solutions satisfy both equations, the question is how do we check our general answer algebraically?  Same as always.  

\medskip

We claim that for any real number $t$, the pair $\left(t, \frac{1}{2} t - \frac{3}{2}\right)$ satisfies both equations.  Substituting $x = t$ and $y =  \frac{1}{2} t - \frac{3}{2}$ into $2x - 4y = 6$ gives $2t - 4\left(\frac{1}{2} t - \frac{3}{2}\right) = 6$.  Simplifying, we get $2t - 2t + 6 = 6$, which is always true.  Similarly, when we make these substitutions in the equation $3x-6y = 9$, we get $3t - 6\left(\frac{1}{2} t - \frac{3}{2}\right) = 9$ which reduces to $3t - 3t + 9 = 9$, so it checks out, too.  

\medskip

Geometrically, $2x-4y = 6$ and $3x-6y=9$ are the same line, which means that they intersect at every point on their graphs.  The reader is encouraged to think about how our parametric solution says exactly that.

\pagebreak

The picture for this system is shown below on the right while the picture for the previous example is shown on the left.

\begin{center}

\begin{tabular}{m{.5in}m{2in}m{.5in}m{2in}}

$~$

&

\begin{mfpic}[10]{-2}{8}{-5}{3}
\arrow \reverse \arrow \polyline{(-2,-2.583), (8,1.583)}
\point[3pt]{(3,-0.5)}
\axes
\tlabel[cc](3,-2){\tiny $\left(3,-\frac{1}{2}\right)$}
\xmarks{-1,1,2,3,4,5,6,7}
\ymarks{-4,-3,-2,-1,1,2}
\tlabel(8,-0.5){\scriptsize $x$}
\tlabel(0.5,3){\scriptsize $y$}
\tcaption{\scriptsize \centerline{$\frac{x}{3} -\frac{4y}{5} = \frac{7}{5}$} \\ \centerline{\boldmath $\frac{2x}{9} + \frac{y}{3} = \frac{1}{2}$}}
\tlpointsep{4pt}
\axislabels {x}{{\tiny $-1 \hspace{7pt}$} -1, {\tiny $1$} 1, {\tiny $2$} 2, {\tiny $4$} 4, {\tiny $5$} 5, {\tiny $6$} 6, {\tiny $7$} 7}
\axislabels {y}{{\tiny $-4 \hspace{7pt}$} -4,{\tiny $-3 \hspace{7pt}$} -3,{\tiny $-2 \hspace{7pt}$} -2,{\tiny $-1 \hspace{7pt}$} -1,{\tiny $1$} 1}
\penwd{1.1pt}
\arrow \reverse \arrow \polyline{(-2.25,3), (8,-3.833)}
\end{mfpic}

&

$~$
 
&

\begin{mfpic}[15]{-1}{5}{-3}{3}
\axes
\xmarks{1,2,3,4}
\ymarks{-2,-1,1,2}
\tlabel(5,-0.5){\scriptsize $x$}
\tlabel(0.5,3){\scriptsize $y$}
\tcaption{\scriptsize \centerline{$2x - 4y = 6$} \\ \centerline{\boldmath $3x-6y = 9$} \\ \centerline{(Same line.)}}
\tlpointsep{4pt}
\axislabels {x}{{\tiny $1$} 1,{\tiny $2$} 2,{\tiny $3$} 3,{\tiny $4$} 4}
\axislabels {y}{{\tiny $-1$} -1,{\tiny $1$} 1, {\tiny $2$} 2}
\penwd{1.1pt}
\arrow \reverse \arrow \polyline{(-1,-2), (5,1)}
\end{mfpic}

\\

\end{tabular}

\end{center}

\item  Multiplying both sides of the first equation by $2$ and the both sides of the second equation by $-3$, we set the stage to eliminate $x$ 

\setlength{\extrarowheight}{2pt}
\[ \begin{array}{lrcr} & 12x + 6y & = & 18  \\ + & (-12x-6y  & = & -36 ) \\ \hline  & 0 & = & -18 \end{array}\] 
\setlength{\extrarowheight}{0pt}

As in the previous example, both $x$ and $y$ dropped out of the equation, but we are left with an irrevocable contradiction, $0 = -18$. This tells us that it is impossible to find a pair $(x,y)$ which satisfies both equations; in other words, the system has no solution.  Graphically, the lines  $6x + 3y =9$ and  $4x + 2y = 12$ are distinct and parallel, so they do not intersect.

\item  We can begin to solve our last system by adding the first two equations  

\setlength{\extrarowheight}{2pt}
\[ \begin{array}{lrcr} & x - y & = & 0  \\ + & (x + y & = & 2 ) \\ \hline  & 2x & = & 2 \end{array}\]  
\setlength{\extrarowheight}{0pt}

which gives $x = 1$.  Substituting this into the first equation gives $1 - y = 0$ so that $y = 1$.  We seem to have determined a solution to our system, $(1,1)$.  While this checks in the first two equations, when we substitute $x=1$ and $y=1$ into the third equation, we get $-2(1) + (1) = -2$ which simplifies to the contradiction $-1 = -2$.  Graphing the lines $x-y=0$, $x+y = 2$, and $-2x+y=-2$, we see that the first two lines do, in fact, intersect at $(1,1)$, however, all three lines never intersect at the same point simultaneously, which is what is required if a solution to the system is to be found.

\begin{center}

\begin{tabular}{m{.5in}m{2in}m{.5in}m{2in}}

$~$

&

\begin{mfpic}[15][7]{-1}{4}{-4}{7}
\arrow \reverse \arrow \polyline{(-1,5), (3.5,-4)}
\axes
\xmarks{1,2,3}
\ymarks{-3,-2,-1,1,2,3,4,5,6}
\tlabel(4,-0.5){\scriptsize $x$}
\tlabel(0.5,7){\scriptsize $y$}
\tcaption{\scriptsize \centerline{$6x + 3y =9$} \\ \centerline{\boldmath $4x + 2y = 12$}}
\tlpointsep{4pt}
\axislabels {x}{{\tiny $1$} 1, {\tiny $2$} 2}
\axislabels {y}{{\tiny $-3 \hspace{7pt}$} -3,{\tiny $-2 \hspace{7pt}$} -2,{\tiny $-1 \hspace{7pt}$} -1,{\tiny $1$} 1,{\tiny $2$} 2,{\tiny $3$} 3,{\tiny $4$} 4,{\tiny $5$} 5,{\tiny $6$} 6}
\penwd{1.1pt}
\arrow \reverse \arrow \polyline{(-0.5,7), (4,-2)}
\end{mfpic}

&

$~$
 
&

\begin{mfpic}[15]{-1}{3}{-3}{3}
\axes
\xmarks{1,2}
\ymarks{-2,-1,1,2}
\tlabel(3,-0.5){\scriptsize $x$}
\tlabel(0.5,3){\scriptsize $y$}
\tcaption{\scriptsize \centerline{$y-x = 0$} \\ \centerline{$y+x = 2$} \\ \centerline{$-2x+y=-2$}}
\tlpointsep{4pt}
\axislabels {y}{{\tiny $-1$} -1,{\tiny $1$} 1}
\arrow \reverse \arrow \polyline{(-1,-1), (3,3)}
\arrow \reverse \arrow \polyline{(-1,3), (3,-1)}
\arrow \reverse \arrow \polyline{(-0.5,-3), (2.5,3)}
\end{mfpic} \\

\end{tabular}

\end{center}

\vspace{-.25in} \qed

\end{enumerate}

\end{ex}

A few remarks about Example \ref{reviewsubelim} are in order.  Notice that some of the systems of linear equations had solutions while others did not.  Those which have solutions are called \index{consistent system}\index{system of equations ! consistent}\textbf{consistent}, those with no solution are called \index{inconsistent system}\index{system of equations ! inconsistent}\textbf{inconsistent}.  We also distinguish between the two different types of behavior among consistent systems. Those which admit free variables are called \index{system of equations ! dependent}\index{dependent system}\textbf{dependent} and those with no free variables are called \index{system of equations ! independent}\index{independent system}\textbf{independent}.\footnote{In the case of systems of linear equations, regardless of the number of equations or variables, consistent independent systems have exactly one solution.  The reader is encouraged to think about why this is the case for linear equations in two variables.  Hint: think geometrically.}   

\medskip

Using this new vocabulary, we classify numbers 1, 2 and 3 in Example \ref{reviewsubelim} as consistent independent systems, number 4 is consistent dependent, and numbers 5 and 6 are inconsistent.\footnote{The adjectives `dependent' and `independent' apply only to \textit{consistent} systems -- they describe the \textit{type} of solutions.  Is there a free variable (dependent) or not (independent)?}  The system in 6 above is called \index{system of equations ! overdetermined}\index{overdetermined system}\textbf{overdetermined}, since we have more equations  than variables.\footnote{If we think if each variable being an unknown quantity, then ostensibly, to recover two unknown quantities, we need two pieces of information - i.e., two equations.  Having more than two equations suggests we have more information than necessary to determine the values of the unknowns.  While this is not necessarily the case, it does explain the choice of terminology `overdetermined'.}  Not surprisingly, a system with more variables than equations is called \index{system of equations ! underdetermined}\index{underdetermined system}\textbf{underdetermined}.  While the system in number 6 above is overdetermined and inconsistent, there exist overdetermined consistent systems (both dependent and independent) and we leave it to the reader to think about what is happening algebraically and geometrically in these cases.  Likewise, there are both consistent and inconsistent underdetermined systems,\footnote{We need more than two variables to give an example of the latter.} but a consistent underdetermined system of linear equations is necessarily dependent.\footnote{Again, experience with systems with more variables helps to see this here, as does a solid course in Linear Algebra.}  

\medskip

We end this section with a story problem.  It is an example of a classic ``mixture'' problem and should be familiar to most readers.  The basic goal here is to create two equations: one which represents \[\text{stuff} + \text{other stuff} = \text{total stuff}\] and the other which represents \[\text{value of stuff} + \text{value of other stuff} =  \text{value of total stuff.}\] 
 
\begin{ex} \label{dudebromixture} The Dude-Bros want to create a highly caffeinated, yet still drinkable, fruit punch for their annual ``Disturb the Neighbors BBQ and Dance Competition''. They plan to add Sasquatch Sweat\textsuperscript{TM} Energy Drink, which has 100 mg.\ of caffeine per fluid ounce, to Frooty Giggle Delight\textsuperscript{TM}, which has only 3 mg.\ of caffeine per fluid ounce.  How much of each component is required to make 5 gallons\footnote{Warning: unit conversion ahead!} of a fruit punch that has 80 mg.\ of caffeine per fluid ounce.

\medskip

{\bf Solution.}  Let $S$ stand for the number of fluid ounces of Sasquatch Sweat\textsuperscript{TM} Energy Drink and let $F$ be the number of fluid ounces of Frooty Giggle Delight\textsuperscript{TM} that will be added together.  The goal is to make 5 gallons and there are 128 fluid ounces per gallon so the first equation is \[S + F = 640.\] That equation describes ``stuff + other stuff = total stuff'' measured in fluid ounces. Now we need to consider the value of the stuff - in this case we need to see how much caffeine is being contributed by each component.  Each fluid ounce of Sasquatch Sweat\textsuperscript{TM} contains 100 mg.\ of caffeine so $S$ fluid ounces would contain $100S$ mg./ of caffeine.

\medskip

Similarly, the $F$ fluid ounces of Frooty Giggle Delight\textsuperscript{TM} add $3F$ mg.\ of caffeine to the total mixture.  Thus when we go to express ``value of stuff + value of other stuff = value of total stuff'' we need to figure out how much caffeine is supposed to be in the end product.  Well, the goal was 5 gallons of punch that had 80 mg.\ of caffeine per fluid ounce so the Dude-Bros need to end up with $5*128*80 = 51200$ mg. of caffeine when they're done.  Hence the second is equation is \[100S + 3F = 51200.\]

\medskip

By turning the first equation into $F = 640 - S$ and substituting that into the second equation we get \[100S + 3(640 - S) = 51200\] which yields $S = \frac{49280}{97} \approx 508.04$ fluid ounces.  Back-substituting this value of $S$ into the first equation gives us $F = \frac{12800}{97} \approx 131.96$ fluid ounces.

\medskip 

The reader should take the time to verify that $S = \frac{49280}{97}$ and $F = \frac{12800}{97}$ do indeed satisfy both equations and thus are the solution to the problem. Those are fairly unattractive numbers so we end this example by discussing a way to verify an approximate answer which is \emph{reasonable} without having to fight with fractions.  Round $S$ down to 508 and round $F$ up to 132. Clearly $508 + 132 = 640$ so the first equation is still satisfied.  Notice that $100*508 + 3*132 = 51196$ which is really close to 51200. Thus the second equation is nearly satisfied which means the values $S = 508$ and $F = 132$, while not precise, are reasonable.\footnote{Just be careful here - sometimes ``close enough for the Dude-Bros'' is not good enough for your Professor!} \qed

\end{ex}

\newpage

\subsection{Exercises}

%% SKIPPED %% \documentclass{ximera}

\begin{document}
	\author{Stitz-Zeager}
	\xmtitle{TITLE}
\mfpicnumber{1} \opengraphsfile{ExercisesforAppLinearSystems} % mfpic settings added 


\label{ExercisesforAppLinearSystems}

In Exercises \ref{reviewsystemfirst} - \ref{reviewsystemlast}, solve the given system using substitution and/or elimination. Classify each system as consistent independent, consistent dependent, or inconsistent. Check your answers both algebraically and graphically.

\begin{multicols}{2}
\begin{enumerate}

\item $\left\{ \begin{array}{rcr} x+2y & = & 5  \\ x  & = & 6  \end{array} \right.$ \label{reviewsystemfirst}  

\item  $\left\{ \begin{array}{rcr} 2y-3x & = & 1  \\ y  & = & -3 \end{array} \right.$  


\setcounter{HW}{\value{enumi}}
\end{enumerate}
\end{multicols}

\begin{multicols}{2}
\begin{enumerate}
\setcounter{enumi}{\value{HW}}

\item  $\left\{ \begin{array}{rcr} \frac{x+2y}{4} & = & -5  \\[5pt] \frac{3x-y}{2}  & = & 1 \end{array} \right.$ 


\item $\left\{ \begin{array}{rcr} \frac{2}{3} x-\frac{1}{5}y & = & 3  \\[5pt]  \frac{1}{2}x+\frac{3}{4}y& = & 1  \end{array} \right.$

\setcounter{HW}{\value{enumi}}
\end{enumerate}
\end{multicols}

\begin{multicols}{2}
\begin{enumerate}
\setcounter{enumi}{\value{HW}}

\item  $\left\{ \begin{array}{rcr} \frac{1}{2}x-\frac{1}{3}y & = & -1  \\ [5pt] 2y-3x & = & 6 \end{array} \right.$  


\item $\left\{ \begin{array}{rcr} x+4y & = & 6  \\ [5pt] \frac{1}{12}x+\frac{1}{3}y& = & \frac{1}{2}  \end{array} \right.$ 

\setcounter{HW}{\value{enumi}}
\end{enumerate}
\end{multicols}



\begin{multicols}{2}
\begin{enumerate}
\setcounter{enumi}{\value{HW}}

\item  $\left\{ \begin{array}{rcr} 3y-\frac{3}{2}x & = & -\frac{15}{2}  \\ [5pt] \frac{1}{2}x-y & = & \frac{3}{2} \end{array} \right.$   


\item $\left\{ \begin{array}{rcr} \frac{5}{6}x+\frac{5}{3}y & = & -\frac{7}{3}  \\ [5pt] -\frac{10}{3}x-\frac{20}{3}y & = & 10  \end{array} \right.$ \label{reviewsystemlast} 


\setcounter{HW}{\value{enumi}}
\end{enumerate}
\end{multicols}

\begin{enumerate}
\setcounter{enumi}{\value{HW}}

\item  A local buffet charges $\$7.50$ per person for the basic buffet and $\$9.25$ for the deluxe buffet (which includes crab legs.)  If 27 diners went out to eat and the total bill was $\$227.00$ before taxes, how many chose the basic buffet and how many chose the deluxe buffet?

\item At The Old Home Fill'er Up and Keep on a-Truckin' Cafe, Mavis mixes two different types of coffee beans to produce a house blend.   The first type costs \$3 per pound and the second costs \$8 per pound.  How much of each type does Mavis use to make 50 pounds of a blend which costs \$6 per pound?

\item  Skippy has a total of $\$$10,000 to split between two investments.  One account offers $3\%$ simple interest, and the other account offers $8\%$ simple interest.  For tax reasons, he can only earn $\$500$ in interest the entire year.  How much money should Skippy invest in each account to earn $\$500$ in interest for the year?

\item A $10 \%$ salt solution is to be mixed with pure water to produce 75 gallons of a $3\%$ salt solution.  How much of each are needed?

\item This exercise is a follow-up to Example \ref{dudebromixture}.  Work with your classmates to explain why mixing 4 gallons of Sasquatch Sweat\textsuperscript{TM} Energy Drink and 1 gallon of Frooty Giggle Delight\textsuperscript{TM} would also produce a mixture that was ``close enough for the Dude-Bros''.

\end{enumerate}

\newpage

\subsection{Answers}

\begin{multicols}{2}
\begin{enumerate}

\item Consistent independent \\
Solution $\left(6, -\frac{1}{2}\right)$

\item Consistent independent \\
Solution $\left(-\frac{7}{3}, -3\right)$ 


\setcounter{HW}{\value{enumi}}
\end{enumerate}
\end{multicols}

\begin{multicols}{2}
\begin{enumerate}
\setcounter{enumi}{\value{HW}}

\item  Consistent independent \\
Solution $\left(-\frac{16}{7}, -\frac{62}{7}\right)$  

\item Consistent independent \\
Solution $\left(\frac{49}{12}, -\frac{25}{18}\right)$

\setcounter{HW}{\value{enumi}}
\end{enumerate}
\end{multicols}

\begin{multicols}{2}
\begin{enumerate}
\setcounter{enumi}{\value{HW}}

\item  Consistent dependent\\
Solution $\left(t, \frac{3}{2}t+3\right)$ \\
for all real numbers $t$

\item  Consistent dependent\\
Solution $\left(6-4t, t\right)$ \\
for all real numbers $t$

\setcounter{HW}{\value{enumi}}
\end{enumerate}
\end{multicols}



\begin{multicols}{2}
\begin{enumerate}
\setcounter{enumi}{\value{HW}}

\item  Inconsistent \\
No solution

\item   Inconsistent \\
No solution


\setcounter{HW}{\value{enumi}}
\end{enumerate}
\end{multicols}

\begin{enumerate}
\setcounter{enumi}{\value{HW}}

\item  $13$ chose the basic buffet and $14$ chose the deluxe buffet.

\item Mavis needs 20 pounds of \$3 per pound coffee and 30 pounds of \$8 per pound coffee.

\item  Skippy needs to invest $\$$6000 in the $3\%$ account and $\$$4000 in the $8 \%$ account.

\item  $22.5$ gallons of the $10 \%$ solution and $52.5$ gallons of pure water.

\end{enumerate}



\end{document}


\closegraphsfile

\end{document}
