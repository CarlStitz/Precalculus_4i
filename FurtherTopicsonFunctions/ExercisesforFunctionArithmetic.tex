\label{ExercisesforFunctionArithmetic}

In Exercises \ref{basicarithonefirst} - \ref{basicarithonelast}, use the pair of functions $f$ and $g$ to find the following values if they exist.

\begin{multicols}{3}
\begin{itemize}

\item  $(f+g)(2)$ 
\item  $(f-g)(-1)$
\item  $(g-f)(1)$

\end{itemize}
\end{multicols}

\begin{multicols}{3}
\begin{itemize}

\item  $(fg)\left(\frac{1}{2}\right)$
\item  $\left(\frac{f}{g}\right)(0)$
\item  $\left(\frac{g}{f}\right)\left(-2\right)$

\end{itemize}
\end{multicols}


\begin{multicols}{2}
\begin{enumerate}

\item  $f(x) = 3x+1$ and  $g(t) = 4-t$ \label{basicarithonefirst}
\item  $f(x) = x^2$ and $g(t) = -2t+1$

\setcounter{HW}{\value{enumi}}
\end{enumerate}
\end{multicols}

\begin{multicols}{2}
\begin{enumerate}
\setcounter{enumi}{\value{HW}}

\item  $f(x) = x^2 - x$ and  $g(t) = 12-t^2$
\item  $f(x) = 2x^3$ and $g(t) = -t^2-2t-3$

\setcounter{HW}{\value{enumi}}
\end{enumerate}
\end{multicols}

\begin{multicols}{2}
\begin{enumerate}
\setcounter{enumi}{\value{HW}}

\item  $f(x) = \sqrt{x+3}$ and  $g(t) = 2t-1$
\item  $f(x) = \sqrt{4-x}$ and $g(t) = \sqrt{t+2}$

\setcounter{HW}{\value{enumi}}
\end{enumerate}
\end{multicols}

\begin{multicols}{2}
\begin{enumerate}
\setcounter{enumi}{\value{HW}}

\item  $f(x) = 2x$ and  $g(t) = \dfrac{1}{2t+1}$
\item  $f(x) = x^2$ and $g(t) = \dfrac{3}{2t-3}$

\setcounter{HW}{\value{enumi}}
\end{enumerate}
\end{multicols}

\begin{multicols}{2}
\begin{enumerate}
\setcounter{enumi}{\value{HW}}

\item  $f(x) = x^2$ and  $g(t) = \dfrac{1}{t^2}$
\item  $f(x) = x^2+1$ and $g(t) = \dfrac{1}{t^2+1}$ \label{basicarithonelast}

\setcounter{HW}{\value{enumi}}
\end{enumerate}
\end{multicols}



Exercises \ref{arithfromgraphfirst} - \ref{arithfromgraphlast} refer to the functions $f$ and $g$ whose graphs are below. 

\begin{multicols}{2}

\begin{mfpic}[19]{-5}{5}{-5}{5}
\tlabel[cc](-3,0.5){\small $\left( -2, 0 \right)$}
\tlabel[cc](2.5,0.5){\small $\left(2, 0 \right)$}
\tlabel[cc](4,-3.5){\small $\left( 4, -3 \right)$}
\tlabel[cc](-4,-3.5){\small $\left(-4, -3 \right)$}
\tlabel[cc](1,3.5){\small $\left(0, 3 \right)$}
\axes
\tlabel[cc](5,-0.5){\scriptsize $x$}
\tlabel[cc](0.5,5){\scriptsize $y$}
\xmarks{-4,-3,-2,-1,1,2,3,4}
\ymarks{-4,-3,-2,-1,1,2,3,4}
\tlpointsep{5pt}
\scriptsize
\axislabels {x}{{$-4 \hspace{7pt}$} -4, {$-3 \hspace{7pt}$} -3, {$-2 \hspace{7pt}$} -2, {$-1 \hspace{7pt}$} -1, {$1$} 1, {$2$} 2, {$3$} 3, {$4$} 4}
\axislabels {y}{{$-4$} -4, {$-3$} -3, {$-2$} -2, {$-1$} -1, {$1$} 1, {$2$} 2, {$4$} 4}
\normalsize
\point[4pt]{(-2,0), (2,0), (4,-3), (-4,-3), (0,3)}
\penwd{1.25pt}
\function{-4,4,.1}{3*cos(3.14159265*x/4)}
\tcaption{$y = f(x)$}
\end{mfpic}


\begin{mfpic}[19]{-5}{5}{-5}{5}
\tlabel[cc](-4,-2.5){\small $\left( -4, -2 \right)$}
\tlabel[cc](-2,0.5){\small $\left(-1, 0 \right)$}
\tlabel[cc](-1,2){\small $\left( 0,2 \right)$}
\axes
\tlabel[cc](5,-0.5){\scriptsize $t$}
\tlabel[cc](0.5,5){\scriptsize $y$}
\xmarks{-4,-3,-2,-1,1,2,3,4}
\ymarks{-4,-3,-2,-1,1,2,3,4}
\tlpointsep{5pt}
\scriptsize
\axislabels {x}{{$-4 \hspace{7pt}$} -4, {$-3 \hspace{7pt}$} -3, {$-2 \hspace{7pt}$} -2, {$-1 \hspace{7pt}$} -1, {$1$} 1, {$2$} 2, {$3$} 3, {$4$} 4}
\axislabels {y}{{$-4$} -4, {$-3$} -3, {$-2$} -2, {$-1$} -1, {$1$} 1, {$4$} 4, {$3$} 3}
\normalsize
\point[4pt]{(-4,-2), (-1,0), (0,2)}
\penwd{1.25pt}
\polyline{(-4,-2), (-1,0), (0,2)}
\arrow \polyline{(0,2), (5,2)}
\tcaption{$y = g(t)$}
\end{mfpic}


\end{multicols}

\begin{multicols}{3}
\begin{enumerate}
\setcounter{enumi}{\value{HW}}

\item $(f + g)(-4)$ \label{arithfromgraphfirst}
\item $(f + g)(0)$
\item $(f- g)(4)$

\setcounter{HW}{\value{enumi}}
\end{enumerate}
\end{multicols}


\begin{multicols}{3}
\begin{enumerate}
\setcounter{enumi}{\value{HW}}

\item $(fg)(-4)$ 
\item $(fg)(-2)$
\item $(fg)(4)$

\setcounter{HW}{\value{enumi}}
\end{enumerate}
\end{multicols}

\enlargethispage{0.5in}

\begin{multicols}{3}
\begin{enumerate}
\setcounter{enumi}{\value{HW}}

\item $\left(\dfrac{f}{g}\right)(0)$
\item $\left(\dfrac{f}{g}\right)(2)$
\item $\left(\dfrac{g}{f}\right)(-1)$ 

\setcounter{HW}{\value{enumi}}
\end{enumerate}
\end{multicols}

\begin{enumerate}
\setcounter{enumi}{\value{HW}}

\item Find the domains of $f+g$, $f-g$,  $fg$, $\dfrac{f}{g}$ and $\dfrac{g}{f}$.  \label{arithfromgraphlast}

\setcounter{HW}{\value{enumi}}
\end{enumerate}

In Exercises \ref{reformarithfirst} - \ref{reformarithlast}, let $f$ be the function defined by \[f = \{(-3, 4), (-2, 2), (-1, 0), (0, 1), (1, 3), (2, 4), (3, -1)\}\] and let $g$ be the function defined by \[g = \{(-3, -2), (-2, 0), (-1, -4), (0, 0), (1, -3), (2, 1), (3, 2)\}\] Compute the indicated value if it exists.


\begin{multicols}{3}
\begin{enumerate}
\setcounter{enumi}{\value{HW}}

\item $(f + g)(-3)$ \label{reformarithfirst}
\item $(f - g)(2)$
\item $(fg)(-1)$

\setcounter{HW}{\value{enumi}}
\end{enumerate}
\end{multicols}

\begin{multicols}{3}
\begin{enumerate}
\setcounter{enumi}{\value{HW}}

\item $(g + f)(1)$
\item $(g - f)(3)$
\item $(gf)(-3)$

\setcounter{HW}{\value{enumi}}
\end{enumerate}
\end{multicols}

\begin{multicols}{3}
\begin{enumerate}
\setcounter{enumi}{\value{HW}}

\item $\left(\frac{f}{g}\right)(-2)$
\item $\left(\frac{f}{g}\right)(-1)$
\item $\left(\frac{f}{g}\right)(2)$

\setcounter{HW}{\value{enumi}}
\end{enumerate}
\end{multicols}

\begin{multicols}{3}
\begin{enumerate}
\setcounter{enumi}{\value{HW}}

\item $\left(\frac{g}{f}\right)(-1)$
\item $\left(\frac{g}{f}\right)(3)$
\item $\left(\frac{g}{f}\right)(-3)$ \label{reformarithlast}

\setcounter{HW}{\value{enumi}}
\end{enumerate}
\end{multicols}

In Exercises \ref{basicarithtwofirst} - \ref{basicarithtwolast}, use the pair of functions $f$ and $g$ to find the domain of the indicated function then find and simplify an expression for it.

\begin{multicols}{4}
\begin{itemize}

\item  $(f+g)(x)$
\item  $(f-g)(x)$
\item  $(fg)(x)$
\item  $\left(\frac{f}{g}\right)(x)$

\end{itemize}
\end{multicols}

\begin{multicols}{2}
\begin{enumerate}
\setcounter{enumi}{\value{HW}}

\item $f(x) = 2x+1$ and $g(x) = x-2$ \label{basicarithtwofirst}
\item $f(x) = 1-4x$ and $g(x) = 2x-1$

\setcounter{HW}{\value{enumi}}
\end{enumerate}
\end{multicols}

\begin{multicols}{2}
\begin{enumerate}
\setcounter{enumi}{\value{HW}}

\item $f(x) = x^2$ and $g(x) = 3x-1$
\item $f(x) = x^2-x$ and $g(x) = 7x$

\setcounter{HW}{\value{enumi}}
\end{enumerate}
\end{multicols}

\begin{multicols}{2}
\begin{enumerate}
\setcounter{enumi}{\value{HW}}

\item $f(x) = x^2-4$ and $g(x) = 3x+6$
\item $f(x) = -x^2+x+6$ and $g(x) = x^2-9$

\setcounter{HW}{\value{enumi}}
\end{enumerate}
\end{multicols}

\begin{multicols}{2}
\begin{enumerate}
\setcounter{enumi}{\value{HW}}

\item $f(x) = \dfrac{x}{2}$ and $g(x) = \dfrac{2}{x}$
\item $f(x) =x-1$ and $g(x) = \dfrac{1}{x-1}$

\setcounter{HW}{\value{enumi}}
\end{enumerate}
\end{multicols}

\begin{multicols}{2}
\begin{enumerate}
\setcounter{enumi}{\value{HW}}

\item $f(x) = x$ and $g(x) = \sqrt{x+1}$
\item $f(x) =\sqrt{x-5}$ and $g(x) = f(x) = \sqrt{x-5}$ \label{basicarithtwolast}

\setcounter{HW}{\value{enumi}}
\end{enumerate}
\end{multicols}

In Exercises \ref{decomposebasicfirst} - \ref{decomposebasiclast}, write the given function as a nontrivial decomposition of functions as directed.

\begin{enumerate}
\setcounter{enumi}{\value{HW}}

\item  For $p(z) = 4z-z^3$, find functions $f$ and $g$ so that $p=f-g$. \label{decomposebasicfirst}
\item  For $p(z) = 4z-z^3$, find functions $f$ and $g$ so that $p=f+g$.
\item  For $g(t) = 3t|2t-1|$, find functions $f$ and $h$  so that $g = fh$.
\item  For $r(x) = \dfrac{3-x}{x+1}$, find functions $f$ and $g$ so $r = \dfrac{f}{g}$.
\item  For $r(x) = \dfrac{3-x}{x+1}$, find functions $f$ and $g$ so $r = fg$. \label{decomposebasiclast}

\setcounter{HW}{\value{enumi}}
\end{enumerate}

\begin{enumerate}
\setcounter{enumi}{\value{HW}}

\item    Can $f(x) = x$ be decomposed as $f = g-h$ where $g(x) = x+\dfrac{1}{x}$ and $h(x) = \dfrac{1}{x}$?

\item   Discuss with your classmates how to phrase the quantities revenue and profit in Definition \ref{revenueprofitdefns} terms of function arithmetic as defined in Definition \ref{functionarithmeticdefn}.
 
\setcounter{HW}{\value{enumi}}
\end{enumerate}



\begin{enumerate}
\setcounter{enumi}{\value{HW}}

\item \label{posnegdecompexercise}  In this exercise, we explore decomposing a function into its positive and negative parts.  Given a function $f$, we define the \index{positive part of a function}\textbf{positive part} of $f$, denoted $f_{+}$ and \index{negative part of a function}\textbf{negative part} of $f$, denoted $f_{-}$ by:

\[ f_{+}(x) = \dfrac{f(x) + |f(x)|}{2}, \qquad \text{and} \qquad f_{-}(x) = \dfrac{f(x) - |f(x)|}{2}. \]

\begin{enumerate}

\item Using a graphing utility, graph each of the functions $f$ below along with $f_{+}$ and $f_{-}$.

\begin{multicols}{3}

\begin{itemize}

\item  $f(x) = x-3$

\item  $f(x) = x^2-x-6$

\item  $f(x) = 4x-x^3$

\end{itemize}

\end{multicols}

Why is $f_{+}$ called the `positive part' of $f$ and $f_{-}$ called the `negative part' of $f$?

\item Show that $f = f_{+} + f_{-}$.

\item Use Definition \ref{absolutevaluepiecewise} to rewrite the expressions for $f_{+}(x)$ and $f_{-}(x)$ as piecewise defined functions.

\end{enumerate}  

\setcounter{HW}{\value{enumi}}
\end{enumerate}


\begin{enumerate}
\setcounter{enumi}{\value{HW}}

\item  Let $U$ be the unit step function defined in Exercise \ref{unitstepexercise} in Section \ref{ConstantandLinearFunctions}.  For each function $f(t)$ below:

\begin{itemize}

\item  Write $(Uf)(t)$ as a piecewise-defined function.

\item  Graph $y = f(t)$ and $y = (Uf)(t)$.

\end{itemize}

\begin{multicols}{3}

\begin{enumerate}

\item $f(t) = t-3$



\item  $f(t) = |t+2|$



\item  $f(t) =(t-1)^2$



\setcounter{HW}{\value{enumii}}

\end{enumerate}

\end{multicols}

\begin{multicols}{3}

\begin{enumerate}

\setcounter{enumii}{\value{HW}}

\item  $f(t) =(t+1)^{-1}$



\item  $f(t) = \sqrt[3]{t-1}$ \vphantom{ $f(t) =(t+1)^{-1}$}



\item  $f(t) = (t-2)^{\frac{2}{3}}$ \vphantom{ $f(t) =(t+1)^{-1}$}



\setcounter{HW}{\value{enumii}}

\end{enumerate}

\end{multicols}

\begin{enumerate}

\setcounter{enumii}{\value{HW}}

\item  Write a general formula for $(Uf)(t)$ for a function $f$.  (Assume the domain of $f$ is $(-\infty, \infty)$.)



\item  Explain how to obtain the graph of $y=(Uf)(t)$ from $y=f(t)$.


\item The function $U(t)$ is used to model a change in state from `off' to `on' (like flipping a light switch.)  How does this relate to your observations?



\item  Use the graph of $y=f(t)$ below to graph $y=(Uf)(t)$.

\begin{center}

\begin{mfpic}[15]{-5}{5}{-4}{4}
\tlabel[cc](-2.25,-3.5){\scriptsize $\left( -2, -3 \right)$}
\tlabel[cc](2,3.5){\scriptsize $\left(2, 3 \right)$}
\tlabel[cc](-4.25,0.5){\scriptsize $\left(-4, 0 \right)$}
\tlabel[cc](0.75,-0.5){\scriptsize $\left(0, 0 \right)$}
\axes
\tcaption{ \scriptsize$y = f(t)$}
\tlabel[cc](5,-0.5){\scriptsize $t$}
\tlabel[cc](0.5,4){\scriptsize $y$}
\xmarks{-4,-3,-2,-1,1,2,3,4}
\ymarks{-3,-2,-1,1,2,3}
\tlpointsep{5pt}
\scriptsize
\axislabels {x}{ {$-3 \hspace{7pt}$} -3, {$-2 \hspace{7pt}$} -2, {$-1 \hspace{7pt}$} -1, {$2$} 2, {$3$} 3, {$4$} 4}
\axislabels {y}{ {$-3$} -3, {$-2$} -2,  {$1$} 1, {$2$} 2, {$3$} 3}
\normalsize
\penwd{1.25pt}
\function{-4,2,.1}{3*sin(3.14159265*x/4)}
\point[4pt]{(-2,-3), (2,3),  (-4,0), (0,0)}
\end{mfpic}

\end{center}


\end{enumerate}

\item Use Example \ref{densityexample} as a guide to help find the following uncertainties.  

\begin{enumerate}

\item  A chemist combines the solutions from two graduated cylinders into a beaker.  The volume of the first solution, $A$, an acid,  is read as $A_{1} = 101 \pm 0.5$ milliliters (mL). The volume of the second solution,  a base, $B$,  is measured to be $B_{1} = 16 \pm 0.5$ mL.    Estimate the percent propagated error in calculating the volume of the combined solution as $V = A_{1} + B_{1} = 101 + 16 = 117$ mL.

\item  A student measures the length, $\ell$, and width, $w$,  of a piece of paper.  They find  $\ell_{1} = 280 \pm 0.5$ millimeters (mm) $w_{1} = 216 \pm 0.5$ mm.    Estimate the percent propagated error in calculating the area of the piece of paper as $A = \ell_{1} \, w_{1} = 280 \times 216 = 60480 \, \text{mm}^2$.

\item  An airplane passenger  observers a car travel a distance $d_{1} = 1320 \pm 2$ feet (ft) in time $t_{1}  = 15 \pm 0.5$ seconds (s).  Estimate the percent propagated error in calculating the speed of the car as $v = \frac{d_{1}}{t_{1}} = \frac{1320}{15} = 88 \, \frac{\text{ft}}{\text{s}}$.

\end{enumerate}

\item\label{AverageCostMarginalCostExercise} Let us return to Example \ref{marginalsetupex} where  $C(x) = .03x^{3} - 4.5x^{2} + 225x + 250$ denotes the cost, in dollars,  of producing $x$ PortaBoy game systems. Recall the \index{average cost}\index{cost ! average}\textbf{average cost}\footnote{First mentioned in Definition \ref{averagecostprofit} in Section \ref{IntroRational}.} is defined as $\overline{C}(x) = \frac{C(x)}{x}$, $x > 0$,  is the cost per item.

\begin{enumerate}

\item\label{ACexercise1} Find and interpret $\overline{C}(75)$.

\item\label{MCexercise1}  Define  the \index{marginal cost}\index{cost ! marginal}\textbf{marginal cost} $MC(x) = C(x+1) - C(x)$.   Find and interpret $MC(75)$.

\item  How do your answers to parts \ref{ACexercise1} and \ref{MCexercise1} compare?

\item Graph $y = \overline{C}(x)$ with help from a graphing utility.  What is happening graphically near $x = 75$?

\item  Use Theorem \ref{functionarithmeticaroc} to show that, in general, $\text{ARoC}[ \overline{C}(x)] = 0$ when $MC(x) = \overline{C}(x)$.

\smallskip

\textbf{HINT:}  Note that, by definition, $MC(x) = C(x+1) - C(x) = \Delta[C(x)]$ when $\Delta x = 1$.  

\smallskip

Hence,  $\text{ARoC}[C(x)] = \frac{\Delta[C(x)]}{\Delta x} = \frac{\Delta[C(x)]}{1} = \Delta[C(x)] = MC(x)$ in this case \ldots 

\end{enumerate}

\setcounter{HW}{\value{enumi}}

\end{enumerate}


\newpage

\subsection{Answers}

\begin{enumerate}

\item For  $f(x) = 3x+1$ and $g(x) = 4-x$

\begin{multicols}{3}
\begin{itemize}

\item  $(f+g)(2) = 9$
\item  $(f-g)(-1) = -7$
\item  $(g-f)(1) = -1$

\end{itemize}
\end{multicols}

\begin{multicols}{3}
\begin{itemize}

\item  $(fg)\left(\frac{1}{2}\right) = \frac{35}{4}$
\item  $\left(\frac{f}{g}\right)(0) = \frac{1}{4}$
\item  $\left(\frac{g}{f}\right)\left(-2\right) = -\frac{6}{5}$

\end{itemize}
\end{multicols}

\item For  $f(x) = x^2$ and $g(x) = -2x+1$

\begin{multicols}{3}
\begin{itemize}

\item  $(f+g)(2) = 1$
\item  $(f-g)(-1) = -2$
\item  $(g-f)(1) = -2$

\end{itemize}
\end{multicols}

\begin{multicols}{3}
\begin{itemize}

\item  $(fg)\left(\frac{1}{2}\right) = 0$
\item  $\left(\frac{f}{g}\right)(0) = 0$
\item  $\left(\frac{g}{f}\right)\left(-2\right) = \frac{5}{4}$

\end{itemize}
\end{multicols}

\item For  $f(x) = x^2 - x$ and  $g(x) = 12-x^2$

\begin{multicols}{3}
\begin{itemize}

\item  $(f+g)(2) = 10$
\item  $(f-g)(-1) = -9$
\item  $(g-f)(1) = 11$

\end{itemize}
\end{multicols}

\begin{multicols}{3}
\begin{itemize}

\item  $(fg)\left(\frac{1}{2}\right) = -\frac{47}{16}$
\item  $\left(\frac{f}{g}\right)(0) = 0$
\item  $\left(\frac{g}{f}\right)\left(-2\right) = \frac{4}{3}$

\end{itemize}
\end{multicols}

\item For $f(x) = 2x^3$ and  $g(x) = -x^2-2x-3$

\begin{multicols}{3}
\begin{itemize}

\item  $(f+g)(2) = 5$
\item  $(f-g)(-1) = 0$
\item  $(g-f)(1) = -8$

\end{itemize}
\end{multicols}

\begin{multicols}{3}
\begin{itemize}

\item  $(fg)\left(\frac{1}{2}\right) = -\frac{17}{16}$
\item  $\left(\frac{f}{g}\right)(0) = 0$
\item  $\left(\frac{g}{f}\right)\left(-2\right) = \frac{3}{16}$

\end{itemize}
\end{multicols}

\item For $f(x) = \sqrt{x+3}$ and  $g(x) = 2x-1$

\begin{multicols}{3}
\begin{itemize}

\item  $(f+g)(2) = 3+\sqrt{5}$
\item  $(f-g)(-1) = 3+\sqrt{2}$
\item  $(g-f)(1) = -1$

\end{itemize}
\end{multicols}

\begin{multicols}{3}
\begin{itemize}

\item  $(fg)\left(\frac{1}{2}\right) = 0$
\item  $\left(\frac{f}{g}\right)(0) = -\sqrt{3}$
\item  $\left(\frac{g}{f}\right)\left(-2\right) = -5$

\end{itemize}
\end{multicols}

\item For $f(x) = \sqrt{4-x}$ and $g(x) = \sqrt{x+2}$

\begin{multicols}{3}
\begin{itemize}

\item  $(f+g)(2) = 2+\sqrt{2}$
\item  $(f-g)(-1) = -1+\sqrt{5}$
\item  $(g-f)(1) = 0$

\end{itemize}
\end{multicols}

\begin{multicols}{3}
\begin{itemize}

\item  $(fg)\left(\frac{1}{2}\right) = \frac{\sqrt{35}}{2}$
\item  $\left(\frac{f}{g}\right)(0) = \sqrt{2}$
\item  $\left(\frac{g}{f}\right)\left(-2\right) = 0$

\end{itemize}
\end{multicols}

\newpage

\item For  $f(x) = 2x$ and  $g(x) = \frac{1}{2x+1}$

\begin{multicols}{3}
\begin{itemize}

\item  $(f+g)(2) = \frac{21}{5}$
\item  $(f-g)(-1) = -1$
\item  $(g-f)(1) = -\frac{5}{3}$

\end{itemize}
\end{multicols}

\begin{multicols}{3}
\begin{itemize}

\item  $(fg)\left(\frac{1}{2}\right) = \frac{1}{2}$
\item  $\left(\frac{f}{g}\right)(0) = 0$
\item  $\left(\frac{g}{f}\right)\left(-2\right) = \frac{1}{12}$

\end{itemize}
\end{multicols}

\item For  $f(x) = x^2$ and $g(x) = \frac{3}{2x-3}$

\begin{multicols}{3}
\begin{itemize}

\item  $(f+g)(2) = 7$
\item  $(f-g)(-1) = \frac{8}{5}$
\item  $(g-f)(1) = -4$

\end{itemize}
\end{multicols}

\begin{multicols}{3}
\begin{itemize}

\item  $(fg)\left(\frac{1}{2}\right) = -\frac{3}{8}$
\item  $\left(\frac{f}{g}\right)(0) = 0$
\item  $\left(\frac{g}{f}\right)\left(-2\right) = -\frac{3}{28}$

\end{itemize}
\end{multicols}

\item For  $f(x) = x^2$ and $g(x) = \frac{1}{x^2}$

\begin{multicols}{3}
\begin{itemize}

\item  $(f+g)(2) =\frac{17}{4}$
\item  $(f-g)(-1) = 0$
\item  $(g-f)(1) = 0$

\end{itemize}
\end{multicols}

\begin{multicols}{3}
\begin{itemize}

\item  $(fg)\left(\frac{1}{2}\right) =1$
\item  $\left(\frac{f}{g}\right)(0)$ is undefined.
\item  $\left(\frac{g}{f}\right)\left(-2\right) = \frac{1}{16}$

\end{itemize}
\end{multicols}

\item For  $f(x) = x^2+1$ and $g(x) = \frac{1}{x^2+1}$

\begin{multicols}{3}
\begin{itemize}

\item  $(f+g)(2) =\frac{26}{5}$
\item  $(f-g)(-1) = \frac{3}{2}$
\item  $(g-f)(1) = -\frac{3}{2}$

\end{itemize}
\end{multicols}

\begin{multicols}{3}
\begin{itemize}

\item  $(fg)\left(\frac{1}{2}\right) =1$
\item  $\left(\frac{f}{g}\right)(0) = 1$
\item  $\left(\frac{g}{f}\right)\left(-2\right) = \frac{1}{25}$

\end{itemize}
\end{multicols}

\setcounter{HW}{\value{enumi}}
\end{enumerate}

\begin{multicols}{3}
\begin{enumerate}
\setcounter{enumi}{\value{HW}}

\item $(f + g)(-4) = -5$   
\item $(f + g)(0) = 5$
\item $(f-g)(4) = -5$

\setcounter{HW}{\value{enumi}}
\end{enumerate}
\end{multicols}


\begin{multicols}{3}
\begin{enumerate}
\setcounter{enumi}{\value{HW}}

\item $(fg)(-4) = 6$ 
\item $(fg)(-2) = 0$
\item $(fg)(4) = -6$

\setcounter{HW}{\value{enumi}}
\end{enumerate}
\end{multicols}

\enlargethispage{0.5in}

\begin{multicols}{3}
\begin{enumerate}
\setcounter{enumi}{\value{HW}}

\item $\left(\dfrac{f}{g}\right)(0) = \dfrac{3}{2}$
\item $\left(\dfrac{f}{g}\right)(2) =  0$
\item $\left(\dfrac{g}{f}\right)(-1) = 0$ 

\setcounter{HW}{\value{enumi}}
\end{enumerate}
\end{multicols}

\begin{enumerate}
\setcounter{enumi}{\value{HW}}

\item The domains of $f+g$, $f-g$ and $fg$ are all $[-4,4]$.  The domain of $\frac{f}{g}$ is $[-4, -1) \cup (-1,4]$ and the domain of $\frac{g}{f}$ is $[-4, -2) \cup (-2,2) \cup (2, 4]$.

\setcounter{HW}{\value{enumi}}
\end{enumerate}


\begin{multicols}{3}
\begin{enumerate}
\setcounter{enumi}{\value{HW}}

\item $(f + g)(-3) = 2$
\item $(f - g)(2) = 3$
\item $(fg)(-1) = 0$

\setcounter{HW}{\value{enumi}}
\end{enumerate}
\end{multicols}

\begin{multicols}{3}
\begin{enumerate}
\setcounter{enumi}{\value{HW}}

\item $(g + f)(1) = 0$
\item $(g - f)(3) = 3$
\item $(gf)(-3) = -8$

\setcounter{HW}{\value{enumi}}
\end{enumerate}
\end{multicols}

\begin{multicols}{3}
\begin{enumerate}
\setcounter{enumi}{\value{HW}}

\item $\left(\frac{f}{g}\right)(-2)$ does not exist
\item $\left(\frac{f}{g}\right)(-1) = 0$
\item $\left(\frac{f}{g}\right)(2) = 4$

\setcounter{HW}{\value{enumi}}
\end{enumerate}
\end{multicols}

\begin{multicols}{3}
\begin{enumerate}
\setcounter{enumi}{\value{HW}}

\item $\left(\frac{g}{f}\right)(-1)$ does not exist
\item $\left(\frac{g}{f}\right)(3) = -2$ 
\item $\left(\frac{g}{f}\right)(-3) = -\frac{1}{2}$ 

\setcounter{HW}{\value{enumi}}
\end{enumerate}
\end{multicols}

\newpage

\begin{enumerate}
\setcounter{enumi}{\value{HW}}

\item For $f(x) = 2x+1$ and $g(x) = x-2$

\begin{multicols}{2}

\begin{itemize}

\item $(f+g)(x) = 3x-1$ \\
      Domain: $(-\infty, \infty)$
      
      \vfill
      
      \columnbreak
      
\item $(f-g)(x) = x+3$ \\
      Domain:  $(-\infty, \infty)$


\end{itemize}

\end{multicols}

\begin{multicols}{2}

\begin{itemize}

\item $(fg)(x) = 2x^2-3x-2$ \\
      Domain: $(-\infty, \infty)$
      
      \vfill
      
      \columnbreak
      
\item $\left(\frac{f}{g}\right)(x) = \frac{2x+1}{x-2}$ \\
      Domain:  $(-\infty, 2) \cup (2, \infty)$


\end{itemize}

\end{multicols}

\item For $f(x) = 1-4x$ and $g(x) = 2x-1$

\begin{multicols}{2}

\begin{itemize}

\item $(f+g)(x) = -2x$ \\
      Domain: $(-\infty, \infty)$
      
      \vfill
      
      \columnbreak
      
\item $(f-g)(x) = 2-6x$ \\
      Domain:  $(-\infty, \infty)$


\end{itemize}

\end{multicols}

\begin{multicols}{2}

\begin{itemize}

\item $(fg)(x) = -8x^2+6x-1$ \\
      Domain: $(-\infty, \infty)$
      
      \vfill
      
      \columnbreak
      
\item $\left(\frac{f}{g}\right)(x) = \frac{1-4x}{2x-1}$ \\
      Domain:  $\left(-\infty, \frac{1}{2} \right) \cup \left(\frac{1}{2}, \infty \right)$


\end{itemize}

\end{multicols}


\item For $f(x) = x^2$ and $g(x) = 3x-1$

\begin{multicols}{2}

\begin{itemize}

\item $(f+g)(x) = x^2+3x-1$ \\
      Domain: $(-\infty, \infty)$
      
      \vfill
      
      \columnbreak
      
\item $(f-g)(x) = x^2-3x+1$ \\
      Domain:  $(-\infty, \infty)$


\end{itemize}

\end{multicols}

\begin{multicols}{2}

\begin{itemize}

\item $(fg)(x) = 3x^3-x^2$ \\
      Domain: $(-\infty, \infty)$
      
      \vfill
      
      \columnbreak
      
\item $\left(\frac{f}{g}\right)(x) = \frac{x^2}{3x-1}$ \\
      Domain:  $\left(-\infty, \frac{1}{3} \right) \cup \left(\frac{1}{3}, \infty \right)$


\end{itemize}

\end{multicols}

\item For $f(x) = x^2-x$ and $g(x) = 7x$

\begin{multicols}{2}

\begin{itemize}

\item $(f+g)(x) = x^2+6x$ \\
      Domain: $(-\infty, \infty)$
      
      \vfill
      
      \columnbreak
      
\item $(f-g)(x) = x^2-8x$ \\
      Domain:  $(-\infty, \infty)$


\end{itemize}

\end{multicols}

\begin{multicols}{2}

\begin{itemize}

\item $(fg)(x) = 7x^3-7x^2$ \\
      Domain: $(-\infty, \infty)$
      
      \vfill
      
      \columnbreak
      
\item $\left(\frac{f}{g}\right)(x) = \frac{x-1}{7}$ \\
      Domain:  $\left(-\infty, 0 \right) \cup \left(0, \infty \right)$


\end{itemize}

\end{multicols}


\item For $f(x) = x^2-4$ and $g(x) = 3x+6$

\begin{multicols}{2}

\begin{itemize}

\item $(f+g)(x) = x^2+3x+2$ \\
      Domain: $(-\infty, \infty)$
      
      \vfill
      
      \columnbreak
      
\item $(f-g)(x) = x^2-3x-10$ \\
      Domain:  $(-\infty, \infty)$


\end{itemize}

\end{multicols}

\begin{multicols}{2}

\begin{itemize}

\item $(fg)(x) = 3x^3+6x^2-12x-24$ \\
      Domain: $(-\infty, \infty)$
      
      \vfill
      
      \columnbreak
      
\item $\left(\frac{f}{g}\right)(x) = \frac{x-2}{3}$ \\
      Domain:  $\left(-\infty, -2 \right) \cup \left(-2, \infty \right)$


\end{itemize}

\end{multicols}

\newpage

\item For $f(x) = -x^2+x+6$ and $g(x) = x^2-9$

\begin{multicols}{2}

\begin{itemize}

\item $(f+g)(x) = x-3$ \\
      Domain: $(-\infty, \infty)$
      
      \vfill
      
      \columnbreak
      
\item $(f-g)(x) = -2x^2+x+15$ \\
      Domain:  $(-\infty, \infty)$


\end{itemize}

\end{multicols}

\begin{multicols}{2}

\begin{itemize}

\item $(fg)(x) = -x^4+x^3+15x^2-9x-54$ \\
      Domain: $(-\infty, \infty)$
      
      \vfill
      
      \columnbreak
      
\item $\left(\frac{f}{g}\right)(x) = -\frac{x+2}{x+3}$ \\
      Domain:  $\left(-\infty, -3 \right) \cup \left(-3, 3 \right) \cup (3, \infty)$


\end{itemize}

\end{multicols}


\item For  $f(x) = \frac{x}{2}$ and $g(x) = \frac{2}{x}$

\begin{multicols}{2}

\begin{itemize}

\item $(f+g)(x) = \frac{x^2+4}{2x}$ \\
      Domain: $(-\infty, 0) \cup (0, \infty)$
      
      \vfill
      
      \columnbreak
      
\item $(f-g)(x) = \frac{x^2-4}{2x}$ \\
      Domain:  $(-\infty,0) \cup (0, \infty)$


\end{itemize}

\end{multicols}

\begin{multicols}{2}

\begin{itemize}

\item $(fg)(x) = 1$ \\
      Domain: $(-\infty,0) \cup (0, \infty)$
      
      \vfill
      
      \columnbreak
      
\item $\left(\frac{f}{g}\right)(x) = \frac{x^2}{4}$ \\
      Domain: $(-\infty,0) \cup (0, \infty)$


\end{itemize}

\end{multicols}



\item For   $f(x) =x-1$ and $g(x) = \frac{1}{x-1}$

\begin{multicols}{2}

\begin{itemize}

\item $(f+g)(x) = \frac{x^2-2x+2}{x-1}$ \\
      Domain: $(-\infty, 1) \cup (1, \infty)$
      
      \vfill
      
      \columnbreak
      
\item $(f-g)(x) = \frac{x^2-2x}{x-1}$ \\
      Domain:  $(-\infty,1) \cup (1, \infty)$


\end{itemize}

\end{multicols}

\begin{multicols}{2}

\begin{itemize}

\item $(fg)(x) = 1$ \\
      Domain: $(-\infty,1) \cup (1, \infty)$
      
      \vfill
      
      \columnbreak
      
\item $\left(\frac{f}{g}\right)(x) =x^2-2x+1$ \\
      Domain: $(-\infty,1) \cup (1, \infty)$


\end{itemize}

\end{multicols}


\item For   $f(x) =x$ and $g(x) = \sqrt{x+1}$

\begin{multicols}{2}

\begin{itemize}

\item $(f+g)(x) = x+\sqrt{x+1}$ \\
      Domain: $[-1,\infty)$
      
      \vfill
      
      \columnbreak
      
\item $(f-g)(x) = x-\sqrt{x+1}$ \\
       Domain: $[-1,\infty)$


\end{itemize}

\end{multicols}

\begin{multicols}{2}

\begin{itemize}

\item $(fg)(x) = x\sqrt{x+1}$ \\
       Domain: $[-1,\infty)$
      
      \vfill
      
      \columnbreak
      
\item $\left(\frac{f}{g}\right)(x) =\frac{x}{\sqrt{x+1}}$ \\
       Domain: $(-1,\infty)$


\end{itemize}

\end{multicols}

\item For   $f(x) = \sqrt{x-5}$ and $g(x) = f(x) = \sqrt{x-5}$

\begin{multicols}{2}

\begin{itemize}

\item $(f+g)(x) = 2\sqrt{x-5}$ \\
      Domain: $[5,\infty)$
      
      \vfill
      
      \columnbreak
      
\item $(f-g)(x) =0$ \\
       Domain: $[5,\infty)$


\end{itemize}

\end{multicols}

\begin{multicols}{2}

\begin{itemize}

\item $(fg)(x) =x-5$ \\
       Domain: $[5,\infty)$
      
      \vfill
      
      \columnbreak
      
\item $\left(\frac{f}{g}\right)(x) =1$ \\
       Domain: $(5,\infty)$


\end{itemize}

\end{multicols}

\setcounter{HW}{\value{enumi}}
\end{enumerate}

\newpage

\begin{enumerate}
\setcounter{enumi}{\value{HW}}

\item One solution is $f(z) = 4z$ and $g(z) = z^3$. 
\item One solution is $f(z) = 4z$ and $g(z) = - z^3$. 
\item One solution is  $f(t) = 3t$ and $h(t) = |2t-1|$ 
\item One solution is $f(x) = 3-x$ and $g(x) = x+1$.  
\item  One solution is $f(x) = 3-x$ and $g(x) = (x+1)^{-1}$.  

\setcounter{HW}{\value{enumi}}
\end{enumerate}

\begin{enumerate}
\setcounter{enumi}{\value{HW}}

\item No.  The equivalence does not hold when $x = 0$.

\setcounter{HW}{\value{enumi}}
\end{enumerate}

\begin{enumerate}
\setcounter{enumi}{\value{HW}}
\addtocounter{enumi}{1}
\item  \begin{enumerate}  

\addtocounter{enumii}{1}

\item $(f_{+} + f_{-})(x) =  f_{+}(x) + f_{-}(x) = \dfrac{f(x) + |f(x)|}{2} + \dfrac{f(x) - |f(x)|}{2} = \dfrac{2f(x)}{2} = f(x)$.

\item   \[ f_{+}(x)  =  \begin{mycases} 
    0 &  \text{if $f(x) < 0$} \\
      f(x) & \text{if $f(x)  \geq 0$} \\
   \end{mycases},  \qquad    f_{-}(x)  =  \begin{mycases} 
    f(x) &  \text{if $f(x) < 0$} \\
      0 & \text{if $f(x)  \geq 0$} \\
   \end{mycases} \]

\end{enumerate}


\setcounter{HW}{\value{enumi}}
\end{enumerate}




\begin{enumerate}
\setcounter{enumi}{\value{HW}}

\item  $~$


\begin{enumerate}

\item $(Uf)(t) =  \begin{cases} 
  0  &  \text{if $t < 0$, } \\
    t-3  & \text{if $t \geq 0$.} \\
   \end{cases}$


\begin{multicols}{2}

\begin{mfpic}[12]{-5}{5}{-5}{5}
\axes
\tlabel[cc](5,-0.5){\scriptsize $t$}
\tlabel[cc](0.5,5){\scriptsize $y$}
\xmarks{-4,-3,-2,-1,1,2,3,4}
\ymarks{-4,-3,-2, -1, 1,2,3,4}
\tlpointsep{4pt}
\scriptsize
\axislabels {x}{ {$-4 \hspace{7pt}$} -4, {$-3 \hspace{7pt}$} -3, {$-2 \hspace{7pt}$} -2, {$-1 \hspace{7pt}$} -1, {$1$} 1, {$2$} 2, {$3$} 3, {$4$} 4}
\axislabels {y}{{$-1$} -1, {$-2$} -2, {$-4$} -4,{$-3$} -3,  {$1$} 1, {$2$} 2, {$3$} 3, {$4$} 4}
\penwd{1.25pt}
\arrow \reverse \arrow \polyline{( -2,-5), (5,2)}
\tcaption{ \scriptsize$y = f(t)$}
\normalsize
\end{mfpic} 


\begin{mfpic}[12]{-5}{5}{-5}{5}
\axes
\tlabel[cc](5,-0.5){\scriptsize $t$}
\tlabel[cc](0.5,5){\scriptsize $y$}
\xmarks{-4,-3,-2,-1,1,2,3,4}
\ymarks{-4,-3,-2, -1, 1,2,3,4}
\tlpointsep{4pt}
\scriptsize
\axislabels {x}{ {$-4 \hspace{7pt}$} -4, {$-3 \hspace{7pt}$} -3, {$-2 \hspace{7pt}$} -2, {$-1 \hspace{7pt}$} -1, {$1$} 1, {$2$} 2, {$3$} 3, {$4$} 4}
\axislabels {y}{{$-1$} -1,{$-2$} -2,{$-3$} -3,{$-4$} -4,{$1$} 1, {$2$} 2, {$3$} 3, {$4$} 4}
\penwd{1.25pt}
\arrow \reverse \polyline{( -5,0), (0,0)}
\arrow \polyline{(0,-3), (5,2)}
\point[4pt]{(0,-3)}
\pointfillfalse
\point[4pt]{(0,0)}
\tcaption{ \scriptsize$y = (Uf)(t)$}
\normalsize
\end{mfpic} 
\end{multicols}


\newpage 


\item $(Uf)(t) =  \begin{cases} 
   0  &  \text{if $t < 0$, } \\
    |t+2|  = t+2 & \text{if $t \geq 0$.} \\
   \end{cases}$



\begin{multicols}{2}

\begin{mfpic}[12]{-5}{5}{-5}{5}
\axes
\tlabel[cc](5,-0.5){\scriptsize $t$}
\tlabel[cc](0.5,5){\scriptsize $y$}
\xmarks{-4,-3,-2,-1,1,2,3,4}
\ymarks{-4,-3,-2, -1, 1,2,3,4}
\tlpointsep{4pt}
\scriptsize
\axislabels {x}{ {$-4 \hspace{7pt}$} -4, {$-3 \hspace{7pt}$} -3, {$-2 \hspace{7pt}$} -2, {$-1 \hspace{7pt}$} -1, {$1$} 1, {$2$} 2, {$3$} 3, {$4$} 4}
\axislabels {y}{{$-1$} -1, {$-2$} -2, {$-4$} -4,{$-3$} -3,  {$1$} 1, {$2$} 2, {$3$} 3, {$4$} 4}
\penwd{1.25pt}
\arrow \reverse \arrow \polyline{( -5,3), (-2,0), (3,5)}
\tcaption{ \scriptsize$y = f(t)$}
\normalsize
\end{mfpic} 


\begin{mfpic}[12]{-5}{5}{-5}{5}
\axes
\tlabel[cc](5,-0.5){\scriptsize $t$}
\tlabel[cc](0.5,5){\scriptsize $y$}
\xmarks{-4,-3,-2,-1,1,2,3,4}
\ymarks{-4,-3,-2, -1, 1,2,3,4}
\tlpointsep{4pt}
\scriptsize
\axislabels {x}{ {$-4 \hspace{7pt}$} -4, {$-3 \hspace{7pt}$} -3, {$-2 \hspace{7pt}$} -2, {$-1 \hspace{7pt}$} -1, {$1$} 1, {$2$} 2, {$3$} 3, {$4$} 4}
\axislabels {y}{{$-1$} -1,{$-2$} -2,{$-3$} -3,{$-4$} -4,{$1$} 1, {$2$} 2, {$3$} 3, {$4$} 4}
\penwd{1.25pt}
\arrow \reverse \polyline{( -5,0), (0,0)}
\arrow \polyline{(0,2), (3,5)}
\point[4pt]{(0,2)}
\pointfillfalse
\point[4pt]{(0,0)}
\tcaption{ \scriptsize$y = (Uf)(t)$}
\normalsize
\end{mfpic} 
\end{multicols}





\item  $(Uf)(t) =  \begin{cases} 
   0  &  \text{if $t < 0$, } \\
   (t-1)^2 & \text{if $t \geq 0$.} \\
   \end{cases}$



\begin{multicols}{2}

\begin{mfpic}[12]{-5}{5}{-5}{5}
\axes
\tlabel[cc](5,-0.5){\scriptsize $t$}
\tlabel[cc](0.5,5){\scriptsize $y$}
\xmarks{-4,-3,-2,-1,1,2,3,4}
\ymarks{-4,-3,-2, -1, 1,2,3,4}
\tlpointsep{4pt}
\scriptsize
\axislabels {x}{ {$-4 \hspace{7pt}$} -4, {$-3 \hspace{7pt}$} -3, {$-2 \hspace{7pt}$} -2, {$-1 \hspace{7pt}$} -1, {$1$} 1, {$2$} 2, {$3$} 3, {$4$} 4}
\axislabels {y}{{$-1$} -1, {$-2$} -2, {$-4$} -4,{$-3$} -3,  {$1$} 1, {$2$} 2, {$3$} 3, {$4$} 4}
\penwd{1.25pt}
\arrow \reverse \arrow \function{-1.2, 3.2, 0.1}{(x-1)**2}
\tcaption{ \scriptsize$y = f(t)$}
\normalsize
\end{mfpic} 


\begin{mfpic}[12]{-5}{5}{-5}{5}
\axes
\tlabel[cc](5,-0.5){\scriptsize $t$}
\tlabel[cc](0.5,5){\scriptsize $y$}
\xmarks{-4,-3,-2,-1,1,2,3,4}
\ymarks{-4,-3,-2, -1, 1,2,3,4}
\tlpointsep{4pt}
\scriptsize
\axislabels {x}{ {$-4 \hspace{7pt}$} -4, {$-3 \hspace{7pt}$} -3, {$-2 \hspace{7pt}$} -2, {$-1 \hspace{7pt}$} -1, {$1$} 1, {$2$} 2, {$3$} 3, {$4$} 4}
\axislabels {y}{{$-1$} -1,{$-2$} -2,{$-3$} -3,{$-4$} -4,{$1$} 1, {$2$} 2, {$3$} 3, {$4$} 4}
\penwd{1.25pt}
\arrow \reverse \polyline{( -5,0), (0,0)}
\arrow \function{0, 3.2, 0.1}{(x-1)**2}
\point[4pt]{(0,1)}
\pointfillfalse
\point[4pt]{(0,0)}
\tcaption{ \scriptsize$y = (Uf)(t)$}
\normalsize
\end{mfpic} 
\end{multicols}




\setcounter{HW}{\value{enumii}}

\end{enumerate}



\begin{enumerate}

\setcounter{enumii}{\value{HW}}

\item  $(Uf)(t) =  \begin{cases} 
  0  &  \text{if $t < 0$, $t \neq -1$ } \\
   (t+1)^{-1} & \text{if $t \geq 0$.} \\
   \end{cases}$
   



\begin{multicols}{2}

\begin{mfpic}[12]{-5}{5}{-5}{5}
\axes
\tlabel[cc](5,-0.5){\scriptsize $t$}
\tlabel[cc](0.5,5){\scriptsize $y$}
\xmarks{-4,-3,-2,-1,1,2,3,4}
\ymarks{-4,-3,-2, -1, 1,2,3,4}
\tlpointsep{4pt}
\scriptsize
\axislabels {x}{ {$-4 \hspace{7pt}$} -4, {$-3 \hspace{7pt}$} -3, {$-2 \hspace{7pt}$} -2, {$-1 \hspace{7pt}$} -1, {$1$} 1, {$2$} 2, {$3$} 3, {$4$} 4}
\axislabels {y}{{$-1$} -1, {$-2$} -2, {$-4$} -4,{$-3$} -3,  {$1$} 1, {$2$} 2, {$3$} 3, {$4$} 4}
\dashed \polyline{(-1,-5), (-1,5)}
\penwd{1.25pt}
\arrow \reverse  \arrow \function{-5, -1.25, 0.1}{(x+1)**(-1)}
\arrow  \reverse \arrow \function{-0.75, 5,  0.1}{(x+1)**(-1)}
\tcaption{ \scriptsize$y = f(t)$}
\normalsize
\end{mfpic} 


\begin{mfpic}[12]{-5}{5}{-5}{5}
\axes
\tlabel[cc](5,-0.5){\scriptsize $t$}
\tlabel[cc](0.5,5){\scriptsize $y$}
\xmarks{-4,-3,-2,-1,1,2,3,4}
\ymarks{-4,-3,-2, -1, 1,2,3,4}
\tlpointsep{4pt}
\scriptsize
\axislabels {x}{ {$-4 \hspace{7pt}$} -4, {$-3 \hspace{7pt}$} -3, {$-2 \hspace{7pt}$} -2, {$-1 \hspace{7pt}$} -1, {$1$} 1, {$2$} 2, {$3$} 3, {$4$} 4}
\axislabels {y}{{$-1$} -1,{$-2$} -2,{$-3$} -3,{$-4$} -4,{$1$} 1, {$2$} 2, {$3$} 3, {$4$} 4}
\penwd{1.25pt}
\arrow \reverse  \polyline{( -5,0), (0,0)}
\arrow \function{0, 5, 0.1}{(x+1)**(-1)}
\point[4pt]{(0,1)}
\pointfillfalse
\point[4pt]{(0,0), (-1,0)}
\tcaption{ \scriptsize$y = (Uf)(t)$  - note the hole at $(-1,0)$}
\normalsize
\end{mfpic} 
\end{multicols}



\item $(Uf)(t) =  \begin{cases} 
   0  &  \text{if $t < 0$, } \\
    \sqrt[3]{t-1} & \text{if $t \geq 0$.} \\
   \end{cases}$
   
   \begin{multicols}{2}

\begin{mfpic}[12]{-5}{5}{-5}{5}
\axes
\tlabel[cc](5,-0.5){\scriptsize $t$}
\tlabel[cc](0.5,5){\scriptsize $y$}
\xmarks{-4,-3,-2,-1,1,2,3,4}
\ymarks{-4,-3,-2, -1, 1,2,3,4}
\tlpointsep{4pt}
\scriptsize
\axislabels {x}{ {$-4 \hspace{7pt}$} -4, {$-3 \hspace{7pt}$} -3, {$-2 \hspace{7pt}$} -2, {$-1 \hspace{7pt}$} -1, {$1$} 1, {$2$} 2, {$3$} 3, {$4$} 4}
\axislabels {y}{{$-1$} -1, {$-2$} -2, {$-4$} -4,{$-3$} -3,  {$1$} 1, {$2$} 2, {$3$} 3, {$4$} 4}
\penwd{1.25pt}
\arrow \reverse \arrow \parafcn{-1.8, 1.5, 0.1}{(1+t**3, t)}
\tcaption{ \scriptsize$y = f(t)$}
\normalsize
\end{mfpic} 


\begin{mfpic}[12]{-5}{5}{-5}{5}
\axes
\tlabel[cc](5,-0.5){\scriptsize $t$}
\tlabel[cc](0.5,5){\scriptsize $y$}
\xmarks{-4,-3,-2,-1,1,2,3,4}
\ymarks{-4,-3,-2, -1, 1,2,3,4}
\tlpointsep{4pt}
\scriptsize
\axislabels {x}{ {$-4 \hspace{7pt}$} -4, {$-3 \hspace{7pt}$} -3, {$-2 \hspace{7pt}$} -2, {$-1 \hspace{7pt}$} -1, {$1$} 1, {$2$} 2, {$3$} 3, {$4$} 4}
\axislabels {y}{{$-1$} -1,{$-2$} -2,{$-3$} -3,{$-4$} -4,{$1$} 1, {$2$} 2, {$3$} 3, {$4$} 4}
\penwd{1.25pt}
\arrow \reverse \polyline{( -5,0), (0,0)}
 \arrow \parafcn{-1, 1.5, 0.1}{(1+t**3, t)}
\point[4pt]{(0,-1)}
\pointfillfalse
\point[4pt]{(0,0)}
\tcaption{ \scriptsize$y = (Uf)(t)$}
\normalsize
\end{mfpic} 
\end{multicols}



\item  $(Uf)(t) =  \begin{cases} 
   0  &  \text{if $t < 0$, } \\
   (t-2)^{\frac{2}{3}} & \text{if $t \geq 0$.} \\
   \end{cases}$
   
   
   \begin{multicols}{2}

\begin{mfpic}[12]{-5}{5}{-5}{5}
\axes
\tlabel[cc](5,-0.5){\scriptsize $t$}
\tlabel[cc](0.5,5){\scriptsize $y$}
\xmarks{-4,-3,-2,-1,1,2,3,4}
\ymarks{-4,-3,-2, -1, 1,2,3,4}
\tlpointsep{4pt}
\scriptsize
\axislabels {x}{ {$-4 \hspace{7pt}$} -4, {$-3 \hspace{7pt}$} -3, {$-2 \hspace{7pt}$} -2, {$-1 \hspace{7pt}$} -1, {$1$} 1, {$2$} 2, {$3$} 3, {$4$} 4}
\axislabels {y}{{$-1$} -1, {$-2$} -2, {$-4$} -4,{$-3$} -3,  {$1$} 1, {$2$} 2, {$3$} 3, {$4$} 4}
\penwd{1.25pt}
\arrow \reverse \arrow \parafcn{-1.9, 1.4, 0.1}{(2+t**3, t**2)}
\tcaption{ \scriptsize$y = f(t)$}
\normalsize
\end{mfpic} 


\begin{mfpic}[12]{-5}{5}{-5}{5}
\axes
\tlabel[cc](5,-0.5){\scriptsize $t$}
\tlabel[cc](0.5,5){\scriptsize $y$}
\xmarks{-4,-3,-2,-1,1,2,3,4}
\ymarks{-4,-3,-2, -1, 1,2,3,4}
\tlpointsep{4pt}
\scriptsize
\axislabels {x}{ {$-4 \hspace{7pt}$} -4, {$-3 \hspace{7pt}$} -3, {$-2 \hspace{7pt}$} -2, {$-1 \hspace{7pt}$} -1, {$1$} 1, {$2$} 2, {$3$} 3, {$4$} 4}
\axislabels {y}{{$-1$} -1,{$-2$} -2,{$-3$} -3,{$-4$} -4,{$1$} 1, {$2$} 2, {$3$} 3, {$4$} 4}
\penwd{1.25pt}
\arrow \reverse \polyline{( -5,0), (0,0)}
\arrow \parafcn{-1.25, 1.44, 0.1}{(2+t**3, t**2)}
\point[4pt]{(0,1.587)}
\pointfillfalse
\point[4pt]{(0,0)}
\tcaption{ \scriptsize$y = (Uf)(t)$}
\normalsize
\end{mfpic} 
\end{multicols}




\setcounter{HW}{\value{enumii}}

\end{enumerate}



\begin{enumerate}

\setcounter{enumii}{\value{HW}}

\item   $(Uf)(t) =  \begin{cases} 
   0  &  \text{if $t < 0$, } \\
    f(t)  & \text{if $t \geq 0$} \\
   \end{cases}$

\item  The graph of $(Uf)(t)$ is $y=0$ for $t < 0$ and $y=f(t)$ for $t \geq 0$.

\item The unit step function keeps the function `off' until $t=0$ then turns the function `on' for $t \geq 0$.


\item  $~$

\begin{center}

\begin{mfpic}[12]{-5}{5}{-4}{4}
\tlabel[cc](2,3.5){\scriptsize $\left(2, 3 \right)$}
\tlabel[cc](0.75,-0.5){\scriptsize $\left(0, 0 \right)$}
\axes
\tcaption{ \scriptsize$y = (Uf)(t)$}
\tlabel[cc](5,-0.5){\scriptsize $t$}
\tlabel[cc](0.5,4){\scriptsize $y$}
\xmarks{-4,-3,-2,-1,1,2,3,4}
\ymarks{-3,-2,-1,1,2,3}
\tlpointsep{5pt}
\scriptsize
\axislabels {x}{  {$-4 \hspace{7pt}$} -4,{$-3 \hspace{7pt}$} -3, {$-2 \hspace{7pt}$} -2, {$-1 \hspace{7pt}$} -1, {$2$} 2, {$3$} 3, {$4$} 4}
\axislabels {y}{ {$-3$} -3, {$-2$} -2,  {$1$} 1, {$2$} 2, {$3$} 3}
\normalsize
\penwd{1.25pt}
\function{0,2,.1}{3*sin(3.14159265*x/4)}
\polyline{(-4,0), (0,0)}
\point[4pt]{(2,3),  (-4,0), (0,0)}
\end{mfpic}

\end{center}


\end{enumerate}

\item \begin{enumerate}

\item   $\Delta V = \Delta[ A +B ] = \Delta A  + \Delta B = \pm 0.5 \, \text{mL} + \pm 0.5 \, \text{mL} = \pm 1 \, \text{mL}$.  

$\frac{\Delta V}{V} = \pm \frac{1}{117} \approx 0.85 \, \%$.

\item   $\Delta A = \Delta [ \ell w] = w_{1} \Delta \ell  + \ell_{1} \Delta w + \Delta \ell \Delta w =  (216 \, \text{mm})(\pm 0.5 \, \text{mm}) +  (280 \, \text{mm})(\pm 0.5 \, \text{mm})  + (\pm 0.5 \, \text{mm})(\pm 0.5 \, \text{mm}) = \pm 248.25 \, \text{mm}^2$. 

 $\frac{\Delta A}{A} = \pm \frac{248.25}{60480} \approx 0.41\, \%$

\item $\Delta v =  \Delta \left[ \frac{d}{t} \right] = \frac{t_{1} \, \Delta d  - d_{1} \,  \Delta t }{t_{1} (t_{1} + \Delta t) } = \frac{ (15 \, \text{s})(\pm 2 \, \text{ft})  - (1320 \, \text{ft})(\pm 0.5 \, \text{s})}{ (15 \, \text{s})  (15 \pm  0.5 \, \text{s}) } = \pm \frac{92}{29} \, \frac{\text{ft}}{\text{s}} \approx 3.17 \, \frac{\text{ft}}{\text{s}}$.  

$\frac{\Delta v}{v} \approx  \pm \frac{3.17}{88} \approx 3.60 \, \%$


\end{enumerate}

\item \begin{enumerate}  \item $\overline{C}(75) \approx 59.58$.  When making $75$ systems, the cost per system is approximately $\$ 59.58$.

\item $MC(75) = C(76) - C(75) = 58.53$.  It costs an additional $\$ 58.53$ to make the $76$th system.

\item  $\overline{C}(75)$ and $MC(75)$ appear to be `pretty close.'

\item  The graph $y = \overline{C}(x)$  has a local (absolute) minimum right near $x = 75$.

\item  Per Theorem \ref{functionarithmeticaroc}, since 

  \[ \begin{array}{lcrl}
  
  \text{ARoC}[ \overline{C}(x)] = \text{ARoC}\left[\dfrac{C(x)}{x}   \right] & = & \dfrac{ \text{ARoC}[C(x)] \,  x  - C(x)  \, \text{ARoC}[x]}{x (x + \Delta x)} & \\[10pt]  
   & = & \dfrac{ \text{ARoC}[C(x)] \,  x  - C(x)(1) }{x (x +\Delta x)} & \text{Since $\text{ARoC}[x] = \frac{\Delta x}{ \Delta x} = 1$}\\[10pt]  \end{array}\]

If $\text{ARoC}[ \overline{C}(x)]  = 0$, then the numerator,  $\text{ARoC}[C(x)] \,  x  - C(x) = 0$.  Solving for $\text{ARoC}[C(x)]$ , we get  $\text{ARoC}[C(x)] = \frac{C(x)}{x} = \overline{C}(x)$.  If we are working with a whole number of items, the smallest meaningful value of $\Delta x$ is $1$, in which case $\text{ARoC}[C(x)]  = MC(x)$.  Hence,  $\text{ARoC}[ \overline{C}(x)] = 0$ when $MC(x) = \overline{C}(x)$, that is, when  the marginal cost and average cost are the same.  At this point, the graph of $y = \overline{C}(x)$ levels off (at a minumum.)\footnote{We'll have more to say on this after Section \ref{AppDerivatives}.} Can you reason why this creates a minimum?



\setcounter{HW}{\value{enumi}}

\end{enumerate}

\setcounter{HW}{\value{enumi}}
\end{enumerate}





