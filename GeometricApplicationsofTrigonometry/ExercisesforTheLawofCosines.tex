\documentclass{ximera}

\begin{document}
	\author{Stitz-Zeager}
	\xmtitle{TITLE}


In Exercises \ref{firstlawofcosines} - \ref{lastlawofcosines}, use the Law of Cosines to find the remaining side(s) and angle(s) if possible.

\begin{multicols}{2}

\begin{enumerate}

\item $a = 7, \; b = 12, \; \gamma = 59.3^{\circ}$ \label{firstlawofcosines}
\item $\alpha = 104^{\circ}, \; b = 25, \; c  = 37$

\setcounter{HW}{\value{enumi}}

\end{enumerate}

\end{multicols}

\begin{multicols}{2} 

\begin{enumerate}

\setcounter{enumi}{\value{HW}}

\item $a = 153, \; \beta = 8.2^{\circ}, \; c = 153$
\item $a = 3, \; b = 4, \; \gamma = 90^{\circ}$

\setcounter{HW}{\value{enumi}}

\end{enumerate}

\end{multicols}

\begin{multicols}{2} 

\begin{enumerate}

\setcounter{enumi}{\value{HW}}

\item $\alpha = 120^{\circ}, \; b = 3, \; c = 4$
\item $a = 7, \; b = 10, \; c = 13$ \label{firstherons}

\setcounter{HW}{\value{enumi}}

\end{enumerate}

\end{multicols}

\begin{multicols}{2} 

\begin{enumerate}

\setcounter{enumi}{\value{HW}}

\item $a = 1, \; b = 2, \; c = 5$
\item $a = 300, \; b = 302, \; c = 48$ \label{secondherons}

\setcounter{HW}{\value{enumi}}

\end{enumerate}

\end{multicols}

\begin{multicols}{2} 

\begin{enumerate}

\setcounter{enumi}{\value{HW}}

\item $a = 5, \; b = 5, \; c = 5$
\item $a = 5, \; b = 12,; c = 13$ \label{thirdherons} \label{lastlawofcosines}

\setcounter{HW}{\value{enumi}}

\end{enumerate}

\end{multicols}

In Exercises \ref{anylawfirst} - \ref{anylawlast}, use any method to solve for the remaining side(s) and angle(s), if possible.

\begin{multicols}{2}

\begin{enumerate}

\setcounter{enumi}{\value{HW}}

\item $a = 18, \; \alpha = 63^{\circ}, \; b = 20$ \label{ambigfirst} \label{anylawfirst}
\item $a = 37, \; b = 45, \; c = 26$

\setcounter{HW}{\value{enumi}}

\end{enumerate}

\end{multicols}

\begin{multicols}{2} 

\begin{enumerate}

\setcounter{enumi}{\value{HW}}

\item $a = 16, \; \alpha = 63^{\circ}, \; b = 20$ \label{ambigsecond}
\item $a = 22, \; \alpha = 63^{\circ}, \; b = 20$ \label{ambigthird}

\setcounter{HW}{\value{enumi}}

\end{enumerate}

\end{multicols}

\begin{multicols}{2} 

\begin{enumerate}

\setcounter{enumi}{\value{HW}}

\item $\alpha = 42^{\circ}, \; b = 117, \; c = 88$
\item $\beta = 7^{\circ}, \; \gamma = 170^{\circ}, \; c = 98.6$ \label{anylawlast}

\setcounter{HW}{\value{enumi}}

\end{enumerate}

\end{multicols}

\begin{enumerate}

\setcounter{enumi}{\value{HW}}

\item Find the area of the triangles given in Exercises \ref{firstherons}, \ref{secondherons} and \ref{thirdherons} above.

\item The hour hand on my antique Seth Thomas schoolhouse clock in 4 inches long and the minute hand is 5.5 inches long.  Find the distance between the ends of the hands when the clock reads four o'clock.  Round your answer to the nearest hundredth of an inch.

\item A geologist wants to measure the diameter of an impact crater.   From her camp, it is 4 miles to the northern-most point of the crater and 2 miles to the southern-most point.  If the angle between the two lines of sight is $117^{\circ}$, what is the diameter of the crater?  Round your answer to the nearest hundredth of a mile.

\item From the Pedimaxus International Airport a tour helicopter can fly to Cliffs of Insanity Point by following a bearing of N$8.2^{\circ}$E for 192 miles and it can fly to Bigfoot Falls by following a bearing of S$68.5^{\circ}$E for 207 miles.\footnote{Please refer to Section \ref{bearings} for an introduction to bearings.}  Find the distance between Cliffs of Insanity Point and Bigfoot Falls.  Round your answer to the nearest mile.  \label{lofcosinesbearingexercise}

\item Cliffs of Insanity Point and Bigfoot Falls from Exericse \ref{lofcosinesbearingexercise} above both lie on a straight stretch of the Great Sasquatch Canyon.  What bearing would the tour helicopter need to follow to go directly from Bigfoot Falls to Cliffs of Insanity Point?  Round your angle to the nearest tenth of a degree.

\item  A naturalist sets off on a hike from a lodge on a bearing of S$80^{\circ}$W.  After 1.5 miles, she changes her bearing to S$17^{\circ}$W and continues hiking for 3 miles.  Find her distance from the lodge at this point.  Round your answer to the nearest hundredth of a mile.  What bearing should she follow to return to the lodge?  Round your angle to the nearest degree.

\item The HMS Sasquatch leaves port on a bearing of N$23^{\circ}$E and travels for 5 miles.  It then changes course and follows a heading of S$41^{\circ}$E for 2 miles.  How far is it from port? Round your answer to the nearest hundredth of a mile. What is its bearing to port?  Round your angle to the nearest degree.

\item  The SS Bigfoot leaves a harbor bound for Nessie Island which is 300 miles away at a bearing of N$32^{\circ}$E.  A storm moves in and after 100 miles, the captain of the Bigfoot finds he has drifted off course.  If his bearing to the harbor is now S$70^{\circ}$W, how far is the SS Bigfoot from Nessie Island?  Round your answer to the nearest hundredth of a mile.  What course should the captain set to head to the island?  Round your angle to the nearest tenth of a degree.

\item From a point 300 feet above level ground in a firetower, a ranger spots two fires in the Yeti National Forest.  The angle of depression\footnote{See Exercise \ref{angleofdepression} in Section \ref{AppRightTrig} for the definition of this angle.} made by the line of sight from the ranger to the first fire is $2.5^{\circ}$ and the angle of depression made by line of sight from the ranger to the second fire is $1.3^{\circ}$.  The angle formed by the two lines of sight is $117^{\circ}$.  Find the distance between the two fires.  Round your answer to the nearest foot. 
\begin{center}
\begin{mfpic}[15]{-5}{5}{-5}{5}
\plotsymbol[5pt]{Asterisk}{(-4.33,2.5),(2.6, 1.5)}
\tlabel[cc](0,-0.5){firetower}
\tlabel[cc](4.5,1.5){fire}
\tlabel[cc](-5.5, 2.5){fire}
\arrow \reverse \arrow \parafcn{35, 145, 5}{1.5*dir(t)}
\tlabel[cc](0,2){$117^{\circ}$}
\point[4pt]{(0,0)}
\dashed \polyline{(0,0), (-4.33,2.5)}
\dashed \polyline{(0,0), (2.6, 1.5)}
\end{mfpic}


\end{center}

HINT: In order to use the $117^{\circ}$ angle between the lines of sight, you will first need to use right angle Trigonometry to find the lengths of the lines of sight.  This will give you a Side-Angle-Side case in which to apply the Law of Cosines.



\item If you apply the Law of Cosines to the ambiguous Angle-Side-Side (ASS) case, the result is a quadratic equation whose variable is that of the missing side. If the equation has no positive real zeros then the information given does not yield a triangle.  If the equation has only one positive real zero then exactly one triangle is formed and if the equation has two distinct positive real zeros then two distinct triangles are formed.  Apply the Law of Cosines to Exercises \ref{ambigfirst}, \ref{ambigsecond} and \ref{ambigthird} above in order to demonstrate this result.  

\item Discuss with your classmates why Heron's Formula yields an area in square units even though four lengths are being multiplied together.

\end{enumerate}

\newpage

\subsection{Answers}

\begin{multicols}{2}

\begin{enumerate}

\item $\begin{array}{lll}
\alpha \approx 35.54^{\circ} & \beta \approx 85.16^{\circ} & \gamma = 59.3^{\circ} \\
a = 7 & b = 12 & c \approx 10.36 \end{array}$

\item $\begin{array}{lll}
\alpha = 104^{\circ} & \beta \approx 29.40^{\circ} & \gamma \approx 46.60^{\circ} \\
a \approx 49.41 & b = 25 & c = 37 \end{array}$

\setcounter{HW}{\value{enumi}}

\end{enumerate}

\end{multicols}

\begin{multicols}{2} 

\begin{enumerate}

\setcounter{enumi}{\value{HW}}

\item $\begin{array}{lll}
\alpha \approx 85.90^{\circ} & \beta = 8.2^{\circ} & \gamma \approx 85.90^{\circ} \\
a = 153 & b \approx 21.88 & c = 153 \end{array}$

\item $\begin{array}{lll}
\alpha \approx 36.87^{\circ} & \beta \approx 53.13^{\circ} & \gamma = 90^{\circ} \\
a = 3 & b = 4 & c = 5 \end{array}$

\setcounter{HW}{\value{enumi}}

\end{enumerate}

\end{multicols}

\begin{multicols}{2} 

\begin{enumerate}

\setcounter{enumi}{\value{HW}}

\item $\begin{array}{lll}
\alpha = 120^{\circ} & \beta \approx 25.28^{\circ} & \gamma \approx 34.72^{\circ} \\
a = \sqrt{37} & b = 3 & c = 4 \end{array}$

\item $\begin{array}{lll}
\alpha \approx 32.31^{\circ} & \beta \approx 49.58^{\circ} & \gamma \approx 98.21^{\circ} \\
a = 7 & b = 10 & c = 13 \end{array}$

\setcounter{HW}{\value{enumi}}

\end{enumerate}

\end{multicols}

\begin{multicols}{2} 

\begin{enumerate}

\setcounter{enumi}{\value{HW}}

\item \begin{tabular}{l}
Information does not \\
produce a triangle \end{tabular}

\item $\begin{array}{lll}
\alpha \approx 83.05^{\circ} & \beta \approx 87.81^{\circ} & \gamma \approx 9.14^{\circ} \\
a = 300 & b = 302 & c = 48 \end{array}$

\setcounter{HW}{\value{enumi}}

\end{enumerate}

\end{multicols}

\begin{multicols}{2} 

\begin{enumerate}

\setcounter{enumi}{\value{HW}}

\item $\begin{array}{lll}
\alpha = 60^{\circ} & \beta = 60^{\circ} & \gamma = 60^{\circ} \\
a = 5 & b = 5 & c = 5 \end{array}$

\item $\begin{array}{lll}
\alpha \approx 22.62^{\circ} & \beta \approx 67.38^{\circ} & \gamma = 90^{\circ} \\
a = 5 & b = 12 & c = 13 \end{array}$

\setcounter{HW}{\value{enumi}}

\end{enumerate}

\end{multicols}

\begin{multicols}{2}

\begin{enumerate}

\setcounter{enumi}{\value{HW}}

\item $\begin{array}{lll}
\alpha = 63^{\circ} & \beta \approx 98.11^{\circ} & \gamma \approx 18.89^{\circ} \\
a = 18 & b = 20 & c \approx 6.54 \end{array}$

$\begin{array}{lll}
\alpha = 63^{\circ} & \beta \approx 81.89^{\circ} & \gamma \approx 35.11^{\circ} \\
a = 18 & b = 20 & c \approx 11.62 \end{array}$

\item $\begin{array}{lll}
\alpha \approx 55.30^{\circ} & \beta \approx 89.40^{\circ} & \gamma \approx 35.30^{\circ} \\
a = 37 & b = 45 & c = 26 \end{array}$

\setcounter{HW}{\value{enumi}}

\end{enumerate}

\end{multicols}

\begin{multicols}{2} 

\begin{enumerate}

\setcounter{enumi}{\value{HW}}

\item \begin{tabular}{l}
Information does not \\
produce a triangle \end{tabular}

\item $\begin{array}{lll}
\alpha = 63^{\circ} & \beta \approx 54.1^{\circ} & \gamma \approx 62.9^{\circ} \\
a = 22 & b = 20 & c \approx 21.98 \end{array}$

\setcounter{HW}{\value{enumi}}

\end{enumerate}

\end{multicols}

\begin{multicols}{2} 

\begin{enumerate}

\setcounter{enumi}{\value{HW}}

\item $\begin{array}{lll}
\alpha = 42^{\circ} & \beta \approx 89.23^{\circ} & \gamma \approx 48.77^{\circ} \\
a \approx 78.30 & b = 117 & c = 88 \end{array}$

\item $\begin{array}{lll}
\alpha \approx 3^{\circ} & \beta = 7^{\circ} & \gamma = 170^{\circ} \\
a \approx 29.72 & b \approx 69.2 & c = 98.6 \end{array}$

\setcounter{HW}{\value{enumi}}

\end{enumerate}

\end{multicols}

\begin{enumerate}
\setcounter{enumi}{\value{HW}}
\item The area of the triangle given in Exercise \ref{firstherons} is $\sqrt{1200} = 20\sqrt{3} \approx 34.64$ square units.\\
The area of the triangle given in Exercise \ref{secondherons} is $\sqrt{51764375} \approx 7194.75$ square units.\\
The area of the triangle given in Exercise \ref{thirdherons} is exactly $30$ square units.

\item The distance between the ends of the hands at four o'clock is about $8.26$ inches.

\item  The diameter of the crater is about 5.22 miles.

\item About 313 miles

\item N$31.8^{\circ}$W

\item She is about 3.92 miles from the lodge and her bearing to the lodge is N$37^{\circ}$E. 

\item  It is about 4.50 miles from port and its heading to port is S$47^{\circ}$W.

\item  It is about 229.61 miles from the island and the captain should set a course of N$16.4^{\circ}$E to reach the island.

\item The fires are about 17456 feet apart. (Try to avoid rounding errors.)

\end{enumerate}



\end{document}
