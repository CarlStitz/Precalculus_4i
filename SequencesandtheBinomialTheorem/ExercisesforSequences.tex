In Exercises \ref{writeoutseqfirst} - \ref{writeoutseqlast},  write out the first four terms of the given sequence.

\begin{multicols}{2}
\begin{enumerate}


\item $a_{n} = 2^{n} - 1 \vphantom{d_{j} = (-1)^{\dfrac{j(j+1)}{2}}}$, $n \geq 0$  \label{writeoutseqfirst}
\item $d_{j} = (-1)^{\frac{j(j+1)}{2}}$, $j \geq 1$

\setcounter{HW}{\value{enumi}}
\end{enumerate}
\end{multicols}

\begin{multicols}{2}
\begin{enumerate}
\setcounter{enumi}{\value{HW}}

\item $\left\{ 5k - 2 \right\}_{k=1}^{\infty} \vphantom{\left\{ \dfrac{n^2+1}{n+1} \right\}_{n=0}^{\infty}}$
\item $\left\{ \dfrac{n^2+1}{n+1} \right\}_{n=0}^{\infty}$

\setcounter{HW}{\value{enumi}}
\end{enumerate}
\end{multicols}

\begin{multicols}{2}
\begin{enumerate}
\setcounter{enumi}{\value{HW}}

\item $\left\{ \dfrac{x^{n}}{n^{2}} \right\}_{n=1}^{\infty}$
\item $\left\{ \dfrac{\ln(n)}{n} \right\}_{n=1}^{\infty} \vphantom{\left\{ \dfrac{x^{n}}{n^{2}} \right\}_{n=1}^{\infty}}$

\setcounter{HW}{\value{enumi}}
\end{enumerate}
\end{multicols}

\begin{multicols}{2}
\begin{enumerate}
\setcounter{enumi}{\value{HW}}
 
\item  $a_{\mbox{\tiny$1$}} = 3$, $a_{n+\mbox{\tiny$1$}} = a_{n} - 1$, $n \geq 1 \vphantom{d_{m} = \dfrac{d_{m\mbox{-\tiny$1$}}}{100}}$
\item  $d_{\mbox{\tiny$0$}} = 12$, $d_{m} = \dfrac{d_{m\mbox{-\tiny$1$}}}{100}$, $m \geq 1$

\setcounter{HW}{\value{enumi}}
\end{enumerate}
\end{multicols}

\begin{multicols}{2}
\begin{enumerate}
\setcounter{enumi}{\value{HW}}

\item  $b_{\mbox{\tiny$1$}} = 2$, $b_{k\mbox{+\tiny$1$}} =3b_{k}+1 \vphantom{\dfrac{c_{j\mbox{-\tiny$1$}}}{(j+1)(j+2)}}$, $k \geq 1$
\item  $c_{\mbox{\tiny$0$}} = -2$, $c_{j} = \dfrac{c_{j\mbox{-\tiny$1$}}}{(j+1)(j+2)}$,  $j \geq 1$

\setcounter{HW}{\value{enumi}}
\end{enumerate}
\end{multicols}

\begin{multicols}{2}
\begin{enumerate}
\setcounter{enumi}{\value{HW}}

\item  $a_{\mbox{\tiny$1$}} = 117$, $a_{n\mbox{+\tiny$1$}} = \dfrac{1}{a_{n}}$, $n \geq 1$
\item  $s_{\mbox{\tiny$0$}} = 1$, $s_{n\mbox{+\tiny$1$}} = x^{n + 1} + s_{n}$, $n \geq 0$

\setcounter{HW}{\value{enumi}}
\end{enumerate}
\end{multicols}


\begin{enumerate}
\setcounter{enumi}{\value{HW}}

\item  $F_{\mbox{\tiny$0$}} = 1$, $F_{\mbox{\tiny$1$}} = 1$, $F_{n} = F_{n\mbox{-\tiny$1$}} + F_{n\mbox{-\tiny$2$}}$, $n \geq 2$  (This is the famous \href{http://en.wikipedia.org/wiki/Fibonacci_number}{\underline{Fibonacci Sequence}} ) \label{writeoutseqlast}

\setcounter{HW}{\value{enumi}}
\end{enumerate}


In Exercises \ref{alggeoneithfirst} - \ref{alggeoneithlast} determine if the given sequence is arithmetic, geometric or neither.  If it is arithmetic, find the common difference $d$; if it is geometric, find the common ratio $r$.

\begin{multicols}{2}
\begin{enumerate}
\setcounter{enumi}{\value{HW}}

 
\item  $\left\{ 3n-5 \right\}_{n=1}^{\infty}$ \label{alggeoneithfirst}

\item  $a_{n} = n^2+3n+2$, $n \geq 1$

\setcounter{HW}{\value{enumi}}
\end{enumerate}
\end{multicols}


\begin{multicols}{2}
\begin{enumerate}
\setcounter{enumi}{\value{HW}}


\item  $\dfrac{1}{3}$, $\dfrac{1}{6}$, $\dfrac{1}{12}$, $\dfrac{1}{24} \vphantom{\left\{ 3 \left(\dfrac{1}{5}\right)^{n-1} \right\}_{n=1}^{\infty}}$, \ldots

\item  $\left\{ 3 \left(\dfrac{1}{5}\right)^{n-1} \right\}_{n=1}^{\infty}$


\item  $17$, $5$, $-7$, $-19$, \ldots

\item  $2$, $22$, $222$, $2222$, \ldots

\item  $0.9$, $9$, $90$, $900 \vphantom{a_{n} = \dfrac{n!}{2}}$, \ldots

\item  $a_{n} = \dfrac{n!}{2}$, $n \geq 0$.  \label{alggeoneithlast}


\setcounter{HW}{\value{enumi}}
\end{enumerate}
\end{multicols}


In Exercises \ref{nthtermfirst} - \ref{nthtermlast}, find an explicit formula for the $n^{\mbox{\scriptsize th}}$ term of the given sequence.\footnote{Use the formulas in Equation \ref{arithgeoformula} as needed.}

\begin{multicols}{3}
\begin{enumerate}
\setcounter{enumi}{\value{HW}}

\item $3$, $5$, $7$, $9 \vphantom{-\dfrac{1}{8}}$, \ldots \label{nthtermfirst}
\item $1$, $-\dfrac{1}{2}$, $\dfrac{1}{4}$, $-\dfrac{1}{8}$, \ldots
\item $1$, $\dfrac{2}{3}$, $\dfrac{4}{5}$, $\dfrac{8}{7}$, \ldots

\setcounter{HW}{\value{enumi}}
\end{enumerate}
\end{multicols}

\begin{multicols}{3}
\begin{enumerate}
\setcounter{enumi}{\value{HW}}

\item $1$, $\dfrac{2}{3}$, $\dfrac{1}{3}$, $\dfrac{4}{27} \vphantom{\dfrac{x^7}{7}}$, \ldots
\item $1$, $\dfrac{1}{4}$, $\dfrac{1}{9}$, $\dfrac{1}{16} \vphantom{-\dfrac{x^7}{7}}$, \ldots
\item $x$, $-\dfrac{x^3}{3}$, $\dfrac{x^5}{5}$, $-\dfrac{x^7}{7}$, \ldots

\item $0.9, 0.99, 0.999, 0.9999, \ldots$
\item $27, 64, 125, 216, \ldots$
\item $1, 0, 1, 0, \ldots$ \label{nthtermlast}

\setcounter{HW}{\value{enumi}}
\end{enumerate}
\end{multicols}



In Exercises \ref{limseqexfirst} - \ref{limseqexlast}, find the indicated limit by using Theorem \ref{passingtocontinuous} and passing to a continuous variable.\footnote{See Example \ref{limitofsequenceexample}.}


\begin{multicols}{3}
\begin{enumerate}
\setcounter{enumi}{\value{HW}}

\item\label{limseqexfirst}  $\ds{\lim_{n \rightarrow \infty}}$ $\dfrac{2n^2 - 3n+1}{4-n^2}$

\item  $\ds{\lim_{k \rightarrow \infty}}$ $\dfrac{k^2 +7k-3}{3k - k^3}$

\item\label{limseqexlast}  $\ds{\lim_{m \rightarrow \infty}}$ $\dfrac{117m^{42} + 3m + 1}{e^{2m} + 6}$

\setcounter{HW}{\value{enumi}}
\end{enumerate}
\end{multicols}


In Exercises \ref{limsqueezefirst} - \ref{limsqueezelast}, use the Squeeze Theorem,  Theorem \ref{squeezeth} to help you determine the limit.\footnote{See part \ref{squeezemotivation} of Example \ref{limitofsequenceexample}.}


\begin{multicols}{3}
\begin{enumerate}
\setcounter{enumi}{\value{HW}}

\item\label{limsqueezefirst}  $\ds{\lim_{n \rightarrow \infty}}$ $\dfrac{(-1)^{n}}{3n+1}$

\item  $\ds{\lim_{k \rightarrow \infty}}$ $1 - \left( - \frac{2}{3}  \right)^{k} $

\item\label{limsqueezelast}  $\ds{\lim_{m \rightarrow \infty}}$ $\dfrac{(-1)^{ \frac{m^2-m}{2}} }{m!}$

\setcounter{HW}{\value{enumi}}
\end{enumerate}
\end{multicols}





\begin{enumerate}
\setcounter{enumi}{\value{HW}}

\item \label{arithmeticandgeometricexercise} Find a sequence which is both arithmetic and geometric.  (Hint: Start with $a_{n} = c$ for all $n$.)

\item Show that a geometric sequence can be transformed into an arithmetic sequence by taking the natural logarithm of the terms.

\item Thomas Robert Malthus is credited with saying, ``The power of population is indefinitely greater than the power in the earth to produce subsistence for man. Population, when unchecked, increases in a geometrical ratio. Subsistence increases only in an arithmetical ratio. A slight acquaintance with numbers will show the immensity of the first power in comparison with the second.''  (See this \href{http://en.wikipedia.org/wiki/Malthus}{\underline{webpage}} for more information.)  Discuss this quote with your classmates from a sequences point of view.
 
\item This classic problem involving sequences shows the power of geometric sequences.  Suppose that a wealthy benefactor agrees to give you one penny today and then double the amount she gives you each day for 30 days.  So, for example, you get two pennies on the second day and four pennies on the third day.  How many pennies do you get on the $30^{\mbox{\scriptsize th}}$ day?  What is the \underline{total} dollar value of the gift you have received?

\item Research the terms `arithmetic mean' and `geometric mean.'  With the help of your classmates, show that a given term of a arithmetic sequence $a_{k}$, $k \geq 2$ is the arithmetic mean of the term immediately preceding, $a_{k\mbox{\tiny$-1$}}$ it and immediately following it, $a_{k\mbox{\tiny$+1$}}$.  State and prove an analogous result for geometric sequences.  

\item Discuss with your classmates how the results of this section might change if we were to examine sequences of other mathematical things like complex numbers or matrices.  Find an explicit formula for the $n^{\mbox{\scriptsize th}}$ term of the sequence $i, -1, -i, 1, i, \ldots$.  List out the first four terms of the matrix sequences we discussed in Exercise \ref{Markovchain} in Section \ref{MatArithmetic}.



\end{enumerate}

\newpage

\subsection{Answers}

\begin{multicols}{2}
\begin{enumerate}

\item $0, 1, 3, 7$
\item $-1, -1, 1, 1$

\setcounter{HW}{\value{enumi}}
\end{enumerate}
\end{multicols}

\begin{multicols}{2}
\begin{enumerate}
\setcounter{enumi}{\value{HW}}

\item $3, 8, 13, 18$
\item $1, 1, \frac{5}{3}, \frac{5}{2}$

\setcounter{HW}{\value{enumi}}
\end{enumerate}
\end{multicols}

\begin{multicols}{2}
\begin{enumerate}
\setcounter{enumi}{\value{HW}}

\item $x, \frac{x^{2}}{4}, \frac{x^{3}}{9}, \frac{x^{4}}{16}$
\item $0, \frac{\ln(2)}{2}, \frac{\ln(3)}{3}, \frac{\ln(4)}{4}$

\setcounter{HW}{\value{enumi}}
\end{enumerate}
\end{multicols}

\begin{multicols}{2}
\begin{enumerate}
\setcounter{enumi}{\value{HW}}

\item $3, 2, 1, 0$
\item $12, 0.12, 0.0012, 0.000012$

\setcounter{HW}{\value{enumi}}
\end{enumerate}
\end{multicols}

\begin{multicols}{2}
\begin{enumerate}
\setcounter{enumi}{\value{HW}}

\item $2, 7, 22, 67$
\item $-2, -\frac{1}{3}, -\frac{1}{36}, -\frac{1}{720}$

\setcounter{HW}{\value{enumi}}
\end{enumerate}
\end{multicols}

\begin{multicols}{2}
\begin{enumerate}
\setcounter{enumi}{\value{HW}}

\item $117, \frac{1}{117}, 117, \frac{1}{117}$
\item $1, x + 1, x^{2} + x + 1, x^{3} + x^{2} + x + 1 $

\setcounter{HW}{\value{enumi}}
\end{enumerate}
\end{multicols}

\begin{multicols}{2}
\begin{enumerate}
\setcounter{enumi}{\value{HW}}

\item $1, 1, 2, 3$

\setcounter{HW}{\value{enumi}}
\end{enumerate}
\end{multicols}

\begin{multicols}{2}
\begin{enumerate}
\setcounter{enumi}{\value{HW}}

\item  arithmetic, $d = 3$

\item  neither


\setcounter{HW}{\value{enumi}}
\end{enumerate}
\end{multicols}

\begin{multicols}{2}
\begin{enumerate}
\setcounter{enumi}{\value{HW}}

\item  geometric, $r = \frac{1}{2}$

\item  geometric, $r = \frac{1}{5}$

\setcounter{HW}{\value{enumi}}
\end{enumerate}
\end{multicols}

\begin{multicols}{2}
\begin{enumerate}
\setcounter{enumi}{\value{HW}}


\item  arithmetic, $d = -12$

\item  neither

\setcounter{HW}{\value{enumi}}
\end{enumerate}
\end{multicols}

\begin{multicols}{2}
\begin{enumerate}
\setcounter{enumi}{\value{HW}}


\item  geometric, $r = 10$

\item  neither


\setcounter{HW}{\value{enumi}}
\end{enumerate}
\end{multicols}

\begin{multicols}{3}
\begin{enumerate}
\setcounter{enumi}{\value{HW}}

\item $a_{n} = 1 + 2n, \; n \geq 1$
\item $a_{n} = \left(-\frac{1}{2}\right)^{n - 1}, \; n \geq 1$
\item $a_{n} = \frac{2^{n - 1}}{2n - 1}, \; n \geq 1$

\setcounter{HW}{\value{enumi}}
\end{enumerate}
\end{multicols}

\begin{multicols}{3}
\begin{enumerate}
\setcounter{enumi}{\value{HW}}

\item $a_{n} = \frac{n}{3^{n - 1}}, \; n \geq 1$
\item $a_{n} = \frac{1}{n^{2}}, \; n \geq 1$
\item $\frac{(-1)^{n - 1}x^{2n - 1}}{2n -1}, \; n \geq 1$

\setcounter{HW}{\value{enumi}}
\end{enumerate}
\end{multicols}

\begin{multicols}{3}
\begin{enumerate}
\setcounter{enumi}{\value{HW}}

\item $a_{n} = \frac{10^{n} - 1}{10^{n}}, \; n \geq 1$
\item $a_{n} = (n + 2)^{3}, \; n \geq 1$
\item $a_{n} = \frac{1 + (-1)^{n-1}}{2}, \; n \geq 1$
 
\setcounter{HW}{\value{enumi}}
\end{enumerate}
\end{multicols}

\begin{multicols}{3}
\begin{enumerate}
\setcounter{enumi}{\value{HW}}

\item  $\ds{\lim_{n \rightarrow \infty}}$ $\frac{2n^2 - 3n+1}{4-n^2} = -2$

\item  $\ds{\lim_{k \rightarrow \infty}}$ $\frac{k^2 +7k-3}{3k - k^3} = 0$

\item\label{limseqexlast}  $\ds{\lim_{m \rightarrow \infty}}$ $\frac{117m^{42} + 3m + 1}{e^{2m} + 6} = 0$

\setcounter{HW}{\value{enumi}}
\end{enumerate}
\end{multicols}

\begin{multicols}{3}
\begin{enumerate}
\setcounter{enumi}{\value{HW}}

\item  $\ds{\lim_{n \rightarrow \infty}}$ $\frac{(-1)^{n}}{3n+1} = 0$

\item  $\ds{\lim_{k \rightarrow \infty}}$ $1 - \left( - \frac{2}{3}  \right)^{k} = 1$

\item  $\ds{\lim_{m \rightarrow \infty}}$ $\frac{(-1)^{ \frac{m^2-m}{2}} }{m!} = 0$

\setcounter{HW}{\value{enumi}}
\end{enumerate}
\end{multicols}



