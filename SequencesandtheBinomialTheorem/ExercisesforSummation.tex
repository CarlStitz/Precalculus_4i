In Exercises \ref{sumfirst} - \ref{sumlast}, find the value of each sum using Definition \ref{sigmanotation}.

\begin{multicols}{4} 
\begin{enumerate}

\item $\displaystyle \sum_{g = 4}^{9} (5g + 3)$  \label{sumfirst}
\item $\displaystyle \sum_{k = 3}^{8} \frac{1}{k}$
\item $\displaystyle \sum_{j = 0}^{5} 2^{j}$
\item $\displaystyle \sum_{k = 0}^{2} (3k - 5)x^{k}$

\setcounter{HW}{\value{enumi}}
\end{enumerate}
\end{multicols}

\begin{multicols}{4}
\begin{enumerate}
\setcounter{enumi}{\value{HW}}

\item $\displaystyle \sum_{i = 1}^{4} \frac{1}{4}(i^{2} + 1)$
\item $\displaystyle \sum_{n = 1}^{100} (-1)^{n}$
\item $\displaystyle \sum_{n = 1}^{5} \frac{(n+1)!}{n!}$
\item $\displaystyle \sum_{j = 1}^{3} \frac{5!}{j! \, (5-j)!}$  \label{sumlast}

\setcounter{HW}{\value{enumi}}
\end{enumerate}
\end{multicols}

In Exercises \ref{writesumfirst} - \ref{writesumlast},  rewrite the sum using summation notation.


\begin{multicols}{2}
\begin{enumerate}
\setcounter{enumi}{\value{HW}}

\item $8 + 11 + 14 + 17 + 20$  \label{writesumfirst}
\item $1 - 2 + 3 - 4 + 5 - 6 + 7 - 8$

\setcounter{HW}{\value{enumi}}
\end{enumerate}
\end{multicols}

\begin{multicols}{2}
\begin{enumerate}
\setcounter{enumi}{\value{HW}}

\item $x - \dfrac{x^{3}}{3} + \dfrac{x^{5}}{5} - \dfrac{x^{7}}{7}$
\item $1 + 2 + 4 + \cdots + 2^{29} \vphantom{x - \dfrac{x^{3}}{3} + \dfrac{x^{5}}{5} - \dfrac{x^{7}}{7}}$

\setcounter{HW}{\value{enumi}}
\end{enumerate}
\end{multicols}

\begin{multicols}{2}
\begin{enumerate}
\setcounter{enumi}{\value{HW}}

\item $2 + \frac{3}{2} + \frac{4}{3} + \frac{5}{4} + \frac{6}{5}$
\item $-\ln(3) + \ln(4) - \ln(5) + \cdots + \ln(20)$

\setcounter{HW}{\value{enumi}}
\end{enumerate}
\end{multicols}

\begin{multicols}{2}
\begin{enumerate}
\setcounter{enumi}{\value{HW}}

\item $1 - \frac{1}{4} + \frac{1}{9} - \frac{1}{16} + \frac{1}{25} - \frac{1}{36}$
\item $\frac{1}{2}(x - 5) + \frac{1}{4}(x - 5)^{2} + \frac{1}{6}(x - 5)^{3} + \frac{1}{8}(x - 5)^{4}$  \label{writesumlast}

\setcounter{HW}{\value{enumi}}
\end{enumerate}
\end{multicols}


In Exercises \ref{findsumformfirst} - \ref{findsumformulalast}, use the formulas in Equation \ref{arithgeosum} to find the sum.

\begin{multicols}{3}
\begin{enumerate}
\setcounter{enumi}{\value{HW}}

\item $\displaystyle \sum_{n = 1}^{10} 5n+3$ \label{findsumformfirst}

\item $\displaystyle \sum_{n = 1}^{20} 2n-1$ 

\item $\displaystyle \sum_{k = 0}^{15} 3-k$ 

\setcounter{HW}{\value{enumi}}
\end{enumerate}
\end{multicols}

\begin{multicols}{3}
\begin{enumerate}
\setcounter{enumi}{\value{HW}}

\item $\displaystyle \sum_{n = 1}^{10} \left(\frac{1}{2}\right)^{n}$

\item $\displaystyle \sum_{n = 1}^{5} \left(\frac{3}{2}\right)^{n}$ 

\item $\displaystyle \sum_{k = 0}^{5} 2\left(\frac{1}{4}\right)^{k}$ 

\setcounter{HW}{\value{enumi}}
\end{enumerate}
\end{multicols}

\begin{multicols}{3}
\begin{enumerate}
\setcounter{enumi}{\value{HW}}

\item  $1+4+7+ \ldots +295$  

\item  $4+2+0-2- \ldots - 146$  

\item $1+3+9+ \ldots + 2187$ 
\setcounter{HW}{\value{enumi}}
\end{enumerate}
\end{multicols}

\begin{multicols}{3}
\begin{enumerate}
\setcounter{enumi}{\value{HW}}

\item  $\frac{1}{2} + \frac{1}{4} + \frac{1}{8} + \ldots + \frac{1}{256}\vphantom{\displaystyle \sum_{n = 1}^{10} -2n + \left(\frac{5}{3}\right)^{n}}$ 

\item $3 - \frac{3}{2} + \frac{3}{4} - \frac{3}{8}+- \dots +\frac{3}{256} \vphantom{\displaystyle \sum_{n = 1}^{10} -2n + \left(\frac{5}{3}\right)^{n}}$



\item $\displaystyle \sum_{n = 1}^{10} -2n + \left(\frac{5}{3}\right)^{n}$ \label{findsumformulalast}

\setcounter{HW}{\value{enumi}}
\end{enumerate}
\end{multicols}



In Exercises \ref{geoseriesexamplefirst} - \ref{geoseriesexamplelast}, use Theorem \ref{geoseries} to find the sum of the given geometric series.\footnote{Remember, when in doubt \ldots}

\begin{multicols}{4}
\begin{enumerate}
\setcounter{enumi}{\value{HW}}
\item $\ds{ \sum_{n = 1}^{\infty} \left( \frac{1}{2} \right)^{n-1}}$  \label{geoseriesexamplefirst}
\item $\ds{ \sum_{n = 0}^{\infty}  \dfrac{(-1)^n \, 3^{n-1}}{4^{n}}}$ 
\item $\ds{ \sum_{m = 2}^{\infty}   \dfrac{3}{2^{m-1}}}$ 
\item $\ds{ \sum_{k =0}^{\infty}  x^{k}}$, $|x|<1$.  \label{geoseriesexamplelast}
\setcounter{HW}{\value{enumi}}
\end{enumerate}

\end{multicols}

\pagebreak

In Exercises \ref{dectofracfirst} - \ref{dectofraclast}, use Theorem \ref{geoseries} to express each repeating decimal as a fraction of integers.

\begin{multicols}{4}

\begin{enumerate}
\setcounter{enumi}{\value{HW}}
\item $0.\overline{7}$ \label{dectofracfirst}
\item $0.\overline{13}$
\item $10.\overline{159}$
\item $-5.8\overline{67}$ \label{dectofraclast}
\setcounter{HW}{\value{enumi}}
\end{enumerate}

\end{multicols}


In Exercises \ref{annuityfirst} - \ref{annuitylast}, use Equation \ref{fvannuity} to compute the future value of the annuity with the given terms.  In all cases, assume the payment is made monthly, the interest rate given is the annual rate, and interest is compounded monthly.

\begin{enumerate}
\setcounter{enumi}{\value{HW}}

\item payments are \$300, interest rate is 2.5\%, term is 17 years. \label{annuityfirst}

\item payments are \$50, interest rate is 1.0\%,  term is 30 years. 

\item payments are \$100, interest rate is 2.0\%, term is 20 years 

\item  payments are \$100, interest rate is 2.0\%,  term is  25 years

\item  payments are \$100, interest rate is 2.0\%,  term is  30 years


\item  payments are \$100, interest rate is 2.0\%,  term is  35 years
\label{annuitylast}   
 
\item Suppose an ordinary annuity offers an annual interest rate of $2 \%$, compounded monthly, for 30 years. What should the monthly payment be to have $\$100,\!000$ at the end of the term? 

\setcounter{HW}{\value{enumi}}
\end{enumerate}

\begin{enumerate}
\setcounter{enumi}{\value{HW}}

\item\label{seriesforfunctionex}  In this exercise,  we  Theorem \ref{geoseries}  to represent  $f(x) = \dfrac{1}{x^2+4}$ as a series.

\begin{enumerate}

\item  Show that $f(x) = \dfrac{\frac{1}{4}}{1 - \left( -\frac{x^2}{4} \right)}$.

\item  Use the formula  in Theorem \ref{geoseries}: $\dfrac{a}{1-r} = \displaystyle{\sum_{k=1}^{\infty} ar^{k-1}}$ to write $f(x)$ as an infinite series.

\item Graph $y = f(x)$ along with some partial sums of the series.  What do you notice?


\end{enumerate}



\item  Using Example \ref{rightsumex} as a guide, find the area between the graph of each function below and the $x$-axis by evaluating the limit of a right endpoint sum.



\begin{enumerate}

\item  $f(x) = 4-x$ over the interval $[0,4]$.   %Ans:  8 $\text{units}^2$

\item  $f(x) = 3x^2$ over the interval $[1,3]$.    %Ans:  26 $\text{units}^2$

\item  $f(x) = 12-x-x^2$ over the interval $[0,3]$.   %Ans:  22.5 $\text{units}^2$


\end{enumerate}




\item Prove the properties listed in Theorem \ref{sigmaprops}.

\item Show that the formula for the future value of an annuity due is \[A = P(1 + i)\left[\frac{(1 + i)^{nt} - 1}{i}\right]\]


\newpage

\item  Discuss with your classmates what goes wrong when trying to find the following sums.\footnote{When in doubt \ldots }


\begin{enumerate}

\begin{multicols}{3}

\item  $\displaystyle{ \sum_{k=1}^{\infty} 2^{k-1}}$


\item  $\displaystyle{ \sum_{k=1}^{\infty} (1.0001)^{k-1}}$

\item  $\displaystyle{ \sum_{k=1}^{\infty} (-1)^{k-1}}$

\end{multicols}

\end{enumerate}


\item  \label{CauchyBoundProofExercise}  In this exercise, we walk through the proof of Cauchy's Bound, Theorem \ref{CauchysBound} in Section \ref{RealZeros}.  

\smallskip

Let $f(x) = a_{n} x^{n} + a_{n-\mbox{\tiny$1$}}x^{n-\mbox{\tiny$1$}} + \ldots + a_{\mbox{\tiny$1$}} x + a_{\mbox{\tiny$0$}}$ be a polynomial of degree $n$ and let $Z$ be the largest zero of $f$ in absolute value and let $M$ be the largest of the numbers: $\frac{|a_{0}|}{|a_{n}|}$, $\frac{|a_{1}|}{|a_{n}|}$, \ldots, $\frac{|a_{n-1}|}{|a_{n}|}$.

\begin{enumerate}

\item  Since $P(Z) = 0$, solve for $Z^{n}$:   $Z^{n} = \frac{a_{n-\mbox{\tiny$1$}}}{a_{n}} \, Z^{n-\mbox{\tiny$1$}} + \ldots + \frac{a_{\mbox{\tiny$1$}}}{a_{n}} \,  Z + \frac{a_{\mbox{\tiny$0$}}}{a_{n}}$.

\item  If $-1 \leq Z \leq 1$, then Cauchy's Bound is immediately satisfied since $Z$ would automatically lie in the interval $\left[-(M+1), M+1\right]$. So we assume $|Z|>1$.  

\smallskip

Under the assumption $|Z|>1$. explain why\footnote{Feel free to use the \href{http://en.wikipedia.org/wiki/Triangle_inequality}{\underline{Triangle Inequality}}, as needed.  See Exercise \ref{triangleinequalityreals} in Section \ref{AbsoluteValueFunctions}.} 

\[\begin{array}{rcl} 

 |Z|^{n} & = &  \left| \frac{a_{n-\mbox{\tiny$1$}}}{a_{n}} \, Z^{n-\mbox{\tiny$1$}} + \ldots + \frac{a_{\mbox{\tiny$1$}}}{a_{n}} \,  Z + \frac{a_{\mbox{\tiny$0$}}}{a_{n}} \right|  \\[10pt]
 
 & \leq &  \frac{|a_{n-1}|}{|a_{n}|} \, |Z|^{n-1} + \ldots + \frac{|a_{1}|}{|a_{n}|} \, |Z| + \frac{|a_{0}|}{|a_{n}|} \\ \end{array} \]

\item  Use the definition of  $M$  along with the Geometric Sum Formula, Equation \ref{arithgeosum} to show:

\[ |Z|^{n} \leq M \left( |Z|^{n-1} + \ldots + |Z| + 1\right) = M \, \dfrac{1 - |Z|^n}{1 - |Z|} =  M \,  \dfrac{ |Z|^n - 1}{|Z| - 1} \]

\item  Now use the fact that $|Z| > 1$ to rearrange the above inequality to get:

\[ |Z| - 1 \leq M \, \dfrac{|Z|^{n} - 1}{|Z|^n} = M \left( 1 - \frac{1}{|Z|^n} \right) \]

\item  Use the fact that $1 - \frac{1}{|Z|^n} < 1$ to get:  

\[ |Z| - 1 \leq  M \left( 1 - \frac{1}{|Z|^n} \right)  < M (1) = M\]

\item  From $|Z| - 1 < M$, we get $|Z| < M+1$.  Hence, $Z$ lies in the interval  $\left[-(M+1), M+1\right]$.

\end{enumerate}



\end{enumerate}
\newpage

\subsection{Answers}

\begin{multicols}{4} 
\begin{enumerate}

\item $213$
\item $\frac{341}{280}$
\item $63$
\item $-5 - 2x + x^{2}$

\setcounter{HW}{\value{enumi}}
\end{enumerate}
\end{multicols}

\begin{multicols}{4} 
\begin{enumerate}
\setcounter{enumi}{\value{HW}}


\item $\frac{17}{2}$
\item $0$
\item  $20$
\item  $25$

\setcounter{HW}{\value{enumi}}
\end{enumerate}
\end{multicols}


\begin{multicols}{4} 
\begin{enumerate}
\setcounter{enumi}{\value{HW}}

\item $\displaystyle \sum_{k = 1}^{5} (3k + 5)$
\item $\displaystyle \sum_{k = 1}^{8} (-1)^{k - 1}k$
\item $\displaystyle \sum_{k = 1}^{4} (-1)^{k - 1} \frac{x^{2k - 1}}{2k - 1}$
\item $\displaystyle \sum_{k = 1}^{30} 2^{k-1}$

\setcounter{HW}{\value{enumi}}
\end{enumerate}
\end{multicols}


\begin{multicols}{4} 
\begin{enumerate}
\setcounter{enumi}{\value{HW}}


\item $\displaystyle \sum_{k = 1}^{5} \frac{k + 1}{k}$
\item $\displaystyle \sum_{k = 3}^{20} (-1)^{k} \ln(k)$
\item $\displaystyle \sum_{k = 1}^{6} \frac{(-1)^{k - 1}}{k^{2}}$
\item $\displaystyle \sum_{k = 1}^{4} \frac{1}{2k}(x - 5)^{k}$

\setcounter{HW}{\value{enumi}}
\end{enumerate}
\end{multicols}


\begin{multicols}{4} 
\begin{enumerate}
\setcounter{enumi}{\value{HW}}

\item $305$

\item  $400$

\item  $-72$

\item $\dfrac{1023}{1024}$

\setcounter{HW}{\value{enumi}}
\end{enumerate}
\end{multicols}

\begin{multicols}{4}
\begin{enumerate}
\setcounter{enumi}{\value{HW}}

\item $\dfrac{633}{32}$

\item $\dfrac{1365}{512}$

\item  $14652$

\item  $-5396$

\setcounter{HW}{\value{enumi}}
\end{enumerate}
\end{multicols}

\begin{multicols}{4}
\begin{enumerate}
\setcounter{enumi}{\value{HW}}

\item  $3280$

\item  $\dfrac{255}{256}$



\item $\dfrac{513}{256}$

\item $\dfrac{17771050}{59049}$

\setcounter{HW}{\value{enumi}}
\end{enumerate}
\end{multicols}

\begin{multicols}{4}
\begin{enumerate}
\setcounter{enumi}{\value{HW}}

\item $\ds{ \sum_{n = 1}^{\infty} \left( \frac{1}{2} \right)^{n-1} = 2}$  
\item $\ds{ \sum_{n = 0}^{\infty}  \dfrac{(-1)^n \, 3^{n-1}}{4^{n}} = \frac{4}{21}}$ 
\item $\ds{ \sum_{m = 2}^{\infty}   \dfrac{3}{2^{m-1}} = 3}$ 
\item $\ds{ \sum_{k =0}^{\infty}  x^{k} = \frac{1}{1-x}}$
\setcounter{HW}{\value{enumi}}
\end{enumerate}
\end{multicols}



\begin{multicols}{4}
\begin{enumerate}
\setcounter{enumi}{\value{HW}}



\item $\dfrac{7}{9}$

\item $\dfrac{13}{99}$


\item $\dfrac{3383}{333}$
\item $-\dfrac{5809}{990}$

\setcounter{HW}{\value{enumi}}
\end{enumerate}
\end{multicols}



\begin{multicols}{4}
\begin{enumerate}
\setcounter{enumi}{\value{HW}}

\item \$76,\!163.67
\item $\$20,\!981.40$

\item $\$29,\!479.69$

\item  $\$38,\!882.12$ 

\setcounter{HW}{\value{enumi}}
\end{enumerate}
\end{multicols}

\begin{multicols}{4}
\begin{enumerate}
\setcounter{enumi}{\value{HW}}


\item $49,\!272.55$

\item  $60,\!754.80$
 
\setcounter{HW}{\value{enumi}}
\end{enumerate}

\end{multicols}

\begin{enumerate}
\setcounter{enumi}{\value{HW}}

\item  For $\$100,\!000$, the monthly payment is $\approx \$202.95$.

\item  \begin{enumerate} \addtocounter{enumii}{1}  \item $f(x) = \dfrac{\frac{1}{4}}{1 - \left( -\frac{x^2}{4} \right)} = \ds{\sum_{k=1}^{\infty} \frac{1}{4} \,\left( -\frac{x^2}{4} \right)^{k-1}} = \ds{\sum_{k=1}^{\infty} \frac{(-1)^{k-1} \, x^{2k-2}}{4^{k}}} $
\item  No matter how many terms are added, the graph of the series seems to only account for a portion of the graph of $y = f(x)$.  This is due to the fact that geometric series converge only when the ratio $|r| < 1$.  In this case, $r =  -\frac{x^2}{4}$ so $|r| < 1$ corresponds to the interval $(-2,2)$.

\end{enumerate}

\item  Using Example \ref{rightsumex} as a guide, find the area between the graph of each function below and the $x$-axis by evaluating the limit of a right endpoint sum.



\begin{enumerate}

\item  $RS_{n} = 8 - \frac{8}{n}$;  8 $\text{units}^2$

\item    $RS_{n} = 26 + \frac{24}{n} + \frac{4}{n^2}$; 26 $\text{units}^2$

\item  $RS_{n} = \frac{45}{2}  - \frac{18}{n} - \frac{9}{2n^2}$;     $\frac{45}{2}$  $\text{units}^2$


\end{enumerate}



\end{enumerate}
