\documentclass{ximera}

\begin{document}
	\author{Stitz-Zeager}
	\xmtitle{TITLE}
\mfpicnumber{1} \opengraphsfile{ExercisesforFundamentalTrigonometricIdentities} % mfpic settings added 



In Exercises \ref{useidsforvaluesfirst01} - \ref{useidsforvalueslast01}, use the Reciprocal and Quotient Identities (Theorem \ref{recipquotidfull}) along with the Pythagorean Identities (Theorem \ref{pythids}), to find the value of the circular function requested below. (Find the exact value unless otherwise indicated.)

\begin{multicols}{3}
\begin{enumerate}

\item  \label{useidsforvaluesfirst01}  If  $\sin(\theta) = \frac{\sqrt{5}}{5}$, find $\csc(\theta)$.

\item  If $\sec(\theta) = - 4$,  find $\cos(\theta)$.

\item  If $\tan(t) = 3$, find $\cot(t)$.

\setcounter{HW}{\value{enumi}}
\end{enumerate}
\end{multicols}

\begin{enumerate}

\setcounter{enumi}{\value{HW}}

\item  If $\theta$ is a Quadrant IV angle with $\cos(\theta) = \frac{5}{13}$, find $\sin(\theta)$.

\item  If $\theta$ is a Quadrant III angle with $\tan(\theta) = 2$, find $\sec(\theta)$.

\item  If $\frac{\pi}{2} < t < \pi$ with $\cot(t) = -2$, find $\csc(t)$.

\item  If $\sec(\theta) = 3$ and $\sin(\theta) < 0$, find $\tan(\theta)$.

\item If $\sin(\theta) = -\frac{2}{3}$ but $\tan(\theta) > 0$, find $\cos(\theta)$.

\item If $0 < t < \frac{\pi}{2}$ and $\sin(t) = 0.42$, find $\cos(t)$, rounded to four decimal places.

\item  If $\theta$ is Quadrant IV angle with $\sec(\theta) = 1.17$, find $\tan(\theta)$, rounded to four decimal places.

\item  \label{useidsforvalueslast01}  If $\pi < t < \frac{3\pi}{2}$ with $\cot(t) = 4.2$, find $\csc(t)$, rounded to four decimal places.

\setcounter{HW}{\value{enumi}}
\end{enumerate}

In Exercises \ref{useidsforvaluesfirst02} - \ref{useidsforvalueslast02}, use the Reciprocal and Quotient Identities (Theorem \ref{recipquotidfull}) along with the Pythagorean Identities (Theorem \ref{pythids}), to find the exact values of the remaining circular functions.  (Compare your methods with how you solved Exercises \ref{findothercircfirst} - \ref{findothercirclast} in Section \ref{TheOtherCircularFunctions}.)


\begin{multicols}{2}

\begin{enumerate}


\setcounter{enumi}{\value{HW}}

\item $\sin(\theta) = \dfrac{3}{5}$ with $\theta$ in Quadrant II \label{useidsforvaluesfirst02}
\item $\tan(\theta) = \dfrac{12}{5}$ with $\theta$ in Quadrant III

\setcounter{HW}{\value{enumi}}

\end{enumerate}

\end{multicols}

\begin{multicols}{2}

\begin{enumerate}

\setcounter{enumi}{\value{HW}}

\item $\csc(\theta) = \dfrac{25}{24}$ with $\theta$ in Quadrant I
\item $\sec(\theta) = 7$ with $\theta$ in Quadrant IV \vphantom{$\dfrac{25}{24}$}

\setcounter{HW}{\value{enumi}}

\end{enumerate}

\end{multicols}

\begin{multicols}{2}

\begin{enumerate}

\setcounter{enumi}{\value{HW}}

\item $\csc(\theta) = -\dfrac{10\sqrt{91}}{91}$ with $\theta$ in Quadrant III
\item $\cot(\theta) = -23$ with $\theta$ in Quadrant II \vphantom{$\dfrac{10}{\sqrt{91}}$}

\setcounter{HW}{\value{enumi}}

\end{enumerate}

\end{multicols}

\begin{multicols}{2}

\begin{enumerate}

\setcounter{enumi}{\value{HW}}

\item  $\tan(\theta) = -2$ with $\theta$ in Quadrant IV.
\item  $\sec(\theta) = -4$ with $\theta$ in Quadrant II.

\setcounter{HW}{\value{enumi}}

\end{enumerate}

\end{multicols}

\begin{multicols}{2}

\begin{enumerate}

\setcounter{enumi}{\value{HW}}

\item $\cot(\theta) = \sqrt{5}$ with $\theta$ in Quadrant III. \vphantom{$\dfrac{25}{24}$}
\item  $\cos(\theta) = \dfrac{1}{3}$ with $\theta$ in Quadrant I.

\setcounter{HW}{\value{enumi}}

\end{enumerate}

\end{multicols}

\begin{multicols}{2}

\begin{enumerate}

\setcounter{enumi}{\value{HW}}

\item  $\cot(t) = 2$ with $0  < t < \dfrac{\pi}{2}$.
\item  $\csc(t) = 5$ with $\dfrac{\pi}{2} < t < \pi$.

\setcounter{HW}{\value{enumi}}

\end{enumerate}

\end{multicols}

\begin{multicols}{2}

\begin{enumerate}

\setcounter{enumi}{\value{HW}}

\item  $\tan(t) = \sqrt{10}$ with $\pi < t < \dfrac{3\pi}{2}$.
\item  $\sec(t) = 2\sqrt{5}$ with $\dfrac{3\pi}{2} < t < 2\pi$.\label{useidsforvalueslast02}


\setcounter{HW}{\value{enumi}}

\end{enumerate}

\end{multicols}


\begin{enumerate}
\setcounter{enumi}{\value{HW}}

\item Skippy claims  $\cos(\theta) + \sin(\theta) = 1$ is an identity because when $\theta = 0$, the equation is true.  Is Skippy correct? Explain.

\setcounter{HW}{\value{enumi}}
\end{enumerate}

\pagebreak

In Exercises \ref{firstcirciden} - \ref{lastcirciden}, verify the identity.  Assume that all quantities are defined.

\begin{multicols}{2}

\begin{enumerate}

\setcounter{enumi}{\value{HW}}

\item $\cos(\theta) \sec(\theta) = 1$ \label{firstcirciden}
\item $\tan(t)\cos(t) = \sin(t)$

\setcounter{HW}{\value{enumi}}

\end{enumerate}

\end{multicols}

\begin{multicols}{2}

\begin{enumerate}

\setcounter{enumi}{\value{HW}}

\item $\sin(\theta) \csc(\theta) = 1$
\item $\tan(t) \cot(t) = 1$

\setcounter{HW}{\value{enumi}}

\end{enumerate}

\end{multicols}

\begin{multicols}{2}

\begin{enumerate}

\setcounter{enumi}{\value{HW}}

\item $\csc(x) \cos(x) = \cot(x)$ \vphantom{$\dfrac{\sin(x)}{\cos^{2}(x)}$}
\item $\dfrac{\sin(t)}{\cos^{2}(t)} = \sec(t) \tan(t)$

\setcounter{HW}{\value{enumi}}

\end{enumerate}

\end{multicols}

\begin{multicols}{2}

\begin{enumerate}

\setcounter{enumi}{\value{HW}}

\item $\dfrac{\cos(\theta)}{\sin^{2}(\theta)} = \csc(\theta) \cot(\theta)$
\item $\dfrac{1+ \sin(x)}{\cos(x)} = \sec(x) + \tan(x)$

\setcounter{HW}{\value{enumi}}

\end{enumerate}

\end{multicols}

\begin{multicols}{2}

\begin{enumerate}

\setcounter{enumi}{\value{HW}}

\item $\dfrac{1 - \cos(\theta)}{\sin(\theta)} = \csc(\theta) - \cot(\theta)$
\item  $\dfrac{\cos(t)}{1 - \sin^{2}(t)} = \sec(t)$

\setcounter{HW}{\value{enumi}}

\end{enumerate}

\end{multicols}

\begin{multicols}{2}

\begin{enumerate}

\setcounter{enumi}{\value{HW}}

\item  $\dfrac{\sin(x)}{1 - \cos^{2}(x)} = \csc(x)$
\item  $\dfrac{\sec(t)}{1 + \tan^{2}(t)} = \cos(t)$

\setcounter{HW}{\value{enumi}}

\end{enumerate}

\end{multicols}

\begin{multicols}{2}

\begin{enumerate}

\setcounter{enumi}{\value{HW}}

\item  $\dfrac{\csc(\theta)}{1 + \cot^{2}(\theta)} = \sin(\theta)$
\item   $\dfrac{\tan(x)}{\sec^{2}(x) - 1} = \cot(x)$

\setcounter{HW}{\value{enumi}}

\end{enumerate}

\end{multicols}

\begin{multicols}{2}

\begin{enumerate}

\setcounter{enumi}{\value{HW}}

\item   $\dfrac{\cot(t)}{\csc^{2}(t) - 1} = \tan(t)$
\item $4 \cos^{2}(\theta) + 4 \sin^{2}(\theta) = 4$

\setcounter{HW}{\value{enumi}}

\end{enumerate}

\end{multicols}

\begin{multicols}{2}

\begin{enumerate}

\setcounter{enumi}{\value{HW}}

\item $9 - \cos^{2}(t) - \sin^{2}(t) = 8$
\item $\tan^{3}(t) = \tan(t)\sec^{2}(t) - \tan(t)$

\setcounter{HW}{\value{enumi}}

\end{enumerate}

\end{multicols}

\begin{multicols}{2}

\begin{enumerate}

\setcounter{enumi}{\value{HW}}

\item $\sin^{5}(x) = \left(1-\cos^{2}(x)\right)^{2} \sin(x)$
\item $\sec^{10}(t) = \left(1 + \tan^{2}(t)\right)^4 \sec^{2}(t)$

\setcounter{HW}{\value{enumi}}

\end{enumerate}

\end{multicols}

\begin{multicols}{2}

\begin{enumerate}

\setcounter{enumi}{\value{HW}}

\item $\cos^{2}(x)\tan^{3}(x) = \tan(x) - \sin(x)\cos(x)$
\item $\sec^{4}(t) - \sec^{2}(t) = \tan^{2}(t) + \tan^{4}(t)$

\setcounter{HW}{\value{enumi}}

\end{enumerate}

\end{multicols}

\begin{multicols}{2}

\begin{enumerate}

\setcounter{enumi}{\value{HW}}

\item $\dfrac{\cos(\theta) + 1}{\cos(\theta) - 1} = \dfrac{1 + \sec(\theta)}{1 - \sec(\theta)}$
\item $\dfrac{\sin(t) + 1}{\sin(t) - 1} = \dfrac{1 + \csc(t)}{1 - \csc(t)}$

\setcounter{HW}{\value{enumi}}

\end{enumerate}

\end{multicols}

\begin{multicols}{2}

\begin{enumerate}

\setcounter{enumi}{\value{HW}}

\item $\dfrac{1 - \cot(x)}{1+ \cot(x)} = \dfrac{\tan(x) - 1}{\tan(x) + 1}$
\item $\dfrac{1 - \tan(t)}{1+ \tan(t)} = \dfrac{\cos(t) - \sin(t)}{\cos(t) + \sin(t)}$

\setcounter{HW}{\value{enumi}}

\end{enumerate}

\end{multicols}

\begin{multicols}{2}

\begin{enumerate}

\setcounter{enumi}{\value{HW}}

\item $\tan(\theta) + \cot(\theta) = \sec(\theta)\csc(\theta)$
\item $\csc(t) - \sin(t) = \cot(t)\cos(t)$

\setcounter{HW}{\value{enumi}}

\end{enumerate}

\end{multicols}

\begin{multicols}{2}

\begin{enumerate}

\setcounter{enumi}{\value{HW}}

\item $\cos(x) - \sec(x) = -\tan(x)\sin(x)$
\item $\cos(x)(\tan(x) + \cot(x)) = \csc(x)$

\setcounter{HW}{\value{enumi}}

\end{enumerate}

\end{multicols}

\begin{multicols}{2}

\begin{enumerate}

\setcounter{enumi}{\value{HW}}

\item $\sin(t)(\tan(t) + \cot(t)) = \sec(t)$ \vphantom{$\dfrac{1}{1-\cos(t)}$}
\item   $\dfrac{1}{1-\cos(\theta)} + \dfrac{1}{1+\cos(\theta)} = 2\csc^{2}(\theta)$

\setcounter{HW}{\value{enumi}}

\end{enumerate}

\end{multicols}

\begin{multicols}{2}

\begin{enumerate}

\setcounter{enumi}{\value{HW}}

\item  $\dfrac{1}{\sec(t) + 1} + \dfrac{1}{\sec(t)-1} = 2 \csc(t) \cot(t)$
\item  $\dfrac{1}{\csc(x) + 1} + \dfrac{1}{\csc(x)-1} = 2 \sec(x) \tan(x)$

\setcounter{HW}{\value{enumi}}

\end{enumerate}

\end{multicols}

\begin{multicols}{2}

\begin{enumerate}

\setcounter{enumi}{\value{HW}}
\small
\item $\dfrac{1}{\csc(t)-\cot(t)} - \dfrac{1}{\csc(t) + \cot(t)} = 2 \cot(t)$
\item $\dfrac{\cos(\theta)}{1 - \tan(\theta)} + \dfrac{\sin(\theta)}{1 - \cot(\theta)} = \sin(\theta) + \cos(\theta)$
\normalsize
\setcounter{HW}{\value{enumi}}

\end{enumerate}

\end{multicols}

\begin{multicols}{2}

\begin{enumerate}

\setcounter{enumi}{\value{HW}}

\item $\dfrac{1}{\sec(t) + \tan(t)} = \sec(t) - \tan(t)$
\item  $\dfrac{1}{\sec(x) - \tan(x)} = \sec(x) + \tan(x)$

\setcounter{HW}{\value{enumi}}

\end{enumerate}

\end{multicols}

\begin{multicols}{2}

\begin{enumerate}

\setcounter{enumi}{\value{HW}}

\item  $\dfrac{1}{\csc(t) - \cot(t)} = \csc(t) + \cot(t)$
\item  $\dfrac{1}{\csc(\theta) + \cot(\theta)} = \csc(\theta) - \cot(\theta)$

\setcounter{HW}{\value{enumi}}

\end{enumerate}

\end{multicols}

\begin{multicols}{2}

\begin{enumerate}

\setcounter{enumi}{\value{HW}}

\item  $\dfrac{1}{1-\sin(x)} = \sec^{2}(x) + \sec(x) \tan(x)$
\item  $\dfrac{1}{1+\sin(t)} = \sec^{2}(t) - \sec(t) \tan(t)$

\setcounter{HW}{\value{enumi}}

\end{enumerate}

\end{multicols}

\begin{multicols}{2}

\begin{enumerate}

\setcounter{enumi}{\value{HW}}

\item  $\dfrac{1}{1-\cos(\theta)} = \csc^{2}(\theta) + \csc(\theta) \cot(\theta)$
\item  $\dfrac{1}{1+\cos(x)} = \csc^{2}(x) - \csc(x) \cot(x)$

\setcounter{HW}{\value{enumi}}

\end{enumerate}

\end{multicols}

\begin{multicols}{2}

\begin{enumerate}

\setcounter{enumi}{\value{HW}}

\item $\dfrac{\cos(t)}{1 + \sin(t)} = \dfrac{1-\sin(t)}{\cos(t)}$
\item $\csc(\theta) - \cot(\theta) = \dfrac{\sin(\theta)}{1 + \cos(\theta)}$

\setcounter{HW}{\value{enumi}}

\end{enumerate}

\end{multicols}

\begin{multicols}{2}

\begin{enumerate}

\setcounter{enumi}{\value{HW}}

\item $\dfrac{1 - \sin(x)}{1 + \sin(x)} = (\sec(x) - \tan(x))^{2}$ \label{lastcirciden}

\setcounter{HW}{\value{enumi}}

\end{enumerate}

\end{multicols}


In Exercises \ref{logcircidenfirst} - \ref{logcircidenlast}, verify the identity.  You may need to consult Sections \ref{AbsoluteValueFunctions} and \ref{PropertiesofLogarithms} for a review of the properties of absolute value and logarithms before proceeding.

\begin{multicols}{2}

\begin{enumerate}
\setcounter{enumi}{\value{HW}}

\item  $\quad \ln|\sec(x)| = -\ln|\cos(x)|$ \label{logcircidenfirst}
\item  $-\ln|\csc(x)| = \ln|\sin(x)|$

\setcounter{HW}{\value{enumi}}
\end{enumerate}

\end{multicols}

\begin{multicols}{2}

\begin{enumerate}

\setcounter{enumi}{\value{HW}}

\item  $-\ln|\sec(x) - \tan(x)| = \ln|\sec(x)+\tan(x)|$
\item  $-\ln|\csc(x) + \cot(x)|= \ln|\csc(x) - \cot(x)|$ \label{logcircidenlast}

\setcounter{HW}{\value{enumi}}
\end{enumerate}

\end{multicols}

\begin{enumerate}
\setcounter{enumi}{\value{HW}}
%\addtocounter{enumi}{51}

\item\label{oneminuscosinelimit}  \begin{enumerate}  \item What indeterminate form is present in the limit $\ds{\lim_{\theta \rightarrow 0}}$ $\frac{1 - \cos(\theta)}{\theta}$?

\item  Graph $f(\theta) = \frac{1 - \cos(\theta)}{\theta}$ near $\theta = 0$. What appears to be $\ds{\lim_{\theta \rightarrow 0}}$ $\frac{1 - \cos(\theta)}{\theta}$?


\item\label{oneminuscosinerewrite}  Verify the identity:  $\frac{1 - \cos(\theta)}{\theta} = \frac{\sin(\theta)}{1 + \cos(\theta)} \, \frac{\sin(\theta)}{\theta}$.

\item  Use the fact\footnote{See Exercise \ref{sintovertexercise3} in Section \ref{TheOtherCircularFunctions}.} that $\ds{\lim_{\theta \rightarrow 0}}$ $\frac{\sin(\theta)}{\theta} = 1$ along with  part \ref{oneminuscosinerewrite} to help you find $\ds{\lim_{\theta \rightarrow 0}}$ $\frac{1 - \cos(\theta)}{\theta}$.

\end{enumerate}

\setcounter{HW}{\value{enumi}}
\end{enumerate}

\newpage

\subsection{Answers}

\begin{multicols}{4}

\begin{enumerate}

\item   $\csc(\theta) = \sqrt{5}$. \vphantom{$\sin(\theta) = -\frac{12}{13}$}

\item  $\cos(\theta) = -\frac{1}{4}$.  \vphantom{$\sin(\theta) = -\frac{12}{13}$}

\item   $\cot(t) = \frac{1}{3}$.  \vphantom{$\sin(\theta) = -\frac{12}{13}$}

\item $\sin(\theta) = -\frac{12}{13}$.  


\setcounter{HW}{\value{enumi}}

\end{enumerate}

\end{multicols}

\begin{multicols}{4}

\begin{enumerate}
\setcounter{enumi}{\value{HW}}

\item $\sec(\theta) = -\sqrt{5}$.  \vphantom{$\cos(\theta) = -\frac{\sqrt{5}}{3}$.}

\item  $\csc(t) = \sqrt{5}$.  \vphantom{$\cos(\theta) = -\frac{\sqrt{5}}{3}$.}

\item  $\tan(\theta) = -2\sqrt{2}$.  \vphantom{$\cos(\theta) = -\frac{\sqrt{5}}{3}$.}

\item $\cos(\theta) = -\frac{\sqrt{5}}{3}$.

\setcounter{HW}{\value{enumi}}

\end{enumerate}

\end{multicols}

\begin{multicols}{3}

\begin{enumerate}
\setcounter{enumi}{\value{HW}}

\item  $\cos(t) \approx 0.9075$.  \vphantom{ $\cos(\theta) = -\frac{\sqrt{5}}{3}$}

\item $\tan(\theta) \approx - 0.6074$.  \vphantom{ $\cos(\theta) = -\frac{\sqrt{5}}{3}$}

\item  $\csc(t) \approx -4.079$.  \vphantom{ $\cos(\theta) = -\frac{\sqrt{5}}{3}$}

\setcounter{HW}{\value{enumi}}

\end{enumerate}

\end{multicols}


\begin{enumerate}

\setcounter{enumi}{\value{HW}}

\item $\sin(\theta) = \frac{3}{5}, \cos(\theta) = -\frac{4}{5}, \tan(\theta) = -\frac{3}{4}, \csc(\theta) = \frac{5}{3}, \sec(\theta) = -\frac{5}{4}, \cot(\theta) = -\frac{4}{3}$

\item $\sin(\theta) = -\frac{12}{13}, \cos(\theta) = -\frac{5}{13}, \tan(\theta) = \frac{12}{5}, \csc(\theta) = -\frac{13}{12}, \sec(\theta) = -\frac{13}{5}, \cot(\theta) = \frac{5}{12}$

\item $\sin(\theta) = \frac{24}{25}, \cos(\theta) = \frac{7}{25}, \tan(\theta) = \frac{24}{7}, \csc(\theta) = \frac{25}{24}, \sec(\theta) = \frac{25}{7}, \cot(\theta) = \frac{7}{24}$

\item $\sin(\theta) = \frac{-4\sqrt{3}}{7}, \cos(\theta) = \frac{1}{7}, \tan(\theta) = -4\sqrt{3}, \csc(\theta) = -\frac{7\sqrt{3}}{12}, \sec(\theta) = 7, \cot(\theta) = -\frac{\sqrt{3}}{12}$

\item $\sin(\theta) = -\frac{\sqrt{91}}{10}, \cos(\theta) = -\frac{3}{10}, \tan(\theta) = \frac{\sqrt{91}}{3}, \csc(\theta) = -\frac{10\sqrt{91}}{91}, \sec(\theta) = -\frac{10}{3}, \cot(\theta) = \frac{3\sqrt{91}}{91}$

\item $\sin(\theta) = \frac{\sqrt{530}}{530}, \cos(\theta) = -\frac{23\sqrt{530}}{530}, \tan(\theta) = -\frac{1}{23}, \csc(\theta) = \sqrt{530}, \sec(\theta) = -\frac{\sqrt{530}}{23}, \cot(\theta) = -23$

\item $\sin(\theta) = -\frac{2\sqrt{5}}{5}, \cos(\theta) = \frac{\sqrt{5}}{5}, \tan(\theta) = -2, \csc(\theta) = -\frac{\sqrt{5}}{2}, \sec(\theta) = \sqrt{5}, \cot(\theta) = -\frac{1}{2}$

\item  $\sin(\theta) = \frac{\sqrt{15}}{4}, \cos(\theta) = -\frac{1}{4}, \tan(\theta) = -\sqrt{15}, \csc(\theta) = \frac{4\sqrt{15}}{15}, \sec(\theta) = -4, \cot(\theta) = -\frac{\sqrt{15}}{15}$

\item $\sin(\theta) = -\frac{\sqrt{6}}{6}, \cos(\theta) = -\frac{\sqrt{30}}{6}, \tan(\theta) = \frac{\sqrt{5}}{5}, \csc(\theta) = -\sqrt{6}, \sec(\theta) = -\frac{\sqrt{30}}{5}, \cot(\theta) = \sqrt{5}$

\item $\sin(\theta) = \frac{2\sqrt{2}}{3}, \cos(\theta) = \frac{1}{3}, \tan(\theta) = 2\sqrt{2}, \csc(\theta) = \frac{3\sqrt{2}}{4}, \sec(\theta) = 3, \cot(\theta) = \frac{\sqrt{2}}{4}$

\item $\sin(t) = \frac{\sqrt{5}}{5}, \cos(t) = \frac{2\sqrt{5}}{5}, \tan(t) = \frac{1}{2}, \csc(t) = \sqrt{5}, \sec(t) = \frac{\sqrt{5}}{2}, \cot(t) = 2$

\item $\sin(t) = \frac{1}{5}, \cos(t) = -\frac{2\sqrt{6}}{5}, \tan(t) = -\frac{\sqrt{6}}{12}, \csc(t) = 5, \sec(t) = -\frac{5\sqrt{6}}{12}, \cot(t) = -2\sqrt{6}$

\item $\sin(t) = -\frac{\sqrt{110}}{11}, \cos(t) = -\frac{\sqrt{11}}{11}, \tan(t) = \sqrt{10}, \csc(t) = -\frac{\sqrt{110}}{10}, \sec(t) = -\sqrt{11}, \cot(t) = \frac{\sqrt{10}}{10}$

\item $\sin(t) = -\frac{\sqrt{95}}{10}, \cos(t) = \frac{\sqrt{5}}{10}, \tan(t) = -\sqrt{19}, \csc(t) = -\frac{2\sqrt{95}}{19}, \sec(t) = 2\sqrt{5}, \cot(t) = -\frac{\sqrt{19}}{19}$

\setcounter{HW}{\value{enumi}}

\end{enumerate}

\begin{enumerate}
\setcounter{enumi}{\value{HW}}

\item No, Skippy is not correct.  In order to be an identity, an equation must hold for \textit{all} applicable angles.  For example,  $\cos(\theta) + \sin(\theta) = 1$ does not hold when $\theta = \pi$.  

\setcounter{HW}{\value{enumi}}
\end{enumerate}

\begin{enumerate}
\setcounter{enumi}{\value{HW}}
\addtocounter{enumi}{51}

\item  \begin{enumerate}  \item  As $\theta \rightarrow 0$,  $\frac{1 - \cos(\theta)}{\theta} \rightarrow \frac{0}{0}$.


\item  The graph of  $f(\theta) = \frac{1 - \cos(\theta)}{\theta}$ approaches $(0,0)$, so   $\ds{\lim_{\theta \rightarrow 0}}$ $\frac{1 - \cos(\theta)}{\theta}$ appears to be $0$.

\addtocounter{enumii}{1}


\item  $\ds{\lim_{\theta \rightarrow 0}}$ $\frac{1 - \cos(\theta)}{\theta}$ $= \ds{\lim_{\theta \rightarrow 0}}$ $\frac{\sin(\theta)}{1 + \cos(\theta)} \, \frac{\sin(\theta)}{\theta} = \left(\frac{0}{1+\cos(0)}\right)(1) = \left(\frac{0}{2}\right)(1) = 0$.
\end{enumerate}

\setcounter{HW}{\value{enumi}}
\end{enumerate}






\end{document}
