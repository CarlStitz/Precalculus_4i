\documentclass{ximera}

\begin{document}
	\author{Stitz-Zeager}
	\xmtitle{TITLE}
\mfpicnumber{1} \opengraphsfile{ExercisesforPolarComplex} % mfpic settings added 


In Exercises \ref{polarcompbasicfirst} - \ref{polarcompbasiclast}, find a polar representation for the complex number $z$.  Identify $\text{Re}(z)$, $\text{Im}(z)$, $|z|$, $\text{arg}(z)$ and $\text{Arg}(z)$.

\begin{multicols}{4}

\begin{enumerate}

\item $z = 9 + 9i$ \label{polarcompbasicfirst}
\item $z = 5 + 5i\sqrt{3}$
\item $z = 6i$
\item $z = -3\sqrt{2} + 3i\sqrt{2}$

\setcounter{HW}{\value{enumi}}

\end{enumerate}

\end{multicols}

\begin{multicols}{4} 

\begin{enumerate}

\setcounter{enumi}{\value{HW}}

\item $z = -6\sqrt{3} + 6i$ \vphantom{$\dfrac{\sqrt{3}}{2}$}
\item $z = -2$ \vphantom{$\dfrac{\sqrt{3}}{2}$}
\item $z = -\dfrac{\sqrt{3}}{2} - \dfrac{1}{2}i$
\item $z = -3-3i$ \vphantom{$\dfrac{\sqrt{3}}{2}$}

\setcounter{HW}{\value{enumi}}

\end{enumerate}

\end{multicols}

\begin{multicols}{4} 

\begin{enumerate}

\setcounter{enumi}{\value{HW}}

\item $z = -5i$
\item $z = 2\sqrt{2} - 2i\sqrt{2}$
\item $z = 6$
\item $z = i\sqrt[3]{7}$

\setcounter{HW}{\value{enumi}}

\end{enumerate}

\end{multicols}

\begin{multicols}{4} 

\begin{enumerate}

\setcounter{enumi}{\value{HW}}

\item $z = 3 + 4i$
\item $z = \sqrt{2} + i$
\item $z = -7 + 24i$
\item $z = -2+6i$

\setcounter{HW}{\value{enumi}}

\end{enumerate}

\end{multicols}

\begin{multicols}{4} 

\begin{enumerate}

\setcounter{enumi}{\value{HW}}

\item $z = -12-5i$
\item $z = -5-2i$
\item $z = 4-2i$
\item $z = 1-3i$ \label{polarcompbasiclast}

\setcounter{HW}{\value{enumi}}

\end{enumerate}

\end{multicols}

In Exercises \ref{rectcompfirst} - \ref{rectcomplast}, find the rectangular form of the given complex number.  Use whatever identities are necessary to find the exact values.

\begin{multicols}{4}

\begin{enumerate}

\setcounter{enumi}{\value{HW}}

\item $z = 6\text{cis}(0)$ \vphantom{$\left(\dfrac{\pi}{6}\right)$} \label{rectcompfirst}
\item $z = 2\text{cis}\left(\dfrac{\pi}{6}\right)$ 
\item $z = 7\sqrt{2}\text{cis}\left(\dfrac{\pi}{4}\right)$
\item $z = 3\text{cis}\left(\dfrac{\pi}{2}\right)$ 

\setcounter{HW}{\value{enumi}}

\end{enumerate}

\end{multicols}

\begin{multicols}{4} 

\begin{enumerate}

\setcounter{enumi}{\value{HW}}

\item $z = 4\text{cis}\left(\dfrac{2\pi}{3}\right)$ 
\item $z = \sqrt{6}\text{cis}\left(\dfrac{3\pi}{4}\right)$ 
\item $z = 9\text{cis}\left(\pi\right)$ \vphantom{$\left(\dfrac{7\pi}{6}\right)$}
\item $z = 3\text{cis}\left(\dfrac{4\pi}{3}\right)$

\setcounter{HW}{\value{enumi}}

\end{enumerate}

\end{multicols}

\begin{multicols}{4} 

\begin{enumerate}

\setcounter{enumi}{\value{HW}}

\item $z = 7\text{cis}\left(-\dfrac{3\pi}{4}\right)$ 
\item \small $z = \sqrt{13}\text{cis}\left(\dfrac{3\pi}{2}\right)$ \normalsize
\item $z = \dfrac{1}{2}\text{cis}\left(\dfrac{7\pi}{4}\right)$ 
\item $z = 12\text{cis}\left(-\dfrac{\pi}{3}\right)$ \vphantom{$\left(\dfrac{7\pi}{6}\right)$}

\setcounter{HW}{\value{enumi}}

\end{enumerate}

\end{multicols}

\begin{multicols}{2} 

\begin{enumerate}

\setcounter{enumi}{\value{HW}}

\item $z = 8\text{cis}\left(\dfrac{\pi}{12}\right)$ \vphantom{$\left(\dfrac{7\pi}{6}\right)$}
\item $z = 2\text{cis}\left(\dfrac{7\pi}{8}\right)$ 

\setcounter{HW}{\value{enumi}}

\end{enumerate}

\end{multicols}

\begin{multicols}{2} 

\begin{enumerate}

\setcounter{enumi}{\value{HW}}

\item $z = 5\text{cis}\left(\arctan\left(\dfrac{4}{3}\right)\right)$
\item $z = \sqrt{10}\text{cis}\left(\arctan\left(\dfrac{1}{3}\right)\right)$ 

\setcounter{HW}{\value{enumi}}

\end{enumerate}

\end{multicols}

\begin{multicols}{2} 

\begin{enumerate}

\setcounter{enumi}{\value{HW}}

\item $z = 15\text{cis}\left(\arctan\left(-2\right)\right)$ 
\item $z=  \sqrt{3}\left(\arctan\left(-\sqrt{2}\right)\right)$

\setcounter{HW}{\value{enumi}}

\end{enumerate}

\end{multicols}

\begin{multicols}{2} 

\begin{enumerate}

\setcounter{enumi}{\value{HW}}

\item $z = 50\text{cis}\left(\pi-\arctan\left(\dfrac{7}{24}\right)\right)$ 
\item  $z = \dfrac{1}{2}\text{cis}\left(\pi+\arctan\left(\dfrac{5}{12}\right)\right)$ \label{rectcomplast}

\setcounter{HW}{\value{enumi}}

\end{enumerate}

\end{multicols}

For Exercises \ref{polarcomparithfirst} - \ref{polarcomparithlast}, use $z = -\frac{3\sqrt{3}}{2} + \frac{3}{2}i$ and $w = 3\sqrt{2} - 3i\sqrt{2}$ to compute the quantity.  Express your answers in polar form using the principal argument.

\begin{multicols}{4}

\begin{enumerate}

\setcounter{enumi}{\value{HW}}

\item $zw$ \vphantom{$\dfrac{z}{w}$} \label{polarcomparithfirst}
\item $\dfrac{z}{w}$
\item $\dfrac{w}{z}$
\item $z^{4}$ \vphantom{$\dfrac{z}{w}$}

\setcounter{HW}{\value{enumi}}

\end{enumerate}

\end{multicols}

\begin{multicols}{4} 

\begin{enumerate}

\setcounter{enumi}{\value{HW}}

\item $w^{3}$ \vphantom{$\dfrac{z^{2}}{w}$}
\item $z^{5}w^{2}$ \vphantom{$\dfrac{z^{2}}{w}$}
\item $z^{3}w^{2}$ \vphantom{$\dfrac{z^{2}}{w}$}
\item $\dfrac{z^{2}}{w}$ 

\setcounter{HW}{\value{enumi}}

\end{enumerate}

\end{multicols}

\begin{multicols}{4} 

\begin{enumerate}

\setcounter{enumi}{\value{HW}}

\item $\dfrac{w}{z^2}$ \vphantom{$\dfrac{z^{2}}{w^{3}}$}
\item $\dfrac{z^3}{w^2}$ 
\item $\dfrac{w^2}{z^3}$ 
\item $\left(\dfrac{w}{z}\right)^6$ \vphantom{$\dfrac{z^{2}}{w^{3}}$} \label{polarcomparithlast}

\setcounter{HW}{\value{enumi}}

\end{enumerate}

\end{multicols}

In Exercises \ref{demoivrefirst} - \ref{demoivrelast}, use DeMoivre's Theorem to find the indicated power of the given complex number.  Express your final answers in rectangular form.

\begin{multicols}{4}

\begin{enumerate}

\setcounter{enumi}{\value{HW}}

\item $\left(-2 + 2i\sqrt{3}\right)^3$ \label{demoivrefirst}
\item $(-\sqrt{3} - i)^3$ 
\item $(-3+3i)^{4}$
\item $(\sqrt{3} + i)^4$

\setcounter{HW}{\value{enumi}}

\end{enumerate}

\end{multicols}

\begin{multicols}{4} 

\begin{enumerate}

\setcounter{enumi}{\value{HW}}

\item $\left(\dfrac{5}{2} + \dfrac{5}{2} i\right)^3$ \vphantom{$\left(\dfrac{\sqrt{2}}{2}\right)$}
\item $\left(-\dfrac{1}{2} - \dfrac{\sqrt{3}}{2} i\right)^{6}$
\item $\left(\dfrac{3}{2} - \dfrac{3}{2} i\right)^3$ \vphantom{$\left(\dfrac{\sqrt{2}}{2}\right)$}
\item $\left(\dfrac{\sqrt{3}}{3} - \dfrac{1}{3} i\right)^4$

\setcounter{HW}{\value{enumi}}

\end{enumerate}

\end{multicols}

\begin{multicols}{4} 

\begin{enumerate}

\setcounter{enumi}{\value{HW}}

\item $\left(\dfrac{\sqrt{2}}{2} + \dfrac{\sqrt{2}}{2} i\right)^4$
\item $(2+2i)^5$ \vphantom{$\left(\dfrac{\sqrt{2}}{2}\right)$}
\item $(\sqrt{3} - i)^{5}$ \vphantom{$\left(\dfrac{\sqrt{2}}{2}\right)$}
\item $(1-i)^8$ \vphantom{$\left(\dfrac{\sqrt{2}}{2}\right)$} \label{demoivrelast}

\setcounter{HW}{\value{enumi}}

\end{enumerate}

\end{multicols}

In Exercises \ref{polarrootsfirst} - \ref{polarrootslast}, find the indicated complex roots.  Express your answers in polar form and then convert them into rectangular form.

\begin{multicols}{2}

\begin{enumerate}

\setcounter{enumi}{\value{HW}}

\item the two square roots of $z = 4i$ \label{polarrootsfirst}
\item the two square roots of $z = -25i$

\setcounter{HW}{\value{enumi}}

\end{enumerate}

\end{multicols}

\begin{multicols}{2} 

\begin{enumerate}

\setcounter{enumi}{\value{HW}}

\item the two square roots of $z = 1 + i\sqrt{3}$
\item the two square roots of $\frac{5}{2} - \frac{5\sqrt{3}}{2}i$

\setcounter{HW}{\value{enumi}}

\end{enumerate}

\end{multicols}

\begin{multicols}{2} 

\begin{enumerate}

\setcounter{enumi}{\value{HW}}

\item  the three cube roots of $z=64$
\item  the three cube roots of $z = -125$

\setcounter{HW}{\value{enumi}}

\end{enumerate}

\end{multicols}

\begin{multicols}{2} 

\begin{enumerate}

\setcounter{enumi}{\value{HW}}

\item  the three cube roots of $z = i$
\item  the three cube roots of $z = -8i$

\setcounter{HW}{\value{enumi}}

\end{enumerate}

\end{multicols}

\begin{multicols}{2} 

\begin{enumerate}

\setcounter{enumi}{\value{HW}}

\item  the four fourth roots of $z=16$
\item  the four fourth roots of $z=-81$

\setcounter{HW}{\value{enumi}}

\end{enumerate}

\end{multicols}

\begin{multicols}{2} 

\begin{enumerate}

\setcounter{enumi}{\value{HW}}

\item the six sixth roots of $z = 64$
\item the six sixth roots of $z = -729$ \label{polarrootslast}

\setcounter{HW}{\value{enumi}}

\end{enumerate}

\end{multicols}

\begin{enumerate}

\setcounter{enumi}{\value{HW}}

\item Use the Sum and Difference Identities in Theorem \ref{circularsumdifference} or the Half Angle Identities in Theorem \ref{halfangle} to convert the three cube roots of $z=\sqrt{2} + i\sqrt{2}$ we found in  Example \ref{nthrootscomplexex}, number \ref{halfanglecuberoot} from polar form to rectangular form.

\item Use a calculator to approximate the rectangular form of the five fifth roots of $1$ we found in Example \ref{nthrootscomplexex}, number \ref{calculatorfifthroot}.

\item According to Theorem \ref{realfactorization} in Section \ref{ComplexZeros}, the polynomial $p(x) = x^{4} + 4$ can be factored into the product linear and irreducible quadratic factors.  In Exercise \ref{factorpolywithnonlinear} in Section \ref{NonLinearEquations}, we showed you how to factor this polynomial into the product of two irreducible quadratic factors using a system of non-linear equations.  Now that we can compute the complex fourth roots of $-4$ directly using  Theorem \ref{nthrootscomplexthm}, we can  apply the Complex Factorization Theorem, Theorem \ref{complexfactorization}, to obtain the linear factorization $p(x) = (x - (1 + i))(x - (1 - i))(x - (-1 + i))(x - (-1 - i))$.  By multiplying the first two factors together and then the second two factors together, thus pairing up the complex conjugate pairs of zeros Theorem \ref{conjugatepairsthm} told us we'd get, we have that $p(x) = (x^{2} - 2x + 2)(x^{2} + 2x + 2)$.  Use the 12 complex $12^{\text{th}}$ roots of 4096 to factor $p(x) = x^{12} - 4096$ into a product of linear and irreducible quadratic factors.


\item Use Exercise \ref{triangleineqforvectorsexercise} from Section \ref{TheDotProduct} to show the the Triangle Inequality $|z + w| \leq |z| + |w|$ holds for all complex numbers $z$ and $w$ as well.  Identify the complex number $z = a + bi$ with the vector $u = \langle a, b \rangle$ and identify the complex number $w = c + di$ with the vector $v = \langle c, d \rangle$ and just follow your nose!


\item Complete the proof of Theorem \ref{modprops} by showing that if $w \neq 0$ than  $\left| \frac{1}{w}\right| = \frac{1}{|w|}$.

\item Recall from Section \ref{ComplexZeros} that given a complex number $z = a+bi$ its complex conjugate, denoted $\overline{z}$, is given by $\overline{z} = a - bi$.

\begin{enumerate}

\item Prove that $\left| \overline{z} \right| = |z|$.

\item Prove that $|z| = \sqrt{z \overline{z}}$

\item Show that $\text{Re}(z) = \dfrac{z + \overline{z}}{2}$ and $\text{Im}(z) = \dfrac{z - \overline{z}}{2i}$

\item Show that if $\theta \in \text{arg}(z)$ then $-\theta \in \text{arg}\left(\overline{z}\right)$. Interpret this result geometrically.

\item Is it always true that $\text{Arg}\left(\overline{z}\right) = -\text{Arg}(z)$?

\end{enumerate}

\item Given a natural number $n \geq 2$, the $n$ complex $n^{\text{th}}$ roots of  $z = 1$ are called the \textbf{\boldmath $n^{\mbox{\textbf{\scriptsize th}}}$ Roots of Unity}. \index{$n^{\textrm{th}}$ Roots of Unity} \index{complex number ! $n^{\textrm{th}}$ Roots of Unity} \index{Roots of Unity} In the following exercises, assume that $n$ is a fixed, but arbitrary, natural number such that $n \geq 2$.

\begin{enumerate}

\item Show that $w = 1$ is an $n^{\text{th}}$ root of unity.

\item Show that if both $w_{\text{\tiny$j$}}$ and $w_{\text{\tiny$k$}}$ are $n^{\text{th}}$ roots of unity then so is their product $w_{\text{\tiny$j$}}w_{\text{\tiny$k$}}$.

\item Show that if $w_{\text{\tiny$j$}}$ is an $n^{\text{th}}$ root of unity then there is an $n^{\text{th}}$ root of unity $w_{\text{\tiny$j'$}}$  so that  $w_{\text{\tiny$j$}}w_{\text{\tiny$j'$}} = 1$. 

\smallskip

 HINT: If $w_{\text{\tiny$j$}} = \text{cis}(\theta)$ let $w_{\text{\tiny$j'$}} = \text{cis}(2\pi - \theta)$. Show $w_{\text{\tiny$j'$}} = \text{cis}(2\pi - \theta)$ is indeed an $n^{\text{th}}$ root of unity.

\end{enumerate}

\item \label{eulerformulaexercise} Another way to express the polar form of a complex number is to use the exponential function.  For real numbers $t$, \href{http://en.wikipedia.org/wiki/Leonhard_Euler}{\underline{Euler}}'s Formula defines $e^{it} = \cos(t) + i \sin(t)$.  

\begin{enumerate}

\item Use Theorem \ref{prodquotpolarcomplex} to show that:

\begin{enumerate}

\item  $e^{ix} e^{iy} = e^{i(x+y)}$ for all real numbers $x$ and $y$.

\item  $\left(e^{ix}\right)^{n} = e^{i(nx)}$ for any real number $x$ and any natural number $n$.

\item $\dfrac{e^{ix}}{e^{iy}} = e^{i(x-y)}$ for all real numbers $x$ and $y$.

\end{enumerate}



\item If $z = r\text{cis}(\theta)$ is the polar form of $z$, show that $z = re^{it}$ where $\theta = t$ radians.

\item Show that $e^{i\pi} + 1 = 0$.  (This famous equation relates the five most important constants in all of Mathematics with the three most fundamental operations in Mathematics.)

\item  \label{expformcosandsin} Show that $\cos(t) = \dfrac{e^{it} + e^{-it}}{2}$ and that $\sin(t) = \dfrac{e^{it} - e^{-it}}{2i}$ for all real numbers $t$. 

\end{enumerate}

\end{enumerate}

\newpage

\subsection{Answers}

\begin{enumerate}

\item $z = 9 + 9i = 9\sqrt{2}\text{cis}\left(\frac{\pi}{4}\right)$, \, $\text{Re}(z) = 9$, \, $\text{Im}(z) = 9$, \, $|z| = 9\sqrt{2}$

$\text{arg}(z) = \left\{\frac{\pi}{4} + 2\pi k \, | \, \text{$k$ is an integer} \right\}$ and $\text{Arg}(z) = \frac{\pi}{4}$.

\item $z = 5+5i\sqrt{3} = 10\text{cis}\left(\frac{\pi}{3}\right)$, \, $\text{Re}(z) = 5$, \, $\text{Im}(z) = 5\sqrt{3}$, \, $|z| = 10$

$\text{arg}(z) = \left\{\frac{\pi}{3} + 2\pi k \, | \, \text{$k$ is an integer} \right\}$ and $\text{Arg}(z) = \frac{\pi}{3}$.

\item $z = 6i = 6\text{cis}\left(\frac{\pi}{2}\right)$, \, $\text{Re}(z) = 0$, \, $\text{Im}(z) = 6$, \, $|z| = 6$

$\text{arg}(z) = \left\{\frac{\pi}{2} + 2\pi k \, | \, \text{$k$ is an integer} \right\}$ and $\text{Arg}(z) = \frac{\pi}{2}$.

\item $z = -3\sqrt{2} + 3i\sqrt{2} = 6\text{cis}\left(\frac{3\pi}{4}\right)$, \, $\text{Re}(z) = -3\sqrt{2}$, \, $\text{Im}(z) =3\sqrt{2}$, \, $|z| = 6$

$\text{arg}(z) = \left\{\frac{3\pi}{4} + 2\pi k \, | \, \text{$k$ is an integer} \right\}$ and $\text{Arg}(z) = \frac{3\pi}{4}$.

\item $z =  -6\sqrt{3} + 6i = 12\text{cis}\left(\frac{5\pi}{6}\right)$, \, $\text{Re}(z) = -6\sqrt{3}$, \, $\text{Im}(z) =6$, \, $|z| = 12$

$\text{arg}(z) = \left\{\frac{5\pi}{6} + 2\pi k \, | \, \text{$k$ is an integer} \right\}$ and $\text{Arg}(z) = \frac{5\pi}{6}$.

\item $z =  -2 = 2\text{cis}\left(\pi\right)$, \, $\text{Re}(z) = -2$, \, $\text{Im}(z) =0$, \, $|z| = 2$

$\text{arg}(z) = \left\{(2k+1)\pi \, | \, \text{$k$ is an integer} \right\}$ and $\text{Arg}(z) = \pi$.

\item $z = -\frac{\sqrt{3}}{2} - \frac{1}{2}i = \text{cis}\left(\frac{7\pi}{6}\right)$, \, $\text{Re}(z) = -\frac{\sqrt{3}}{2}$, \, $\text{Im}(z) = -\frac{1}{2}$, \, $|z| = 1$

$\text{arg}(z) = \left\{\frac{7\pi}{6} + 2\pi k \, | \, \text{$k$ is an integer} \right\}$ and $\text{Arg}(z) = -\frac{5\pi}{6}$.

\item $z = -3-3i = 3\sqrt{2}\text{cis}\left(\frac{5\pi}{4}\right)$, \, $\text{Re}(z) = -3$, \, $\text{Im}(z) =-3$, \, $|z| = 3\sqrt{2}$

$\text{arg}(z) = \left\{\frac{5\pi}{4} + 2\pi k \, | \, \text{$k$ is an integer} \right\}$ and $\text{Arg}(z) = -\frac{3\pi}{4}$.

\item $z = -5i = 5\text{cis}\left(\frac{3\pi}{2}\right)$, \, $\text{Re}(z) = 0$, \, $\text{Im}(z) = -5$, \, $|z| = 5$

$\text{arg}(z) = \left\{\frac{3\pi}{2} + 2\pi k \, | \, \text{$k$ is an integer} \right\}$ and $\text{Arg}(z) = -\frac{\pi}{2}$.

\item $z = 2\sqrt{2} - 2i\sqrt{2} = 4\text{cis}\left(\frac{7\pi}{4}\right)$, \, $\text{Re}(z) = 2\sqrt{2}$, \, $\text{Im}(z) = -2\sqrt{2}$, \, $|z| = 4$

$\text{arg}(z) = \left\{\frac{7\pi}{4} + 2\pi k \, | \, \text{$k$ is an integer} \right\}$ and $\text{Arg}(z) = -\frac{\pi}{4}$.

\item $z =6 = 6\text{cis}\left(0\right)$, \, $\text{Re}(z) = 6$, \, $\text{Im}(z) = 0$, \, $|z| = 6$

$\text{arg}(z) = \left\{2\pi k \, | \, \text{$k$ is an integer} \right\}$ and $\text{Arg}(z) =0$.

\item $z = i \sqrt[3]{7} = \sqrt[3]{7}\text{cis}\left(\frac{\pi}{2}\right)$, \, $\text{Re}(z) =0$, \, $\text{Im}(z) = \sqrt[3]{7}$, \, $|z| = \sqrt[3]{7}$

$\text{arg}(z) = \left\{\frac{\pi}{2} + 2\pi k \, | \, \text{$k$ is an integer} \right\}$ and $\text{Arg}(z) = \frac{\pi}{2}$.

\item $z = 3+4i = 5\text{cis}\left(\arctan\left(\frac{4}{3}\right)\right)$, \, $\text{Re}(z) = 3$, \, $\text{Im}(z) = 4$, \, $|z| = 5$

$\text{arg}(z) = \left\{\arctan\left(\frac{4}{3}\right) + 2\pi k \, | \, \text{$k$ is an integer} \right\}$ and $\text{Arg}(z) =\arctan\left(\frac{4}{3}\right) $.

\item $z = \sqrt{2}+i = \sqrt{3}\text{cis}\left(\arctan\left(\frac{\sqrt{2}}{2}\right)\right)$, \, $\text{Re}(z) = \sqrt{2}$, \, $\text{Im}(z) = 1$, \, $|z| = \sqrt{3}$

$\text{arg}(z) = \left\{\arctan\left(\frac{\sqrt{2}}{2}\right) + 2\pi k \, | \, \text{$k$ is an integer} \right\}$ and $\text{Arg}(z) =\arctan\left(\frac{\sqrt{2}}{2}\right) $.

\item $z = -7 + 24i = 25\text{cis}\left(\pi - \arctan\left(\frac{24}{7}\right)\right)$, \, $\text{Re}(z) = -7$, \, $\text{Im}(z) = 24$, \, $|z| = 25$

$\text{arg}(z) = \left\{\pi - \arctan\left(\frac{24}{7}\right) + 2\pi k \, | \, \text{$k$ is an integer} \right\}$ and $\text{Arg}(z) =\pi - \arctan\left(\frac{24}{7}\right) $.

\item $z = -2 + 6i = 2\sqrt{10}\text{cis}\left(\pi - \arctan\left(3\right)\right)$, \, $\text{Re}(z) = -2$, \, $\text{Im}(z) = 6$, \, $|z| =2\sqrt{10}$

$\text{arg}(z) = \left\{\pi - \arctan\left(3\right) + 2\pi k \, | \, \text{$k$ is an integer} \right\}$ and $\text{Arg}(z) =\pi - \arctan\left(3\right) $.

\item $z = -12 -5i = 13\text{cis}\left(\pi + \arctan\left(\frac{5}{12}\right)\right)$, \, $\text{Re}(z) = -12$, \, $\text{Im}(z) = -5$, \, $|z| = 13$

$\text{arg}(z) = \left\{\pi +\arctan\left(\frac{5}{12}\right) + 2\pi k \, | \, \text{$k$ is an integer} \right\}$ and $\text{Arg}(z) =  \arctan\left(\frac{5}{12}\right) -\pi $.

\item $z = -5-2i = \sqrt{29}\text{cis}\left(\pi + \arctan\left(\frac{2}{5}\right)\right)$, \, $\text{Re}(z) = -5$, \, $\text{Im}(z) = -2$, \, $|z| = \sqrt{29}$

$\text{arg}(z) = \left\{\pi +\arctan\left(\frac{2}{5}\right) + 2\pi k \, | \, \text{$k$ is an integer} \right\}$ and $\text{Arg}(z) =  \arctan\left(\frac{2}{5}\right) -\pi $.

\item $z =4-2i = 2\sqrt{5}\text{cis}\left(\arctan\left(-\frac{1}{2}\right)\right)$, \, $\text{Re}(z) =4$, \, $\text{Im}(z) = -2$, \, $|z| = 2\sqrt{5}$

$\text{arg}(z) = \left\{\arctan\left(-\frac{1}{2}\right) + 2\pi k \, | \, \text{$k$ is an integer} \right\}$ and $\text{Arg}(z) = \arctan\left(-\frac{1}{2}\right) = -\arctan\left(\frac{1}{2}\right) $.

\item $z =1-3i = \sqrt{10}\text{cis}\left(\arctan\left(-3\right)\right)$, \, $\text{Re}(z) =1$, \, $\text{Im}(z) = -3$, \, $|z| =\sqrt{10}$

$\text{arg}(z) = \left\{\arctan\left(-3\right) + 2\pi k \, | \, \text{$k$ is an integer} \right\}$ and $\text{Arg}(z) =  \arctan\left(-3\right) = -\arctan(3)$.

\setcounter{HW}{\value{enumi}}

\end{enumerate}

\begin{multicols}{2}

\begin{enumerate}

\setcounter{enumi}{\value{HW}}

\item $z = 6\text{cis}(0) = 6$
\item $z = 2\text{cis}\left(\frac{\pi}{6}\right) = \sqrt{3} + i$

\setcounter{HW}{\value{enumi}}

\end{enumerate}

\end{multicols}

\begin{multicols}{2} 

\begin{enumerate}

\setcounter{enumi}{\value{HW}}

\item $z = 7\sqrt{2}\text{cis}\left(\frac{\pi}{4}\right) = 7+7i$
\item $z = 3\text{cis}\left(\frac{\pi}{2}\right) = 3i$ 

\setcounter{HW}{\value{enumi}}

\end{enumerate}

\end{multicols}

\begin{multicols}{2} 

\begin{enumerate}

\setcounter{enumi}{\value{HW}}

\item $z = 4\text{cis}\left(\frac{2\pi}{3}\right) = -2+2i\sqrt{3}$
\item $z = \sqrt{6}\text{cis}\left(\frac{3\pi}{4}\right) = -\sqrt{3}+i\sqrt{3}$ 

\setcounter{HW}{\value{enumi}}

\end{enumerate}

\end{multicols}

\begin{multicols}{2} 

\begin{enumerate}

\setcounter{enumi}{\value{HW}}

\item $z = 9\text{cis}\left(\pi\right) = -9$
\item $z = 3\text{cis}\left(\frac{4\pi}{3}\right) = -\frac{3}{2} - \frac{3i\sqrt{3}}{2}$

\setcounter{HW}{\value{enumi}}

\end{enumerate}

\end{multicols}

\begin{multicols}{2} 

\begin{enumerate}

\setcounter{enumi}{\value{HW}}

\item $z = 7\text{cis}\left(-\frac{3\pi}{4}\right) = -\frac{7\sqrt{2}}{2} - \frac{7\sqrt{2}}{2}i$ 
\item $z = \sqrt{13}\text{cis}\left(\frac{3\pi}{2}\right) = -i\sqrt{13}$ 

\setcounter{HW}{\value{enumi}}

\end{enumerate}

\end{multicols}

\begin{multicols}{2} 

\begin{enumerate}

\setcounter{enumi}{\value{HW}}

\item $z = \frac{1}{2}\text{cis}\left(\frac{7\pi}{4}\right) = \frac{\sqrt{2}}{4} - i\frac{\sqrt{2}}{4}$ 
\item $z = 12\text{cis}\left(-\frac{\pi}{3}\right) = 6 - 6i\sqrt{3}$ 

\setcounter{HW}{\value{enumi}}

\end{enumerate}

\end{multicols}

\begin{multicols}{2} 

\begin{enumerate}

\setcounter{enumi}{\value{HW}}

\item $z = 8\text{cis}\left(\frac{\pi}{12}\right) = 4\sqrt{2+\sqrt{3}}+4i\sqrt{2-\sqrt{3}}$ 
\item $z = 2\text{cis}\left(\frac{7\pi}{8}\right) = -\sqrt{2 + \sqrt{2}} + i\sqrt{2 - \sqrt{2}}$ 

\setcounter{HW}{\value{enumi}}

\end{enumerate}

\end{multicols}

\begin{multicols}{2} 

\begin{enumerate}

\setcounter{enumi}{\value{HW}}

\item $z = 5\text{cis}\left(\arctan\left(\frac{4}{3}\right)\right) = 3 + 4i$ 
\item $z = \sqrt{10}\text{cis}\left(\arctan\left(\frac{1}{3}\right)\right) = 3+i$ 

\setcounter{HW}{\value{enumi}}

\end{enumerate}

\end{multicols}

\begin{multicols}{2} 

\begin{enumerate}

\setcounter{enumi}{\value{HW}}

\item $z = 15\text{cis}\left(\arctan\left(-2\right)\right) = 3\sqrt{5} -6i\sqrt{5}$ 
\item $z=  \sqrt{3}\text{cis}\left(\arctan\left(-\sqrt{2}\right)\right) = 1-i\sqrt{2}$

\setcounter{HW}{\value{enumi}}

\end{enumerate}

\end{multicols}

\begin{multicols}{2} 

\begin{enumerate}

\setcounter{enumi}{\value{HW}}

\item $z = 50\text{cis}\left(\pi-\arctan\left(\frac{7}{24}\right)\right) = -48 + 14i$ 
\item $z = \frac{1}{2}\text{cis}\left(\pi+\arctan\left(\frac{5}{12}\right)\right) = -\frac{6}{13} - \frac{5i}{26}$

\setcounter{HW}{\value{enumi}}

\end{enumerate}

\end{multicols}

\pagebreak

In Exercises \ref{polarcomparithfirst} - \ref{polarcomparithlast}, we have that $z = -\frac{3\sqrt{3}}{2} + \frac{3}{2}i = 3\text{cis}\left(\frac{5\pi}{6}\right)$ and $w = 3\sqrt{2} - 3i\sqrt{2} = 6\text{cis}\left(-\frac{\pi}{4}\right)$ so we get the following.  

\begin{multicols}{3}

\begin{enumerate}

\setcounter{enumi}{\value{HW}}

\item $zw = 18\text{cis}\left(\frac{7\pi}{12}\right)$
\item $\frac{z}{w} = \frac{1}{2}\text{cis}\left(-\frac{11\pi}{12}\right)$
\item $\frac{w}{z} = 2\text{cis}\left(\frac{11\pi}{12}\right)$

\setcounter{HW}{\value{enumi}}

\end{enumerate}

\end{multicols}

\begin{multicols}{3} 

\begin{enumerate}

\setcounter{enumi}{\value{HW}}

\item $z^{4} = 81\text{cis}\left(-\frac{2\pi}{3}\right)$
\item $w^{3} = 216\text{cis}\left(-\frac{3\pi}{4}\right)$
\item $z^{5}w^{2} = 8748\text{cis}\left(-\frac{\pi}{3}\right)$

\setcounter{HW}{\value{enumi}}

\end{enumerate}

\end{multicols}

\begin{multicols}{3} 

\begin{enumerate}

\setcounter{enumi}{\value{HW}}

\item $z^3w^2 = 972 \text{cis}(0)$
\item $\frac{z^2}{w} =\frac{3}{2}\text{cis}\left(-\frac{\pi}{12}\right)$
\item $\frac{w}{z^2} =\frac{2}{3}\text{cis}\left(\frac{\pi}{12}\right)$

\setcounter{HW}{\value{enumi}}

\end{enumerate}

\end{multicols}

\begin{multicols}{3} 

\begin{enumerate}

\setcounter{enumi}{\value{HW}}

\item $\frac{z^3}{w^2} =\frac{3}{4}\text{cis}(\pi)$
\item $\frac{w^2}{z^3} =\frac{4}{3}\text{cis}(\pi)$
\item $\left(\frac{w}{z}\right)^6 =64\text{cis}\left(-\frac{\pi}{2} \right)$

\setcounter{HW}{\value{enumi}}

\end{enumerate}

\end{multicols}

\begin{multicols}{3}

\begin{enumerate}

\setcounter{enumi}{\value{HW}}

\item $\left(-2 + 2i\sqrt{3}\right)^3 = 64$
\item $(-\sqrt{3} - i)^3 =-8i$
\item $(-3+3i)^{4}=-324$

\setcounter{HW}{\value{enumi}}

\end{enumerate}

\end{multicols}

\begin{multicols}{3}

\begin{enumerate}

\setcounter{enumi}{\value{HW}}

\item $(\sqrt{3} + i)^4 =-8 + 8i\sqrt{3}$ \vphantom{$\left(\frac{\sqrt{2}}{2}\right)^{2}$}
\item $\left(\frac{5}{2} + \frac{5}{2} i\right)^3=-\frac{125}{4}+\frac{125}{4} i$ \vphantom{$\left(\frac{\sqrt{2}}{2}\right)^{2}$}
\item $\left(-\frac{1}{2} - \frac{i \sqrt{3}}{2}\right)^{6}=1$

\setcounter{HW}{\value{enumi}}

\end{enumerate}

\end{multicols}

\begin{multicols}{3}

\begin{enumerate}

\setcounter{enumi}{\value{HW}}

\item $\left(\frac{3}{2} - \frac{3}{2} i\right)^3=-\frac{27}{4}-\frac{27}{4} i$ \vphantom{$\left(\frac{\sqrt{2}}{2}\right)^{2}$}
\item $\left(\frac{\sqrt{3}}{3} - \frac{1}{3} i\right)^4 =-\frac{8}{81} - \frac{8i\sqrt{3}}{81}$
\item $\left(\frac{\sqrt{2}}{2} + \frac{\sqrt{2}}{2} i\right)^4=-1$

\setcounter{HW}{\value{enumi}}

\end{enumerate}

\end{multicols}

\begin{multicols}{3}

\begin{enumerate}

\setcounter{enumi}{\value{HW}}

\item $(2+2i)^5 = -128-128i$
\item $(\sqrt{3} - i)^{5} =  -16\sqrt{3} - 16i$
\item  $(1-i)^8=16$

\setcounter{HW}{\value{enumi}}

\end{enumerate}

\end{multicols}

\begin{enumerate}

\setcounter{enumi}{\value{HW}}

\item Since $z=4i = 4\text{cis}\left(\frac{\pi}{2}\right)$ we have 

\begin{multicols}{2}

$w_{\text{\tiny$0$}} = 2\text{cis}\left(\frac{\pi}{4}\right) = \sqrt{2} +i\sqrt{2}$

$w_{\text{\tiny$1$}} = 2\text{cis}\left(\frac{5\pi}{4}\right) = -\sqrt{2} - i\sqrt{2}$

\end{multicols}

\item Since $z=-25i = 25\text{cis}\left(\frac{3\pi}{2}\right)$ we have 

\begin{multicols}{2}

$w_{\text{\tiny$0$}} = 5\text{cis}\left(\frac{3\pi}{4}\right) = -\frac{5\sqrt{2}}{2} +\frac{5\sqrt{2}}{2} i$

$w_{\text{\tiny$1$}} = 5\text{cis}\left(\frac{7\pi}{4}\right) = \frac{5\sqrt{2}}{2} - \frac{5\sqrt{2}}{2} i$

\end{multicols}

\item Since $z=1 + i\sqrt{3} = 2\text{cis}\left(\frac{\pi}{3}\right)$ we have 

\begin{multicols}{2}

$w_{\text{\tiny$0$}} = \sqrt{2}\text{cis}\left(\frac{\pi}{6}\right) = \frac{\sqrt{6}}{2} +\frac{\sqrt{2}}{2} i$

$w_{\text{\tiny$1$}} = \sqrt{2}\text{cis}\left(\frac{7\pi}{6}\right) = -\frac{\sqrt{6}}{2}-\frac{\sqrt{2}}{2} i$

\end{multicols}

\item Since $z=\frac{5}{2} - \frac{5\sqrt{3}}{2}i = 5\text{cis}\left(\frac{5\pi}{3}\right)$ we have 

\begin{multicols}{2}

$w_{\text{\tiny$0$}} =\sqrt{5}\text{cis}\left(\frac{5\pi}{6}\right) = -\frac{\sqrt{15}}{2} + \frac{\sqrt{5}}{2}i$

$w_{\text{\tiny$1$}} = \sqrt{5}\text{cis}\left(\frac{11\pi}{6}\right) = \frac{\sqrt{15}}{2} - \frac{\sqrt{5}}{2}i$

\end{multicols}

\item Since $z = 64 = 64\text{cis}\left(0\right)$ we have 

\begin{multicols}{3}

$w_{\text{\tiny$0$}} = 4\text{cis}\left(0\right) = 4$

$w_{\text{\tiny$1$}} =4\text{cis}\left(\frac{2\pi}{3}\right) = -2 + 2i\sqrt{3}$

$w_{\text{\tiny$2$}} = 4\text{cis}\left(\frac{4\pi}{3}\right) =  -2 - 2i\sqrt{3}$

\end{multicols}

\pagebreak

\item Since $z = -125 = 125\text{cis}\left(\pi\right)$ we have 

\begin{multicols}{3}

$w_{\text{\tiny$0$}} = 5\text{cis}\left(\frac{\pi}{3}\right) = \frac{5}{2} + \frac{5\sqrt{3}}{2} i$

$w_{\text{\tiny$1$}} =5\text{cis}\left(\pi\right) = -5$

$w_{\text{\tiny$2$}} = 5\text{cis}\left(\frac{5\pi}{3}\right) = \frac{5}{2} - \frac{5\sqrt{3}}{2} i$

\end{multicols}

\item Since $z = i = \text{cis}\left(\frac{\pi}{2}\right)$ we have 

\begin{multicols}{3}

$w_{\text{\tiny$0$}} = \text{cis}\left(\frac{\pi}{6}\right) = \frac{\sqrt{3}}{2} + \frac{1}{2}i$

$w_{\text{\tiny$1$}} = \text{cis}\left(\frac{5\pi}{6}\right) = -\frac{\sqrt{3}}{2} + \frac{1}{2}i$

$w_{\text{\tiny$2$}} = \text{cis}\left(\frac{3\pi}{2}\right) = -i$

\end{multicols}

\item Since $z = -8i = 8\text{cis}\left(\frac{3\pi}{2}\right)$ we have 

\begin{multicols}{3}

$w_{\text{\tiny$0$}} = 2\text{cis}\left(\frac{\pi}{2}\right) = 2i$

$w_{\text{\tiny$1$}} = 2\text{cis}\left(\frac{7\pi}{6}\right) = -\sqrt{3} -i$

$w_{\text{\tiny$2$}} = \text{cis}\left(\frac{11\pi}{6}\right) = \sqrt{3}-i$

\end{multicols}


\item Since $z=16 = 16\text{cis}\left(0 \right)$ we have 

\begin{multicols}{2}

$w_{\text{\tiny$0$}} =2\text{cis}\left(0\right) =2$

$w_{\text{\tiny$1$}} = 2\text{cis}\left(\frac{\pi}{2}\right) = 2i$

\end{multicols}

\begin{multicols}{2}

$w_{\text{\tiny$2$}} = 2\text{cis}\left(\pi\right) = -2$

$w_{\text{\tiny$3$}} = 2\text{cis}\left(\frac{3\pi}{2}\right) = -2i$

\end{multicols}


\item Since $z=-81 = 81\text{cis}\left(\pi \right)$ we have 

\begin{multicols}{2}

$w_{\text{\tiny$0$}} =3\text{cis}\left(\frac{\pi}{4}\right) = \frac{3\sqrt{2}}{2} + \frac{3\sqrt{2}}{2}i$

$w_{\text{\tiny$1$}} = 3\text{cis}\left(\frac{3\pi}{4}\right) =-\frac{3\sqrt{2}}{2} + \frac{3\sqrt{2}}{2}i$


\end{multicols}

\begin{multicols}{2}

$w_{\text{\tiny$2$}} = 3\text{cis}\left(\frac{5\pi}{4}\right) =-\frac{3\sqrt{2}}{2} - \frac{3\sqrt{2}}{2}i$

$w_{\text{\tiny$3$}} = 3\text{cis}\left(\frac{7\pi}{4}\right) =\frac{3\sqrt{2}}{2} - \frac{3\sqrt{2}}{2}i$

\end{multicols}





\item Since $z = 64 = 64\text{cis}(0)$ we have 


\begin{multicols}{3}

$w_{\text{\tiny$0$}} = 2\text{cis}(0) = 2$

$w_{\text{\tiny$1$}} = 2\text{cis}\left(\frac{\pi}{3}\right) = 1 + \sqrt{3}i$

$w_{\text{\tiny$2$}} = 2\text{cis}\left(\frac{2\pi}{3}\right) = -1 + \sqrt{3}i$

\end{multicols}

\begin{multicols}{3}

$w_{\text{\tiny$3$}} = 2\text{cis}\left(\pi\right) = -2$

$w_{\text{\tiny$4$}} = 2\text{cis}\left(-\frac{2\pi}{3}\right) = -1 - \sqrt{3}i$

$w_{\text{\tiny$5$}} = 2\text{cis}\left(-\frac{\pi}{3}\right) = 1 - \sqrt{3}i$

\end{multicols}


\item Since $z = -729 = 729 \text{cis}(\pi)$ we have 


\begin{multicols}{3}

$w_{\text{\tiny$0$}} = 3\text{cis}\left(\frac{\pi}{6}\right) = \frac{3\sqrt{3}}{2} + \frac{3}{2}i$

$w_{\text{\tiny$1$}} = 3\text{cis}\left(\frac{\pi}{2}\right) = 3i$

$w_{\text{\tiny$2$}} = 3\text{cis}\left(\frac{5\pi}{6}\right) = -\frac{3\sqrt{3}}{2} + \frac{3}{2}i$

\end{multicols}

\begin{multicols}{3}

$w_{\text{\tiny$3$}} = 3\text{cis}\left(\frac{7\pi}{6}\right) =  -\frac{3\sqrt{3}}{2}-\frac{3}{2}i$

$w_{\text{\tiny$4$}} = 3\text{cis}\left(-\frac{3\pi}{2}\right) = -3i$

$w_{\text{\tiny$5$}} = 3\text{cis}\left(-\frac{11\pi}{6}\right) = \frac{3\sqrt{3}}{2} - \frac{3}{2}i$

\end{multicols}


\item Note: In the answers for $w_{\text{\tiny$0$}}$ and $w_{\text{\tiny$2$}}$ the first rectangular form comes from applying the appropriate Sum or Difference Identity ($\frac{\pi}{12} = \frac{\pi}{3} - \frac{\pi}{4}$ and $\frac{17\pi}{12} = \frac{2\pi}{3} + \frac{3\pi}{4}$, respectively) and the second comes from using the Half-Angle Identities. 

$w_{\text{\tiny$0$}} = \sqrt[3]{2} \text{cis}\left(\frac{\pi}{12}\right) = \sqrt[3]{2}\left( \frac{\sqrt{6} + \sqrt{2}}{4} + i\left( \frac{\sqrt{6} - \sqrt{2}}{4} \right) \right) = \sqrt[3]{2}\left( \frac{\sqrt{2 + \sqrt{3}}}{2} + i\frac{\sqrt{2 - \sqrt{3}}}{2} \right)$ 
 
$w_{\text{\tiny$1$}} = \sqrt[3]{2} \text{cis}\left(\frac{3\pi}{4}\right) = \sqrt[3]{2} \left( -\frac{\sqrt{2}}{2} + \frac{\sqrt{2}}{2}i \right)$ 

$w_{\text{\tiny$2$}} = \sqrt[3]{2} \text{cis}\left(\frac{17\pi}{12}\right) = \sqrt[3]{2}\left( \frac{\sqrt{2} - \sqrt{6}}{4} + i\left( \frac{-\sqrt{2} - \sqrt{6}}{4} \right) \right) = \sqrt[3]{2}\left( \frac{\sqrt{2 - \sqrt{3}}}{2} + i\frac{\sqrt{2 + \sqrt{3}}}{2} \right)$

\item $w_{\text{\tiny$0$}} = \text{cis}(0) = 1$

$w_{\text{\tiny$1$}} = \text{cis}\left(\frac{2\pi}{5}\right) \approx 0.309 + 0.951i$

$w_{\text{\tiny$2$}} = \text{cis}\left(\frac{4\pi}{5}\right) \approx -0.809 + 0.588i$

$w_{\text{\tiny$3$}} = \text{cis}\left(\frac{6\pi}{5}\right) \approx -0.809 - 0.588i$

$w_{\text{\tiny$4$}} = \text{cis}\left(\frac{8\pi}{5}\right) \approx 0.309 - 0.951i$

\item $p(x) = x^{12} - 4096 = (x - 2)(x + 2)(x^{2} + 4)(x^{2} - 2x + 4)(x^{2} + 2x + 4)(x^{2} - 2\sqrt{3}x + 4)(x^{2} + 2\sqrt{3} + 4)$

\end{enumerate}


\end{document}
