\mfpicnumber{1}

\opengraphsfile{PolarConics}

\setcounter{footnote}{0}

\label{PolarConics}

In this section, we revisit our friends the Conic Sections which we began studying in Chapter \ref{TheConicSections}.  Our first task is to formalize the notion of rotating axes so this subsection is actually a follow-up to Example  \ref{rotationmatrixex} in Section \ref{MatArithmetic}. In that example, we saw that the graph of $y = \frac{2}{x}$ is actually a hyperbola.  

\smallskip

More specifically, the graph of $y = \frac{1}{x}$ is the hyperbola  obtained by rotating the graph of $x^2-y^2=4$ counter-clockwise through a $45^{\circ}$ angle.  Armed with polar coordinates, we can generalize the process of rotating axes as shown below.


\subsection{Rotation of Axes}
\label{rotationaxes}

Consider the $x$- and $y$-axes below along with the dashed $x'$- and $y'$-axes obtained by rotating the $x$- and $y$-axes counter-clockwise through an angle $\theta$ and consider the point $P(x,y)$.  The coordinates $(x,y)$ are rectangular coordinates and are based on the $x$- and $y$-axes.  

\smallskip

Suppose we wished to find rectangular coordinates based on the $x'$- and $y'$-axes.  That is, we wish to determine $P(x',y')$.  While this seems like a formidable challenge, it is nearly trivial if we use polar coordinates.  

\smallskip

Consider the angle $\phi$ whose initial side is the positive $x'$-axis and whose terminal side contains the point $P$.   We relate $P(x,y)$ and $P(x',y')$ by converting them to polar coordinates. 

\begin{center}
\begin{mfpic}[20]{-5}{5}{-5}{5}
\axes
\tlabel[cc](5,-0.5){\scriptsize $x$}
\tlabel[cc](0.5,5){\scriptsize $y$}
\tlabel[cc](3.75,3.5){\scriptsize $x'$}
\tlabel[cc](-3.5,3.75){\scriptsize $y'$}
\xmarks{-4,-3,-2,-1,1,2,3,4}
\ymarks{-4,-3,-2,-1,1,2,3,4}
\point[3pt]{(1,3)}
\gclear \tlabelrect[cc](1,3.5){\scriptsize $P(x,y) = P(x',y')$}
\dotted \polyline{(0,0), (1,3)}
\dashed \arrow \rotatepath{(0,0),45} \polyline{(-5,0), (5,0)}
\dashed \arrow \rotatepath{(0,0),45} \polyline{(0,-5), (0,5)}
\rotatepath{(0,0),45} \polyline{(2.83,-0.15),(2.83,0.15)}
\rotatepath{(0,0),135} \polyline{(1.41,-0.15),(1.41,0.15)}
\dotted \rotatepath{(0,0),45} \polyline{(2.83,0),(2.83,1.41)}
\dotted \rotatepath{(0,0),45} \polyline{(0,1.41),(2.83,1.41)}
\arrow \parafcn{5,40,5}{1.5*dir(t)}
\arrow \parafcn{95,130,5}{2*dir(t)}
\arrow \parafcn{50,69,5}{1.5*dir(t)}
\tlabel[cc](1.75,0.75){\scriptsize $\theta$}
\tlabel[cc](-.75,2.25){\scriptsize $\theta$}
\tlabel[cc](1,1.5){\scriptsize $\phi$}
\end{mfpic}

\end{center}

 Converting $P(x,y)$ to polar coordinates with $r > 0$ yields $x = r\cos(\theta + \phi)$ and $y = r\sin(\theta + \phi)$.  To convert the point $P(x',y')$ into polar coordinates, we first match the polar axis with the positive $x'$-axis, choose the same $r>0$ (since the origin is the same in both systems) and get $x' = r\cos(\phi)$ and $y' = r\sin(\phi)$.  
 
 \smallskip
 
 Using the sum formulas for sine and cosine, we have

\[ \begin{array}{rcll}

x & = & r\cos(\theta + \phi) & \\[3pt]
  & = & r\cos(\theta)\cos(\phi) - r\sin(\theta) \sin(\phi) & \text{Sum formula for cosine} \\[3pt]
  & = & (r\cos(\phi))\cos(\theta) - (r\sin(\phi))\sin(\theta) & \\[3pt]
  & = & x' \cos(\theta) - y' \sin(\theta) & \text{Since $x' = r\cos(\phi)$ and $y' = r\sin(\phi)$}\\ \end{array}\]

Similarly, using the sum formula for sine we get $y = x'\sin(\theta) + y'\cos(\theta)$.  These equations enable us to easily convert points with $x'y'$-coordinates back into $xy$-coordinates.  They also enable us to easily convert equations in the variables $x$ and $y$ into equations in the variables in terms of $x'$ and $y'$.\footnote{Just like in Section \ref{PolarCoordinates}, the equations $x = r\cos(\theta)$ and $y = r\sin(\theta)$ make it easy to convert \textit{points} from polar coordinates into rectangular coordinates, and they make it easy to convert \textit{equations} from rectangular coordinates into polar coordinates.}  

\smallskip

If we want equations which enable us to convert points with $xy$-coordinates into $x'y'$-coordinates, we need to solve the system

\vspace{-.05in}

\[ \left\{ \begin{array}{rcl} x' \cos(\theta) - y' \sin(\theta) & = & x \\ x'\sin(\theta) + y'\cos(\theta) & = & y \\ \end{array} \right.\]

for $x'$ and $y'$. Perhaps the cleanest way\footnote{We could, of course, interchange the roles of $x$ and $x'$, $y$ and $y'$ and replace $\phi$ with $-\phi$ to get $x'$ and $y'$ in terms of $x$ and $y$, but that seems like cheating.  The matrix $A$ introduced here is revisited in the Exercises.} to solve this system is to write it as a matrix equation. Using the machinery developed in Section \ref{MatMethods}, we write the above system as the matrix equation $AX' = X$ where

\vspace{-.05in}

\[ \begin{array}{ccc}

A = \left[ \begin{array}{rr} \cos(\theta) & -\sin(\theta) \\ \sin(\theta) & \cos(\theta) \\ \end{array} \right], 

&

X'= \left[ \begin{array}{c}x' \\ y'\\ \end{array} \right],

&

X= \left[ \begin{array}{c}x \\ y\\ \end{array} \right]


\end{array} \]

Since $\det(A) = (\cos(\theta))(\cos(\theta)) - (-\sin(\theta))(\sin(\theta)) = \cos^{2}(\theta) + \sin^{2}(\theta) = 1$, the determinant of $A$ is not zero so $A$ is invertible and  $X' = A^{-1}X$.  Using the  formula given in Equation \ref{2by2inverse} with $\det(A) = 1$, we find

\vspace{-.05in}

\[ A^{-1} = \left[ \begin{array}{rr} \cos(\theta) & \sin(\theta) \\ -\sin(\theta) & \cos(\theta) \\ \end{array} \right] \]

so that

\[ \begin{array}{rcl}

X'& = & A^{-1} X \\ [5pt]
\left[ \begin{array}{c}x' \\ y'\\ \end{array} \right] & = & \left[ \begin{array}{rr} \cos(\theta) & \sin(\theta) \\ -\sin(\theta) & \cos(\theta) \\ \end{array} \right]\left[ \begin{array}{c}x \\ y\\ \end{array} \right] \\ [15pt]

\left[ \begin{array}{c}x' \\ y'\\ \end{array} \right] & = & \left[ \begin{array}{c}x \cos(\theta) + y\sin(\theta) \\- x \sin(\theta) + y\cos(\theta)  \\ \end{array} \right] \\ \end{array} \]

From which we get $x' = x \cos(\theta) + y\sin(\theta)$ and $y'=- x \sin(\theta) + y\cos(\theta)$.   To summarize, 

\medskip

\colorbox{ResultColor}{\bbm  

\begin{thm} \label{rotatecoordinatesthm}  \textbf{Rotation of Axes:} \index{rotation of axes} Suppose the positive $x$ and $y$ axes are rotated counter-clockwise through an angle $\theta$ to produce the axes $x'$ and $y'$, respectively.  Then the coordinates $P(x,y)$ and $P(x',y')$ are related by the following systems of equations
\[ \begin{array}{ccc}

\left\{  \begin{array}{rcl}x & = & x' \cos(\theta) - y' \sin(\theta) \\ y& = &  x'\sin(\theta) + y'\cos(\theta)  \\ \end{array} \right. &

\text{and}

&

\left\{ \begin{array}{rcl} x' & = & x \cos(\theta) + y \sin(\theta) \\ y' & = &  -x\sin(\theta) +y\cos(\theta)  \\ \end{array} \right.  

\\ \end{array} \]

\end{thm}

\ebm}

\medskip

We put the formulas in  Theorem \ref{rotatecoordinatesthm} to good use in the following example.

\begin{ex} \label{rotatedaxesex1}  Suppose the $x$- and $y$- axes are both rotated counter-clockwise through the angle $\theta = \frac{\pi}{3}$ to produce the $x'$- and $y'$- axes, respectively. 


\begin{enumerate}

\item  Let $P(x,y) = (2,-4)$ and find $P(x',y')$.  Check your answer algebraically and graphically.

\item  \label{rotatedellipseex} Convert the equation $21x^2+10xy\sqrt{3}+31y^2=144$ to an equation in $x'$ and $y'$ and graph.

\end{enumerate}

{\bf Solution.}

\begin{enumerate}

 \item  If $P(x,y) = (2,-4)$ then $x=2$ and $y=-4$.  Using these values for $x$ and $y$ along with  $\theta = \frac{\pi}{3}$, Theorem \ref{rotatecoordinatesthm} gives
$x' = x \cos(\theta) + y \sin(\theta) = 2 \cos\left(\frac{\pi}{3}\right) + (-4)\sin\left(\frac{\pi}{3}\right)$ which simplifies to $x' = 1-2\sqrt{3}$.  

\smallskip

Similarly,  $y' = -x\sin(\theta) + y\cos(\theta) = (-2)\sin\left(\frac{\pi}{3}\right) + (-4)\cos\left(\frac{\pi}{3}\right)$ which gives $y' = -\sqrt{3}-2 = -2-\sqrt{3}$. Hence $P(x',y') = \left(1-2\sqrt{3}, -2-\sqrt{3}\right)$.  

\smallskip

To check our answer algebraically,  we convert $P(x',y') = \left(1-2\sqrt{3},-2-\sqrt{3}\right)$ back into $x$ and $y$ coordinates using the formulas in   Theorem \ref{rotatecoordinatesthm}.  We get

\[ \begin{array}{rcl}
x  & = &  x' \cos(\theta) - y' \sin(\theta) \\ [3pt]
   & = & (1-2\sqrt{3}) \cos\left(\frac{\pi}{3}\right) - (-2-\sqrt{3})\sin\left(\frac{\pi}{3}\right) \\ [3pt]
   & = & \left(\frac{1}{2} - \sqrt{3} \right) -\left(-\sqrt{3} - \frac{3}{2}\right)\\ [3pt]
   & = & 2 \end{array} \] 
   
Similarly, using  $y =  x'\sin(\theta) + y'\cos(\theta)$, we obtain $y = -4$ as required.  

\smallskip

To check our answer graphically, we sketch in the $x'$-axis and $y'$-axis to see if the new coordinates   $P(x',y') = \left(1-2\sqrt{3},-2-\sqrt{3}\right) \approx (-2.46,-3.73)$ seem reasonable.  Our graph is below.


\begin{center}
\begin{mfpic}[18]{-5}{5}{-5}{5}
\axes
\tlabel[cc](5,-0.5){\scriptsize $x$}
\tlabel[cc](0.5,5){\scriptsize $y$}
\tlabel[cc](2.75,4){\scriptsize $x'$}
\tlabel[cc](-4,2.75){\scriptsize $y'$}
\xmarks{-4,-3,-2,-1,1,2,3,4}
\ymarks{-4,-3,-2,-1,1,2,3,4}
\point[3pt]{(2,-4)}
\gclear \tlabelrect[cc](2,-4.5){\scriptsize $P(x,y) = (2,-4)$}
\gclear \tlabelrect[cc](2,-5){\scriptsize  $P(x',y') \approx (-2.46,-3.73)$}
\dashed \arrow \rotatepath{(0,0),60} \polyline{(-5,0), (5,0)}
\dashed \arrow \rotatepath{(0,0),60} \polyline{(0,-5), (0,5)}
\dotted \rotatepath{(0,0),60} \polyline{(-2.46,0), (-2.46,-3.73)}
\dotted \rotatepath{(0,0),60} \polyline{(0,-3.73), (-2.46,-3.73)}
\rotatepath{(0,0),240} \polyline{(1,-0.15),(1,0.15)}
\rotatepath{(0,0),240} \polyline{(2,-0.15),(2,0.15)}
\rotatepath{(0,0),240} \polyline{(3,-0.15),(3,0.15)}
\rotatepath{(0,0),240} \polyline{(4,-0.15),(4,0.15)}
\rotatepath{(0,0),330} \polyline{(1,-0.15),(1,0.15)}
\rotatepath{(0,0),330} \polyline{(2,-0.15),(2,0.15)}
\rotatepath{(0,0),330} \polyline{(3,-0.15),(3,0.15)}
\rotatepath{(0,0),330} \polyline{(4,-0.15),(4,0.15)}
\arrow \parafcn{5,55,5}{1.5*dir(t)}
\arrow \parafcn{95,145,5}{1.5*dir(t)}
\tlabel[cc](1.75,0.75){\scriptsize $\frac{\pi}{3}$}
\tlabel[cc](-.75,1.75){\scriptsize $\frac{\pi}{3}$}
\end{mfpic}

\end{center}

\item   To convert the equation $21x^2+10xy\sqrt{3}+31y^2=144$ to an equation in the variables $x'$ and $y'$, we substitute  $x = x' \cos\left(\frac{\pi}{3}\right) - y' \sin\left(\frac{\pi}{3}\right) = \frac{x'}{2} - \frac{y'\sqrt{3}}{2}$ and  $y =  x'\sin\left(\frac{\pi}{3}\right) + y'\cos\left(\frac{\pi}{3}\right) = \frac{x'\sqrt{3}}{2} + \frac{y'}{2}$.

\smallskip

 While this is by no means a trivial task, it is nothing more than a hefty dose of Intermediate Algebra. While we leave most of the details to the reader, a good starting point is to verify: 

\[ x^2 = \frac{(x')^2}{4} -\frac{x'y' \sqrt{3}}{2} +\frac{3(y')^2}{4}, \quad xy = \frac{(x')^2 \sqrt{3}}{4} -\frac{x'y'}{2} -\frac{(y')^2 \sqrt{3}}{4}, \quad y^2 = \frac{3(x')^2}{4} +\frac{x'y'\sqrt{3}}{2} + \frac{(y')^2}{4} \]

To our surprise and delight, the equation $21x^2+10xy\sqrt{3}+31y^2=144$ in $xy$-coordinates reduces to $36(x')^2 + 16(y')^2 = 144$, or $\frac{(x')^2}{4} + \frac{(y')^2}{9} = 1$ in $x'y'$-coordinates.  

\smallskip

That is, the curve is an ellipse centered at $(0,0)$ with vertices along the $y'$-axis with ($x'y'$-coordinates) $(0, \pm 3)$ and whose minor axis has endpoints with ($x'y'$-coordinates) $(\pm 2, 0)$ as seen below.


\begin{center}
\begin{mfpic}[18]{-5}{5}{-5}{5}
\axes
\tlabel[cc](5,-0.5){\scriptsize $x$}
\tlabel[cc](0.5,5){\scriptsize $y$}
\tlabel[cc](2.75,4){\scriptsize $x'$}
\tlabel[cc](-4,2.75){\scriptsize $y'$}
\xmarks{-4,-3,-2,-1,1,2,3,4}
\ymarks{-4,-3,-2,-1,1,2,3,4}
\point[4pt]{\plr{(2,60)}, \plr{(-2,60)}, \plr{(3,150)}, \plr{(-3,150)}}
\dashed \arrow \rotatepath{(0,0),60} \polyline{(-5,0), (5,0)}
\dashed \arrow \rotatepath{(0,0),60} \polyline{(0,-5), (0,5)}
\rotatepath{(0,0),60} \polyline{(1,-0.15),(1,0.15)}
\rotatepath{(0,0),60} \polyline{(2,-0.15),(2,0.15)}
\rotatepath{(0,0),60} \polyline{(3,-0.15),(3,0.15)}
\rotatepath{(0,0),60} \polyline{(4,-0.15),(4,0.15)}
\rotatepath{(0,0),150} \polyline{(1,-0.15),(1,0.15)}
\rotatepath{(0,0),150} \polyline{(2,-0.15),(2,0.15)}
\rotatepath{(0,0),150} \polyline{(3,-0.15),(3,0.15)}
\rotatepath{(0,0),150} \polyline{(4,-0.15),(4,0.15)}
\rotatepath{(0,0),240} \polyline{(1,-0.15),(1,0.15)}
\rotatepath{(0,0),240} \polyline{(2,-0.15),(2,0.15)}
\rotatepath{(0,0),240} \polyline{(3,-0.15),(3,0.15)}
\rotatepath{(0,0),240} \polyline{(4,-0.15),(4,0.15)}
\rotatepath{(0,0),330} \polyline{(1,-0.15),(1,0.15)}
\rotatepath{(0,0),330} \polyline{(2,-0.15),(2,0.15)}
\rotatepath{(0,0),330} \polyline{(3,-0.15),(3,0.15)}
\rotatepath{(0,0),330} \polyline{(4,-0.15),(4,0.15)}
\arrow \parafcn{5,55,5}{4*dir(t)}
\arrow \parafcn{95,145,5}{4*dir(t)}
\penwd{1.25pt}
\rotatepath{(0,0),60} \ellipse{(0,0), 2, 3}
\tlabel[cc](4,1.75){\scriptsize $\frac{\pi}{3}$}
\tlabel[cc](-1.75,4){\scriptsize $\frac{\pi}{3}$}
\tcaption{$21x^2+10xy\sqrt{3}+31y^2=144$}
\end{mfpic}

\end{center}

\end{enumerate}

\qed

\end{ex}

Thanks to the elimination of  the $xy$' term from the equation $21x^2+10xy\sqrt{3}+31y^2=144$ in  Example \ref{rotatedaxesex1} number \ref{rotatedellipseex}, we   were able to graph the equation on the $x'y'$-plane using what we know from Chapter \ref{TheConicSections}. 

\smallskip
	
 It is natural to wonder if, given an equation of the form $Ax^2 + Bxy +Cy^2 + Dx + Ey + F = 0$, with $B \neq 0$, is there an angle $\theta$ so that if we rotate the $x$ and $y$-axes counter-clockwise through that angle $\theta$, the equation in the rotated  variables $x'$ and $y'$ contains no $x'y'$ term.
 
 \smallskip
 
  To find out, we make the usual substitutions  $x = x' \cos(\theta) - y' \sin(\theta)$ and  $y =  x'\sin(\theta) + y'\cos(\theta)$ into the equation $Ax^2 + Bxy +Cy^2 + Dx + Ey + F = 0$ and set the coefficient of the $x'y'$ term equal to $0$.  
  
  \smallskip
  
  Terms containing $x'y'$ in this expression will come from the first three terms of the equation: $Ax^2$, $Bxy$ and $Cy^2$.  We leave it to the reader to verify that 

\[ \begin{array}{rcl}



x^2 & = &(x')^2 \cos^{2}(\theta) - 2x'y'\cos(\theta) \sin(\theta) + (y')^2 \sin(\theta) \\ [3pt]


xy & = &  (x')^2\cos(\theta) \sin(\theta)+ x'y' \left(\cos^{2}(\theta)-\sin^2(\theta)\right)-(y')^2\cos(\theta)\sin(\theta) \\ [3pt]


y^2 & = & (x')^2 \sin^{2}(\theta) +2x'y' \cos(\theta) \sin(\theta) +(y')^2 \cos^{2}(\theta) \\

\end{array} \]

The contribution to the $x'y'$-term from $Ax^2$ is $-2A\cos(\theta) \sin(\theta)$, from $Bxy$ it is $B \left(\cos^{2}(\theta)-\sin^2(\theta)\right)$, and from $Cy^2$ it is $2C \cos(\theta) \sin(\theta)$.  Equating the $x'y'$-term to $0$, we get

\[ \begin{array}{rcll}

-2A\cos(\theta) \sin(\theta) + B \left(\cos^{2}(\theta)-\sin^2(\theta)\right) + 2C \cos(\theta) \sin(\theta) & = & 0 & \\ [3pt]
-A \sin(2\theta) + B \cos(2\theta) + C \sin(2\theta) & = & 0 & \text{Double Angle Identities} \\ [3pt]
\end{array} \]

From this, we get $B \cos(2\theta) = (A-C)\sin(2\theta)$. Our goal is to solve for $\theta$ in terms of  $A$, $B$ and $C$. 

\smallskip

 Since we are assuming $B \neq 0$, we can divide both sides of this equation by $B$.  To solve for $\theta$ we would like to divide both sides of the equation by $\sin(2\theta)$, provided of course that  we have assurances that $\sin(2\theta) \neq 0$.  
 
 \smallskip
 
 If  $\sin(2\theta) = 0$, then we would have  $B \cos(2\theta) = 0$, and since $B \neq 0$, this would force $\cos(2\theta) = 0$.  Since no angle $\theta$ can have both $\sin(2\theta) = 0$ and $\cos(2\theta) = 0$, we can safely assume\footnote{The reader is invited to think about the case $\sin(2\theta) = 0$ geometrically.  What happens to the axes in this case?} $\sin(2\theta) \neq 0$.  
 
 \smallskip
 
Hence, we get $\frac{\cos(2\theta)}{\sin(2\theta)} = \frac{A-C}{B}$, or $\cot(2\theta) = \frac{A-C}{B}$.  We have just proved the following theorem.

 \smallskip
\colorbox{ResultColor}{\bbm

\begin{thm} \label{rotatedconicthm}  The equation  $Ax^2 + Bxy +Cy^2 + Dx + Ey + F = 0$ with $B \neq 0$ can be transformed into an equation in variables $x'$ and $y'$ without any $x'y'$ terms by rotating the $x$- and $y$- axes counter-clockwise through an angle $\theta$ which satisfies $\cot(2\theta) = \frac{A-C}{B}$.

\end{thm}

\ebm}
 \smallskip

We put Theorem \ref{rotatedconicthm} to good use in the following example.

\begin{ex} \label{graphrotatedconicex} Graph the following equations.

\begin{enumerate}

\item \label{rotatedhyperbolaex}  $5x^2+26xy+5y^2-16x\sqrt{2}+16y\sqrt{2}-104 = 0$

\item \label{rotatedparabolaex} $16x^2+24xy+9y^2 +15x-20y = 0$

\end{enumerate}

{\bf Solution.}  

\begin{enumerate}

\item Since the equation $5x^2+26xy+5y^2-16x\sqrt{2}+16y\sqrt{2}-104 = 0$ is already given to us in the form required by Theorem \ref{rotatedconicthm}, we identify $A = 5$, $B = 26$ and $C = 5$ so that $\cot(2\theta) = \frac{A-C}{B} = \frac{5-5}{26} = 0$.  

\smallskip

This means $\cot(2\theta) = 0$ which gives $\theta = \frac{\pi}{4} + \frac{\pi}{2} k$ for integers $k$.  We choose $\theta = \frac{\pi}{4}$ so that our rotation equations are $x = \frac{x' \sqrt{2}}{2} -\frac{y' \sqrt{2}}{2}$ and $y = \frac{x' \sqrt{2}}{2} + \frac{y' \sqrt{2}}{2}$.  The reader should verify that 

\vspace{-.1in}

\[ x^2 = \frac{(x')^2}{2} - x'y' +\frac{(y')^2}{2}, \quad xy = \frac{(x')^2}{2} -\frac{(y')^2}{2}, \quad y^2 = \frac{(x')^2}{2} + x'y' +\frac{(y')^2}{2} \]

Making the other substitutions, we get that $5x^2+26xy+5y^2-16x\sqrt{2}+16y\sqrt{2}-104 = 0$ reduces to $18(x')^2-8(y')^2+32y'-104 = 0$, or $\frac{(x')^2}{4} -\frac{(y'-2)^2}{9} = 1$.  

\smallskip

Hence, we have a hyperbola centered at the $x'y'$-coordinates $(0,2)$ opening in the $x'$ direction with vertices $(\pm 2,2)$ (in $x'y'$-coordinates)  and asymptotes $y' = \pm \frac{3}{2} x' + 2$. We graph this equation below. 

\begin{center}

\begin{mfpic}[18]{-5}{5}{-4}{5}
\axes
\tlabel[cc](5,-0.5){\scriptsize $x$}
\tlabel[cc](0.5,5){\scriptsize $y$}
\tlabel[cc](3.75,3.25){\scriptsize $x'$}
\tlabel[cc](-3.25,3.75){\scriptsize $y'$}
\xmarks{-4,-3,-2,-1,1,2,3,4}
\ymarks{-3,-2,-1,1,2,3,4}
\dashed \arrow \rotatepath{(0,0),45} \polyline{(-5,0), (5,0)}
\dashed \arrow \rotatepath{(0,0),45} \polyline{(0,-5), (0,5)}
\dotted \rotatepath{(0,0),45} \function{-2,2,0.1}{1.5*x+2}
\dotted \rotatepath{(0,0),45} \function{-2,2,0.1}{2-1.5*x}
\dotted \rotatepath{(0,0),45} \function{-2,2,0.1}{2}
\rotatepath{(0,0),45} \polyline{(1,-0.15),(1,0.15)}
\rotatepath{(0,0),45} \polyline{(2,-0.15),(2,0.15)}
\rotatepath{(0,0),45} \polyline{(3,-0.15),(3,0.15)}
\rotatepath{(0,0),45} \polyline{(4,-0.15),(4,0.15)}
\rotatepath{(0,0),135} \polyline{(1,-0.15),(1,0.15)}
\rotatepath{(0,0),135} \polyline{(2,-0.15),(2,0.15)}
\rotatepath{(0,0),135} \polyline{(3,-0.15),(3,0.15)}
\rotatepath{(0,0),135} \polyline{(4,-0.15),(4,0.15)}
\rotatepath{(0,0),225} \polyline{(1,-0.15),(1,0.15)}
\rotatepath{(0,0),225} \polyline{(2,-0.15),(2,0.15)}
\rotatepath{(0,0),225} \polyline{(3,-0.15),(3,0.15)}
\rotatepath{(0,0),225} \polyline{(4,-0.15),(4,0.15)}
\rotatepath{(0,0),315} \polyline{(1,-0.15),(1,0.15)}
\rotatepath{(0,0),315} \polyline{(2,-0.15),(2,0.15)}
\rotatepath{(0,0),315} \polyline{(3,-0.15),(3,0.15)}
\rotatepath{(0,0),315} \polyline{(4,-0.15),(4,0.15)}
\point[3pt]{(0,2.83), (-2.83,0)}
\arrow \parafcn{5,40,5}{1.25*dir(t)}
\tlabel[cc](2.35,0.4){\scriptsize $\theta = \frac{\pi}{4}$}
\tcaption{$5x^2+26xy+5y^2-16x\sqrt{2}+16y\sqrt{2}-104 = 0$}
\penwd{1.25pt}
\arrow \reverse \arrow \rotatepath{(0,0),45} \parafcn{-0.78,0.78,0.1}{(2*sec(t), 2+3*tan(t))}
\arrow \reverse \arrow \rotatepath{(0,0),45} \parafcn{-0.78,0.78,0.1}{(0-2*sec(t), 2+3*tan(t))}
\end{mfpic}


\end{center}

\item  From $16x^2+24xy+9y^2 +15x-20y = 0$, we get $A = 16$, $B=24$ and $C = 9$ so that $\cot(2\theta) = \frac{7}{24}$.  Since this isn't one of the values of the common angles, we will need to use inverse functions. 

\smallskip

Ultimately, we need to find $\cos(\theta)$ and $\sin(\theta)$, which means we have two options. If we use the arccotangent function immediately, after the usual calculations we get $\theta = \frac{1}{2} \text{arccot}\left(\frac{7}{24}\right)$.  To get $\cos(\theta)$ and $\sin(\theta)$ from this, we would need to use half angle identities.  

\smallskip

Alternatively, we can start with $\cot(2\theta) = \frac{7}{24}$, use a double angle identity, and then go after $\cos(\theta)$ and $\sin(\theta)$.  We adopt the second approach. 

\smallskip

From $\cot(2\theta) = \frac{7}{24}$, we have $\tan(2\theta) = \frac{24}{7}$.  Using the double angle identity for tangent, we have $\frac{2\tan(\theta)}{1-\tan^{2}(\theta)} = \frac{24}{7}$, which gives $24 \tan^{2}(\theta) + 14 \tan(\theta) - 24=0$.  

\smallskip

Factoring, we get $2(3\tan(\theta)+4)(4\tan(\theta)-3) = 0$ which gives $\tan(\theta) = -\frac{4}{3}$ or $\tan(\theta) = \frac{3}{4}$.  While either of these values of $\tan(\theta)$ satisfies the equation $\cot(2\theta) = \frac{7}{24}$, we choose $\tan(\theta) = \frac{3}{4}$, since this produces an acute angle,\footnote{As usual, there are infinitely many solutions to $\tan(\theta) = \frac{3}{4}$.  We choose the acute angle  $\theta=\arctan\left(\frac{3}{4}\right)$. The reader is encouraged to think about why there is always at least one acute answer to $\cot(2\theta) = \frac{A-C}{B}$ and what this means geometrically in terms of what we are trying to accomplish by rotating the axes.  The reader is also encouraged to keep a sharp lookout for the angles which satisfy $\tan(\theta) = -\frac{4}{3}$ in our final graph.  (Hint:  $\left(\frac{3}{4}\right) \left(-\frac{4}{3}\right) = -1$.)}  $\theta = \arctan\left(\frac{3}{4}\right)$.  

\smallskip

To find the rotation equations, we need $\cos(\theta) = \cos\left(\arctan\left(\frac{3}{4}\right)\right)$ and $\sin(\theta) = \sin\left(\arctan\left(\frac{3}{4}\right)\right)$. Using the techniques developed in Section \ref{TheInverseTrigonometricFunctions} we get $\cos(\theta) = \frac{4}{5}$ and $\sin(\theta) = \frac{3}{5}$. 

\smallskip

Our rotation equations are $x = x' \cos(\theta) - y' \sin(\theta) = \frac{4x'}{5} - \frac{3y'}{5}$ and $y = x' \sin(\theta) + y'\cos(\theta) = \frac{3x'}{5} + \frac{4y'}{5}$. 

\smallskip

 As usual, we now substitute these quantities into $16x^2+24xy+9y^2 +15x-20y = 0$ and simplify.  As a first step, the reader can verify


\[ x^2 = \frac{16(x')^2}{25} - \frac{24 x'y'}{25} +\frac{9(y')^2}{25}, \quad xy = \frac{12(x')^2}{25} +\frac{7 x'y'}{25} - \frac{12(y')^2}{25}, \quad y^2 = \frac{9(x')^2}{25} + \frac{24x'y'}{25} +\frac{16(y')^2}{25} \]

Once the dust settles, we get $25(x')^2 - 25y' = 0$, or $y' = (x')^2$, whose graph is a parabola opening along the positive $y'$-axis with vertex $(0,0)$. We graph this equation below. 

\end{enumerate}
\begin{center}


\begin{mfpic}[18]{-4}{5}{-5}{5}
\axes
\tlabel[cc](5,-0.5){\scriptsize $x$}
\tlabel[cc](0.5,5){\scriptsize $y$}
\tlabel[cc](4.25,2.35){\scriptsize $x'$}
\tlabel[cc](-2.5,4.5){\scriptsize $y'$}
\xmarks{-4,-3,-2,-1,1,2,3,4}
\ymarks{-3,-2,-1,1,2,3,4}
\dashed \arrow \rotatepath{(0,0),36.87} \polyline{(-5,0), (5,0)}
\dashed \arrow \rotatepath{(0,0),36.87} \polyline{(0,-5), (0,5)}
\rotatepath{(0,0),36.87} \polyline{(1,-0.15),(1,0.15)}
\rotatepath{(0,0),36.87} \polyline{(2,-0.15),(2,0.15)}
\rotatepath{(0,0),36.87} \polyline{(3,-0.15),(3,0.15)}
\rotatepath{(0,0),36.87} \polyline{(4,-0.15),(4,0.15)}
\rotatepath{(0,0),126.87} \polyline{(1,-0.15),(1,0.15)}
\rotatepath{(0,0),126.87} \polyline{(2,-0.15),(2,0.15)}
\rotatepath{(0,0),126.87} \polyline{(3,-0.15),(3,0.15)}
\rotatepath{(0,0),126.87} \polyline{(4,-0.15),(4,0.15)}
\rotatepath{(0,0),216.87} \polyline{(1,-0.15),(1,0.15)}
\rotatepath{(0,0),216.87} \polyline{(2,-0.15),(2,0.15)}
\rotatepath{(0,0),216.87} \polyline{(3,-0.15),(3,0.15)}
\rotatepath{(0,0),216.87} \polyline{(4,-0.15),(4,0.15)}
\rotatepath{(0,0),306.87} \polyline{(1,-0.15),(1,0.15)}
\rotatepath{(0,0),306.87} \polyline{(2,-0.15),(2,0.15)}
\rotatepath{(0,0),306.87} \polyline{(3,-0.15),(3,0.15)}
\rotatepath{(0,0),306.87} \polyline{(4,-0.15),(4,0.15)}
\point[3pt]{(0,0)}
\arrow \parafcn{5,30,5}{2*dir(t)}
\tlabel[cc](3.75,0.5){\scriptsize $\theta = \arctan\left(\frac{3}{4}\right)$}
\tcaption{$16x^2+24xy+9y^2 +15x-20y = 0$}
\penwd{1.25pt}
\arrow \reverse \arrow \rotatepath{(0,0),36.87} \function{-2,2,0.1}{x**2}
\end{mfpic}

\end{center}

\vspace{-.15in} \qed


\end{ex}


Note  that even though the coefficients of $x^2$ and $y^2$ were both positive numbers in parts \ref{rotatedhyperbolaex} and  \ref{rotatedparabolaex} of Example  \ref{graphrotatedconicex}, the graph in part  \ref{rotatedhyperbolaex} turned out to be a hyperbola and the graph in part \ref{rotatedparabolaex} worked out to be a parabola.  

\smallskip

Whereas in Chapter \ref{TheConicSections}, we could easily pick out which conic section we were dealing with based on the presence (or absence) of quadratic terms and their coefficients, Example  \ref{graphrotatedconicex} demonstrates the situation is much more complicated when an  $xy$ term is present. 

\smallskip

 Nevertheless,  it is possible to determine which conic section we have by looking at a special, familiar  \textit{combination} of the coefficients of the quadratic terms.  We have the following theorem.

\smallskip
\colorbox{ResultColor}{\bbm
\begin{thm} \label{conicclassification} $~$

Suppose the equation $Ax^2 + Bxy + Cy^2 + Dx + Ey + F = 0$ describes a non-degenerate conic section.\footnote{Recall that this means its graph is either a circle, parabola, ellipse or hyperbola.  See page \pageref{degenerateconics}.}

\begin{itemize}

\item If $B^2 - 4AC > 0$ then the graph of the equation is a hyperbola.

\item If $B^2 - 4AC =0$ then the graph of the equation is a parabola.

\item If $B^2 - 4AC < 0$ then the graph of the equation is an ellipse or circle.

\end{itemize}

\end{thm}
\ebm}
\smallskip 

As you may expect, the quantity $B^2 - 4AC$ mentioned in Theorem \ref{conicclassification} is called the \textbf{discriminant}\index{discriminant ! of a conic} of the conic section.  While we will not attempt to explain the deep Mathematics which produces this `coincidence', we will at least work through the proof of Theorem \ref{conicclassification} mechanically to show that it is true.\footnote{We hope that someday you get to see \emph{why} this works the way it does.}  

\smallskip

First note that if the coefficient $B=0$ in the equation $Ax^2 + Bxy + Cy^2 + Dx + Ey + F = 0$, Theorem \ref{conicclassification} reduces to the result presented in Exercise \ref{conicsclassificationnoxytermex} in Section \ref{Hyperbolas}.

\smallskip

Hence,  we proceed under the assumption that $B \neq 0$.  We rotate the $xy$-axes counter-clockwise through an angle $\theta$ which satisfies $\cot(2\theta) = \frac{A-C}{B}$ to produce an equation with no $x'y'$-term in accordance with  Theorem \ref{rotatedconicthm}:  $A'(x')^2 +C(y')^2 + Dx' + Ey' + F'= 0$.  

\smallskip

In this form, we can invoke Exercise \ref{conicsclassificationnoxytermex} in Section \ref{Hyperbolas} once more using the product $A'C'$.  Our goal is to find the product $A'C'$ in terms of the coefficients $A$, $B$ and $C$ in the original equation.  

\smallskip

We substitute $x = x' \cos(\theta) - y' \sin(\theta)$  $y =  x'\sin(\theta) + y'\cos(\theta)$ into  $Ax^2 + Bxy + Cy^2 + Dx + Ey + F = 0$.  After gathering like terms, the coefficient $A'$ on $(x')^2$ and the coefficient $C'$ on $(y')^2$ are
\vspace{-.1in}
\enlargethispage{.2in}
\[ \begin{array}{rcll}

A' & = & A \cos^{2}(\theta) + B \cos(\theta) \sin(\theta) + C \sin^{2}(\theta) \\ [3pt]
C' & = & A \sin^{2}(\theta) - B \cos(\theta) \sin(\theta) + C \cos^{2}(\theta) \\
 
\end{array} \]

In order to make use of the condition $\cot(2\theta) = \frac{A-C}{B}$, we rewrite our formulas for $A'$ and $C'$ using the power reduction formulas.  After some regrouping, we get

\[ \begin{array}{rcll}

2A' & = &  \left[ (A+C) + (A-C)\cos(2\theta)\right]  + B\sin(2\theta)  \\ [3pt]
2C' & = &  \left[ (A+C)-(A-C)\cos(2\theta)\right] - B\sin(2\theta)  \\
 
\end{array} \]

Next, we try to make sense of the product \[(2A')(2C') = \left\{ \left[ (A+C) + (A-C)\cos(2\theta)\right]  + B\sin(2\theta)\right\} \left\{\left[ (A+C)-(A-C)\cos(2\theta)\right] - B\sin(2\theta)\right\}\]  We break this product into pieces.  First, we use the difference of squares to multiply the `first' quantities in each factor to get

\[ \begin{array}{rcl}
  \left[ (A+C) + (A-C)\cos(2\theta)\right] \left[ (A+C)-(A-C)\cos(2\theta)\right] & = &  (A+C)^2 - (A-C)^2 \cos^{2}(2\theta) \\
\end{array} \]

Next, we add the product of the `outer' and `inner' quantities in each factor to get

\[ \begin{array}{rcl}

- B\sin(2\theta)\left[ (A+C) + (A-C)\cos(2\theta)\right] &&\\
+ B\sin(2\theta)\left[ (A+C)-(A-C)\cos(2\theta)\right] & = & -2B(A-C)\cos(2\theta)\sin(2\theta)  \\

\end{array} \]

The product of the `last' quantity in each factor is $(B\sin(2\theta))(- B\sin(2\theta)) = -B^2\sin^2(2\theta)$.  

\smallskip

Putting all of this together yields  


\[ \begin{array}{rcl}

4A'C' & = & (A+C)^2 - (A-C)^2 \cos^{2}(2\theta) -2B(A-C)\cos(2\theta)\sin(2\theta) -B^2\sin^2(2\theta) \\

\end{array} \]

From $\cot(2\theta) = \frac{A-C}{B}$, we get $\frac{\cos(2\theta)}{\sin(2\theta)} = \frac{A-C}{B}$, or $(A-C)\sin(2\theta)=B\cos(2\theta)$. 

\smallskip

Using this substitution twice along with the Pythagorean Identity $\cos^{2}(2\theta) = 1 - \sin^{2}(2\theta)$ we get:

\[ \begin{array}{rcl}

4A'C' & = & (A+C)^2 - (A-C)^2 \cos^{2}(2\theta) -2B(A-C)\cos(2\theta)\sin(2\theta) -B^2\sin^2(2\theta) \\ [3pt]
      & = & (A+C)^2 - (A-C)^2 \left[ 1-\sin^{2}(2\theta)\right] -2B\cos(2\theta) B \cos(2\theta) -B^2\sin^2(2\theta) \\ [3pt]
      & = & (A+C)^2 - (A-C)^2  + (A-C)^2 \sin^{2}(2\theta) -2B^2\cos^{2}(2\theta) -B^2\sin^2(2\theta) \\ [3pt]   
      & = & (A+C)^2 - (A-C)^2  +\left[(A-C) \sin(2\theta)\right]^2 -2B^2\cos^{2}(2\theta) -B^2\sin^2(2\theta)  \\ [3pt]    
      & = & (A+C)^2 - (A-C)^2  +\left[B \cos(2\theta)\right]^2 -2B^2\cos^{2}(2\theta) -B^2\sin^2(2\theta)  \\ [3pt]      
      & = & (A+C)^2 - (A-C)^2  +B^2\cos^{2}(2\theta) -2B^2\cos^{2}(2\theta) -B^2\sin^2(2\theta)  \\ [3pt]   
      & = & (A+C)^2 - (A-C)^2  -B^2\cos^{2}(2\theta) -B^2\sin^2(2\theta)  \\ [3pt] 
		  & = & (A+C)^2 - (A-C)^2  -B^2\left[\cos^{2}(2\theta)+ \sin^2(2\theta)\right]  \\ [3pt]  
		  & = & (A+C)^2 - (A-C)^2  -B^2 (1) \\ [3pt] 
	  	& = & \left(A^2 + 2AC+C^2\right) - \left(A^2 - 2AC+C^2\right)  -B^2  \\ [3pt] 			    			
	  	& = & 4AC  -B^2  \\ [3pt] 		
\end{array} \]

Hence, $B^2 - 4AC = -4 A'C'$, so the quantity $B^2 - 4AC$ has the opposite sign of $A'C'$.  The result now follows by applying  Exercise \ref{conicsclassificationnoxytermex} in Section \ref{Hyperbolas}.

\begin{ex} \label{conicdiscex}  Use Theorem \ref{conicclassification} to classify the graphs of the following non-degenerate conics.

\begin{enumerate}

\item   $21x^2+10xy\sqrt{3}+31y^2=144$  

\item   $5x^2+26xy+5y^2-16x\sqrt{2}+16y\sqrt{2}-104 = 0$  

\item   $16x^2+24xy+9y^2 +15x-20y = 0$  

\end{enumerate}

{\bf Solution.} This is a straightforward application of Theorem \ref{conicclassification}.

 \begin{enumerate}
 
\item  We have $A = 21$, $B = 10\sqrt{3}$ and $C = 31$ so $B^2 - 4AC = (10\sqrt{3})^2 - 4(21)(31) = -2304 < 0$.  Theorem \ref{conicclassification} predicts the graph is an ellipse, which checks with our work from Example \ref{rotatedaxesex1} number \ref{rotatedellipseex}.
 
\item  Here,  $A = 5$, $B = 26$ and $C = 5$, so $B^2 - 4AC = 26^2 - 4(5)(5) = 576 > 0$. Theorem \ref{conicclassification} classifies the graph as a hyperbola, which matches our answer to Example \ref{graphrotatedconicex} number \ref{rotatedhyperbolaex}. 

\item  Finally,  we have $A = 16$, $B = 24$ and $C = 9$ which gives $24^2 - 4(16)(9) = 0$. Theorem \ref{conicclassification} tells us that the graph is a parabola,  matching our result from Example \ref{graphrotatedconicex} number \ref{rotatedparabolaex}. \qed

\end{enumerate}

\end{ex}


\subsection{The Polar Form of Conics}
\label{polarformofconics}

Here, we revisit the conic sections from a more unified perspective starting with a  `new' definition below.

\smallskip

\colorbox{ResultColor}{\bbm
\begin{defn} \label{focusdirectrixeccentrityconic} Given a fixed line $L$,  a point $F$  not on $L$, and a positive number $e$, a conic section is the set of all points $P$ such that

\[ \dfrac{\text{the distance from $P$ to $F$}}{\text{the distance from $P$ to $L$}} = e\]

The line $L$ is called the \textbf{directrix}\index{directrix ! of a conic section in polar form} of the conic section, the point $F$ is called a \textbf{focus}\index{focus ! of a conic section in polar form} of the conic section, and the constant $e$ is called the \textbf{eccentricity}\index{eccentricity} of the conic section.

\end{defn}
\ebm}
\smallskip 

We have seen the notions of focus and directrix before in the definition of a parabola, Definition \ref{paraboladefn}.   There, a parabola is defined as the set of points equidistant from the focus and directrix, giving an eccentricity $e = 1$ according to Definition \ref{focusdirectrixeccentrityconic}.  

\smallskip

We have also seen the concept of eccentricity before.  It was introduced for ellipses in Definition \ref{ellipseeccentricity} in Section \ref{Ellipses}, and later extended to hyperbolas in Exercise \ref{hyperbolaeccentricity} in Section \ref{Hyperbolas}.  There, $e$ was also defined as a ratio of distances, though in these cases the distances involved were measurements from the center to a focus and from the center to a vertex.  

\smallskip

One way to reconcile the `old' ideas of focus, directrix and eccentricity with the `new' ones presented in Definition \ref{focusdirectrixeccentrityconic} is to derive equations for the conic sections using Definition \ref{focusdirectrixeccentrityconic}  and compare these parameters with what we know from Chapter \ref{TheConicSections}. 

\smallskip

We begin by assuming the conic section has eccentricity $e$, a focus $F$ at the origin and that the  directrix is the vertical line $x = -d$ as in the figure below.  

\begin{center}
\begin{mfpic}[20]{-5}{5}{-1}{6}
\axes
\tlabel[cc](0.5,6){\scriptsize $y$}
\tlabel[cc](5,-0.5){\scriptsize $x$}
\tlabel[cc](3.75,4){\scriptsize $P(r,\theta)$}
\dashed \polyline{(3,4), (-4,4)}
\dashed \polyline{(0,0), (3,4)}
\plotsymbol[3pt]{Asterisk}{(0, 0)}
\point[3pt]{(3,4)}
\arrow \reverse \arrow \polyline{(-4,-0.5), (-4,6)}
\tlabel[cc](-4,-1){\scriptsize $x = -d$}
\arrow \reverse \arrow \polyline{(0.25,4.5), (2.75,4.5)}
\gclear \tlabelrect[cc](1.5,4.5){\scriptsize $r\cos(\theta)$}
\arrow \reverse \arrow \polyline{(-3.75,4.5), (-0.25,4.5)}
\gclear \tlabelrect[cc](-2,4.5){\scriptsize $d$}
\tlabel[cc](0.75,-0.5){\scriptsize $O = F$}
\arrow \parafcn{5,48,5}{1.5*dir(t)}
\tlabel[cc](1.75,0.75){\scriptsize $\theta$}
\tlabel[cc](1.5,2.5){\scriptsize $r$}
\end{mfpic}
\end{center}

Using a polar coordinate representation $P(r,\theta)$ for a point on the conic with $r > 0$, we get

\[ e =  \dfrac{\text{the distance from $P$ to $F$}}{\text{the distance from $P$ to $L$}} = \dfrac{r}{d+r\cos(\theta)} \]

so that $r = e(d+r\cos(\theta))$.  Solving this equation for $r$, yields 


\[ r = \dfrac{ed}{1-e\cos(\theta)}\]

At this point, we convert the equation $r = e(d+r\cos(\theta))$ back into a rectangular equation in the variables $x$ and $y$.  If $e > 0$, but $e\neq 1$, the usual conversion process outlined in Section \ref{PolarCoordinates} gives\footnote{Turn $r = e(d+r\cos(\theta))$ into $r = e(d + x)$ and square both sides to get $r^{2} = e^{2}(d + x)^{2}$.  Replace $r^{2}$ with $x^{2} + y^{2}$, expand $(d + x)^{2}$, combine like terms, complete the square on $x$ and clean things up.}

\[ \left( \frac{\left(1-e^2\right)^2}{e^2d^2}\right) \left(x - \dfrac{e^2 d}{1-e^2}\right)^2 + \left(\frac{1-e^2}{e^2d^2}\right) y^2 = 1\]


If $0 < e < 1$, then $0< 1-e^2 < 1$ and, hence, $(1-e^2)^2 < 1-e^2$.  We leave it to the reader to show that this means we have  the equation of an ellipse centered at  $\left(\frac{e^2 d}{1-e^2}, 0\right)$  with major axis along the $x$-axis. 

\smallskip

Using the notation from Section \ref{Ellipses}, we have $a^2 = \frac{e^2 d^2}{\left(1-e^2\right)^2}$ and $b^2 = \frac{e^2 d^2}{1-e^2}$, so the major axis has length $\frac{2ed}{1-e^2}$ and the minor axis has length $\frac{2ed}{\sqrt{1-e^2}}$.  

\smallskip

Moreover, we find that one focus is $(0,0)$ and working through the formula given in  Definition \ref{ellipseeccentricity} gives the eccentricity to be $e$, as required. 

\smallskip

 If $e > 1$, then $1 - e^2 < 0$ but $(1-e^2)^2 > 0$ so the equation generates a hyperbola with center $\left(\frac{e^2 d}{1-e^2}, 0\right)$ whose transverse axis lies along the $x$-axis.  
 
 \smallskip
 
 Since such hyperbolas have the form $\frac{(x-h)^2}{a^2} -\frac{y^2}{b^2} = 1$, we need to take the \textit{opposite} reciprocal of the coefficient of $y^2$ to find $b^2$. 
 
 \smallskip
 
Doing this, we obtain\footnote{ Since $1 - e^2 < 0$ here, we rewrite $\left(1-e^2\right)^2 = \left(e^2-1\right)^2$ to help simplify things later on.}  $a^2 = \frac{e^2 d^2}{\left(1-e^2\right)^2} = \frac{e^2d^2}{\left(e^2-1\right)^2}$ and  $b^2 = -\frac{e^2 d^2}{1-e^2} = \frac{e^2d^2}{e^2-1}$, so the transverse axis has length  $\frac{2ed}{e^2-1}$ and the conjugate axis has length $\frac{2ed}{\sqrt{e^2-1}}$.  
 
 \smallskip
 
 Additionally, we verify that one focus is at $(0,0)$, and the formula given in Exercise \ref{hyperbolaeccentricity} in Section \ref{Hyperbolas} gives  the eccentricity is $e$ in this case as well. 
 
 \smallskip
 
 If $e=1$, the equation $r = \frac{ed}{1-e\cos(\theta)}$ reduces to $r = \frac{d}{1-\cos(\theta)}$ which translates to $y^2 = 2d\left(x + \frac{d}{2}\right)$.   
 
 \smallskip
 
The equation $y^2 = 2d\left(x + \frac{d}{2}\right)$ describes a parabola with vertex $\left(-\frac{d}{2}, 0\right)$ opening to the right.  

\smallskip

In the language of Section \ref{Parabolas}, $4p = 2d$ so $p = \frac{d}{2}$, the focus is $(0,0)$, the focal diameter is $2d$ and the directrix is $x = -d$, as required. 

\smallskip

 Hence, we have shown that in all cases, our `new' understanding  of `conic section', `focus', `eccentricity' and `directrix' as presented in Definition \ref{focusdirectrixeccentrityconic} correspond with the `old' definitions given in Chapter \ref{TheConicSections}. 
 
\smallskip

Before we summarize our findings, we note that in order to arrive at our general equation of a conic $r = \frac{ed}{1-e\cos(\theta)}$, we assumed that the directrix was the line $x = -d$ for $d > 0$. 

\smallskip

 We could have just as easily chosen the directrix to be $x = d$, $y = -d$ or $y = d$.  As the reader can verify, in these cases we obtain the forms  $r = \frac{ed}{1+e\cos(\theta)}$,  $r = \frac{ed}{1-e\sin(\theta)}$ and  $r = \frac{ed}{1+e\sin(\theta)}$, respectively. 
 
 \smallskip
 
 The key thing to remember is that in any of these cases, the directrix is always perpendicular to the major axis of an ellipse and it is always perpendicular to the transverse axis of the hyperbola.
 
 \smallskip
 
  For parabolas, knowing the focus is $(0,0)$ and the directrix also tells us which way the parabola opens.  
 
 \smallskip
 
 We have established the following theorem.

\smallskip

\colorbox{ResultColor}{\bbm

\begin{thm}  \label{polarformofconicsthm}  Suppose $e$ and $d$ are positive numbers.  Then

\begin{itemize}

\item  the graph of $r = \dfrac{ed}{1 - e\cos(\theta)}$ is the graph of a conic section with directrix $x = -d$.

\item  the graph of $r = \dfrac{ed}{1 + e\cos(\theta)}$ is the graph of a conic section with directrix $x = d$.

\item  the graph of $r = \dfrac{ed}{1 - e\sin(\theta)}$ is the graph of a conic section with directrix $y = -d$.

\item  the graph of $r = \dfrac{ed}{1 + e\sin(\theta)}$ is the graph of a conic section with directrix $y = d$.


\end{itemize}

In each case above, $(0,0)$ is a focus of the conic and the number $e$ is the eccentricity of the conic. 

\begin{itemize}

\item If $0 < e < 1$, the graph is an ellipse.   The quantities $\dfrac{2ed}{1-e^2}$ and  $\dfrac{2ed}{\sqrt{1-e^2}}$ are the lengths of the major and minor axes, respectively.

\item If $e = 1$, the graph is a parabola whose focal diameter is $2d$.

\item If $e > 1$, the graph is a hyperbola.  The quantities    $\dfrac{2ed}{e^2-1}$ and $\dfrac{2ed}{\sqrt{e^2-1}}$ are the lengths of the transverse and conjugate axes, respectively.

\end{itemize} 

\smallskip

\end{thm}

\ebm}

\smallskip

We test out Theorem \ref{polarformofconicsthm} in the next example.

\smallskip

\begin{ex}  \label{polarconicgraphex}  Sketch the graphs of the following equations.


\begin{multicols}{3}
\begin{enumerate}

\item  \label{polarparabola} $r  = \dfrac{4}{1-\sin(\theta)}$

\item  $r = \dfrac{12}{3 - \cos(\theta)}$  

\item  $r = \dfrac{6}{1 + 2\sin(\theta)}$


\end{enumerate}
\end{multicols}



{\bf Solution.}  \begin{enumerate}  \item From $r  = \frac{4}{1-\sin(\theta)}$, we first note $e =1$ which means we have a parabola on our hands.  

\smallskip

Since $ed = 4$, we have $d=4$ and given the form of the equation, the directrix at $y = -4$.  

\smallskip

Since the focus is at $(0,0)$, we know that the vertex is located at the point (in rectangular coordinates) $(0,-2)$ and must open upwards.  

\smallskip

With $d=4$, we have a focal diameter of $2d = 8$, so the parabola contains the points $(\pm 4, 0)$.  

\smallskip

Putting all this together, we graph  $r  = \frac{4}{1-\sin(\theta)}$ below on the left.


\item  We first rewrite $r = \frac{12}{3 - \cos(\theta)}$ in the form found in Theorem \ref{polarformofconicsthm}, namely $r = \frac{4}{1 - (1/3) \cos(\theta)}$.  

\smallskip

Since $e = \frac{1}{3}$ satisfies $0 < e < 1$, we know that the graph of this equation is an ellipse.  

\smallskip

Since $ed= 4$, we have $d = 12$ and, based on the form of the equation, the directrix is $x = -12$.  

\smallskip

Hence,  the ellipse has its major axis along the $x$-axis, which means we can find the vertices of the ellipse by finding where the ellipse intersects the $x$-axis.   

\smallskip

Since  $r(0) = 6$ and $r(\pi) = 3$,  our vertices are the rectangular points  $(-3,0)$ and $(6,0)$.  

\smallskip

The center of the ellipse is the midpoint of the vertices,  which in this case is $\left(\frac{3}{2}, 0\right)$.\footnote{As a quick check, we have from Theorem \ref{polarformofconicsthm} the major axis should have length $\frac{2ed}{1-e^2} = \frac{(2)(4)}{1-(1/3)^2} = 9$.}  

\smallskip

We know one focus is $(0,0)$, which is $\frac{3}{2}$ from the center $\left(\frac{3}{2}, 0\right)$ and this allows us to find the other focus $(3, 0)$, even though we are not asked to do so.  

\smallskip

Finally, we know from Theorem \ref{polarformofconicsthm} that the length of the minor axis is $\frac{2ed}{\sqrt{1-e^2}} =$ {\scriptsize $\frac{4}{\sqrt{1-(1/3)^2}}$} $= 6\sqrt{3}$ which means the endpoints of the minor axis are $\left(\frac{3}{2}, \pm 3\sqrt{2}\right)$. 

\smallskip

We now have everything we need to graph $r = \frac{12}{3 - \cos(\theta)}$ below on the right.



\begin{center}

\begin{tabular}{cc}

\begin{mfpic}[13]{-5}{5}{-5}{4}
\axes
\xmarks{-4,-3,-2,-1,1,2,3,4}
\ymarks{-5,-4,-3,-2,-1,1,2,3}
\point[4pt]{(-4,0), (4,0), (0,-2)}
\plotsymbol[3pt]{Asterisk}{(0, 0)}
\arrow \reverse \arrow \polyline{(-5,-4), (5,-4)}
\gclear \tlabelrect(0,-5){\scriptsize $y = -4$}
\tlabelsep{5pt}
\scriptsize
\axislabels {x}{{$-4 \hspace{7pt}$} -4, {$-3 \hspace{7pt} $} -3, {$-2\hspace{7pt} $} -2, {$-1 \hspace{7pt}$} -1, {$1$} 1, {$2$} 2, {$3$} 3, {$4$} 4}
\axislabels {y}{{$-3$} -3, {$-2$} -2, {$-1$} -1, {$1$} 1, {$2$} 2, {$3$} 3}
\normalsize
\tcaption{$r = \frac{4}{1-\sin(\theta)}$}
\penwd{1.25pt}
\arrow \reverse \arrow \plrfcn{-200,20,5}{4/(1-sind(t))}
\end{mfpic} 

&

\begin{mfpic}[13]{-4}{7}{-6}{6}
\axes
\tlabel[cc](7,-0.5){\scriptsize $x$}
\tlabel[cc](0.5,6){\scriptsize $y$}
\dashed \polyline{(-6,0), (-4,0)}
\xmarks{-3,-2,-1,1,2,3,4,5,6}
\ymarks{-5,-4,-3,-2,-1,1,2,3,4,5}
\point[4pt]{(-3,0), (6,0), (1.5, 4.25), (1.5,-4.25)}
\plotsymbol[3pt]{Asterisk}{(0, 0), (3,0)}
\plotsymbol[3pt]{Cross}{(1.5,0)}
\arrow \reverse \arrow \polyline{(-6,-6), (-6,6)}
\gclear \tlabelrect(-6,-4){\scriptsize $x = -12$}
\tlabelsep{5pt}
\scriptsize
\axislabels {x}{ {$-3 \hspace{7pt} $} -3, {$-2\hspace{7pt} $} -2, {$-1 \hspace{7pt}$} -1, {$1$} 1, {$2$} 2, {$3$} 3, {$4$} 4, {$5$} 5, {$6$} 6}
\axislabels {y}{{$-4$} -4, {$-3$} -3, {$-2$} -2, {$-1$} -1, {$1$} 1, {$2$} 2, {$3$} 3, {$4$} 4}
\normalsize
\tcaption{$r = \frac{12}{3-\cos(\theta)}$}
\penwd{1.25pt}
\ellipse{(1.5,0),4.5,4.25}
\end{mfpic} \\

\end{tabular}
\end{center}



\item From $r = \frac{6}{1 + 2\sin(\theta)}$ we get $e = 2 > 1$ so the graph is a hyperbola.  

\smallskip

Since $ed = 6$, we get $d=3$, and from the form of the equation, we know the directrix is $y = 3$.  

\smallskip

Hence,  the transverse axis of the hyperbola lies along the $y$-axis, so we can find the vertices by looking where the hyperbola intersects the $y$-axis.  

\smallskip

We find $r\left( \frac{\pi}{2} \right) = 2$ and $r\left(\frac{3\pi}{2}\right) = -6$.  These two points correspond to the rectangular points $(0,2)$ and $(0,6)$ which puts the center of the hyperbola at $(0,4)$.  

\smallskip

Since one focus is at $(0,0)$,  $4$ units away from the center, we know the other focus is at $(0,8)$.  

\smallskip

According to Theorem \ref{polarformofconicsthm}, the conjugate axis has a length of $\frac{2ed}{\sqrt{e^2-1}} = \frac{(2)(6)}{\sqrt{2^2-1}} = 4 \sqrt{3}$.  This together with the location of the vertices give the slopes of the asymptotes as: $\pm \frac{2}{2\sqrt{3}}  = \pm \frac{\sqrt{3}}{3}$.  

\smallskip

Since the center of the hyperbola is $(0,4)$, the asymptotes are $y = \pm \frac{\sqrt{3}}{3} x + 4$.  

\smallskip

Using all of our work, we graph the hyperbola below.



\begin{center}

\begin{mfpic}[13]{-6}{6}{-1}{9}
\axes
\tlabel[cc](6,-0.5){\scriptsize $x$}
\tlabel[cc](0.5,9){\scriptsize $y$}
\dashed \function{-5,5,0.1}{0.577*x+4}
\dashed \function{-5,5,0.1}{4-0.577*x}
\xmarks{-5,-4,-3,-2,-1,1,2,3,4,5}
\ymarks{1,2,3,5,6,7,8}
\point[4pt]{(0,2), (0,6)}
\plotsymbol[3pt]{Asterisk}{(0, 0), (0,8)}
\plotsymbol[3pt]{Cross}{(0,4)}
\arrow \reverse \arrow \polyline{(-6,3), (6,3)}
\gclear \tlabelrect(0,3){\scriptsize $y = 3$}
\tlabelsep{5pt}
\scriptsize
\axislabels {x}{ {$-5 \hspace{7pt} $} -5,{$-4 \hspace{7pt} $} -4,{$-3 \hspace{7pt} $} -3, {$-2\hspace{7pt} $} -2, {$-1 \hspace{7pt}$} -1, {$1$} 1, {$2$} 2, {$3$} 3, {$4$} 4, {$5$} 5}
\axislabels {y}{{$1$} 1, {$2$} 2, {$4$} 4, {$5$} 5, {$6$} 6, {$7$} 7, {$8$} 8}
\normalsize
\tcaption{$r = \frac{6}{1 + 2\sin(\theta)}$}
\penwd{1.25pt}
\arrow \reverse \arrow \plrfcn{5,175,5}{6/(1+2*sind(t))}
\arrow \reverse \arrow \plrfcn{-125,-55,5}{6/(1+2*sind(t))}
\end{mfpic}

\end{center}

\vspace{-0.15in} \qed

\end{enumerate}

\end{ex}


In light of Section \ref{rotationaxes}, the reader may wonder what the rotated form of the conic sections would look like in polar form.  

\smallskip

We know from Exercise \ref{polargraphtransformations} in Section \ref{PolarGraphs} that replacing $\theta$ with $(\theta - \phi)$ in an expression $r = f(\theta)$ rotates the graph of $r = f(\theta)$ counter-clockwise by an angle $\phi$.  

\smallskip

For instance, to graph $r = \frac{4}{1-\sin\left(\theta - \frac{\pi}{4}\right)}$ all we need to do is rotate the graph of  $r = \frac{4}{1-\sin\left(\theta\right)}$, which we obtained in Example \ref{polarconicgraphex} number \ref{polarparabola}, counter-clockwise by $\frac{\pi}{4}$ radians, as shown below.

\begin{center}
\begin{mfpic}[12]{-5}{5}{-5}{4}
\axes
\xmarks{-4,-3,-2,-1,1,2,3,4}
\ymarks{-5,-4,-3,-2,-1,1,2,3}
\point[4pt]{\plr{(4,225)}, \plr{(4,45)}, \plr{(2,315)}}
\plotsymbol[3pt]{Asterisk}{(0, 0)}
\tlabelsep{5pt}
\scriptsize
\axislabels {x}{{$-4 \hspace{7pt}$} -4, {$-3 \hspace{7pt} $} -3, {$-2\hspace{7pt} $} -2, {$-1 \hspace{7pt}$} -1, {$1$} 1, {$2$} 2, {$3$} 3, {$4$} 4}
\axislabels {y}{{$-3$} -3, {$-2$} -2, {$-1$} -1, {$1$} 1, {$2$} 2, {$3$} 3}
\normalsize

\dashed \arrow \rotatepath{(0,0),45} \polyline{(-5,0), (5,0)}
\dashed \arrow \rotatepath{(0,0),45} \polyline{(0,-5), (0,5)}
\rotatepath{(0,0),45} \polyline{(1,-0.15),(1,0.15)}
\rotatepath{(0,0),45} \polyline{(2,-0.15),(2,0.15)}
\rotatepath{(0,0),45} \polyline{(3,-0.15),(3,0.15)}
\rotatepath{(0,0),45} \polyline{(4,-0.15),(4,0.15)}
\rotatepath{(0,0),135} \polyline{(1,-0.15),(1,0.15)}
\rotatepath{(0,0),135} \polyline{(2,-0.15),(2,0.15)}
\rotatepath{(0,0),135} \polyline{(3,-0.15),(3,0.15)}
\rotatepath{(0,0),135} \polyline{(4,-0.15),(4,0.15)}
\rotatepath{(0,0),225} \polyline{(1,-0.15),(1,0.15)}
\rotatepath{(0,0),225} \polyline{(2,-0.15),(2,0.15)}
\rotatepath{(0,0),225} \polyline{(3,-0.15),(3,0.15)}
\rotatepath{(0,0),225} \polyline{(4,-0.15),(4,0.15)}
\rotatepath{(0,0),315} \polyline{(1,-0.15),(1,0.15)}
\rotatepath{(0,0),315} \polyline{(2,-0.15),(2,0.15)}
\rotatepath{(0,0),315} \polyline{(3,-0.15),(3,0.15)}
\rotatepath{(0,0),315} \polyline{(4,-0.15),(4,0.15)}

\tcaption{$r = \frac{4}{1-\sin\left(\theta - \frac{\pi}{4}\right)}$}
\penwd{1.25pt}
\arrow \reverse \arrow \plrfcn{85,185,5}{0-4/sind(t-45)}
\arrow \reverse \arrow \plrfcn{-155,65,5}{4/(1-sind(t-45))}
\end{mfpic} 

\end{center}



Using rotations, we can greatly simplify the form of the conic sections presented in Theorem \ref{polarformofconicsthm}, since any three of the forms given there can be obtained from the fourth by rotating through some multiple of $\frac{\pi}{2}$.  

\smallskip

Moreover, since rotations do not affect lengths, all of the formulas for lengths Theorem \ref{polarformofconicsthm} remain intact. 

\smallskip

The formula in Theorem \ref{mostgeneralpolarformconic} below captures all the conic sections that have a focus at $(0,0)$.  It also includes circles centered at the origin by extending the concept of eccentricity to include $e=0$.  

\smallskip

While substituting $e=0$ into the equation given in Theorem \ref{mostgeneralpolarformconic} quickly reduces to a circle centered at the origin, the reader is best advised to think about this idea in light of Definition \ref{ellipseeccentricity} in Section \ref{Ellipses}.

\smallskip



\smallskip
\colorbox{ResultColor}{\bbm
\begin{thm} \label{mostgeneralpolarformconic}  Given constants $\ell > 0$, $e \geq 0$ and $\phi$, the graph of the equation \[ r = \dfrac{\ell}{1 - e\cos(\theta - \phi)}\] is a conic section with eccentricity $e$ and one focus at $(0,0)$.

\begin{itemize}

\item  If $e = 0$, the graph is a circle centered at $(0,0)$ with radius $\ell$.

\item  If $e \neq 0$, then the conic has a focus at $(0,0)$. 

\smallskip

 Defining $d = \dfrac{\ell}{e}$, the directrix contains the point with polar coordinates $(-d,\phi)$.

 \text{\tiny $\bullet$}  If $0 < e < 1$, the graph is an ellipse.   The quantities $\dfrac{2ed}{1-e^2}$ and  $\dfrac{2ed}{\sqrt{1-e^2}}$ are the lengths of the major and minor axes, respectively.

 \text{\tiny $\bullet$}  If $e = 1$, the graph is a parabola whose focal diameter is $2d$.

\text{\tiny $\bullet$}  If $e > 1$, the graph is a hyperbola.  The quantities    $\dfrac{2ed}{e^2-1}$ and $\dfrac{2ed}{\sqrt{e^2-1}}$ are the lengths of the transverse and conjugate axes, respectively.



\end{itemize} 

\smallskip

\end{thm}
\ebm}


\newpage

\subsection{Exercises}

\documentclass{ximera}

\begin{document}
	\author{Stitz-Zeager}
	\xmtitle{TITLE}


Graph the following equations.

\begin{multicols}{2}

\begin{enumerate}

\item  $x^2+2xy+y^2 -x\sqrt{2}+y\sqrt{2} -6= 0$

\item  $7x^2-4xy\sqrt{3}+3y^2-2x-2y\sqrt{3}-5= 0$
\setcounter{HW}{\value{enumi}}
\end{enumerate}
\end{multicols}


\begin{multicols}{2}

\begin{enumerate}
\setcounter{enumi}{\value{HW}}

\item  $5x^2+6xy+5y^2 - 4\sqrt{2}x+4\sqrt{2}y = 0$ 
\item  $x^2+ 2\sqrt{3}xy+3y^2+ 2\sqrt{3}x-2y-16 = 0$ 

\setcounter{HW}{\value{enumi}}
\end{enumerate}
\end{multicols}


\begin{multicols}{2}
\begin{enumerate}
\setcounter{enumi}{\value{HW}}



\item  $13x^2-34xy\sqrt{3}+47y^2 - 64=0$
\item  $x^2-2\sqrt{3} xy-y^2+8=0$
\setcounter{HW}{\value{enumi}}
\end{enumerate}
\end{multicols}






\begin{multicols}{2}
\begin{enumerate}
\setcounter{enumi}{\value{HW}}

\item  $x^2-4xy+4y^2-2x\sqrt{5}-y\sqrt{5}=0$

\item  $8x^2+12xy+17y^2 - 20 = 0$
\setcounter{HW}{\value{enumi}}
\end{enumerate}
\end{multicols}


Graph the following equations.

\begin{multicols}{2}

\begin{enumerate}
\setcounter{enumi}{\value{HW}}
\item  $r = \dfrac{2}{1-\cos(\theta)}$

\item  $r = \dfrac{3}{2 + \sin(\theta)}$
\setcounter{HW}{\value{enumi}}
\end{enumerate}
\end{multicols}

\begin{multicols}{2}

\begin{enumerate}
\setcounter{enumi}{\value{HW}}
\item  $r = \dfrac{3}{2-\cos(\theta)}$

\item  $r = \dfrac{2}{1 + \sin(\theta)}$
\setcounter{HW}{\value{enumi}}
\end{enumerate}
\end{multicols}

\begin{multicols}{2}

\begin{enumerate}
\setcounter{enumi}{\value{HW}}

\item   $r = \dfrac{4}{1+3\cos(\theta)}$

\item  $r = \dfrac{2}{1-2\sin(\theta)}$

\setcounter{HW}{\value{enumi}}
\end{enumerate}
\end{multicols}

\begin{multicols}{2}

\begin{enumerate}
\setcounter{enumi}{\value{HW}}
\item  $r = \dfrac{2}{1 + \sin(\theta - \frac{\pi}{3})}$

\item  $r = \dfrac{6}{3 - \cos\left(\theta + \frac{\pi}{4}\right)}$
\setcounter{HW}{\value{enumi}}
\end{enumerate}
\end{multicols}


 The matrix $A(\theta) = \left[ \begin{array}{rr} \cos(\theta) & -\sin(\theta) \\ \sin(\theta) & \cos(\theta) \\ \end{array} \right]$  is called a \textbf{rotation matrix}\index{matrix ! rotation}\index{rotation matrix}. 
 
 \smallskip
 
 We've seen this matrix most recently  used in the proof of Theorem \ref{rotatecoordinatesthm}.

\begin{enumerate}
\setcounter{enumi}{\value{HW}}
\item  Show the matrix from Example \ref{rotationmatrixex} in Section \ref{MatArithmetic} is none other than $A\left(\frac{\pi}{4}\right)$.

\item  Discuss with your classmates how to use $A(\theta)$ to rotate points in the plane.

\item  Using the even / odd identities for cosine and sine,  show $A(\theta)^{-1} = A(-\theta)$.  Interpret this geometrically.



\end{enumerate}

\newpage

\subsection{Answers}

\begin{multicols}{2}

\begin{enumerate}

\item  $x^2+2xy+y^2 -x\sqrt{2}+y\sqrt{2} -6= 0$ \\ becomes $(x')^2 = -(y'-3)$  after rotating \\ counter-clockwise through $\theta = \frac{\pi}{4}$. 

\begin{mfpic}[18]{-5}{5}{-5}{5}

\axes

\tlabel[cc](5,-0.5){\scriptsize $x$}
\tlabel[cc](0.5,5){\scriptsize $y$}
\tlabel[cc](4,3.5){\scriptsize $x'$}
\tlabel[cc](-3,4){\scriptsize $y'$}
\xmarks{-4,-3,-2,-1,1,2,3,4}
\ymarks{-4,-3,-2,-1,1,2,3,4}
\point[4pt]{\plr{(3,135)}, \plr{(1.73,45)}, \plr{(-1.73,45)}}
\dashed \arrow \rotatepath{(0,0),45} \polyline{(-5,0), (5,0)}
\dashed \arrow \rotatepath{(0,0),45} \polyline{(0,-5), (0,5)}
\rotatepath{(0,0),45} \polyline{(1,-0.15),(1,0.15)}
\rotatepath{(0,0),45} \polyline{(2,-0.15),(2,0.15)}
\rotatepath{(0,0),45} \polyline{(3,-0.15),(3,0.15)}
\rotatepath{(0,0),45} \polyline{(4,-0.15),(4,0.15)}
\rotatepath{(0,0),135} \polyline{(1,-0.15),(1,0.15)}
\rotatepath{(0,0),135} \polyline{(2,-0.15),(2,0.15)}
\rotatepath{(0,0),135} \polyline{(3,-0.15),(3,0.15)}
\rotatepath{(0,0),135} \polyline{(4,-0.15),(4,0.15)}
\rotatepath{(0,0),225} \polyline{(1,-0.15),(1,0.15)}
\rotatepath{(0,0),225} \polyline{(2,-0.15),(2,0.15)}
\rotatepath{(0,0),225} \polyline{(3,-0.15),(3,0.15)}
\rotatepath{(0,0),225} \polyline{(4,-0.15),(4,0.15)}
\rotatepath{(0,0),315} \polyline{(1,-0.15),(1,0.15)}
\rotatepath{(0,0),315} \polyline{(2,-0.15),(2,0.15)}
\rotatepath{(0,0),315} \polyline{(3,-0.15),(3,0.15)}
\rotatepath{(0,0),315} \polyline{(4,-0.15),(4,0.15)}
\point[3pt]{(0,0)}
\arrow \parafcn{5,40,5}{4*dir(t)}
\gclear \tlabelrect[cc](4,1.5){\scriptsize $\theta = \frac{\pi}{4}$}
\tcaption{ $x^2+2xy+y^2 -x\sqrt{2}+y\sqrt{2} -6= 0$}
\penwd{1.25pt}
\arrow \reverse \arrow \rotatepath{(0,0),45} \function{-2.75,2.75,0.1}{3-(x**2)}
\end{mfpic}


\item  $7x^2-4xy\sqrt{3}+3y^2-2x-2y\sqrt{3}-5= 0$ \\ becomes $\frac{(x'-2)^2}{9}+(y')^2 = 1$ after rotating \\ counter-clockwise through $\theta = \frac{\pi}{3}$

\begin{mfpic}[18]{-4}{6}{-5}{5}

\axes

\tlabel[cc](6,-0.5){\scriptsize $x$}
\tlabel[cc](0.5,5){\scriptsize $y$}
\tlabel[cc](3.5,5){\scriptsize $x'$}
\tlabel[cc](-4,3){\scriptsize $y'$}
\xmarks{-3,-2,-1,1,2,3,4,5}
\ymarks{-4,-3,-2,-1,1,2,3,4}
\point[4pt]{\plr{(-1,60)}, \plr{(5,60)}, (1.87,1.23), (0.13,2.23)}
\dashed \arrow \rotatepath{(0,0),60} \polyline{(-4,0), (6,0)}
\dashed \arrow \rotatepath{(0,0),60} \polyline{(0,-5), (0,5)}
\rotatepath{(0,0),60} \polyline{(1,-0.15),(1,0.15)}
\rotatepath{(0,0),60} \polyline{(2,-0.15),(2,0.15)}
\rotatepath{(0,0),60} \polyline{(3,-0.15),(3,0.15)}
\rotatepath{(0,0),60} \polyline{(4,-0.15),(4,0.15)}
\rotatepath{(0,0),60} \polyline{(5,-0.15),(5,0.15)}
\rotatepath{(0,0),150} \polyline{(1,-0.15),(1,0.15)}
\rotatepath{(0,0),150} \polyline{(2,-0.15),(2,0.15)}
\rotatepath{(0,0),150} \polyline{(3,-0.15),(3,0.15)}
\rotatepath{(0,0),150} \polyline{(4,-0.15),(4,0.15)}
\rotatepath{(0,0),240} \polyline{(1,-0.15),(1,0.15)}
\rotatepath{(0,0),240} \polyline{(2,-0.15),(2,0.15)}
\rotatepath{(0,0),240} \polyline{(3,-0.15),(3,0.15)}
\rotatepath{(0,0),330} \polyline{(1,-0.15),(1,0.15)}
\rotatepath{(0,0),330} \polyline{(2,-0.15),(2,0.15)}
\rotatepath{(0,0),330} \polyline{(3,-0.15),(3,0.15)}
\rotatepath{(0,0),330} \polyline{(4,-0.15),(4,0.15)}
\point[3pt]{(0,0)}
\arrow \parafcn{5,55,5}{5.5*dir(t)}
\gclear \tlabelrect[cc](4.5,3){\scriptsize $\theta = \frac{\pi}{3}$}
\tcaption{$7x^2-4xy\sqrt{3}+3y^2-2x-2y\sqrt{3}-5= 0$}
\penwd{1.25pt}
\rotatepath{(0,0),60} \ellipse{(2,0),3,1}
\end{mfpic}

\setcounter{HW}{\value{enumi}}
\end{enumerate}

\end{multicols}



\begin{multicols}{2}

\begin{enumerate}

\setcounter{enumi}{\value{HW}}


\item  $5x^2+6xy+5y^2 - 4\sqrt{2}x+4\sqrt{2}y = 0$ \\ becomes $(x')^2+\frac{(y'+2)^2}{4} = 1$  after rotating \\ counter-clockwise through $\theta = \frac{\pi}{4}$. 

\begin{mfpic}[18]{-5}{5}{-5}{5}

\axes

\tlabel[cc](5,-0.5){\scriptsize $x$}
\tlabel[cc](0.5,5){\scriptsize $y$}
\tlabel[cc](4,3.5){\scriptsize $x'$}
\tlabel[cc](-3,4){\scriptsize $y'$}
\xmarks{-4,-3,-2,-1,1,2,3,4}
\ymarks{-4,-3,-2,-1,1,2,3,4}
\point[4pt]{\plr{(0,0)}, \plr{(-2,135)}, \plr{(-4,135)}, \plr{(2.24,288)}, \plr{(2.24,341)}}
\dashed \arrow \rotatepath{(0,0),45} \polyline{(-5,0), (5,0)}
\dashed \arrow \rotatepath{(0,0),45} \polyline{(0,-5), (0,5)}
\rotatepath{(0,0),45} \polyline{(1,-0.15),(1,0.15)}
\rotatepath{(0,0),45} \polyline{(2,-0.15),(2,0.15)}
\rotatepath{(0,0),45} \polyline{(3,-0.15),(3,0.15)}
\rotatepath{(0,0),45} \polyline{(4,-0.15),(4,0.15)}
\rotatepath{(0,0),135} \polyline{(1,-0.15),(1,0.15)}
\rotatepath{(0,0),135} \polyline{(2,-0.15),(2,0.15)}
\rotatepath{(0,0),135} \polyline{(3,-0.15),(3,0.15)}
\rotatepath{(0,0),135} \polyline{(4,-0.15),(4,0.15)}
\rotatepath{(0,0),225} \polyline{(1,-0.15),(1,0.15)}
\rotatepath{(0,0),225} \polyline{(2,-0.15),(2,0.15)}
\rotatepath{(0,0),225} \polyline{(3,-0.15),(3,0.15)}
\rotatepath{(0,0),225} \polyline{(4,-0.15),(4,0.15)}
\rotatepath{(0,0),315} \polyline{(1,-0.15),(1,0.15)}
\rotatepath{(0,0),315} \polyline{(2,-0.15),(2,0.15)}
\rotatepath{(0,0),315} \polyline{(3,-0.15),(3,0.15)}
\rotatepath{(0,0),315} \polyline{(4,-0.15),(4,0.15)}
\point[3pt]{(0,0)}
\arrow \parafcn{5,40,5}{4*dir(t)}
\gclear \tlabelrect[cc](4,1.5){\scriptsize $\theta = \frac{\pi}{4}$}
\tcaption{ $5x^2+6xy+5y^2 - 4\sqrt{2}x+4\sqrt{2}y = 0$}
\penwd{1.25pt}
\rotatepath{(0,0),45} \ellipse{(0,-2),1,2}
\end{mfpic}

\item   $x^2+ 2\sqrt{3}xy+3y^2+ 2\sqrt{3}x-2y-16 = 0$  \\ becomes$(x')^2 = y'+4$ after rotating \\ counter-clockwise through $\theta = \frac{\pi}{3}$

\begin{mfpic}[18]{-4}{6}{-5}{5}

\axes

\tlabel[cc](6,-0.5){\scriptsize $x$}
\tlabel[cc](0.5,5){\scriptsize $y$}
\tlabel[cc](3.5,5){\scriptsize $x'$}
\tlabel[cc](-4,3){\scriptsize $y'$}
\xmarks{-3,-2,-1,1,2,3,4,5}
\ymarks{-4,-3,-2,-1,1,2,3,4}
\point[4pt]{\plr{(-4,150)}, \plr{(-2,60)}, \plr{(2,60)}}
\dashed \arrow \rotatepath{(0,0),60} \polyline{(-4,0), (6,0)}
\dashed \arrow \rotatepath{(0,0),60} \polyline{(0,-5), (0,5)}
\rotatepath{(0,0),60} \polyline{(1,-0.15),(1,0.15)}
\rotatepath{(0,0),60} \polyline{(2,-0.15),(2,0.15)}
\rotatepath{(0,0),60} \polyline{(3,-0.15),(3,0.15)}
\rotatepath{(0,0),60} \polyline{(4,-0.15),(4,0.15)}
\rotatepath{(0,0),60} \polyline{(5,-0.15),(5,0.15)}
\rotatepath{(0,0),150} \polyline{(1,-0.15),(1,0.15)}
\rotatepath{(0,0),150} \polyline{(2,-0.15),(2,0.15)}
\rotatepath{(0,0),150} \polyline{(3,-0.15),(3,0.15)}
\rotatepath{(0,0),150} \polyline{(4,-0.15),(4,0.15)}
\rotatepath{(0,0),240} \polyline{(1,-0.15),(1,0.15)}
\rotatepath{(0,0),240} \polyline{(2,-0.15),(2,0.15)}
\rotatepath{(0,0),240} \polyline{(3,-0.15),(3,0.15)}
\rotatepath{(0,0),330} \polyline{(1,-0.15),(1,0.15)}
\rotatepath{(0,0),330} \polyline{(2,-0.15),(2,0.15)}
\rotatepath{(0,0),330} \polyline{(3,-0.15),(3,0.15)}
\rotatepath{(0,0),330} \polyline{(4,-0.15),(4,0.15)}
\point[3pt]{(0,0)}
\arrow \parafcn{5,55,5}{5.5*dir(t)}
\gclear \tlabelrect[cc](4.5,3){\scriptsize $\theta = \frac{\pi}{3}$}
\tcaption{$x^2+ 2\sqrt{3}xy+3y^2+ 2\sqrt{3}x-2y-16 = 0$}
\penwd{1.25pt}
\arrow \reverse \arrow \rotatepath{(0,0),60} \function{-3,3,0.1}{0-4+(x**2)}
\end{mfpic}

\setcounter{HW}{\value{enumi}}
\end{enumerate}
\end{multicols}

\begin{multicols}{2}

\begin{enumerate}
\setcounter{enumi}{\value{HW}}

\item  $13x^2-34xy\sqrt{3}+47y^2 - 64=0$ \\ becomes $(y')^2 - \frac{(x')^2}{16}  =1 $  after rotating \\ counter-clockwise through $\theta = \frac{\pi}{6}$. 

\begin{mfpic}[18]{-5}{5}{-5}{5}

\axes

\tlabel[cc](5,-0.5){\scriptsize $x$}
\tlabel[cc](0.5,5){\scriptsize $y$}
\tlabel[cc](4.75,2.25){\scriptsize $x'$}
\tlabel[cc](-2.25,4.5){\scriptsize $y'$}
\xmarks{-4,-3,-2,-1,1,2,3,4}
\ymarks{-4,-3,-2,-1,1,2,3,4}
\point[4pt]{\plr{(1,120)}, \plr{(1,300)}}
\dashed \arrow \rotatepath{(0,0),30} \polyline{(-5,0), (5,0)}
\dashed \arrow \rotatepath{(0,0),30} \polyline{(0,-5), (0,5)}
\dotted \rotatepath{(0,0),30} \function{-4,4,0.1}{0.25*x}
\dotted \rotatepath{(0,0),30} \function{-4,4,0.1}{0-0.25*x}
\rotatepath{(0,0),30} \polyline{(1,-0.15),(1,0.15)}
\rotatepath{(0,0),30} \polyline{(2,-0.15),(2,0.15)}
\rotatepath{(0,0),30} \polyline{(3,-0.15),(3,0.15)}
\rotatepath{(0,0),30} \polyline{(4,-0.15),(4,0.15)}
\rotatepath{(0,0),120} \polyline{(1,-0.15),(1,0.15)}
\rotatepath{(0,0),120} \polyline{(2,-0.15),(2,0.15)}
\rotatepath{(0,0),120} \polyline{(3,-0.15),(3,0.15)}
\rotatepath{(0,0),120} \polyline{(4,-0.15),(4,0.15)}
\rotatepath{(0,0),210} \polyline{(1,-0.15),(1,0.15)}
\rotatepath{(0,0),210} \polyline{(2,-0.15),(2,0.15)}
\rotatepath{(0,0),210} \polyline{(3,-0.15),(3,0.15)}
\rotatepath{(0,0),210} \polyline{(4,-0.15),(4,0.15)}
\rotatepath{(0,0),300} \polyline{(1,-0.15),(1,0.15)}
\rotatepath{(0,0),300} \polyline{(2,-0.15),(2,0.15)}
\rotatepath{(0,0),300} \polyline{(3,-0.15),(3,0.15)}
\rotatepath{(0,0),300} \polyline{(4,-0.15),(4,0.15)}
\point[3pt]{(0,0)}
\arrow \parafcn{95,115,5}{4*dir(t)}
\gclear \tlabelrect[cc](-1,4.5){\scriptsize $\theta = \frac{\pi}{6}$}
\tcaption{$13x^2-34xy\sqrt{3}+47y^2 - 64=0$}
\penwd{1.25pt}
\arrow \reverse \arrow \rotatepath{(0,0),30} \parafcn{-45,45,5}{(4*tand(t), 1/cosd(t))}
\arrow \reverse \arrow \rotatepath{(0,0),30} \parafcn{-45,45,5}{(4*tand(t), 0-1/cosd(t))}
\end{mfpic}


\item   $x^2-2\sqrt{3} xy-y^2+8=0$ \\ becomes $\frac{(x')^2}{4} - \frac{(y')^2}{4} = 1$ after rotating \\ counter-clockwise through $\theta = \frac{\pi}{3}$

\begin{mfpic}[18]{-4}{6}{-5}{5}

\axes

\tlabel[cc](6,-0.5){\scriptsize $x$}
\tlabel[cc](0.5,5){\scriptsize $y$}
\tlabel[cc](3.5,5){\scriptsize $x'$}
\tlabel[cc](-4,3){\scriptsize $y'$}
\xmarks{-3,-2,-1,1,2,3,4,5}
\ymarks{-4,-3,-2,-1,1,2,3,4}
\point[4pt]{\plr{(-4,150)}, \plr{(-2,60)}, \plr{(2,60)}}
\dashed \arrow \rotatepath{(0,0),60} \polyline{(-4,0), (6,0)}
\dashed \arrow \rotatepath{(0,0),60} \polyline{(0,-5), (0,5)}
\dotted \rotatepath{(0,0),60} \function{-3,3,0.1}{x}
\dotted \rotatepath{(0,0),60} \function{-3,3,0.1}{-x}
\rotatepath{(0,0),60} \polyline{(1,-0.15),(1,0.15)}
\rotatepath{(0,0),60} \polyline{(2,-0.15),(2,0.15)}
\rotatepath{(0,0),60} \polyline{(3,-0.15),(3,0.15)}
\rotatepath{(0,0),60} \polyline{(4,-0.15),(4,0.15)}
\rotatepath{(0,0),60} \polyline{(5,-0.15),(5,0.15)}
\rotatepath{(0,0),150} \polyline{(1,-0.15),(1,0.15)}
\rotatepath{(0,0),150} \polyline{(2,-0.15),(2,0.15)}
\rotatepath{(0,0),150} \polyline{(3,-0.15),(3,0.15)}
\rotatepath{(0,0),150} \polyline{(4,-0.15),(4,0.15)}
\rotatepath{(0,0),240} \polyline{(1,-0.15),(1,0.15)}
\rotatepath{(0,0),240} \polyline{(2,-0.15),(2,0.15)}
\rotatepath{(0,0),240} \polyline{(3,-0.15),(3,0.15)}
\rotatepath{(0,0),330} \polyline{(1,-0.15),(1,0.15)}
\rotatepath{(0,0),330} \polyline{(2,-0.15),(2,0.15)}
\rotatepath{(0,0),330} \polyline{(3,-0.15),(3,0.15)}
\rotatepath{(0,0),330} \polyline{(4,-0.15),(4,0.15)}
\point[3pt]{(0,0)}
\arrow \parafcn{5,55,5}{5.5*dir(t)}
\gclear \tlabelrect[cc](4.5,3){\scriptsize $\theta = \frac{\pi}{3}$}
\tcaption{$x^2-2\sqrt{3} xy-y^2+8=0$}
\penwd{1.25pt}
\arrow \reverse \arrow \rotatepath{(0,0),60} \parafcn{-55,55,5}{(2/cosd(t), 2*tand(t))}
\arrow \reverse \arrow \rotatepath{(0,0),60} \parafcn{-55,55,5}{(-2/cosd(t), 2*tand(t))}
\end{mfpic}
\setcounter{HW}{\value{enumi}}
\end{enumerate}

\end{multicols}


\begin{multicols}{2}

\begin{enumerate}
\setcounter{enumi}{\value{HW}}

\item  $x^2-4xy+4y^2-2x\sqrt{5}-y\sqrt{5}=0$ \\ becomes $(y')^2=x$  after rotating \\ counter-clockwise through $\theta = \arctan\left(\frac{1}{2}\right)$. 

\begin{mfpic}[18]{-5}{5}{-5}{5}

\axes

\tlabel[cc](5,-0.5){\scriptsize $x$}
\tlabel[cc](0.5,5){\scriptsize $y$}
\tlabel[cc](4.75,2.25){\scriptsize $x'$}
\tlabel[cc](-2.25,4.5){\scriptsize $y'$}
\xmarks{-4,-3,-2,-1,1,2,3,4}
\ymarks{-4,-3,-2,-1,1,2,3,4}
\point[4pt]{\plr{(4.47,53.12)}, \plr{(4.47,0)}}
\dashed \arrow \rotatepath{(0,0),26.56} \polyline{(-5,0), (5,0)}
\dashed \arrow \rotatepath{(0,0),26.56} \polyline{(0,-5), (0,5)}
\rotatepath{(0,0),26.56} \polyline{(1,-0.15),(1,0.15)}
\rotatepath{(0,0),26.56} \polyline{(2,-0.15),(2,0.15)}
\rotatepath{(0,0),26.56} \polyline{(3,-0.15),(3,0.15)}
\rotatepath{(0,0),26.56} \polyline{(4,-0.15),(4,0.15)}
\rotatepath{(0,0),116.56} \polyline{(1,-0.15),(1,0.15)}
\rotatepath{(0,0),116.56} \polyline{(2,-0.15),(2,0.15)}
\rotatepath{(0,0),116.56} \polyline{(3,-0.15),(3,0.15)}
\rotatepath{(0,0),116.56} \polyline{(4,-0.15),(4,0.15)}
\rotatepath{(0,0),206.56} \polyline{(1,-0.15),(1,0.15)}
\rotatepath{(0,0),206.56} \polyline{(2,-0.15),(2,0.15)}
\rotatepath{(0,0),206.56} \polyline{(3,-0.15),(3,0.15)}
\rotatepath{(0,0),206.56} \polyline{(4,-0.15),(4,0.15)}
\rotatepath{(0,0),296.56} \polyline{(1,-0.15),(1,0.15)}
\rotatepath{(0,0),296.56} \polyline{(2,-0.15),(2,0.15)}
\rotatepath{(0,0),296.56} \polyline{(3,-0.15),(3,0.15)}
\rotatepath{(0,0),296.56} \polyline{(4,-0.15),(4,0.15)}
\point[3pt]{(0,0)}
\arrow \parafcn{5,20,5}{4*dir(t)}
\tlabel(4,.75){\scriptsize $\theta = \arctan\left(\frac{1}{2}\right)$}
\tcaption{$x^2-4xy+4y^2-2x\sqrt{5}-y\sqrt{5}=0$}
\penwd{1.25pt}
\arrow \reverse \arrow \rotatepath{(0,0),26.56} \parafcn{-2.25,2.25,0.1}{(t**2,t)}
\end{mfpic}


\item   $8x^2+12xy+17y^2 - 20 = 0$ \\ becomes $(x')^2 + \frac{(y')^2}{4} = 1$ after rotating \\ counter-clockwise through $\theta = \arctan(2)$

\begin{mfpic}[18]{-5}{5}{-5}{5}

\axes

\tlabel[cc](5,-0.5){\scriptsize $x$}
\tlabel[cc](0.5,5){\scriptsize $y$}
\tlabel[cc](2.75,4.25){\scriptsize $x'$}
\tlabel[cc](-4.25,2.75){\scriptsize $y'$}
\xmarks{-4,-3,-2,-1,1,2,3,4}
\ymarks{-4,-3,-2,-1,1,2,3,4}
\point[4pt]{\plr{(2,153)}, \plr{(-2,153)},\plr{(1,63)}, \plr{(-1,63)} }
\dashed \arrow \rotatepath{(0,0),63} \polyline{(-5,0), (5,0)}
\dashed \arrow \rotatepath{(0,0),63} \polyline{(0,-5), (0,5)}
\rotatepath{(0,0),63} \polyline{(1,-0.15),(1,0.15)}
\rotatepath{(0,0),63} \polyline{(2,-0.15),(2,0.15)}
\rotatepath{(0,0),63} \polyline{(3,-0.15),(3,0.15)}
\rotatepath{(0,0),63} \polyline{(4,-0.15),(4,0.15)}
\rotatepath{(0,0),153} \polyline{(1,-0.15),(1,0.15)}
\rotatepath{(0,0),153} \polyline{(2,-0.15),(2,0.15)}
\rotatepath{(0,0),153} \polyline{(3,-0.15),(3,0.15)}
\rotatepath{(0,0),153} \polyline{(4,-0.15),(4,0.15)}
\rotatepath{(0,0),243} \polyline{(1,-0.15),(1,0.15)}
\rotatepath{(0,0),243} \polyline{(2,-0.15),(2,0.15)}
\rotatepath{(0,0),243} \polyline{(3,-0.15),(3,0.15)}
\rotatepath{(0,0),243} \polyline{(4,-0.15),(4,0.15)}
\rotatepath{(0,0),333} \polyline{(1,-0.15),(1,0.15)}
\rotatepath{(0,0),333} \polyline{(2,-0.15),(2,0.15)}
\rotatepath{(0,0),333} \polyline{(3,-0.15),(3,0.15)}
\rotatepath{(0,0),333} \polyline{(4,-0.15),(4,0.15)}
\point[3pt]{(0,0)}
\arrow \parafcn{5,58,5}{2.5*dir(t)}
\gclear \tlabelrect[cc](2.5,1){\scriptsize $\theta = \arctan(2)$}
\tcaption{$8x^2+12xy+17y^2 - 20 = 0$}
\penwd{1.25pt}
\rotatepath{(0,0),63} \ellipse{(0,0),1,2}
\end{mfpic}

\setcounter{HW}{\value{enumi}}
\end{enumerate}

\end{multicols}

\newpage



\begin{multicols}{2} 

\begin{enumerate}
\setcounter{enumi}{\value{HW}}
\item  $r = \frac{2}{1-\cos(\theta)}$ is a parabola \\ directrix $x = -2$ ,  vertex $(-1,0)$ \\ focus $(0,0)$,   focal diameter $4$ \\

\begin{mfpic}[18]{-5}{5}{-5}{5}
\axes
\tlabel[cc](5,-0.5){\scriptsize $x$}
\tlabel[cc](0.5,5){\scriptsize $y$}
\xmarks{-4,-3,-2,-1,1,2,3,4}
\ymarks{-4,-3,-2,-1,1,2,3,4}
\plotsymbol[3pt]{Asterisk}{(0, 0)}
\point[4pt]{(0,2), (0,-2), (-1,0)}
\penwd{1.25pt}
\arrow \reverse \arrow \polyline{(-2,-5), (-2,5)}

\arrow \reverse \arrow \plrfcn{60,300,5}{2/(1-cosd(t))}

\tlabelsep{5pt}
\scriptsize
\axislabels {x}{{$-4 \hspace{7pt} $} -4,{$-3 \hspace{7pt} $} -3, {$-2\hspace{7pt} $} -2, {$-1 \hspace{7pt}$} -1, {$1$} 1, {$2$} 2, {$3$} 3, {$4$} 4}
\axislabels {y}{{$-4$} -4,{$-3$} -3,{$-2$} -2,{$-1$} -1,{$1$} 1, {$2$} 2, {$3$} 3, {$4$} 4}
\normalsize
\end{mfpic}

\vspace{.15in}

\item $r = \frac{3}{2 + \sin(\theta)} = \frac{\frac{3}{2}}{1 + \frac{1}{2} \sin(\theta)}$ is an ellipse \\ directrix $y = 3$ , vertices $(0,1)$, $(0,-3)$ \\ center $(0,-2)$ ,  foci $(0,0)$, $(0,-2)$ \\ minor axis length $2\sqrt{3}$ \\

\begin{mfpic}[18]{-5}{5}{-5}{5}
\axes
\tlabel[cc](5,-0.5){\scriptsize $x$}
\tlabel[cc](0.5,5){\scriptsize $y$}
\xmarks{-4,-3,-2,-1,1,2,3,4}
\ymarks{-4,-3,-2,1,2,3,4}
\plotsymbol[3pt]{Asterisk}{(0, 0), (0,-2)}
\plotsymbol[3pt]{Cross}{(0,-1)}
\point[4pt]{(0,1), (0,-3), (1.73,-1), (-1.73,-1)}
\arrow \reverse \arrow \polyline{(-5,3), (5,3)}
\penwd{1.25pt}
\plrfcn{0,360,5}{3/(2+sind(t))}

\tlabelsep{5pt}
\scriptsize
\axislabels {x}{{$-4 \hspace{7pt} $} -4,{$-3 \hspace{7pt} $} -3, {$-2\hspace{7pt} $} -2, {$-1 \hspace{7pt}$} -1, {$1$} 1, {$2$} 2, {$3$} 3, {$4$} 4}
\axislabels {y}{{$-4$} -4,{$-2$} -2,{$-1$} -1,{$1$} 1, {$2$} 2, {$3$} 3, {$4$} 4}
\normalsize
\end{mfpic}
\setcounter{HW}{\value{enumi}}
\end{enumerate}
\end{multicols}

\begin{multicols}{2} 

\begin{enumerate}
\setcounter{enumi}{\value{HW}}


\item $r = \frac{3}{2 - \cos(\theta)} = \frac{\frac{3}{2}}{1 - \frac{1}{2} \cos(\theta)}$ is an ellipse \\ directrix $x = -3$ , vertices $(-1,0)$, $(3,0)$ \\ center $(1,0)$ ,  foci $(0,0)$, $(2,0)$ \\ minor axis length $2\sqrt{3}$ \\

\begin{mfpic}[18]{-5}{5}{-5}{5}
\axes
\tlabel[cc](5,-0.5){\scriptsize $x$}
\tlabel[cc](0.5,5){\scriptsize $y$}
\xmarks{-4,-3,-2,-1,2,3,4}
\ymarks{-4,-3,-2,-1,1,2,3,4}
\plotsymbol[3pt]{Asterisk}{(0, 0), (2,0)}
\plotsymbol[3pt]{Cross}{(1,0)}
\point[4pt]{(-1,0), (3,0), (1,1.73), (1,-1.73)}
\arrow \reverse \arrow \polyline{(-3,-5), (-3,5)}
\penwd{1.25pt}
\plrfcn{0,360,5}{3/(2-cosd(t))}

\tlabelsep{5pt}
\scriptsize
\axislabels {x}{{$-4 \hspace{7pt} $} -4,{$-3 \hspace{7pt} $} -3, {$-2\hspace{7pt} $} -2, {$-1 \hspace{7pt}$} -1, {$1$} 1, {$2$} 2, {$3$} 3, {$4$} 4}
\axislabels {y}{{$-4$} -4,{$-2$} -2,{$-1$} -1,{$1$} 1, {$2$} 2, {$3$} 3, {$4$} 4}
\normalsize
\end{mfpic}




\item  $r = \frac{2}{1+\sin(\theta)}$ is a parabola \\ directrix $y=2$ ,  vertex $(0,1)$ \\ focus $(0,0)$,   focal diameter $4$ \\

\begin{mfpic}[18]{-5}{5}{-5}{5}
\axes
\tlabel[cc](5,-0.5){\scriptsize $x$}
\tlabel[cc](0.5,5){\scriptsize $y$}
\xmarks{-4,-3,-2,-1,1,2,3,4}
\ymarks{-4,-3,-2,-1,1,2,3,4}
\plotsymbol[3pt]{Asterisk}{(0, 0)}
\point[4pt]{(-2,0), (2,0), (0,1)}
\arrow \reverse \arrow \polyline{(-5,2), (5,2)}
\penwd{1.25pt}
\arrow \reverse \arrow \plrfcn{-30,210,5}{2/(1+sind(t))}

\tlabelsep{5pt}
\scriptsize
\axislabels {x}{{$-4 \hspace{7pt} $} -4,{$-3 \hspace{7pt} $} -3, {$-2\hspace{7pt} $} -2, {$-1 \hspace{7pt}$} -1, {$1$} 1, {$2$} 2, {$3$} 3, {$4$} 4}
\axislabels {y}{{$-4$} -4,{$-3$} -3,{$-2$} -2,{$-1$} -1,{$1$} 1, {$2$} 2, {$3$} 3, {$4$} 4}
\normalsize
\end{mfpic}

\setcounter{HW}{\value{enumi}}
\end{enumerate}
\end{multicols}


\begin{multicols}{2}
\begin{enumerate}
\setcounter{enumi}{\value{HW}}


\item  $r = \frac{4}{1+3\cos(\theta)}$ is a hyperbola \\ directrix $x = \frac{4}{3}$, vertices $(1,0)$, $(2,0)$ \\ center $\left(\frac{3}{2}, 0\right)$, foci $(0,0)$, $(3,0)$ \\ conjugate axis length $2\sqrt{2}$ \\


\begin{mfpic}[18]{-5}{5}{-5}{5}
\axes
\tlabel[cc](5,-0.5){\scriptsize $x$}
\tlabel[cc](0.5,5){\scriptsize $y$}
\xmarks{-4,-3,-2,-1,1,2,3,4}
\ymarks{-4,-3,-2,-1,1,2,3,4}
\plotsymbol[3pt]{Asterisk}{(0, 0), (3,0)}
\plotsymbol[3pt]{Cross}{(1.5,0)}
\point[4pt]{(1,0), (2,0)}
\arrow \reverse \arrow \polyline{(1.33,-5), (1.33,5)}
\dotted \function{0,3,0.1}{2.828*(x-1.5)}
\dotted \function{0,3,0.1}{0-2.828*(x-1.5)}
\tlabelsep{5pt}
\scriptsize
\axislabels {x}{{$-4 \hspace{7pt} $} -4,{$-3 \hspace{7pt} $} -3, {$-2\hspace{7pt} $} -2, {$-1 \hspace{7pt}$} -1, {$1$} 1, {$2$} 2, {$3$} 3, {$4$} 4}
\axislabels {y}{{$-4$} -4,{$-3$} -3,{$-2$} -2,{$-1$} -1,{$1$} 1, {$2$} 2, {$3$} 3, {$4$} 4}
\penwd{1.25pt}
\arrow \reverse \arrow \plrfcn{-88,88,5}{4/(1+3*cosd(t))}
\arrow \reverse \arrow \plrfcn{129,231.5,5}{4/(1+3*cosd(t))}
\normalsize
\end{mfpic}


\item  $r = \frac{2}{1-2\sin(\theta)}$ is a hyperbola \\ directrix $y = -1$, vertices $\left(0,-\frac{2}{3}\right)$, $(0,-2)$ \\ center $\left(0, -\frac{4}{3} \right)$, foci $(0,0)$, $\left(0, -\frac{8}{3}\right)$ \\ conjugate axis length $\frac{2\sqrt{3}}{3}$ \\


\begin{mfpic}[18]{-5}{5}{-5}{5}
\axes
\tlabel[cc](5,-0.5){\scriptsize $x$}
\tlabel[cc](0.5,5){\scriptsize $y$}
\xmarks{-4,-3,-2,-1,1,2,3,4}
\ymarks{-4,-3,-2,-1,1,2,3,4}
\plotsymbol[3pt]{Asterisk}{(0, 0), (0,-2.67)}
\plotsymbol[3pt]{Cross}{(0, -1.33)}
\point[4pt]{(0, -0.67), (0,-2)}
\arrow \reverse \arrow \polyline{(-5,-1), (5,-1)}
\dotted \function{-3,3,0.1}{0.577*x-1.33}
\dotted \function{-3,3,0.1}{-1.33-0.577x}
\tlabelsep{5pt}
\scriptsize
\axislabels {x}{{$-4 \hspace{7pt} $} -4,{$-3 \hspace{7pt} $} -3, {$-2\hspace{7pt} $} -2, {$-1 \hspace{7pt}$} -1, {$1$} 1, {$2$} 2, {$3$} 3, {$4$} 4}
\axislabels {y}{{$-4$} -4,{$-3$} -3,{$-2$} -2,{$-1$} -1,{$1$} 1, {$2$} 2, {$3$} 3, {$4$} 4}
\penwd{1.25pt}
\arrow \reverse \arrow \plrfcn{45,135,5}{2/(1-2*sind(t))}
\arrow \reverse \arrow \plrfcn{168,372,5}{2/(1-2*sind(t))}
\normalsize
\end{mfpic}


\setcounter{HW}{\value{enumi}}
\end{enumerate}
\end{multicols}



\begin{multicols}{2}
\begin{enumerate}
\setcounter{enumi}{\value{HW}}

\item   $r = \frac{2}{1 + \sin(\theta - \frac{\pi}{3})}$  is  \\
 the parabola $r = \frac{2}{1 + \sin(\theta)}$ \\
rotated through $\phi = \frac{\pi}{3}$  \\



\begin{mfpic}[18]{-4}{6}{-5}{5}

\axes

\tlabel[cc](6,-0.5){\scriptsize $x$}
\tlabel[cc](0.5,5){\scriptsize $y$}
\tlabel[cc](3.5,5){\scriptsize $x'$}
\tlabel[cc](-4,3){\scriptsize $y'$}
\xmarks{-3,-2,-1,1,2,3,4,5}
\ymarks{-4,-3,-2,-1,1,2,3,4}
\point[4pt]{\plr{(-2,60)}, \plr{(2,60)}}
\plotsymbol[3pt]{Asterisk}{(0, 0)}
\dashed \arrow \rotatepath{(0,0),60} \polyline{(-4,0), (6,0)}
\dashed \arrow \rotatepath{(0,0),60} \polyline{(0,-5), (0,5)}
\arrow \reverse \arrow \rotatepath{(0,0),60} \polyline{(-4,2), (4,2)}
\rotatepath{(0,0),60} \polyline{(1,-0.15),(1,0.15)}
\rotatepath{(0,0),60} \polyline{(2,-0.15),(2,0.15)}
\rotatepath{(0,0),60} \polyline{(3,-0.15),(3,0.15)}
\rotatepath{(0,0),60} \polyline{(4,-0.15),(4,0.15)}
\rotatepath{(0,0),60} \polyline{(5,-0.15),(5,0.15)}
\rotatepath{(0,0),150} \polyline{(1,-0.15),(1,0.15)}
\rotatepath{(0,0),150} \polyline{(2,-0.15),(2,0.15)}
\rotatepath{(0,0),150} \polyline{(3,-0.15),(3,0.15)}
\rotatepath{(0,0),150} \polyline{(4,-0.15),(4,0.15)}
\rotatepath{(0,0),240} \polyline{(1,-0.15),(1,0.15)}
\rotatepath{(0,0),240} \polyline{(2,-0.15),(2,0.15)}
\rotatepath{(0,0),240} \polyline{(3,-0.15),(3,0.15)}
\rotatepath{(0,0),330} \polyline{(1,-0.15),(1,0.15)}
\rotatepath{(0,0),330} \polyline{(2,-0.15),(2,0.15)}
\rotatepath{(0,0),330} \polyline{(3,-0.15),(3,0.15)}
\rotatepath{(0,0),330} \polyline{(4,-0.15),(4,0.15)}
\point[3pt]{(0,0)}
\arrow \parafcn{5,55,5}{5.5*dir(t)}
\gclear \tlabelrect[cc](4.5,3){\scriptsize $\phi = \frac{\pi}{3}$}
\penwd{1.25pt}
\arrow \reverse \arrow \plrfcn{25,275,5}{2/(1+sind(t-60))}
\end{mfpic}



\item  $r = \frac{6}{3 - \cos\left(\theta + \frac{\pi}{4}\right)}$  is the ellipse \\
$r = \frac{6}{3 - \cos\left(\theta \right)} = \frac{2}{1 - \frac{1}{3} \cos\left(\theta \right)}$ \\
rotated through $\phi = -\frac{\pi}{4}$  \\



\begin{mfpic}[18]{-5}{5}{-5}{5}
\axes
\tlabel[cc](5,-0.5){\scriptsize $x$}
\tlabel[cc](0.5,5){\scriptsize $y$}
\tlabel[cc](4,-3.5){\scriptsize $x'$}
\tlabel[cc](4,3){\scriptsize $y'$}
\xmarks{-4,-3,-2,-1,1,2,3,4}
\ymarks{-5,-4,-3,-2,-1,1,2,3,4}
\point[4pt]{\plr{(-1.5,-45)}, \plr{(3,-45)}, (-0.97, -2.03), (2.03, 0.97) }
\plotsymbol[3pt]{Asterisk}{(0, 0), \plr{(1.5,-45)}}


\tlabelsep{5pt}
\scriptsize
\axislabels {x}{{$-4 \hspace{7pt}$} -4, {$-3 \hspace{7pt} $} -3, {$-2\hspace{7pt} $} -2, {$-1 \hspace{7pt}$} -1, {$1$} 1, {$2$} 2, {$3$} 3, {$4$} 4}
\axislabels {y}{{$-4$} -4,{$-3$} -3, {$-2$} -2, {$-1$} -1, {$1$} 1, {$2$} 2, {$3$} 3, {$4$} 4}
\normalsize

\dashed \arrow \rotatepath{(0,0),-45} \polyline{(-5,0), (5,0)}
\dashed \arrow \rotatepath{(0,0),-45} \polyline{(0,-5), (0,5)}
\rotatepath{(0,0),-45} \polyline{(1,-0.15),(1,0.15)}
\rotatepath{(0,0),-45} \polyline{(2,-0.15),(2,0.15)}
\rotatepath{(0,0),-45} \polyline{(3,-0.15),(3,0.15)}
\rotatepath{(0,0),-45} \polyline{(4,-0.15),(4,0.15)}
\rotatepath{(0,0),-135} \polyline{(1,-0.15),(1,0.15)}
\rotatepath{(0,0),-135} \polyline{(2,-0.15),(2,0.15)}
\rotatepath{(0,0),-135} \polyline{(3,-0.15),(3,0.15)}
\rotatepath{(0,0),-135} \polyline{(4,-0.15),(4,0.15)}
\rotatepath{(0,0),-225} \polyline{(1,-0.15),(1,0.15)}
\rotatepath{(0,0),-225} \polyline{(2,-0.15),(2,0.15)}
\rotatepath{(0,0),-225} \polyline{(3,-0.15),(3,0.15)}
\rotatepath{(0,0),-225} \polyline{(4,-0.15),(4,0.15)}
\rotatepath{(0,0),-315} \polyline{(1,-0.15),(1,0.15)}
\rotatepath{(0,0),-315} \polyline{(2,-0.15),(2,0.15)}
\rotatepath{(0,0),-315} \polyline{(3,-0.15),(3,0.15)}
\rotatepath{(0,0),-315} \polyline{(4,-0.15),(4,0.15)}
\arrow \parafcn{-5,-40,-5}{4*dir(t)}
\gclear \tlabelrect[cc](4.5,-1.75){\scriptsize $\phi = -\frac{\pi}{4}$}
\penwd{1.25pt}
\plrfcn{0,360,5}{6/(3-cosd(t+45))}

\end{mfpic} 


\setcounter{HW}{\value{enumi}}

\end{enumerate}
\end{multicols}


\end{document}


\closegraphsfile