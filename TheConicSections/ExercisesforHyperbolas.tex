\documentclass{ximera}

\begin{document}
	\author{Stitz-Zeager}
	\xmtitle{TITLE}


\label{ExercisesforHyperbolas}

In Exercises \ref{graphhyperbolafirst} - \ref{graphhyperbolalast}, graph the hyperbola in the $xy$-plane.  Find the center, the lines which contain the transverse and conjugate axes, the vertices, the foci and the equations of the asymptotes.

\begin{multicols}{2}
\begin{enumerate}

\item $\dfrac{x^{2}}{16} - \dfrac{y^{2}}{9} = 1$ \label{graphhyperbolafirst} \label{oddhypeone}

\item $\dfrac{y^{2}}{9} - \dfrac{x^{2}}{16} = 1$ 



\setcounter{HW}{\value{enumi}}
\end{enumerate}
\end{multicols}

\begin{multicols}{2}
\begin{enumerate}
\setcounter{enumi}{\value{HW}}

\item $\dfrac{(x - 2)^{2}}{4} - \dfrac{(y + 3)^{2}}{9} = 1$  \label{oddhypethree}
\item $\dfrac{(y - 3)^{2}}{11} - \dfrac{(x - 1)^{2}}{10} = 1$


\setcounter{HW}{\value{enumi}}
\end{enumerate}
\end{multicols}

\begin{multicols}{2}
\begin{enumerate}
\setcounter{enumi}{\value{HW}}


\item $\dfrac{(x + 4)^{2}}{16} - (y - 4)^{2}= 1$  \label{oddhypefive}
\item  $\dfrac{(x+1)^2}{9} - \dfrac{(y-3)^2}{4} = 1$


\setcounter{HW}{\value{enumi}}
\end{enumerate}
\end{multicols}

\begin{multicols}{2}
\begin{enumerate}
\setcounter{enumi}{\value{HW}}
  
\item  $\dfrac{(y+2)^2}{16} - \dfrac{(x-5)^2}{20} = 1$  \label{oddhypeseven}
\item  $\dfrac{(x-4)^2}{8} - \dfrac{(y-2)^2}{18} = 1$ \label{graphhyperbolalast}

\setcounter{HW}{\value{enumi}}
\end{enumerate}
\end{multicols}

In Exercises \ref{stdfrmhypfirst} - \ref{stdfrmhyplast}, put the equation in standard form.  Find the center, the lines which contain the transverse and conjugate axes, the vertices, the foci and the equations of the asymptotes.\footnote{ \ldots assuming the equation were graphed in the $xy$-plane \ldots}

\begin{multicols}{2}
\begin{enumerate}
\setcounter{enumi}{\value{HW}}

\item $12x^{2} - 3y^{2} + 30y - 111 = 0$  \label{stdfrmhypfirst}  \label{oddhypenine}
\item $18y^{2} - 5x^{2} +  72y + 30x - 63= 0$

\setcounter{HW}{\value{enumi}}
\end{enumerate}
\end{multicols}

\begin{multicols}{2}
\begin{enumerate}
\setcounter{enumi}{\value{HW}}
 
\item $9x^2-25y^2-54x-50y-169 = 0$  \label{oddhypeeleven}
\item $-6x^2+5y^2-24x+40y+26=0$  \label{stdfrmhyplast}

\setcounter{HW}{\value{enumi}}
\end{enumerate}
\end{multicols}

\begin{enumerate}
\setcounter{enumi}{\value{HW}}

\item For each of the odd numbered equations given in Exercises \ref{oddhypeone} - \ref{oddhypeeleven}, find two or more explicit functions of $x$ represented by each of the equations.  (See Example \ref{horizontalparabolaex} in Section \ref{Parabolas}.)

\setcounter{HW}{\value{enumi}}
\end{enumerate}

In Exercises \ref{semihyperbolafunctionfirst} - \ref{semihyperbolafunctionlast}, graph each function by recognizing it as a portion of a hyperbola.

\begin{multicols}{2}
\begin{enumerate}
\setcounter{enumi}{\value{HW}}

\item   $f(x) = \sqrt{x^2-4}$ \label{semihyperbolafunctionfirst}
\item   $g(x) = -\sqrt{x^2-4x}$

\setcounter{HW}{\value{enumi}}
\end{enumerate}
\end{multicols}

\begin{multicols}{2}
\begin{enumerate}
\setcounter{enumi}{\value{HW}}

\item  $f(x) = -2\sqrt{x^2+2x-3}$
\item  $g(x) = -2 + 2\sqrt{x^2-9}$ \label{semihyperbolafunctionlast}

\setcounter{HW}{\value{enumi}}
\end{enumerate}
\end{multicols}

\enlargethispage{0.25in}

In Exercises \ref{buildhypefromgraphfirst} - \ref{buildhypefromgraphlast}, find an equation for the hyperbola whose graph is given.

\begin{multicols}{2}
\begin{enumerate}
\setcounter{enumi}{\value{HW}}

\item $~$ \label{buildhypefromgraphfirst}  % $\dfrac{x^2}{16} - \dfrac{y^2}{16} = 1$

\begin{mfpic}[8][13]{-6}{6}{-5}{5}
\axes
\tlabel[cc](6,-0.5){\scriptsize $x$}
\tlabel[cc](0.5,5){\scriptsize $y$}
\tlabel[cc](3.25, 3){\scriptsize $(5,3)$}
\tlabel[cc](-2.25, 0.75){\scriptsize $(-4,0)$}
\tlabel[cc](2.75, 0.75){\scriptsize $(4,0)$}
\xmarks{-5 step 1 until 5}
\ymarks{-4 step 1 until 4}
\tlpointsep{4pt}
\scriptsize
\axislabels {x}{   {$-2 \hspace{7pt}$} -2, {$2$} 2}
\axislabels {y}{ {$-4$} -4, {$-2$} -2, {$-3$} -3, {$-1$} -1,  {$2$} 2,  {$3$} 3,  {$4$} 4  }
\penwd{1.25pt}
\arrow \reverse \function{-6, -4, 0.1}{sqrt((x**2) - 16)}
\arrow \reverse \function{-6, -4, 0.1}{-sqrt((x**2) - 16)}
\arrow  \function{4, 6, 0.1}{sqrt((x**2) - 16)}
\arrow \function{4, 6, 0.1}{-sqrt((x**2) - 16)}
\point[4pt]{(4,0), (-4,0), (5,3)}
\normalsize
\end{mfpic} 

\vfill

\columnbreak

\item $~$  \label{buildhypefromgraphlast} % $\dfrac{(y-4)^2}{4} - \dfrac{(x-4)^2}{3} = 1$

\begin{mfpic}[13][10]{-3}{9}{-2}{10}
\axes
\tlabel[cc](9,-0.5){\scriptsize $x$}
\tlabel[cc](0.5,10){\scriptsize $y$}
\tlabel[cc](1.25, -1){\scriptsize $(1,0)$}
\tlabel[cc](6.5,-1){\scriptsize $(7,0)$}
\tlabel[cc](4, 5.25){\scriptsize $(4,6)$}
\tlabel[cc](4,2.75){\scriptsize $(4,2)$}
\xmarks{-2 step 1 until 8}
\ymarks{1 step 1 until 8}
\tlpointsep{4pt}
\scriptsize
\axislabels {x}{ {$3$} 3, {$4$} 4, {$5$} 5,  {$8$} 8, {$-1 \hspace{6pt}$} -1, {$-2 \hspace{6pt}$} -2}
\axislabels {y}{{$1$} 1, {$3$} 3, {$2$} 2,{$4$} 4, {$5$} 5,  {$6$} 6, {$7$} 7, {$8$} 8}
\penwd{1.25pt}
\arrow \reverse \arrow \function{-1,9,0.1}{4+0.666*sqrt(3*(x**2)-24*x+57)}
\arrow \reverse \arrow \function{-1,9,0.1}{4-0.666*sqrt(3*(x**2)-24*x+57)}
\point[4pt]{(4,2), (4,6), (1,0), (7,0)}
\normalsize
\end{mfpic} 

\setcounter{HW}{\value{enumi}}
\end{enumerate}
\end{multicols}

\newpage

In Exercises \ref{buildhypfirst} - \ref{buildhyplast},  find the standard form of the equation of the hyperbola which has the given properties.

\begin{enumerate}

\setcounter{enumi}{\value{HW}}

\item Center $(3, 7)$, Vertex $(3, 3)$, Focus $(3, 2)$  \label{buildhypfirst}
\item Vertex $(0, 1)$, Vertex $(8, 1)$, Focus $(-3, 1)$

\item Foci $(0, \pm 8)$, Vertices $(0, \pm 5)$.
\item Foci $(\pm 5, 0)$, length of the Conjugate Axis $6$


\item Vertices $(3,2)$, $(13,2)$; Endpoints of the Conjugate Axis $(8,4)$, $(8,0)$
\item Vertex $(-10, 5)$, Asymptotes $y = \pm \frac{1}{2}(x - 6) + 5$ \label{buildhyplast}

\setcounter{HW}{\value{enumi}}
\end{enumerate}


In Exercises \ref{generalconicfirst} - \ref{generalconiclast}, find the standard form of the equation using the guidelines on page \pageref{idconocsrulesofthumb} and then graph the conic section.

\begin{multicols}{2}
\begin{enumerate}
\setcounter{enumi}{\value{HW}}

\item $x^2-2x-4y-11=0$  \label{generalconicfirst}
\item $x^2 + y^2-8x+4y+11=0$

\setcounter{HW}{\value{enumi}}
\end{enumerate}
\end{multicols}

\begin{multicols}{2}
\begin{enumerate}
\setcounter{enumi}{\value{HW}}

\item  $9x^2 + 4y^2-36x+24y + 36=0$

\item $9x^2-4y^2-36x-24y-36=0$


\setcounter{HW}{\value{enumi}}
\end{enumerate}
\end{multicols}


\begin{multicols}{2}
\begin{enumerate}
\setcounter{enumi}{\value{HW}}

\item  $y^2+8y-4x+16=0$

\item  $4x^2+y^2-8x+4=0$


\setcounter{HW}{\value{enumi}}
\end{enumerate}
\end{multicols}


\begin{multicols}{2}
\begin{enumerate}
\setcounter{enumi}{\value{HW}}

\item   $4x^2+9y^2-8x+54y+49=0$

\item  $x^2 + y^2-6x+4y+14=0$

\setcounter{HW}{\value{enumi}}
\end{enumerate}
\end{multicols}


\begin{multicols}{2}
\begin{enumerate}
\setcounter{enumi}{\value{HW}}

\item  $2x^2+ 4y^2+12x-8y+25=0$

\item   $4x^2-5y^2-40x-20y+160=0$  \label{generalconiclast}


\setcounter{HW}{\value{enumi}}
\end{enumerate}
\end{multicols}




\begin{enumerate}

\setcounter{enumi}{\value{HW}}



\item The location of an earthquake's epicenter $-$ the point on the surface of the Earth directly above where the earthquake actually occurred $-$ can be determined by a process similar to how we located Sasquatch in Example \ref{FindtheSasquatch}.  (As we said back in Exercise \ref{Richterexercise} in Section \ref{LogarithmicFunctions}, earthquakes are complicated events and it is not our intent to provide a complete discussion of the science involved in them.  Instead, we refer the interested reader to a course in Geology or the U.S. Geological Survey's Earthquake Hazards Program found \href{http://earthquake.usgs.gov/}{\underline{here}}.)  Our technique works only for relatively small distances because we need to assume that the Earth is flat in order to use hyperbolas in the plane.  The P-waves (``P'' stands for Primary) of an earthquake in Sasquatchia travel at 6 kilometers per second.\footnote{Depending on the composition of the crust at a specific location, P-waves can travel between 5 kps and 8 kps.}  Station A records the waves first. Then Station B, which is 100 kilometers due north of Station A, records the waves 2 seconds later.  Station C, which is 150 kilometers due west of Station A records the waves 3 seconds after that (a total of 5 seconds after Station A). Where is the epicenter?

\item \label{hyperbolaeccentricity} The notion of eccentricity introduced for ellipses in Definition \ref{ellipseeccentricity} in Section \ref{Ellipses} is the same for hyperbolas in that we can define the eccentricity $e$ of a hyperbola as 

\[  e = \dfrac{\mbox{distance from the center to a focus}}{\mbox{distance from the center to a vertex}} \]
  

\begin{enumerate}

\item  With the help of your classmates, explain why $e > 1$ for any hyperbola.

\item  Find the equation of the hyperbola with vertices $(\pm 3,0)$ and eccentricity $e = 2$.

\item  With the help of your classmates, find the eccentricity of each of the hyperbolas in  Exercises \ref{graphhyperbolafirst} - \ref{graphhyperbolalast}.  What role does eccentricity play in the shape of the graphs?

\end{enumerate}

\item  On page \pageref{paraboloid} in Section \ref{Parabolas}, we discussed paraboloids of revolution when studying the design of satellite dishes and parabolic mirrors.  In much the same way, `natural draft' cooling towers are often shaped as \index{hyperboloid} \textbf{hyperboloids of revolution}.  Each vertical cross section of these towers is a hyperbola.  Suppose the a natural draft cooling tower has the cross section below. Suppose the tower is 450 feet wide at the base,  275 feet wide at the top, and 220 feet at its narrowest point (which occurs 330 feet above the ground.)  Determine the height of the tower to the nearest foot.
\begin{center}

\begin{mfpic}[20]{-3}{3}{0}{5}
\curve{(3,0), (1.5,3), (2,5)}
\curve{(-3,0), (-1.5,3), (-2,5)}
\point[4pt]{(3,0), (1.5,3), (2,5), (-3,0), (-1.5,3), (-2,5)}
\arrow \reverse \arrow \polyline{(-2.75,0), (2.75,0)}
\tlabel[cc](0,-0.5){\scriptsize $450$ ft}
\arrow \reverse \arrow \polyline{(-1.25,3), (1.25,3)}
\tlabel[cc](0,2.5){\scriptsize $220$ ft}
\arrow \reverse \arrow \polyline{(-1.75,5), (1.75,5)}
\tlabel[cc](0,5.5){\scriptsize $275$ ft}
\arrow \reverse \arrow \polyline{(5,0.25), (5,2.75)}
\dotted \polyline{(1.5,3), (5,3)}
\gclear \tlabelrect[cc]{(5,1.5)}{\scriptsize $330$ ft}
\end{mfpic}

\end{center} 

\item With the help of your classmates, research the Cassegrain Telescope.  It uses the reflective property of the hyperbola as well as that of the parabola to make an ingenious telescope.

\item \label{conicsclassificationnoxytermex} With the help of your classmates show that if $Ax^2 + Cy^2 + Dx + Ey + F = 0$ determines a non-degenerate conic\footnote{Recall that this means its graph is either a circle, parabola, ellipse or hyperbola.} then

\begin{itemize}

\item  $AC < 0$ means that the graph is a hyperbola

\item  $AC = 0$ means that the graph is a parabola

\item  $AC > 0$ means that the graph is an ellipse or circle

\end{itemize}

\textbf{NOTE:}  This result will be generalized in Theorem \ref{conicclassification} in Section \ref{rotationaxes}.

\end{enumerate}

\newpage

\subsection{Answers}

\begin{enumerate}

\item \begin{multicols}{2} \raggedcolumns
$\dfrac{x^{2}}{16} - \dfrac{y^{2}}{9} = 1$

Center $(0, 0)$\\
Transverse axis on $y = 0$\\
Conjugate axis on $x = 0$\\
Vertices $(4, 0), (-4, 0)$\\
Foci $(5, 0), (-5, 0)$\\
Asymptotes $y = \pm \frac{3}{4} x$\\

\begin{mfpic}[12][9]{-7}{7}{-7}{7}
\axes
\tlabel(7,-0.5){\scriptsize $x$}
\tlabel(0.5,7){\scriptsize $y$}
\xmarks{-6 step 1 until 6}
\ymarks{-6 step 1 until 6}
\point[4pt]{(4,0),(-4,0)}
\dotted[1pt, 3pt] \polyline{(-4,3), (4,3), (4, -3), (-4,-3), (-4,3)}
\arrow \reverse \arrow \dashed \function{-7,7,0.1}{0.75*x}
\arrow \reverse \arrow \dashed \function{-7,7,0.1}{-0.75*x}
\plotsymbol[4pt]{Cross}{(0,0)}
\plotsymbol[4pt]{Asterisk}{(5,0), (-5,0)}
\tlpointsep{4pt}
\tiny
\axislabels {x}{{$-6 \hspace{6pt}$} -6, {$-5 \hspace{6pt}$} -5, {$-4 \hspace{6pt}$} -4, {$-3 \hspace{6pt}$} -3, {$-2 \hspace{6pt}$} -2, {$-1 \hspace{6pt}$}{-1}, {$1$} 1, {$2$} 2, {$3$} 3, {$4$} 4, {$5$} 5, {$6$} 6}
\axislabels {y}{{$-6$} -6, {$-5$} -5, {$-4$} -4, {$-3$} -3, {$-2$} -2, {$-1$}{-1}, {$1$} 1, {$2$} 2, {$3$} 3, {$4$} 4, {$5$} 5, {$6$} 6}
\normalsize
\penwd{1.25pt}
\arrow \reverse \arrow \parafcn{-5,5,0.1}{(sqrt(16 + (1.778*(t**2))),t)}
\arrow \reverse \arrow \parafcn{-5,5,0.1}{(-sqrt(16 + (1.778*(t**2))),t)}
\end{mfpic}

\end{multicols}


\item \begin{multicols}{2} \raggedcolumns
$\dfrac{y^{2}}{9} - \dfrac{x^{2}}{16} = 1$

Center $(0, 0)$\\
Transverse axis on $x = 0$\\
Conjugate axis on $y = 0$\\
Vertices $(0, 3), (0, -3)$\\
Foci $(0, 5), (0, -5)$\\
Asymptotes $y = \pm \frac{3}{4} x$\\

\begin{mfpic}[12][9]{-7}{7}{-7}{7}
\axes
\tlabel(7,-0.5){\scriptsize $x$}
\tlabel(0.5,7){\scriptsize $y$}
\xmarks{-6 step 1 until 6}
\ymarks{-6 step 1 until 6}
\point[4pt]{(0,3),(0,-3)}
\dotted[1pt, 3pt] \polyline{(-4,3), (4,3), (4, -3), (-4,-3), (-4,3)}
\arrow \reverse \arrow \dashed \function{-7,7,0.1}{0.75*x}
\arrow \reverse \arrow \dashed \function{-7,7,0.1}{-0.75*x}
\plotsymbol[4pt]{Cross}{(0,0)}
\plotsymbol[4pt]{Asterisk}{(0,5), (0,-5)}
\tlpointsep{4pt}
\tiny
\axislabels {x}{{$-6 \hspace{6pt}$} -6, {$-5 \hspace{6pt}$} -5, {$-4 \hspace{6pt}$} -4, {$-3 \hspace{6pt}$} -3, {$-2 \hspace{6pt}$} -2, {$-1 \hspace{6pt}$}{-1}, {$1$} 1, {$2$} 2, {$3$} 3, {$4$} 4, {$5$} 5, {$6$} 6}
\axislabels {y}{{$-6$} -6, {$-5$} -5, {$-4$} -4, {$-3$} -3, {$-2$} -2, {$-1$}{-1}, {$1$} 1, {$2$} 2, {$3$} 3, {$4$} 4, {$5$} 5, {$6$} 6}
\normalsize
\penwd{1.25pt}
\arrow \reverse \arrow \function{-7,7,0.1}{sqrt(9 + (0.5625*(x**2)))}
\arrow \reverse \arrow \function{-7,7,0.1}{-sqrt(9 + (0.5625*(x**2)))}
\end{mfpic}

\end{multicols}

\item \begin{multicols}{2} \raggedcolumns
$\dfrac{(x - 2)^{2}}{4} - \dfrac{(y + 3)^{2}}{9} = 1$

Center $(2, -3)$\\
Transverse axis on $y = -3$\\
Conjugate axis on $x = 2$\\
Vertices $(0, -3), (4, -3)$\\
Foci $(2 + \sqrt{13}, -3), (2 - \sqrt{13}, -3)$\\
Asymptotes $y = \pm \frac{3}{2}(x - 2) - 3$\\

\begin{mfpic}[12][9]{-4}{8}{-11}{5}
\axes
\tlabel(8,-0.5){\scriptsize $x$}
\tlabel(0.5,5){\scriptsize $y$}
\xmarks{-3 step 1 until 7}
\ymarks{-10 step 1 until 4}
\point[4pt]{(0,-3),(4,-3)}
\dotted[1pt, 3pt] \polyline{(0,0), (4,0), (4, -6), (0,-6), (0,0)}
\arrow \function{4,7,0.1}{-3 + sqrt((2.25*((x - 2)**2)) - 9)}
\arrow \reverse \function{-3,0,0.1}{-3 + sqrt((2.25*((x - 2)**2)) - 9)}
\arrow \function{4,7,0.1}{-3 - sqrt((2.25*((x - 2)**2)) - 9)}
\arrow \reverse \function{-3,0,0.1}{-3 - sqrt((2.25*((x - 2)**2)) - 9)}
\arrow \reverse \arrow \dashed \function{-3,7,0.1}{-1.5*x}
\arrow \reverse \arrow \dashed \function{-3,7,0.1}{(1.5*x) - 6}
\plotsymbol[4pt]{Cross}{(2,-3)}
\plotsymbol[4pt]{Asterisk}{(5.60555,-3), (-1.60555,-3)}
\tlpointsep{4pt}
\tiny
\axislabels {x}{{$-3 \hspace{6pt}$} -3, {$-2 \hspace{6pt}$} -2, {$-1 \hspace{6pt}$}{-1}, {$1$} 1, {$2$} 2, {$3$} 3, {$4$} 4, {$5$} 5, {$6$} 6, {$7$} 7}
\axislabels {y}{{$-10$} -10, {$-9$} -9, {$-8$} -8, {$-7$} -7, {$-6$} -6, {$-5$} -5, {$-4$} -4, {$-3$} -3, {$-2$} -2, {$-1$}{-1}, {$1$} 1, {$2$} 2, {$3$} 3, {$4$} 4}
\normalsize
\penwd{1.25pt}
\arrow \function{4,7,0.1}{-3 + sqrt((2.25*((x - 2)**2)) - 9)}
\arrow \reverse \function{-3,0,0.1}{-3 + sqrt((2.25*((x - 2)**2)) - 9)}
\arrow \function{4,7,0.1}{-3 - sqrt((2.25*((x - 2)**2)) - 9)}
\arrow \reverse \function{-3,0,0.1}{-3 - sqrt((2.25*((x - 2)**2)) - 9)}
\end{mfpic}

\end{multicols}

\pagebreak

\item \begin{multicols}{2} \raggedcolumns
$\dfrac{(y - 3)^{2}}{11} - \dfrac{(x - 1)^{2}}{10} = 1$

Center $(1, 3)$\\
Transverse axis on $x = 1$\\
Conjugate axis on $y = 3$\\
Vertices $(1, 3 + \sqrt{11}), (1, 3 - \sqrt{11})$\\
Foci $(1, 3 + \sqrt{21}), (1, 3 - \sqrt{21})$\\
Asymptotes $y = \pm \frac{\sqrt{110}}{10}(x - 1) + 3$\\

\begin{mfpic}[12][9]{-6}{8}{-4}{10}
\axes
\tlabel(8,-0.5){\scriptsize $x$}
\tlabel(0.5,10){\scriptsize $y$}
\xmarks{-5 step 1 until 7}
\ymarks{-3 step 1 until 9}
\point[4pt]{(1,6.16228),(1,-0.16228)}
\dotted[1pt, 3pt] \polyline{(-2.3166,6.16228), (4.3166,6.16228), (4.3166,-0.16228), (-2.3166,-0.16228), (-2.3166,6.16228)}
\arrow \reverse \arrow \dashed \function{-6,8,0.1}{0.95346*(x - 1) + 3}
\arrow \reverse \arrow \dashed \function{-6,8,0.1}{-0.95346*(x - 1) + 3}
\plotsymbol[4pt]{Cross}{(1,3)}
\plotsymbol[4pt]{Asterisk}{(1,7.58258), (1,-1.58258)}
\tlpointsep{4pt}
\tiny
\axislabels {x}{{$-5 \hspace{6pt}$} -5, {$-4 \hspace{6pt}$} -4, {$-3 \hspace{6pt}$} -3, {$-2 \hspace{6pt}$} -2, {$-1 \hspace{6pt}$}{-1}, {$1$} 1, {$2$} 2, {$3$} 3, {$4$} 4, {$5$} 5, {$6$} 6, {$7$} 7}
\axislabels {y}{{$-3$} -3, {$-2$} -2, {$-1$}{-1}, {$1$} 1, {$2$} 2, {$3$} 3, {$4$} 4, {$5$} 5, {$6$} 6, {$7$} 7, {$8$} 8, {$9$} 9}
\normalsize
\penwd{1.25pt}
\arrow \reverse \arrow \function{-5.5,7.5,0.1}{3 + sqrt(10 + (0.90909*((x-1)**2)))}
\arrow \reverse \arrow \function{-5.5,7.5,0.1}{3 - sqrt(10 + (0.90909*((x-1)**2)))}
\end{mfpic}

\end{multicols}

\item \begin{multicols}{2} \raggedcolumns
$\dfrac{(x + 4)^{2}}{16} - \dfrac{(y - 4)^{2}}{1} = 1$

Center $(-4, 4)$\\
Transverse axis on $y = 4$\\
Conjugate axis on $x = -4$\\
Vertices $(-8, 4), (0, 4)$\\
Foci $(-4 + \sqrt{17}, 4), (-4 - \sqrt{17}, 4)$\\
Asymptotes $y = \pm \frac{1}{4}(x +4) +4$\\

\begin{mfpic}[12][12]{-12}{4}{-1}{6}
\axes
\tlabel(4,-0.5){\scriptsize $x$}
\tlabel(0.5,6){\scriptsize $y$}
\xmarks{-11 step 1 until 3}
\ymarks{1 step 1 until 5}
\point[4pt]{(-8,4),(0,4)}
\dotted[1pt, 3pt] \polyline{(-8,5), (0,5), (0,3), (-8,3), (-8,5)}
\arrow \reverse \arrow \dashed \function{-12,4,0.1}{0.25*x + 5}
\arrow \reverse \arrow \dashed \function{-12,4,0.1}{-0.25*x + 3}
\plotsymbol[4pt]{Cross}{(-4,4)}
\plotsymbol[4pt]{Asterisk}{(0.123106,4), (-8.123106,4)}
\tlpointsep{4pt}
\tiny
\axislabels {x}{{$-11 \hspace{6pt}$} -11, {$-10 \hspace{6pt}$} -10, {$-9 \hspace{6pt}$} -9, {$-8 \hspace{6pt}$} -8, {$-7 \hspace{6pt}$} -7, {$-6 \hspace{6pt}$} -6, {$-5 \hspace{6pt}$} -5, {$-4 \hspace{6pt}$} -4, {$-3 \hspace{6pt}$} -3, {$-2 \hspace{6pt}$} -2, {$-1 \hspace{6pt}$}{-1}, {$1$} 1, {$2$} 2, {$3$} 3}
\axislabels {y}{{$1$} 1, {$2$} 2, {$3$} 3, {$4$} 4, {$5$} 5}
\normalsize
\penwd{1.25pt}
\arrow \function{0,4,0.1}{4 + sqrt((0.0625*((x + 4)**2)) - 1)}
\arrow \reverse \function{-12,-8,0.1}{4 + sqrt((0.0625*((x + 4)**2)) - 1)}
\arrow \function{0,4,0.1}{4 - sqrt((0.0625*((x + 4)**2)) - 1)}
\arrow \reverse \function{-12,-8,0.1}{4 - sqrt((0.0625*((x + 4)**2)) - 1)}
\end{mfpic}

\end{multicols}

\item \begin{multicols}{2} \raggedcolumns
$\dfrac{(x+1)^2}{9} - \dfrac{(y-3)^2}{4} = 1$

Center $(-1, 3)$\\
Transverse axis on $y=3$\\
Conjugate axis on $x=-1$\\
Vertices $(2, 3), (-4, 3)$\\
Foci $\left(-1+\sqrt{13}, 3\right), \left(-1-\sqrt{13}, 3\right)$\\
Asymptotes $y = \pm \frac{2}{3} (x+1)+3$\\

\begin{mfpic}[12]{-8}{6}{-1}{6}
\axes
\tlabel(6,-0.5){\scriptsize $x$}
\tlabel(0.5,6){\scriptsize $y$}
\xmarks{-7 step 1 until 5}
\ymarks{1 step 1 until 5}
\point[4pt]{(2,3),(-4,3)}
\dotted \polyline{(-4,1), (-4,5), (2, 5), (2,1),(-4,1)}
\arrow \reverse \arrow \dashed \function{-7,5,0.1}{(2/3)*(x+1)+3}
\arrow \reverse \arrow \dashed \function{-7,5,0.1}{3-(2/3)*(x+1)}
\plotsymbol[4pt]{Cross}{(-1,3)}
\plotsymbol[4pt]{Asterisk}{(2.6056,3), (-4.6056,3)}
\tlpointsep{4pt}
\tiny
\axislabels {x}{{$-7 \hspace{6pt}$} -7,{$-6 \hspace{6pt}$} -6,{$-5 \hspace{6pt}$} -5, {$-4 \hspace{6pt}$} -4, {$-3 \hspace{6pt}$} -3, {$-2 \hspace{6pt}$} -2, {$-1 \hspace{6pt}$}{-1}, {$1$} 1, {$2$} 2, {$3$} 3, {$4$} 4, {$5$} 5}
\axislabels {y}{{$1$} 1, {$2$} 2, {$3$} 3, {$4$} 4, {$5$} 5}
\normalsize
\penwd{1.25pt}
\arrow \reverse \arrow \parafcn{-1.4,1.4,0.1}{(3*cosh(t)-1,2*sinh(t)+3)}
\arrow \reverse \arrow \parafcn{-1.4,1.4,0.1}{(-3*cosh(t)-1,2*sinh(t)+3)}
\end{mfpic}

\end{multicols}

\item \begin{multicols}{2} \raggedcolumns
$\dfrac{(y+2)^2}{16} - \dfrac{(x-5)^2}{20} = 1$

Center $(5, -2)$\\
Transverse axis on $x=5$\\
Conjugate axis on $y=-2$\\
Vertices $(5,2), (5,-6)$\\
Foci $\left(5,4 \right), \left(5,-8\right)$\\
Asymptotes $y = \pm \frac{2\sqrt{5}}{5} (x-5)-2$\\

\begin{mfpic}[10]{-2}{12}{-9}{5}
\axes
\tlabel(12,-0.5){\scriptsize $x$}
\tlabel(0.5,5){\scriptsize $y$}
\xmarks{-1 step 1 until 11}
\ymarks{-8 step 1 until 4}
\point[4pt]{(5,2),(5,-6)}
\dotted \polyline{(0.5279,-6), (0.5279,2), (9.4721, 2), (9.4721,-6),(0.5279,-6)}
\arrow \reverse \arrow \dashed \function{-2,12,0.1}{0.89442*(x-5)-2}
\arrow \reverse \arrow \dashed \function{-2,12,0.1}{0-2-0.89442*(x-5)}
\plotsymbol[4pt]{Cross}{(5,-2)}
\plotsymbol[4pt]{Asterisk}{(5,4), (5,-8)}
\tlpointsep{4pt}
\tiny
\axislabels {x}{{$-1 \hspace{6pt}$}{-1}, {$1$} 1, {$2$} 2, {$3$} 3, {$4$} 4, {$5$} 5, {$6$} 6, {$7$} 7, {$8$} 8, {$9$} 9, {$10$} 10, {$11$} 11}
\axislabels {y}{{$-8$} -8,{$-7$} -7,{$-6$} -6,{$-5$} -5,{$-4$} -4,{$-3$} -3,{$-2$} -2,{$-1$} -1,{$1$} 1, {$2$} 2, {$3$} 3, {$4$} 4}
\normalsize
\penwd{1.25pt}
\arrow \reverse \arrow \parafcn{-1.25,1.25,0.1}{(4.4721*sinh(t)+5,4*cosh(t)-2)}
\arrow \reverse \arrow \parafcn{-1.25,1.25,0.1}{(4.4721*sinh(t)+5,0-4*cosh(t)-2)}
\end{mfpic}

\end{multicols}

\pagebreak

\item \begin{multicols}{2} \raggedcolumns
$\dfrac{(x-4)^2}{8} - \dfrac{(y-2)^2}{18} = 1$

Center $(4, 2)$\\
Transverse axis on $y=2$\\
Conjugate axis on $x=4$\\
Vertices $\left(4+2\sqrt{2},2\right), \left(4-2\sqrt{2},2\right)$\\
Foci $\left(4+\sqrt{26},2 \right), \left(4-\sqrt{26},2\right)$\\
Asymptotes $y = \pm \frac{3}{2} (x-4)+2$\\

\begin{mfpic}[10]{-3}{11}{-4}{10}
\axes
\tlabel(9,-0.5){\scriptsize $x$}
\tlabel(0.5,10){\scriptsize $y$}
\xmarks{-2 step 1 until 10}
\ymarks{-3 step 1 until 9}
\point[4pt]{(6.8284,2),(1.1716,2)}
\dotted \polyline{(1.1716,-2.2426), (1.1716,6.2426), (6.8284, 6.2426), (6.8284,-2.2426),(1.1716,-2.2426)}
\arrow \reverse \arrow \dashed \function{-1,9,0.1}{1.5*(x-4)+2}
\arrow \reverse \arrow \dashed \function{-1,9,0.1}{2-1.5*(x-4)}
\plotsymbol[4pt]{Cross}{(4,2)}
\plotsymbol[4pt]{Asterisk}{(9.0990,2), (-1.0990,2)}
\tlpointsep{4pt}
\tiny
\axislabels {x}{{$-2 \hspace{6pt}$} -2, {$-1 \hspace{6pt}$} -1, {$1$} 1, {$2$} 2, {$3$} 3, {$4$} 4, {$5$} 5, {$6$} 6, {$7$} 7, {$8$} 8, {$9$} 9, {$10$} 10}
\axislabels {y}{ -3,{$-2$} -2,{$-1$} -1,{$1$} 1, {$2$} 2, {$3$} 3, {$4$} 4, {$5$} 5, {$6$} 6, {$7$} 7, {$8$} 8, {$9$} 9}
\normalsize
\penwd{1.25pt}
\arrow \reverse \arrow \parafcn{-1.2,1.2,0.1}{(2.8284*cosh(t)+4,4.2426*sinh(t)+2)}
\arrow \reverse \arrow \parafcn{-1.2,1.2,0.1}{(0-2.8284*cosh(t)+4,4.2426*sinh(t)+2)}
\end{mfpic}

\end{multicols}

\setcounter{HW}{\value{enumi}}
\end{enumerate}

\begin{multicols}{2}
\begin{enumerate}
\setcounter{enumi}{\value{HW}}


\item $\dfrac{x^{2}}{3} - \dfrac{(y - 5)^{2}}{12} = 1$

Center $(0, 5)$\\
Transverse axis on $y = 5$\\
Conjugate axis on $x = 0$\\
Vertices $(\sqrt{3}, 5), (-\sqrt{3}, 5)$\\
Foci $(\sqrt{15}, 5), (-\sqrt{15}, 5)$\\
Asymptotes $y = \pm 2x + 5$

\item $\dfrac{(y + 2)^{2}}{5} - \dfrac{(x - 3)^{2}}{18} = 1$

Center $(3, -2)$\\
Transverse axis on $x = 3$\\
Conjugate axis on $y = -2$\\
Vertices $(3, -2 + \sqrt{5}), (3, -2 - \sqrt{5})$\\
Foci $(3, -2 + \sqrt{23}), (3, -2 - \sqrt{23})$\\
Asymptotes $y = \pm \frac{\sqrt{10}}{6}(x - 3) - 2$


\setcounter{HW}{\value{enumi}}
\end{enumerate}
\end{multicols}


\begin{multicols}{2}
\begin{enumerate}
\setcounter{enumi}{\value{HW}}

\item $\dfrac{(x-3)^{2}}{25} - \dfrac{(y+1)^{2}}{9} = 1$

Center $(3, -1)$\\
Transverse axis on $y=-1$\\
Conjugate axis on $x=3$\\
Vertices $(8, -1), (-2, -1)$\\
Foci $\left(3+\sqrt{34}, -1 \right), \left(3-\sqrt{34}, -1 \right)$\\
Asymptotes $y = \pm \frac{3}{5}(x - 3) - 1$



\item $\dfrac{(y+4)^{2}}{6} - \dfrac{(x+2)^{2}}{5} = 1$

Center $(-2, -4)$\\
Transverse axis on $x=-2$\\
Conjugate axis on $y=-4$\\
Vertices $\left(-2,-4+\sqrt{6} \right), \left(-2,-4-\sqrt{6} \right)$\\
Foci $\left(-2, -4+\sqrt{11} \right), \left(-2, -4-\sqrt{11} \right)$\\
Asymptotes $y = \pm \frac{\sqrt{30}}{5}(x + 2) - 4$


\setcounter{HW}{\value{enumi}}
\end{enumerate}
\end{multicols}


\begin{enumerate}
\setcounter{enumi}{\value{HW}}

\item $~$


For number \ref{oddhypeone}:

\begin{itemize}

\item  $f(x) = \frac{3}{4} \sqrt{x^2-16}$ represents the upper half of the hyperbola.

\item  $g(x) =  -\frac{3}{4} \sqrt{x^2-16}$ represents the lower half of the hyperbola.

\end{itemize}

For number \ref{oddhypethree}:

\begin{itemize}

\item  $f(x) = -3 + \frac{3}{2} \sqrt{x^2-4x} $ represents the upper half of the hyperbola.

\item  $g(x) = -3 -  \frac{3}{2} \sqrt{x^2-4x} $ represents the lower half of the hyperbola.

\end{itemize}
For number \ref{oddhypefive}:

\begin{itemize}

\item  $f(x) = 4 + \frac{1}{4} \sqrt{x^2+8x} $ represents the upper half of the hyperbola.

\item  $g(x) = 4 - \frac{1}{4} \sqrt{x^2+8x}  $ represents the lower half of the hyperbola.

\end{itemize}


For number \ref{oddhypeseven}:

\begin{itemize}

\item  $f(x) = -2 + \frac{2}{5} \sqrt{5x^2-50x+225}$ represents the upper half of the hyperbola.

\item  $g(x) =-2 - \frac{2}{5} \sqrt{5x^2-50x+225}$ represents the lower half of the hyperbola.

\end{itemize}

For number \ref{oddhypenine}:

\begin{itemize}

\item  $f(x) = 5+2 \sqrt{x^2-3}$ represents the upper half of the hyperbola.

\item  $g(x) = 5-2 \sqrt{x^2-3}$ represents the lower half of the hyperbola.

\end{itemize}

For number \ref{oddhypeeleven}:

\begin{itemize}

\item  $f(x) = -1+ \frac{3}{5} \sqrt{x^2-6x-16}$ represents the upper half of the hyperbola.

\item  $g(x) = -1- \frac{3}{5} \sqrt{x^2-6x-16} $ represents the lower half of the hyperbola.

\end{itemize}


\setcounter{HW}{\value{enumi}}
\end{enumerate}

\begin{multicols}{2}
\begin{enumerate}
\setcounter{enumi}{\value{HW}}

\item  $f(x) = \sqrt{x^2-4}$

\begin{mfpic}[15]{-4}{4}{-1}{5}
\axes
\tlabel[cc](4,-0.5){\scriptsize $x$}
\tlabel[cc](0.5,5){\scriptsize $y$}
\tlabel[cc](-2, -0.5){\scriptsize $(-2,0)$}
\tlabel[cc](2, -0.5){\scriptsize $(2, 0)$}
\xmarks{-3 step 1 until 3}
\ymarks{0 step 1 until 4}
\tlpointsep{4pt}
\scriptsize
%\axislabels {x}{ {$-3 \hspace{7pt}$} -3,  {$1$} 1,  {$3$} 3}
\axislabels {y}{ {$1$} 1, {$2$} 2, {$3$} 3}
\penwd{1.25pt}
\arrow \reverse \ \function{-4,-2,0.1}{sqrt(x^2-4)}
\arrow \function{2,4,0.1}{sqrt(x^2-4)}
\point[4pt]{(-2,0), (2,0)}
\normalsize
\end{mfpic} 

\vfill

\columnbreak

\item $g(x) = - \sqrt{x^2-4x}$

\begin{mfpic}[15]{-3}{7}{-4.5}{1.5}
\axes
\tlabel[cc](7,-0.5){\scriptsize $x$}
\tlabel[cc](0.5,1.5){\scriptsize $y$}
\tlabel[cc](-0.75, 0.5){\scriptsize $(0,0)$}
\tlabel[cc](4, 0.5){\scriptsize $(4,0)$}
\xmarks{-2 step 1 until 6}
\ymarks{-4 step 1 until -1}
\tlpointsep{4pt}
\scriptsize
\axislabels {x}{{$1$} 1, {$2$} 2, {$3$} 3, {$5$} 5, {$6$} 6, {$-1 \hspace{6pt}$} -1,{$-2 \hspace{6pt}$} -2 }
\axislabels {y}{{$-3$} -3, {$-4$} -4}
\penwd{1.25pt}
\arrow \reverse \function{-3, 0, 0.1}{-sqrt((x**2)-4*x)}
\arrow  \function{4, 7, 0.1}{-sqrt((x**2)-4*x)}
\point[4pt]{(0,0), (4,0)}
\normalsize
\end{mfpic} 

\setcounter{HW}{\value{enumi}}
\end{enumerate}
\end{multicols}

\begin{multicols}{2}
\begin{enumerate}
\setcounter{enumi}{\value{HW}}

\item  $f(x) = -2\sqrt{x^2+2x-3}$

\begin{mfpic}[15]{-5}{3}{-6}{2}
\axes
\tlabel[cc](3,-0.5){\scriptsize $x$}
\tlabel[cc](0.5,2){\scriptsize $y$}
\tlabel[cc](-3, 0.5){\scriptsize $(-3,0)$}
\tlabel[cc](1, 0.5){\scriptsize $(1,0)$}
\xmarks{-4 step 1 until 2}
\ymarks{-5 step 1 until 1}
\tlpointsep{4pt}
\scriptsize
\axislabels {x}{ {$-1 \hspace{7pt}$} -1, {$-2 \hspace{7pt}$} -2,{$-4 \hspace{7pt}$} -4, {$2$} 2}
\axislabels {y}{ {$1$} 1, {$-2$} -2,{$-3$} -3,  {$-1$} -1,  {$-5$} -5,  {$-4$} -4}
\penwd{1.25pt}
\arrow \reverse \function{-4.5,-3,0.1}{-2*sqrt((x**2)+(2*x)-3)}
\arrow \function{1,2.5,0.1}{-2*sqrt((x**2)+(2*x)-3)}
\point[4pt]{(-3,0), (1,0)}
\normalsize
\end{mfpic} 


\item  $g(x) = -2 + 2\sqrt{x^2-9}$

\begin{mfpic}[15]{-5}{5}{-3}{5}
\axes
\tlabel[cc](5,-0.5){\scriptsize $x$}
\tlabel[cc](0.5,5){\scriptsize $y$}
\tlabel[cc](-3, -2.5){\scriptsize $(-3,-2)$}
\tlabel[cc](3, -2.5){\scriptsize $(3,-2)$}
\xmarks{-4 step 1 until 4}
\ymarks{-2 step 1 until 4}
\tlpointsep{4pt}
\scriptsize
\axislabels {x}{ {$-1 \hspace{7pt}$} -1, {$-2 \hspace{7pt}$} -2,{$-4 \hspace{7pt}$} -4, {$2$} 2, {$1$} 1, {$4$} 4}
\axislabels {y}{ {$4$} 4, {$2$} 2,{$3$} 3,{$1$} 1, {$-1$} -1, {$-2$} -2}
\penwd{1.25pt}
\arrow \reverse \function{-5,-3,0.1}{-2+2*sqrt((x**2)-9)}
\arrow  \function{3,5,0.1}{-2+2*sqrt((x**2)-9)}
\point[4pt]{(-3,-2), (3,-2)}
\normalsize
\end{mfpic} 

\setcounter{HW}{\value{enumi}}
\end{enumerate}
\end{multicols}

\begin{multicols}{2}
\begin{enumerate}
\setcounter{enumi}{\value{HW}}

\item  $\dfrac{x^2}{16} - \dfrac{y^2}{16} = 1$

\item $\dfrac{(y-4)^2}{4} - \dfrac{(x-4)^2}{3} = 1$


\setcounter{HW}{\value{enumi}}
\end{enumerate}
\end{multicols}


\begin{multicols}{2}
\begin{enumerate}
\setcounter{enumi}{\value{HW}}

\item $\dfrac{(y - 7)^{2}}{16} - \dfrac{(x - 3)^{2}}{9} = 1$
\item $\dfrac{(x - 4)^{2}}{16} - \dfrac{(y - 1)^{2}}{33} = 1$

\setcounter{HW}{\value{enumi}}
\end{enumerate}
\end{multicols}

\begin{multicols}{2}
\begin{enumerate}
\setcounter{enumi}{\value{HW}}

\item $\dfrac{y^{2}}{25} - \dfrac{x^{2}}{39} = 1$
\item $\dfrac{x^{2}}{16} - \dfrac{y^{2}}{9} = 1$

\setcounter{HW}{\value{enumi}}
\end{enumerate}
\end{multicols}

\begin{multicols}{2}
\begin{enumerate}
\setcounter{enumi}{\value{HW}}

\item $\dfrac{(x - 8)^{2}}{25} - \dfrac{(y - 2)^{2}}{4} = 1$
\item $\dfrac{(x - 6)^{2}}{256} - \dfrac{(y - 5)^{2}}{64} = 1$

\setcounter{HW}{\value{enumi}}
\end{enumerate}
\end{multicols}


\begin{multicols}{2}
\begin{enumerate}
\setcounter{enumi}{\value{HW}}

\item $(x-1)^2 = 4(y+3)$ \\

\begin{mfpic}[15]{-4}{5}{-5}{1}
\axes
\xmarks{-3 step 1 until 4}
\ymarks{-4 step 1 until 0}
\arrow \reverse \arrow \polyline{(-5,-4),(5,-4)}
\plotsymbol[4pt]{Asterisk}{(1,-2)}
\tlabel(5,-0.5){\scriptsize $x$}
\tlabel(0.5,1){\scriptsize $y$}
\point[4pt]{(3,-2),(1,-3),(-1,-2)}
\tlpointsep{4pt}
\tiny
\axislabels {x}{{$-3 \hspace{7pt}$} -3, {$-2 \hspace{7pt}$} -2, {$-1 \hspace{7pt}$} -1, {$1$} 1, {$2$} 2, {$3$} 3, {$4$} 4}
\axislabels {y}{{$-4$} -4, {$-3$} -3, {$-2$} -2, {$-1$} -1}
\normalsize
\penwd{1.25pt}
\arrow \reverse \arrow \function{-3,5,0.1}{((x -1)**2)/4 - 3}
\end{mfpic}


\vfill

\columnbreak


\item $(x-4)^2+(y+2)^2 = 9$ \\

\begin{mfpic}[20]{-1}{8}{-6}{2}
\axes
\plotsymbol[4pt]{Cross}{(4,-2)}
\xmarks{1,4,7}
\ymarks{-5,-2,1}
\tlabel(8,-0.5){\scriptsize $x$}
\tlabel(0.5,2){\scriptsize $y$}
\tlpointsep{4pt}
\tiny
\axislabels {x}{{$1$} 1,{$4$} 4,{$7$} 7}
\axislabels {y}{{$-5$} -5, {$-2$} -2, {$1$} 1 }
\normalsize
\penwd{1.25pt}
\circle{(4,-2),3}
\end{mfpic}

\setcounter{HW}{\value{enumi}}
\end{enumerate}
\end{multicols}

\begin{multicols}{2}
\begin{enumerate}
\setcounter{enumi}{\value{HW}}

\item $\dfrac{(x - 2)^{2}}{4} + \dfrac{(y + 3)^{2}}{9} = 1$\\

\begin{mfpic}[15]{-1}{5}{-7}{1}
\axes
\tlabel(5,-0.5){\scriptsize $x$}
\tlabel(0.5,1){\scriptsize $y$}
\xmarks{1 step 1 until 4}
\ymarks{-6 step 1 until 0}
\plotsymbol[4pt]{Asterisk}{(2, -0.7639), (2,-5.23606)}
\plotsymbol[4pt]{Cross}{(2,-3)}
\point[4pt]{(2,0), (2,-6), (0,-3), (4,-3)}
\tlpointsep{4pt}
\scriptsize
\axislabels {x}{{$1$} 1, {$2$} 2, {$3$} 3, {$4$} 4}
\axislabels {y}{{$-6$} -6, {$-5$} -5, {$-4$} -4, {$-3$} -3, {$-2$} -2, {$-1$} -1}
\normalsize
\penwd{1.25pt}
\ellipse{(2,-3),2,3}
\end{mfpic} 

\vfill

\columnbreak

\item $\dfrac{(x - 2)^{2}}{4} - \dfrac{(y + 3)^{2}}{9} = 1$ \\

\begin{mfpic}[12][9]{-4}{8}{-11}{5}
\axes
\tlabel(8,-0.5){\scriptsize $x$}
\tlabel(0.5,5){\scriptsize $y$}
\xmarks{-3 step 1 until 7}
\ymarks{-10 step 1 until 4}
\point[4pt]{(0,-3),(4,-3)}
\dotted[1pt, 3pt] \polyline{(0,0), (4,0), (4, -6), (0,-6), (0,0)}
\arrow \reverse \arrow \dashed \function{-3,7,0.1}{-1.5*x}
\arrow \reverse \arrow \dashed \function{-3,7,0.1}{(1.5*x) - 6}
\plotsymbol[4pt]{Cross}{(2,-3)}
\plotsymbol[4pt]{Asterisk}{(5.60555,-3), (-1.60555,-3)}
\tlpointsep{4pt}
\tiny
\axislabels {x}{{$-3 \hspace{6pt}$} -3, {$-2 \hspace{6pt}$} -2, {$-1 \hspace{6pt}$}{-1}, {$1$} 1, {$2$} 2, {$3$} 3, {$4$} 4, {$5$} 5, {$6$} 6, {$7$} 7}
\axislabels {y}{{$-10$} -10, {$-9$} -9, {$-8$} -8, {$-7$} -7, {$-6$} -6, {$-5$} -5, {$-4$} -4, {$-3$} -3, {$-2$} -2, {$-1$}{-1}, {$1$} 1, {$2$} 2, {$3$} 3, {$4$} 4}
\normalsize
\penwd{1.25pt}
\arrow \function{4,7,0.1}{-3 + sqrt((2.25*((x - 2)**2)) - 9)}
\arrow \reverse \function{-3,0,0.1}{-3 + sqrt((2.25*((x - 2)**2)) - 9)}
\arrow \function{4,7,0.1}{-3 - sqrt((2.25*((x - 2)**2)) - 9)}
\arrow \reverse \function{-3,0,0.1}{-3 - sqrt((2.25*((x - 2)**2)) - 9)}
\end{mfpic}


\setcounter{HW}{\value{enumi}}
\end{enumerate}
\end{multicols}

\newpage

\begin{multicols}{2}
\begin{enumerate}
\setcounter{enumi}{\value{HW}}

\item $(y + 4)^{2} = 4x$ \\

\begin{mfpic}[15]{-2}{5}{-9}{1}
\axes
\xmarks{-1 step 1 until 4}
\ymarks{-8 step 1 until 0}
\arrow \reverse \arrow \polyline{(-1,-9),(-1,1)}
\plotsymbol[4pt]{Asterisk}{(1,-4)}
\tlabel(5,-0.5){\scriptsize $x$}
\tlabel(0.5,1){\scriptsize $y$}
\point[4pt]{(0,-4),(1,-2),(1,-6)}
\tlpointsep{4pt}
\tiny
\axislabels {x}{{$-1 \hspace{7pt}$} -1, {$1$} 1, {$2$} 2, {$3$} 3, {$4$} 4}
\axislabels {y}{{$-8$} -8, {$-7$} -7, {$-6$} -6, {$-5$} -5, {$-4$} -4, {$-3$} -3, {$-2$} -2, {$-1$} -1}
\normalsize
\penwd{1.25pt}
\arrow \function{0,5,0.1}{-4-(2*sqrt(x))}
\arrow \function{0,5,0.1}{-4+(2*sqrt(x))}
\end{mfpic}

\vfill

\columnbreak

\item $\dfrac{(x-1)^2}{1}+\dfrac{y^2}{4}=0$ \\
The graph is the point $(1,0)$ only.

\setcounter{HW}{\value{enumi}}
\end{enumerate}
\end{multicols}


\begin{multicols}{2}
\begin{enumerate}
\setcounter{enumi}{\value{HW}}

\item $\dfrac{(x-1)^2}{9}+\dfrac{(y+3)^2}{4} = 1$ \\


\begin{mfpic}[18]{-3}{5}{-6}{1}
\axes
\tlabel(5,-0.25){\scriptsize $x$}
\tlabel(0.25,1){\scriptsize $y$}
\xmarks{-2 step 1 until 4}
\ymarks{-5 step 1 until -1}
\plotsymbol[4pt]{Asterisk}{(3.2361,-3), (-1.2361,-3)}
\plotsymbol[4pt]{Cross}{(1,-3)}
\point[4pt]{(4,-3), (-2,-3), (1,-1), (1,-5)}
\tlpointsep{4pt}
\scriptsize
\axislabels {x}{{$-2 \hspace{7pt}$} -2, {$-1 \hspace{7pt}$} -1, {$1$} 1, {$2$} 2, {$3$} 3, {$4$} 4}
\axislabels {y}{{$-5$} -5, {$-4$} -4, {$-3$} -3, {$-2$} -2, {$-1$} -1}
\normalsize
\penwd{1.25pt}
\ellipse{(1,-3),3,2}
\end{mfpic} 

\vfill

\columnbreak

\item  $(x-3)^2+(y+2)^2=-1$ \\
There is no graph.

\setcounter{HW}{\value{enumi}}
\end{enumerate}
\end{multicols}


\begin{multicols}{2}
\begin{enumerate}
\setcounter{enumi}{\value{HW}}

\item $\dfrac{(x+3)^2}{2}+\dfrac{(y-1)^2}{1} = -\dfrac{3}{4}$ \\
There is no graph.

\vfill

\columnbreak

\item $\dfrac{(y+2)^2}{16} - \dfrac{(x-5)^2}{20} = 1$ \\

\begin{mfpic}[10]{-2}{12}{-9}{5}
\axes
\tlabel(12,-0.5){\scriptsize $x$}
\tlabel(0.5,5){\scriptsize $y$}
\xmarks{-1 step 1 until 11}
\ymarks{-8 step 1 until 4}
\point[4pt]{(5,2),(5,-6)}
\dotted \polyline{(0.5279,-6), (0.5279,2), (9.4721, 2), (9.4721,-6),(0.5279,-6)}
\arrow \reverse \arrow \dashed \function{-2,12,0.1}{0.89442*(x-5)-2}
\arrow \reverse \arrow \dashed \function{-2,12,0.1}{0-2-0.89442*(x-5)}
\plotsymbol[4pt]{Cross}{(5,-2)}
\plotsymbol[4pt]{Asterisk}{(5,4), (5,-8)}
\tlpointsep{4pt}
\tiny
\axislabels {x}{{$-1 \hspace{6pt}$}{-1}, {$1$} 1, {$2$} 2, {$3$} 3, {$4$} 4, {$5$} 5, {$6$} 6, {$7$} 7, {$8$} 8, {$9$} 9, {$10$} 10, {$11$} 11}
\axislabels {y}{{$-8$} -8,{$-7$} -7,{$-6$} -6,{$-5$} -5,{$-4$} -4,{$-3$} -3,{$-2$} -2,{$-1$} -1,{$1$} 1, {$2$} 2, {$3$} 3, {$4$} 4}
\normalsize
\penwd{1.25pt}
\arrow \reverse \arrow \parafcn{-1.25,1.25,0.1}{(4.4721*sinh(t)+5,4*cosh(t)-2)}
\arrow \reverse \arrow \parafcn{-1.25,1.25,0.1}{(4.4721*sinh(t)+5,0-4*cosh(t)-2)}
\end{mfpic}

\setcounter{HW}{\value{enumi}}
\end{enumerate}
\end{multicols}

\begin{enumerate}
\setcounter{enumi}{\value{HW}}

\addtocounter{enumi}{1}

\item By placing Station A at $(0, -50)$ and Station B at $(0, 50)$, the two second time difference yields the hyperbola $\frac{y^{2}}{36} - \frac{x^{2}}{2464} = 1$ with foci A and B and center $(0, 0)$.  Placing Station C at $(-150, -50)$ and using foci A and C gives us a center of $(-75, -50)$ and the hyperbola $\frac{(x + 75)^{2}}{225} - \frac{(y + 50)^{2}}{5400} = 1$.  The point of intersection of these two hyperbolas which is closer to A than B and closer to A than C is $(-57.8444, -9.21336)$ so that is the epicenter.  

\item  \begin{enumerate} \setcounter{enumii}{1} \item $\dfrac{x^2}{9} - \dfrac{y^2}{27} = 1$. \end{enumerate}

\item  The tower may be modeled (approximately)\footnote{The exact value underneath $(y - 330)^{2}$ is $\frac{52707600}{1541}$ in case you need more precision.} by $\frac{x^2}{12100} - \frac{(y-330)^2}{34203} = 1$.  To find the height, we plug in $x = 137.5$ which yields $y \approx 191$ or $y \approx 469$.  Since the top of the tower is above the narrowest point, we get the tower is approximately 469 feet tall.

\end{enumerate}


\end{document}
